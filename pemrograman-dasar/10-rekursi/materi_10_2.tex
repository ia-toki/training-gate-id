\documentclass{beamer}
\usetheme{tokitex}

\usepackage{pgf}
\usepackage{graphics}
\usepackage{multirow}
\usepackage{multicol}
\usepackage{tabto}
\usepackage{tikz}
\usepackage[english,bahasa]{babel}
\newtranslation[to=bahasa]{Section}{Bagian}
\newtranslation[to=bahasa]{Subsection}{Subbagian}

\usepackage{listings}
\usepackage{color}

\definecolor{dkgreen}{rgb}{0,0.6,0}
\definecolor{gray}{rgb}{0.5,0.5,0.5}
\definecolor{mauve}{rgb}{0.58,0,0.82}

\lstset{frame=tb,
    language=pascal,
    showstringspaces=false,
    columns=flexible,
    basicstyle={\small\ttfamily},
    numbers=none,
    numberstyle=\tiny\color{gray},
    keywordstyle=\color{blue},
    commentstyle=\color{dkgreen},
    stringstyle=\color{mauve},
    breaklines=true,
    breakatwhitespace=true,
    tabsize=2
}

\title{Rekursi Lanjutan}
\author{Tim Olimpiade Komputer Indonesia}
\date{}

\begin{document}

\begin{frame}
\titlepage
\end{frame}

\begin{frame}
\frametitle{Pendahuluan}
Melalui dokumen ini, kalian akan:
\begin{itemize}
    \item Mempelajari konsep rekursi yang bercabang.
    \item Belajar merancang fungsi/prosedur rekursif yang lebih sulit.
\end{itemize}
\end{frame}

\begin{frame}
\frametitle{Soal: Fibonacci}
Deskripsi:
\begin{itemize}
        \item Deret fibonacci adalah deret di mana suatu bilangan adalah penjumlahan dari dua bilangan sebelumnya kecuali dua bilangan pertama
        \item Jika $f_N$ adalah bilangan fibonacci ke-$N$, maka $f_0$ = 0, $f_1$ = 1, dan $f_N$ = $f_{N-1}$ + $f_{N-2}$ untuk $N > 1$
        \item Beberapa bilangan pertama dari deret fibonacci adalah 0, 1, 1, 2, 3, 5, 8, 13, 21 dst.
        \item Bantu Pak Dengklek cari bilangan fibonacci ke-$N$
        \item Contoh : Bilangan fibonacci ke-6 adalah 8. Perhatikan bahwa indeks dimulai dari 0
\end{itemize}
\end{frame}

\begin{frame}
\frametitle{Soal: Fibonacci (lanj.) }
Format masukan:
\begin{itemize}
        \item Sebuah baris berisi sebuah bilangan $N$
\end{itemize}
Format keluaran:
\begin{itemize}
        \item Sebuah baris berisi bilangan fibonacci ke-$N$
\end{itemize}
Batasan:
\begin{itemize}
        \item $0 \le N \le 20$
\end{itemize}
\end{frame}

\begin{frame}
\frametitle{Solusi}
\begin{itemize}
    \item Permasalahan: Bagaimana cara mendapatkan nilai dari dua bilangan fibonacci sebelum bilangan fibonacci ke-$N$?
    \item Apakah kita bisa melakukan rekursi untuk mencari bilangan fibonacci ke-${N-1}$ dan ke-${N-2}$?
\end{itemize}
\end{frame}

\begin{frame}
\frametitle{Penjelasan solusi Rekursi}
\begin{itemize}
    \item Tahap 1 : Base Case
    \begin {itemize}
        \item Jika kita lihat di batasan soal, kita akan menjawab soal dengan $f_0$ sampai $f_{20}$
        \item Pertanyaan : Dari $f_0$ sampai $f_{20}$, mana yang paling gampang dihitung?
        \item Yang paling gampang dihitung adalah bilangan $f_0$ dan $f_1$, jadi keduanya adalah base casenya
        \item Nilai dari $f_0$ = 0 dan $f_1$ = 1 atau dengan kata lain $f_N = N$ untuk $N \le 1$
    \end {itemize}
\end{itemize}
\end{frame}

\begin{frame}
\frametitle{Penjelasan solusi Rekursi (lanj.) }
\begin{itemize}
    \item Tahap 2 : Bukan Base Case
    \begin {itemize}
        \item Bagaimana kalau fungsi kita menerima $N$ dengan $N > 1$?
        \item Seperti yang sudah didefinisikan, $f_N$ = $f_{N-1}$ + $f_{N-2}$ untuk $N > 1$
        \item Contoh: $f_5$ = $f_4$ + $f_3$
        \item Dengan ini, kita telah mengecilkan permasalahan dari $f_5$ menjadi $f_4$ dan $f_3$ di mana keduanya ini akan dijumlahkan
        \item $f_4$ dan $f_3$ sendiri akan dikecilkan terus menerus sampai menjadi $base case$ yaitu $f_0$ atau $f_1$
    \end {itemize}
\end{itemize}
\end{frame}

\begin{frame}[fragile]
\frametitle{Contoh Solusi: fibonacci\_rekursi.pas}
Perhatikan contoh berikut:
\begin{lstlisting}
  (* Fibonacci dengan rekursi : *)
   function fibonacci(N:longint):longint;
   begin
       if (N <= 1) then (* Base Case *)
           fibonacci := N
       else
           fibonacci := fibonacci(N-1) + fibonacci(N-2); (* Proses Mengecilkan Masalah*)
   end;
\end{lstlisting}
\end{frame}

\begin{frame}[fragile]
\frametitle{Penjelasan solusi rekursi}
Alur eksekusi rekursi dapat dimodelkan dengan pohon rekursi. Berikut adalah contoh pohonnya untuk $f_4$.
\begin{tikzpicture}[level/.style={sibling distance=45mm/#1}]
\node [circle,draw] (z){$f_4$}
  child {node [circle,draw] (a) {$f_3$}
    child {node [circle,draw] (b) {$f_2$}
      child {node [circle,draw] (j) {$f_1$}}
      child {node [circle,draw] (j) {$f_0$}}
      }
    child {node [circle,draw] (g) {$f_1$}}
  }
  child {node [circle,draw] (j) {$f_2$}
      child {node [circle,draw] (k) {$f_1$}}
      child {node [circle,draw] (k) {$f_0$}}
  };
\end{tikzpicture}
\end{frame}

\begin{frame}
\frametitle{Penjelasan solusi rekursi (lanj.)}
Kira-kira seperti inilah yang terjadi pada program ketika menghitung $f_4$
\begin {itemize}
    \item Kita akan panggil fibonacci(4)
    \item fibonacci(4) akan mengecek, apakah N=4 adalah $base case$
    \item Ternyata bukan! Karena baru $base case$ jika N $\le$ 1
    \item Untuk menjawab nilai $f_4$, fibonacci(4)  butuh nilai dari $f_3$ dan $f_2$
    \item Dijalankanlah "fibonacci := fibonacci(3) + fibonacci(2)"
    \item Alur rekursi berjalan berurutan. Karena fibonacci(3) dipanggil duluan maka itulah yang dieksekusi duluan
\end{itemize}
\end{frame}

\begin{frame}
\frametitle{Penjelasan solusi rekursi (lanj.)}
\begin {itemize}
    \item fibonacci(3) akan menjalankan "fibonacci := fibonacci(2) + fibonacci(1)"
    \item fibonacci(2) akan menjalankan "fibonacci := fibonacci(1) + fibonacci(0)"
    \item Ternyata ketika fibonacci(1), 1 termasuk $base case$ sehingga fibonacci(1) = 1
    \item Ternyata ketika fibonacci(0), 0 termasuk $base case$ sehingga fibonacci(0) = 0
    \item Balik lagi ke fibonacci(2), nilai dari fibonacci(2) menjadi 1 + 0 = 1
\end{itemize}
\end{frame}

\begin{frame}
\frametitle{Penjelasan solusi rekursi (lanj.)}
\begin {itemize}
    \item Balik lagi ke fibonacci(3), nilai fibonacci(2) sudah ada nilainya tetapi fibonacci(1) belum sehingga akan dipanggil dan akan menghasilkan 1 (karena $base case$). Nilai dari fibonacci(3) menjadi 1 + 1 = 2
    \item Balik lagi ke fibonacci(4), nilai fibonacci(3) sudah ada nilainya tetapi fibonacci(2) belum sehingga akan dipanggil dan akan menghasilkan 1 (alur yang terjadi sama seperti sebelumnya). Nilai dari fibonacci(4) menjadi 2 + 1 = 3
\end{itemize}
Tantangan : Cobalah membuat pohon rekursi dan alurnya untuk menghitung $f_5$
\end{frame}

\begin{frame}
\frametitle{Kompleksitas solusi}
\begin {itemize}
   \item Perhatikan pohon rekursi yang sebelumnya. Berapa kali fungsi akan dipanggil?
   \item Setiap fungsi yang dipanggil dia akan bercabang 2 dan kedalamannya adalah $N$
   \item Walaupun totalnya tidak tepat $2^{N}$, namun secara kasar kita bisa sebutkan kira-kira fungsi akan dipanggil $2^{N}$ kali
   \item Kompleksitasnya menjadi O($2^{N}$)
\end {itemize}
\end{frame}

\begin{frame}
\frametitle{Masalah}
\begin{itemize}
    \item Perhatikan kembali pohon rekursi yang sebelumnya. Terlihat $f_2$ dihitung dua kali
    \item Ketika $f_N$ cukup besar maka akan ada banyak fungsi dengan parameter yang sama namun dihitung berkali-kali. Hal inilah yang membuat kompleksitasnya boros
    \item Kita bisa mereduksi kompleksitas rekursi fibonacci menjadi O($N$) dengan teknik yang akan kita pelajari pada pemrograman lanjut
    \item Kita juga bisa membuat solusi O($N$)-nya dengan menghitung nilai fibonacci dengan $loop$. Dapatkah Anda membuatnya?
\end{itemize}
\end{frame}

\begin{frame}
\frametitle{$<$TODO: Konten Wajib$>$}
\begin{itemize}
    \item diberikan N. cetak seluruh permutasi dari 1..N
    \item jelaskan tahap demi tahap, seperti pada berkas "Rekursi (2).html" (lihat  https://www.dropbox.com/sh/zxcd1uct6rf3mio/AAD5v7W-EaypSSh7uP8D5FRVa?dl=0 sebagai acuan)
\end{itemize}
\end{frame}

\end{document}
