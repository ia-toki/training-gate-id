\documentclass{beamer}
\usetheme{tokitex}

\usepackage{pgf}
\usepackage{graphics}
\usepackage{multirow}
\usepackage{multicol}
\usepackage{tabto}
\usepackage[english,bahasa]{babel}
\newtranslation[to=bahasa]{Section}{Bagian}
\newtranslation[to=bahasa]{Subsection}{Subbagian}

\usepackage{listings}
\usepackage{color}

\definecolor{dkgreen}{rgb}{0,0.6,0}
\definecolor{gray}{rgb}{0.5,0.5,0.5}
\definecolor{mauve}{rgb}{0.58,0,0.82}

\lstset{frame=tb,
  language=pascal,
  showstringspaces=false,
  columns=flexible,
  basicstyle={\small\ttfamily},
  numbers=none,
  numberstyle=\tiny\color{gray},
  keywordstyle=\color{blue},
  commentstyle=\color{dkgreen},
  stringstyle=\color{mauve},
  breaklines=true,
  breakatwhitespace=true,
  tabsize=2
}

\title{Rekursi Lanjutan}
\author{Tim Olimpiade Komputer Indonesia}

\begin{document}

\begin{frame}
\titlepage
\end{frame}

\begin{frame}
\frametitle{Pendahuluan}
Melalui dokumen ini, kalian akan:
\begin{itemize}
	\item Mempelajari konsep rekursi yang bercabang.
	\item Belajar merancang fungsi/prosedur rekursif yang lebih sulit.
\end{itemize}
\end{frame}

\begin{frame}
\frametitle{$<$TODO: Konten Wajib$>$}
\begin{itemize}
	\item jelaskan cari fibonacci ke-n dengan rekursif
	\item beri contoh kode, jelaskan alur eksekusi (dengan diagram seperti pohon)
	\item jelaskan analisis kompleksitasnya, yaitu O(2\^n)
\end{itemize}
\end{frame}

\begin{frame}
\frametitle{$<$TODO: Konten Wajib$>$}
\begin{itemize}
	\item diberikan N. cetak seluruh permutasi dari 1..N
	\item jelaskan tahap demi tahap, seperti pada berkas "Rekursi (2).html" (cek folder referensi)
\end{itemize}
\end{frame}

\end{document}