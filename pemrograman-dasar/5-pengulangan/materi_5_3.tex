\documentclass{beamer}
\usetheme{tokitex}
\usepackage{graphics}
\usepackage{multirow}
\usepackage{tabto}
\usepackage[english,bahasa]{babel}
\newtranslation[to=bahasa]{Section}{Bagian}
\newtranslation[to=bahasa]{Subsection}{Subbagian}

\usepackage{listings}
\usepackage{color}

\definecolor{dkgreen}{rgb}{0,0.6,0}
\definecolor{gray}{rgb}{0.5,0.5,0.5}
\definecolor{mauve}{rgb}{0.58,0,0.82}

\lstset{frame=tb,
  aboveskip=0mm,
  belowskip=0mm,
  language=pascal,
  showstringspaces=false,
  columns=flexible,
  basicstyle={\small\ttfamily},
  numbers=none,
  numberstyle=\tiny\color{gray},
  keywordstyle=\color{blue},
  commentstyle=\color{dkgreen},
  stringstyle=\color{mauve},
  breaklines=true,
  breakatwhitespace=true,
  tabsize=2
}

\title{Analisis Kompleksitas}
\author{Tim Olimpiade Komputer Indonesia}

\begin{document}

\begin{frame}
\titlepage
\end{frame}

\begin{frame}
\frametitle{Pendahuluan}
Melalui dokumen ini, kalian akan:
\begin{itemize}
	\item Memahami konsep analisis kompleksitas.
	\item Mampu menganalisis kompleksitas untuk memperkirakan \textit{runtime} eksekusi program.
\end{itemize}
\end{frame}

\begin{frame}
\frametitle{$<$TODO: Konten Wajib$>$}
\begin{itemize}
	\item pengenalan notasi Big-Oh
	\item penjelasan tidak perlu terlalu teoritis, cukup menekankan:
	\begin{itemize}
		\item konstanta pada Big-Oh bisa dibuang, jelaskan juga kenapa
		\item ambil suku terbesar saja, contoh $O(N^2 + N^3)$ bisa dianggap $O(N^3)$, jelaskan juga kenapa
	\end{itemize}
	\item berikan contoh perhitungan kompleksitas
\end{itemize}
\end{frame}

\begin{frame}
\frametitle{$<$TODO: Contoh Wajib$>$}
\begin{itemize}
	\item hitung kompleksitas for-loop biasa untuk mencetak angka 1..N
	\item for loop 2 lapis untuk mencetak matriks N*N, kompleksitasnya $O(N^2)$
	\item for loop 2 lapis untuk mencetak matriks N*M, kompleksitasnya $O(NM)$
	\item loop "while (N $>$ 0) \{ N = N/2; \}", kompleksitasnya $O(\log{N})$, lalu jelaskan kenapa tidak ditulis $O(\log_2{N})$
	\item for loop yang isinya loop while seperti poin sebelumnya, kompleksitasnya $O(N\log{N})$
	\item loop "while (i*i $<$ N) \{ i++; \}", kompleksitasnya $O(\sqrt{N})$
\end{itemize}
\end{frame}

\begin{frame}
\frametitle{$<$TODO: Konten Wajib$>$}
\begin{itemize}
	\item Klasifikasi growth function: constant, logarithmic, linear, polynomial, exponential
	\item berikan grafik untuk menunjukkan pertumbuhan fungsi
\end{itemize}
\end{frame}

\begin{frame}
\frametitle{$<$TODO: Konten Wajib$>$}
\begin{itemize}
	\item Jelaskan tentang konvensi 1 detik = 100 juta proses
	\item Jelaskan tentang: diberikan N dan kompleksitas algoritma itu, ramalkan kira-kira berapa detik yang dibutuhkan algoritma itu untuk berjalan.
	\item kaitkan dengan timelimit dari soal programming biasanya
\end{itemize}
\end{frame}

\end{document}