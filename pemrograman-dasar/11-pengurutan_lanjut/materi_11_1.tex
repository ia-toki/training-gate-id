\documentclass{beamer}
\usetheme{tokitex}

\usepackage{pgf}
\usepackage{graphics}
\usepackage{multirow}
\usepackage{multicol}
\usepackage{tabto}
\usepackage[english,bahasa]{babel}
\newtranslation[to=bahasa]{Section}{Bagian}
\newtranslation[to=bahasa]{Subsection}{Subbagian}

\usepackage{listings}
\usepackage{color}

\definecolor{dkgreen}{rgb}{0,0.6,0}
\definecolor{gray}{rgb}{0.5,0.5,0.5}
\definecolor{mauve}{rgb}{0.58,0,0.82}

\lstset{frame=tb,
  language=pascal,
  showstringspaces=false,
  columns=flexible,
  basicstyle={\small\ttfamily},
  numbers=none,
  numberstyle=\tiny\color{gray},
  keywordstyle=\color{blue},
  commentstyle=\color{dkgreen},
  stringstyle=\color{mauve},
  breaklines=true,
  breakatwhitespace=true,
  tabsize=2
}

\title{Pencarian Lanjut - Merge Sort}
\author{Tim Olimpiade Komputer Indonesia}

\begin{document}

\begin{frame}
\titlepage
\end{frame}

\begin{frame}
\frametitle{Pendahuluan}
Melalui dokumen ini, kalian akan:
\begin{itemize}
	\item Mempelajari konsep \textit{merge sort}.
	\item Mengimplementasikan \textit{merge sort}.
\end{itemize}
\end{frame}

\section{Konsep}
\frame{\sectionpage}

\begin{frame}
\frametitle{$<$TODO: Konten Wajib$>$}
\begin{itemize}
	\item pengenalan strategi divide and conquer (divide, conquer, combine)
	\item ide dasar: divide (bagi array menjadi 2), conquer (kalau sudah tinggal 1 elemen berarti sudah sorted), combine (gabungkan 2 array sorted jadi 1 array sorted)
	\item cara merge 2 array sorted
	\item kasih demonstrasi cara merge
	\item analisis kompleksitas
	\item bisa gunakan berkas "TEKS : Sorting Standard.html" (lihat folder referensi pada materi sesi-9) sebagai acuan
\end{itemize}
\end{frame}

\section{Implementasi}
\frame{\sectionpage}

\begin{frame}
\frametitle{$<$TODO: Konten Wajib$>$}
\begin{itemize}
	\item buat fungsi rekursif
	\item jelaskan per bagian kodenya
	\item jelaskan bahwa merge sort butuh memori tambahan untuk menampung hasil sort sementara
\end{itemize}
\end{frame}

\end{document}