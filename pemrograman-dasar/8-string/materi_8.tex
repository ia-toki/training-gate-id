\documentclass{beamer}
\usetheme{tokitex}
\usepackage{graphics}
\usepackage{multirow}
\usepackage{multicol}
\usepackage{tabto}
\usepackage[english,bahasa]{babel}
\newtranslation[to=bahasa]{Section}{Bagian}
\newtranslation[to=bahasa]{Subsection}{Subbagian}

\usepackage{listings}
\usepackage{color}

\definecolor{dkgreen}{rgb}{0,0.6,0}
\definecolor{gray}{rgb}{0.5,0.5,0.5}
\definecolor{mauve}{rgb}{0.58,0,0.82}

\lstset{frame=tb,
  language=pascal,
  showstringspaces=false,
  columns=flexible,
  basicstyle={\small\ttfamily},
  numbers=none,
  numberstyle=\tiny\color{gray},
  keywordstyle=\color{blue},
  commentstyle=\color{dkgreen},
  stringstyle=\color{mauve},
  breaklines=true,
  breakatwhitespace=true,
  tabsize=2
}

\title{Pendalaman String\newline (Pascal)}
\author{Tim Olimpiade Komputer Indonesia}

\begin{document}

\begin{frame}
\titlepage
\end{frame}

\begin{frame}
\frametitle{Pendahuluan}
Melalui dokumen ini, kalian akan:
\begin{itemize}
	\item Memahami lebih dalam soal string.
	\item Mengenal fungsi-fungsi dasar untuk pengolahan string.
\end{itemize}
\vfill
Seluruh pembahasan string pada dokumen ini mengacu pada tipe data \textbf{string} pada \textbf{Pascal}.
\end{frame}

\section{Pengolahan String}
\frame{\sectionpage}

\begin{frame}
\frametitle{Pengolahan String}
Pascal menyediakan berbagai fungsi dan prosedur dasar pengolah string, beberapa di antaranya adalah:
\begin{itemize}
	\item length
	\item pos
	\item copy
	\item delete
	\item insert
	\item str
	\item val
\end{itemize}
\end{frame}

\begin{frame}[fragile]
\frametitle{Pengolahan String:\newline length}
\begin{block}{length(S)}
Merupakan fungsi yang mengembalikan panjang dari string $S$.
\end{block}

Contoh:
\begin{lstlisting}
s1 := 'bebek';
s2 := '';
s3 := 'Pak Dengklek';

writeln(length(s1)); (* mencetak 5 *)
writeln(length(s2)); (* mencetak 0 *)
writeln(length(s3)); (* mencetak 12 *)
\end{lstlisting}
\end{frame}

\begin{frame}[fragile]
\frametitle{Pengolahan String:\newline pos}
\begin{block}{pos(T, S)}
Merupakan fungsi yang mencari dan mengembalikan posisi terawal substring $T$ dari suatu string $S$.

Jika tidak ditemukan, dikembalikan nilai 0.
\end{block}
Contoh:
\begin{lstlisting}
s := 'Pak Dengklek berternak';
t1 := 'Dengklek';
t2 := 'pak';
t3 := 'klek';

writeln(pos(t1, s)); (* mencetak 4 *)
writeln(pos(t2, s)); (* mencetak 0, tidak ditemukan  *)
writeln(pos(t3, s)); (* mencetak 9 *)
\end{lstlisting}
\end{frame}

\begin{frame}[fragile]
\frametitle{Pengolahan String:\newline copy}
\begin{block}{copy(S, pos, cnt)}
Merupakan fungsi yang mengembalikan substring dari indeks $pos$ sebanyak $cnt$ karakter dari string $S$.
\end{block}
Contoh:
\begin{lstlisting}
s := 'Pak Dengklek berternak';

writeln(copy(s, 1, 6)); (* mencetak 'Pak De' *)
writeln(copy(s, 3, 1)); (* mencetak 'k' *)
\end{lstlisting}
\end{frame}

\begin{frame}[fragile]
\frametitle{Pengolahan String:\newline delete}
\begin{block}{delete(S, pos, cnt)}
Merupakan prosedur yang menghapus substring dari indeks $pos$ sebanyak $cnt$ karakter dari string $S$.

Parameter $S$ dipanggil dengan \textit{by reference}.
\end{block}
Contoh:
\begin{lstlisting}
s := 'Pak Dengklek berternak';

delete(s, 2, 3);
writeln(s); (* mencetak 'PDengklek berternak' *)
\end{lstlisting}
\end{frame}

\begin{frame}[fragile]
\frametitle{Pengolahan String:\newline insert}
\begin{block}{insert(T, S, pos)}
Merupakan prosedur yang menyisipkan string $T$ ke dalam string $S$ mulai dari indeks $pos$.

Parameter $S$ dipanggil dengan \textit{by reference}.
\end{block}
Contoh:
\begin{lstlisting}
s := 'Pak Dengklek berternak';
t := 'dan Bu ';

insert(t, s, 5);
writeln(s); (* mencetak 'Pak dan Bu Dengklek berternak' *)
\end{lstlisting}
\end{frame}

\begin{frame}[fragile]
\frametitle{Pengolahan String:\newline str}
\begin{block}{str(v, S)}
Merupakan prosedur yang mengkonversi suatu data numerik $v$ menjadi string, dan ditampung ke dalam string $S$.

Parameter $S$ dipanggil dengan \textit{by reference}.
\end{block}
Contoh:
\begin{lstlisting}
s := '';
nilai := 781;

str(nilai, s);
writeln(s);           (* mencetak '781' *)
writeln(length(s)); (* mencetak 3 *)
\end{lstlisting}
\end{frame}

\begin{frame}[fragile]
\frametitle{Pengolahan String:\newline val}
\begin{block}{val(S, v, e)}
Merupakan prosedur yang mengkonversi suatu string $S$ menjadi data numerik, dan ditampung ke dalam variabel $v$.

Jika terjadi \textbf{error}, variabel $e$ akan berisi nilai yang tidak nol.

Parameter $v$ dan $e$ dipanggil dengan \textit{by reference}.
\end{block}
Contoh:
\begin{lstlisting}
s := '123';
nilai := 0;
e := 0;

val(s, nilai, e);
writeln(nilai + 5); (* mencetak 128 *)
\end{lstlisting}

\textbf{Error} bisa terjadi misalnya ketika s bernilai '1a23';
\end{frame}

\begin{frame}[fragile]
\frametitle{Pengolahan String:\newline val (lanj.)}
Parameter e sebenarnya tidak harus ada, sehingga bisa saja ditulis:
\begin{lstlisting}
s := '123';
nilai := 0;

val(s, nilai);
writeln(nilai + 5); (* mencetak 128 *)
\end{lstlisting}
\end{frame}

\begin{frame}[fragile]
\frametitle{Operasi Tambahan: konkatenasi}
\begin{itemize}
	\item Konkatenasi adalah penggabungan string.
	\item Pada Pascal, hal ini dapat dilakukan cukup dengan operasi '+', layaknya operasi numerik.
	\item Contoh:
	\begin{lstlisting}
	s := 'Pak';
	t := 'Dengklek';
	
	gabung := s + t;
	writeln(gabung); (* mencetak 'PakDengklek' *)
	\end{lstlisting}
\end{itemize}
\end{frame}


\section{Wujud Asli String}
\frame{\sectionpage}

\begin{frame}
\frametitle{Fakta Tentang String...}
Sebenarnya, \textbf{string} adalah \textbf{\alert{array of char}}! 
\end{frame}

\begin{frame}
\frametitle{Fakta Tentang String... (lanj.)}
\begin{itemize}
	\item Pascal membungkus \textbf{array[0..255] of char} menjadi \textbf{string}, kemudian menambahkan fungsi dan prosedur dasar untuk pengolahan string.
	\item Elemen ke-0 dari string tidak digunakan untuk menampung karakter, melainkan untuk menyimpan \alert{panjang dari string tersebut}.
	\item Artinya, tipe data \textbf{string} hanya bisa menampung maksimal 255 karakter yang menyusunnya.
	\item Gunakan tipe data \textbf{ansistring} untuk menampung karakter yang lebih banyak dari itu, atau buat \textbf{array of char} sendiri $:)$
\end{itemize}
\end{frame}

\begin{frame}[fragile]
\frametitle{Bukti}
\begin{itemize}
	\item Coba jalankan potongan kode berikut!
	\begin{lstlisting}
	s := 'tes';
	writeln(byte(s[0]));
	s := 'tes lagi'
	writeln(byte(s[0]));
	\end{lstlisting}
\end{itemize}
\end{frame}

\begin{frame}
\frametitle{Selanjutnya...}
\begin{itemize}
	\item Pembelajaran kalian tentang Bahasa Pascal sudah cukup untuk bisa menuliskan algoritma-algoritma kompleks.
	\item Berikutnya kita akan mempelajari hal-hal yang lebih berkaitan dengan \textbf{algoritma}, bukan sekedar belajar bahasa $:)$. 
\end{itemize}
\end{frame}

\end{document}