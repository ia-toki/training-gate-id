\documentclass{beamer}
\usepackage{graphics}
\usepackage{multirow}
\usepackage{tabto}

\usepackage{listings}
\usepackage{color}

\definecolor{dkgreen}{rgb}{0,0.6,0}
\definecolor{gray}{rgb}{0.5,0.5,0.5}
\definecolor{mauve}{rgb}{0.58,0,0.82}

\lstset{frame=tb,
  language=pascal,
  aboveskip=3mm,
  belowskip=3mm,
  showstringspaces=false,
  columns=flexible,
  basicstyle={\small\ttfamily},
  numbers=none,
  numberstyle=\tiny\color{gray},
  keywordstyle=\color{blue},
  commentstyle=\color{dkgreen},
  stringstyle=\color{mauve},
  breaklines=true,
  breakatwhitespace=true,
  tabsize=3
}

\title{Ekspresi dan Input/Output}
\author{Tim Olimpiade Komputer Indonesia}

\begin{document}

\begin{frame}
\titlepage
\end{frame}

\begin{frame}
\frametitle{Pendahuluan}
Melalui dokumen ini, kalian akan:
\begin{itemize}
	\item Mengenal ekspresi.
	\item Mengenal input dan output.
\end{itemize}
\end{frame}

\begin{frame}[fragile]
\frametitle{Kilas Balik: Assignment}
\begin{itemize}
	\item Cukup membosankan bila kita hanya bisa mengisi suatu variabel dengan suatu nilai yang pasti. Kadang-kadang kita membutuhkan hal yang lebih ekspresif. Contoh:
	\begin{lstlisting}
	a := 5;
	b := 2;
	jumlah := a + b;
	\end{lstlisting}
	\item Kenyataannya, ya! Hal ini sangat mungkin diwujudkan pada pemrograman.
	\item Perintah "a + b" biasa disebut sebagai ekspresi.
\end{itemize}
\end{frame}

\begin{frame}
\frametitle{Mengenal Ekspresi}
\begin{itemize}
	\item Ekspresi terdiri dari dua komponen: \alert{operator} dan \alert{operand}.
	\item Operand menyatakan nilai yang akan dioperasikan, misalnya suatu bilangan atau suatu ekspresi lagi.
	\item Operator menyatakan bagaimana operand akan dioperasikan, apakah ditambah, dikali, atau dibagi.
	\item Pada ekspresi "1 + 2", 1 dan 2 merupakan operand, dan '+' merupakan operator.
	\item Pada ekspresi "5 - (1 + 2)", 1 dan (1 + 2) merupakan operand, dan '-' merupakan operator. Lebih jauh lagi, (1 + 2) sendiri terdiri dari operand 1 dan 2, serta operator '+'.
\end{itemize}
\end{frame}

\begin{frame}
\frametitle{Operasi Numerik}
\begin{itemize}
	\item Operasi pada bilangan yang dapat dilakukan adalah penjumlahan (+), pengurangan (-), perkalian (*), pembagian (/), pembagian integer (div), dan modulo (mod).
	\item Jika kedua operand merupakan bilangan bulat, hasil pengoperasian selalu bilangan bulat juga, kecuali untuk operasi pembagian yang selalu menghasilkan floating point. Contoh:
	\begin{itemize}
		\item 3-1 = 2
		\item 10/5 = 2.0000000
		\item 7/2 = 3.5000000
	\end{itemize}
	\item Ketika setidaknya salah satu dari operand ada yang bertipe data floating point, pengoperasian akan selalu menghasilkan floating point.
\end{itemize}
\end{frame}

\begin{frame}
\frametitle{Operasi Numerik (lanj.)}
\begin{itemize}
	\item Operasi pembagian integer didefinisikan sebagai: membagi, lalu dibulatkan ke bawah. Contoh:
	\begin{itemize}
		\item 7 div 2 = 3
		\item 10 div 2 = 5
		\item 3 div 5 = 0
	\end{itemize}
	\item Operasi modulo adalah mengambil sisa bagi dari operand pertama terhadap operand kedua. Contoh:
	\begin{itemize}
		\item 7 mod 2 = 1
		\item 10 mod 2 = 0
		\item 3 mod 5 = 3
		\item 8 mod 3 = 2
	\end{itemize}
	\item Operasi div dan mod hanya bisa dilakukan apabila kedua operand memiliki tipe data bilangan bulat.
\end{itemize}
\end{frame}

\begin{frame}[fragile]
\frametitle{Contoh Program: kuadrat1.pas}
\begin{itemize}
	\item Setelah memahami tentang operasi numerik, coba perhatikan program berikut dan cari tahu apa outputnya!
	\begin{lstlisting}
		var
		    a,b,c,x,hasil:longint;
		begin
		    a := 1;
		    b := 3;
		    c := -2;
		    x := 2;	    

		    hasil := a*x*x + b*x + c;
		    writeln('ax^2 + bx + c = ', hasil);
		end.
	\end{lstlisting}
\end{itemize}
\end{frame}

\begin{frame}
\frametitle{Presendensi Pengerjaan}
\begin{itemize}
	\item Seperti pada ilmu matematika, ada juga presendensi pengerjaan pada ekspresi numerik. Tabel berikut menunjukkan prioritasnya:
	
	\begin{tabular}{|c|c|}
	\hline Prioritas & Operasi \\ 
	\hline 1 & *,/,div,mod \\ 
	\hline 2 & +,- \\ 
	\hline 
	\end{tabular} 
	\item Jika ada beberapa operasi yang memiliki prioritas sama, operasi yang terletak di posisi lebih kiri akan dikerjakan lebih dahulu.
\end{itemize}
\end{frame}

\begin{frame}[fragile]
\frametitle{Contoh Program: numerik.pas}
\begin{itemize}
	\item Kita juga bisa menggunakan tanda kurung untuk mengatur prioritas pengerjaan suatu ekspresi.
	\item Perhatikan contoh berikut dan coba jalankan programnya:
	\begin{lstlisting}
		var
		    hasil1,hasil2:longint;
		begin
		    hasil1 := 3+5 div 4;
		    hasil2 := (3+5) div 4;
		    writeln(hasil1);
		    writeln(hasil2);
		end.
	\end{lstlisting}
	\item Isi dari variabel hasil1 adalah 4, karena operasi "5 div 4" memiliki prioritas yang lebih tinggi untuk dikerjakan, dan menghasilkan nilai 1. Barulah "3 + 1" dilaksanakan kemudian. 
\end{itemize}
\end{frame}

\begin{frame}
\frametitle{Fungsi Dasar Numerik}
\begin{itemize}
	\item Terdapat pula fungsi-fungsi dasar yang telah disediakan Pascal untuk membantu perhitungan:
	\begin{itemize}
		\item trunc: mengambil bagian depan penanda desimal dari suatu bilangan pecahan. Contoh: trunc(3.14) akan menghasilkan 3.00.
		\item frac: mengambil bagian belakang penanda desimal dari suatu bilangan pecahan. Contoh: frac(3.14) akan menghasilkan 0.14.
		\item round: membulatkan suatu bilangan pecahan bilangan bulat terdekat (hasilnya adalah bilangan bertipe integer). Contoh: round(1.2) akan menghasilkan 1, sementara round(1.87) akan menghasilkan 2.
		\item sqr: mengkuadratkan suatu bilangan. Contoh: sqr(3) akan menghasilkan 9.
		\item sqrt: mendapatkan akar kuadrat dari suatu bilangan. Contoh: sqrt(9) akan menghasilkan 3.00, dan sqrt(3) akan menghasilkan 1.73205....
	\end{itemize}
\end{itemize}
\end{frame}

\begin{frame}[fragile]
\frametitle{Contoh Program: kuadrat2.pas}
\begin{itemize}
	\item Program "kuadrat1.pas" bisa ditulis juga dengan menggunakan fungsi sqr(x), sehingga menjadi:
	\begin{lstlisting}
		var
		    a,b,c,x,hasil:longint;
		begin
		    a := 1;
		    b := 3;
		    c := -2;
		    x := 2;	    

		    hasil := a*sqr(x) + b*x + c;
		    writeln('ax^2 + bx + c = ', hasil);
		end.
	\end{lstlisting}
\end{itemize}
\end{frame}

\begin{frame}
\frametitle{Operasi Relasional}
\begin{itemize}
	\item Kita juga bisa melakukan operasi relasional, yaitu:
	\begin{itemize}
		\item kurang dari ($<$)
		\item lebih dari ($>$)
		\item sama dengan ($=$)
		\item kurang dari atau sama dengan ($<=$)
		\item lebih dari atau sama dengan ($>=$)
		\item tidak sama dengan ($<>$)
	\end{itemize}
	\item Operasi relasional harus melibatkan dua operand (ingat bahwa operand bisa jadi berupa ekspresi lagi), dan menghasilkan sebuah nilai kebenaran. Pada Pascal, nilai kebenaran dinyatakan dengan tipe data \alert{boolean}.
\end{itemize}
\end{frame}

\begin{frame}[fragile]
\frametitle{Contoh Program: relasional.pas}
\begin{itemize}
	\item Perhatikan contoh berikut dan coba jalankan programnya:
	\begin{lstlisting}
		begin
		    writeln(2 > 1);
		    writeln(2 < 1);
		    writeln(2 = 1);
		    writeln(2 >= 1);
		    writeln(1 = 1);
		    writeln(1 <> 1);
		    writeln(1 <> 2);
		end.
	\end{lstlisting}
\end{itemize}
\end{frame}

\begin{frame}
\frametitle{Operasi Relasional}
\begin{itemize}
	\item Operasi relasional dapat dilakukan pada setiap tipe data ordinal, sehingga bisa juga diterapkan pada char.
	\item Tentu saja, ekspresi "'a' $<$ 'b'" akan bernilai TRUE, sementara "'a' $>$ 'z'" bernilai FALSE.
	\item Lebih jauh lagi, string sebenarnya merupakan untaian char. Operasi relasional juga bisa diterapkan pada string, dan Pascal akan membandingkan karakter demi karakter dari kiri ke kanan. 
	\item Hasilnya, keputusan apakah suatu string "lebih kecil dari" string lainnya ditentukan seperti urutan pada kamus.
	\item Misalnya "'aa' $<$ 'ab'" akan bernilai TRUE, dan "'aa' $<$ 'a'" bernilai FALSE. 
\end{itemize}
\end{frame}

\begin{frame}[fragile]
\frametitle{Contoh Program: relasional2.pas}
\begin{itemize}
	\item Perhatikan contoh berikut dan coba jalankan programnya:
		\begin{lstlisting}
			begin
			    writeln('a' > 'A');
			    writeln('a' < 'A');
			    writeln('a' >= 'A');
			    writeln('a' = 'A');

			    writeln('a' < 'aa');
			    writeln('abcb' > 'abca');
			    writeln('abc' = 'abc');
			    writeln('abc' <= 'abc');
			end.
		\end{lstlisting}
\end{itemize}
\end{frame}



\begin{frame}
\frametitle{Tentang Input dan Output}
\begin{itemize}
	\item Untuk menghasilkan output yang bervariasi, perlu ada suatu input dari luar program.
	\item Dengan demikian, diperlukan mekanisme untuk melakukan pembacaan input dari luar program.
\end{itemize}
\end{frame}

\begin{frame}
\frametitle{Membaca Input}
\begin{itemize}
	\item Input bagi suatu program bisa berasal dari berbagai sumber, misalnya \textit{standart input} (layar) atau suatu \textit{file}.
	\item Pada Pascal, dikenal dua fungsi yang umum untuk membaca input: \alert{read} dan \alert{readln}.
\end{itemize}
\end{frame}

\begin{frame}
\frametitle{Membaca Input: readln}
\begin{itemize}
	\item a
\end{itemize}
\end{frame}

\begin{frame}
\frametitle{Operasi Output}
\begin{itemize}
	\item bisa lewat file, stdio, dsb. ajarin write/writeln
\end{itemize}
\end{frame}

\begin{frame}
\frametitle{Contoh Program}
\begin{itemize}
	\item IO pakai basa-basi
\end{itemize}
\end{frame}

\begin{frame}
\frametitle{Penjelasan Tentang STDIO}
\begin{itemize}
	\item ada 2 stream berbeda
\end{itemize}
\end{frame}

\begin{frame}
\frametitle{Tentang Basa Basi}
\begin{itemize}
	\item OSN/IOI ga butuh basa basi
\end{itemize}
\end{frame}

\end{document}