\documentclass{beamer}
\usetheme{tokitex}

\usepackage{tikz}
\usepackage{graphics}
\usepackage{multirow}
\usepackage{tabto}
\usepackage{xspace}
\usepackage{amsmath}
\usepackage{hyperref}
\usepackage{wrapfig}
\usepackage{mathtools}

\usepackage{tikz}
\usepackage{clrscode3e}
\usepackage{gensymb}

\usepackage[english,bahasa]{babel}
\newtranslation[to=bahasa]{Section}{Bagian}
\newtranslation[to=bahasa]{Subsection}{Subbagian}

\usepackage{listings, lstautogobble}
\usepackage{color}

\definecolor{dkgreen}{rgb}{0,0.6,0}
\definecolor{gray}{rgb}{0.5,0.5,0.5}
\definecolor{mauve}{rgb}{0.58,0,0.82}

\lstset{frame=tb,
  language=c++,
  aboveskip=0mm,
  belowskip=0mm,
  showstringspaces=false,
  columns=fullflexible,
  keepspaces=true,
  basicstyle={\small\ttfamily},
  numbers=none,
  numberstyle=\tiny\color{gray},
  keywordstyle=\color{blue},
  commentstyle=\color{dkgreen},
  stringstyle=\color{mauve},
  breaklines=true,
  breakatwhitespace=true,
  lineskip={-3pt}
}

\usepackage{caption}
\captionsetup[figure]{labelformat=empty}

\newcommand{\progTerm}[1]{\textbf{#1}}
\newcommand{\foreignTerm}[1]{\textit{#1}}
\newcommand{\newTerm}[1]{\alert{\textbf{#1}}}
\newcommand{\emp}[1]{\alert{#1}}
\newcommand{\statement}[1]{"#1"}

\newcommand{\floor}[1]{\lfloor #1 \rfloor}
\newcommand{\ceil}[1]{\lceil #1 \rceil}
\newcommand{\abs}[1]{\left\lvert#1\right\rvert}
\newcommand{\norm}[1]{\left\lVert#1\right\rVert}

% Getting tired of writing \foreignTerm all the time
\newcommand{\farray}{\foreignTerm{array}\xspace}
\newcommand{\fArray}{\foreignTerm{Array}\xspace}
\newcommand{\foverhead}{\foreignTerm{overhead}\xspace}
\newcommand{\fOverhead}{\foreignTerm{Overhead}\xspace}
\newcommand{\fsubarray}{\foreignTerm{subarray}\xspace}
\newcommand{\fSubarray}{\foreignTerm{Subarray}\xspace}
\newcommand{\fbasecase}{\foreignTerm{base case}\xspace}
\newcommand{\fBasecase}{\foreignTerm{Base case}\xspace}
\newcommand{\ftopdown}{\foreignTerm{top-down}\xspace}
\newcommand{\fTopdown}{\foreignTerm{Top-down}\xspace}
\newcommand{\fbottomup}{\foreignTerm{bottom-up}\xspace}
\newcommand{\fBottomup}{\foreignTerm{Bottom-up}\xspace}
\newcommand{\fpruning}{\foreignTerm{pruning}\xspace}
\newcommand{\fPruning}{\foreignTerm{Pruning}\xspace}

\newcommand{\fgraph}{\foreignTerm{graph}\xspace}
\newcommand{\fGraph}{\foreignTerm{Graph}\xspace}
\newcommand{\froot}{\foreignTerm{root}\xspace}
\newcommand{\fRoot}{\foreignTerm{Root}\xspace}
\newcommand{\fnode}{\foreignTerm{node}\xspace}
\newcommand{\fNode}{\foreignTerm{Node}\xspace}
\newcommand{\fedge}{\foreignTerm{edge}\xspace}
\newcommand{\fEdge}{\foreignTerm{Edge}\xspace}
\newcommand{\fcycle}{\foreignTerm{cycle}\xspace}
\newcommand{\fCycle}{\foreignTerm{Cycle}\xspace}
\newcommand{\fdegree}{\foreignTerm{degree}\xspace}
\newcommand{\fDegree}{\foreignTerm{Degree}\xspace}
\newcommand{\fadjacencylist}{\foreignTerm{adjacency list}\xspace}
\newcommand{\fAdjacencylist}{\foreignTerm{Adjacency list}\xspace}
\newcommand{\fadjacencymatrix}{\foreignTerm{adjacency matrix}\xspace}
\newcommand{\fAdjacencymatrix}{\foreignTerm{Adjacency matrix}\xspace}
\newcommand{\fedgelist}{\foreignTerm{edge list}\xspace}
\newcommand{\fEdgelist}{\foreignTerm{Edge list}\xspace}
\newcommand{\flist}{\foreignTerm{list}\xspace}
\newcommand{\fList}{\foreignTerm{List}\xspace}
\newcommand{\fgraphtraversal}{\foreignTerm{graph traversal}\xspace}
\newcommand{\fGraphtraversal}{\foreignTerm{Graph traversal}\xspace}
\newcommand{\ftree}{\foreignTerm{tree}\xspace}
\newcommand{\fTree}{\foreignTerm{Tree}\xspace}
\newcommand{\fsubtree}{\foreignTerm{subtree}\xspace}
\newcommand{\fSubtree}{\foreignTerm{Subtree}\xspace}
\newcommand{\fparent}{\foreignTerm{parent}\xspace}
\newcommand{\fParent}{\foreignTerm{Parent}\xspace}
\newcommand{\fsibling}{\foreignTerm{sibling}\xspace}
\newcommand{\fSibling}{\foreignTerm{Sibling}\xspace}
\newcommand{\fpath}{\foreignTerm{path}\xspace}
\newcommand{\fPath}{\foreignTerm{Path}\xspace}
\newcommand{\fconnectedcomponent}{\foreignTerm{connected component}\xspace}
\newcommand{\fConnectedcomponent}{\foreignTerm{Connected component}\xspace}
\newcommand{\fbridge}{\foreignTerm{bridge}\xspace}
\newcommand{\fBridge}{\foreignTerm{Bridge}\xspace}
\newcommand{\farticulationpoint}{\foreignTerm{articulation point}\xspace}
\newcommand{\fArticulationpoint}{\foreignTerm{Articulation point}\xspace}
\newcommand{\ftreeedge}{\foreignTerm{tree edge}\xspace}
\newcommand{\fTreeedge}{\foreignTerm{Tree edge}\xspace}
\newcommand{\fbackedge}{\foreignTerm{back edge}\xspace}
\newcommand{\fBackedge}{\foreignTerm{Back edge}\xspace}
\newcommand{\fforwardedge}{\foreignTerm{forward edge}\xspace}
\newcommand{\fForwardedge}{\foreignTerm{Forward edge}\xspace}
\newcommand{\fcrossedge}{\foreignTerm{cross edge}\xspace}
\newcommand{\fCrossedge}{\foreignTerm{Cross edge}\xspace}
\newcommand{\fdiscoverytime}{\foreignTerm{discovery time}\xspace}
\newcommand{\fDiscoverytime}{\foreignTerm{Discovery time}\xspace}
\newcommand{\flowlink}{\foreignTerm{low link}\xspace}
\newcommand{\fLowlink}{\foreignTerm{Low link}\xspace}
\newcommand{\fstack}{\foreignTerm{stack}\xspace}
\newcommand{\fStack}{\foreignTerm{Stack}\xspace}
\newcommand{\for}{\foreignTerm{or}\xspace}
\newcommand{\fOr}{\foreignTerm{Or}\xspace}
\newcommand{\fand}{\foreignTerm{and}\xspace}
\newcommand{\fAnd}{\foreignTerm{And}\xspace}
\newcommand{\fcentroid}{\foreignTerm{centroid}\xspace}
\newcommand{\fCentroid}{\foreignTerm{Centroid}\xspace}

\newcommand{\fDivideAndConquer}{\foreignTerm{Divide and conquer}\xspace}
\newcommand{\fdivideAndConquer}{\foreignTerm{divide and conquer}\xspace}
\newcommand{\fMergeSort}{\foreignTerm{Merge sort}\xspace}
\newcommand{\fmergeSort}{\foreignTerm{merge sort}\xspace}
\newcommand{\fQuickSort}{\foreignTerm{Quicksort}\xspace}
\newcommand{\fquickSort}{\foreignTerm{quicksort}\xspace}
\newcommand{\fpivot}{\foreignTerm{pivot}\xspace}
\newcommand{\fPivot}{\foreignTerm{Pivot}\xspace}
\newcommand{\fbruteForce}{\foreignTerm{brute force}\xspace}
\newcommand{\fBruteForce}{\foreignTerm{Brute force}\xspace}
\newcommand{\fCompleteSearch}{\foreignTerm{complete search}\xspace}
\newcommand{\fExhaustiveSearch}{\foreignTerm{exhaustive search}\xspace}
\newcommand{\fbinarySearch}{\foreignTerm{binary search}\xspace}
\newcommand{\fBinarySearch}{\foreignTerm{Binary search}\xspace}
\newcommand{\fternarySearch}{\foreignTerm{ternary search}\xspace}
\newcommand{\fTernarySearch}{\foreignTerm{Ternary search}\xspace}
\newcommand{\funimodal}{\foreignTerm{unimodal}\xspace}
\newcommand{\fUnimodal}{\foreignTerm{Unimodal}\xspace}
\newcommand{\fGreedy}{\foreignTerm{Greedy}\xspace}
\newcommand{\fgreedy}{\foreignTerm{greedy}\xspace}
\newcommand{\fgreedyChoice}{\foreignTerm{greedy choice}\xspace}
\newcommand{\fGreedyChoice}{\foreignTerm{Greedy choice}\xspace}

\newcommand{\fdp}{\foreignTerm{dynamic programming}\xspace}
\newcommand{\fDp}{\foreignTerm{Dynamic programming}\xspace}
\newcommand{\fbitmask}{\foreignTerm{bitmask}\xspace}
\newcommand{\fBitmask}{\foreignTerm{Bitmask}\xspace}
\newcommand{\fstate}{\foreignTerm{state}\xspace}
\newcommand{\fState}{\foreignTerm{State}\xspace}
\newcommand{\fsubmask}{\foreignTerm{submask}\xspace}
\newcommand{\fSubmask}{\foreignTerm{Submask}\xspace}

\newcommand{\pheap}{\foreignTerm{heap}\xspace}
\newcommand{\pHeap}{\foreignTerm{Heap}\xspace}
\newcommand{\pBinaryHeap}{\foreignTerm{Binary Heap}\xspace}
\newcommand{\pbinaryHeap}{\foreignTerm{binary heap}\xspace}
\newcommand{\pHeapsort}{\foreignTerm{Heapsort}\xspace}
\newcommand{\pheapsort}{\foreignTerm{heapsort}\xspace}
\newcommand{\pdjs}{\foreignTerm{disjoint set}\xspace}
\newcommand{\pDjs}{\foreignTerm{Disjoint set}\xspace}

\newcommand{\fdotProduct}{\foreignTerm{dot product}\xspace}
\newcommand{\fDotProduct}{\foreignTerm{Dot product}\xspace}
\newcommand{\fcrossProduct}{\foreignTerm{cross product}\xspace}
\newcommand{\fCrossProduct}{\foreignTerm{Cross product}\xspace}
\newcommand{\fconvexHull}{\foreignTerm{convex hull}\xspace}
\newcommand{\fConvexHull}{\foreignTerm{Convex hull}\xspace}
\newcommand{\fgrahamScan}{\foreignTerm{graham scan}\xspace}
\newcommand{\fGrahamScan}{\foreignTerm{Graham scan}\xspace}
\newcommand{\flineSweep}{\foreignTerm{line sweep}\xspace}
\newcommand{\fLineSweep}{\foreignTerm{Line sweep}\xspace}

\newcommand{\fset}{\foreignTerm{set}\xspace}
\newcommand{\fSet}{\foreignTerm{Set}\xspace}
\newcommand{\fprefixSum}{\foreignTerm{prefix sum}\xspace}
\newcommand{\fPrefixSum}{\foreignTerm{Prefix sum}\xspace}
\newcommand{\ffenwickTree}{\foreignTerm{fenwick tree}\xspace}
\newcommand{\fFenwickTree}{\foreignTerm{Fenwick tree}\xspace}
\newcommand{\frangeSumQuery}{\foreignTerm{range sum query}\xspace}
\newcommand{\fRangeSumQuery}{\foreignTerm{Range sum query}\xspace}
\newcommand{\fquery}{\foreignTerm{query}\xspace}
\newcommand{\fQuery}{\foreignTerm{Query}\xspace}
\newcommand{\fsegmentTree}{\foreignTerm{segment tree}\xspace}
\newcommand{\fSegmentTree}{\foreignTerm{Segment tree}\xspace}
\newcommand{\fbinaryTree}{\foreignTerm{binary tree}\xspace}
\newcommand{\fBinaryTree}{\foreignTerm{Binary tree}\xspace}
\newcommand{\flazyPropagation}{\foreignTerm{lazy propagation}\xspace}
\newcommand{\fLazyPropagation}{\foreignTerm{Lazy propagation}\xspace}
\newcommand{\fsparseTable}{\foreignTerm{sparse table}\xspace}
\newcommand{\fSparseTable}{\foreignTerm{Sparse table}\xspace}

\newcommand{\ftrail}{\foreignTerm{trail}\xspace}
\newcommand{\fTrail}{\foreignTerm{Trail}\xspace}
\newcommand{\feulerTour}{\foreignTerm{euler tour}\xspace}
\newcommand{\fEulerTour}{\foreignTerm{Euler tour}\xspace}
\newcommand{\feulerTourTree}{\foreignTerm{euler tour tree}\xspace}
\newcommand{\fEulerTourTree}{\foreignTerm{Euler tour tree}\xspace}

\newcommand{\fmaxflow}{\foreignTerm{maximum flow}\xspace}
\newcommand{\fMaxflow}{\foreignTerm{Maximum flow}\xspace}
\newcommand{\fmincut}{\foreignTerm{minimum cut}\xspace}
\newcommand{\fMincut}{\foreignTerm{Minimum cut}\xspace}
\newcommand{\fflow}{\foreignTerm{flow}\xspace}
\newcommand{\fFlow}{\foreignTerm{Flow}\xspace}
\newcommand{\fsource}{\foreignTerm{source}\xspace}
\newcommand{\fSource}{\foreignTerm{Source}\xspace}
\newcommand{\fsink}{\foreignTerm{sink}\xspace}
\newcommand{\fSink}{\foreignTerm{Sink}\xspace}
\newcommand{\fbackEdge}{\foreignTerm{back-edge}\xspace}
\newcommand{\fBackEdge}{\foreignTerm{Back-edge}\xspace}
\newcommand{\fresidualCapacity}{\foreignTerm{residual capacity}\xspace}
\newcommand{\fResidualCapacity}{\foreignTerm{Residual capacity}\xspace}
\newcommand{\fbottleneck}{\foreignTerm{bottleneck}\xspace}
\newcommand{\fBottleneck}{\foreignTerm{Bottleneck}\xspace}
\newcommand{\faugmentingPath}{\foreignTerm{augmenting path}\xspace}
\newcommand{\fAugmentingPath}{\foreignTerm{Augmenting path}\xspace}


\title{Brute Force}
\author{Tim Olimpiade Komputer Indonesia}
\date{}

\begin{document}

\begin{frame}
\titlepage
\end{frame}

\begin{frame}
\frametitle{Pendahuluan}
Melalui dokumen ini, kalian akan:
\begin{itemize}
  \item Mempelajari konsep \fbruteForce.
  \item Mampu mengerjakan persoalan dengan pendekatan \fbruteForce.
\end{itemize}
\end{frame}

\begin{frame}
\frametitle{Konsep}
\begin{itemize}
  \item \newTerm{\fBruteForce} bukan suatu algoritma khusus, melainkan suatu strategi penyelesaian masalah.
  \item Sebutan lainnya adalah \fCompleteSearch dan \fExhaustiveSearch.
  \item Prinsip dari strategi ini hanya satu, yaitu...
\end{itemize}
\end{frame}

\begin{frame}
\frametitle{Konsep (lanj.)}
\begin{center}
  \huge coba semua kemungkinan!
\end{center}
\end{frame}

\begin{frame}
\frametitle{Sifat Brute Force}
\begin{itemize}
  \item \fBruteForce \emp{menjamin} solusi pasti benar, karena seluruh kemungkinan dijelajahi.
  \item Akibatnya, umumnya \fbruteForce bekerja dengan lambat.
  \item Terutama ketika banyak kemungkinan solusi yang perlu dicoba.
\end{itemize}
\end{frame}

\begin{frame}
\frametitle{Soal: Subset Sum}
\begin{itemize}
  \item Diberikan $N$ buah bilangan $\{a_1, a_2, ..., a_N\}$ dan bilangan $K$.
  \item Apakah terdapat subhimpunan sedemikian sehingga jumlahan dari elemen-elemennya sama dengan $K$? 
  \item Bila ya, maka keluarkan "YA". Selain itu keluarkan "TIDAK".
\end{itemize}

Batasan:
\begin{itemize} 
  \item $1 \leq N \leq 15$
  \item $1 \leq K \leq 10^9$
  \item $1 \leq a_i \leq 10^9$
\end{itemize}
\end{frame}

\begin{frame}
\frametitle{Solusi}
\begin{itemize}
  \item Untuk setiap elemen, kita memiliki 2 pilihan, yaitu memilih elemen tersebut atau tidak memilihnya.
  \item Kita akan menelusuri semua kemungkinan pilihan.
  \item Jika jumlahan dari elemen-elemen yang dipilih sama dengan $K$, maka terdapat solusi.
  \item Hal ini dapat dengan mudah diimplementasikan secara rekursif.
\end{itemize}
\end{frame}

\begin{frame}
\frametitle{Performa?}
\begin{itemize}
  \item Terdapat $2^N$ kemungkinan konfigurasi "pilih/tidak pilih".
  \item Kompleksitas solusi adalah $O(2^N)$.
  \item Untuk nilai $N$ terbesar, $2^N = 2^{15} = 32.768$.
  \item Masih jauh di bawah 100 juta, yaitu banyaknya operasi komputer perdetik pada umumnya.
\end{itemize}
\end{frame}

\begin{frame}
\frametitle{Implementasi}
\begin{codebox}
\Procname{$\proc{solve}(i, sum)$}
\li \If $i > N$ \Then
\li   \Return $(sum \isequal K)$
    \End
\zi
\li $option1 \gets \proc{solve}(i+1, sum + a_i)$ \Comment Pilih elemen $a_i$
\li $option2 \gets \proc{solve}(i+1, sum)$ \Comment Tidak pilih elemen $a_i$
\li \Return $option1$ or $option2$
\end{codebox}

\begin{codebox}
\Procname{$\proc{solveSubsetSum}()$}
\li \Return $\proc{solve}(1, 0)$
\end{codebox}

\end{frame}

\begin{frame}
\frametitle{Optimisasi}
\begin{itemize}
  \item Bisakah solusi tersebut menjadi lebih cepat?
  \item Perhatikan kasus ketika nilai $sum$ telah melebihi $K$.
  \item Karena semua $a_i$ bernilai positif, maka $sum$ tidak akan mengecil.
  \item Karena itu, bila $sum$ sudah melebihi $K$, \emp{dipastikan} tidak akan tercapai sebuah solusi.
\end{itemize}
\end{frame}

\begin{frame}
\frametitle{Solusi Teroptimisasi}
\begin{codebox}
\Procname{$\proc{solve}(i, sum)$}
\li \If $i > N$ \Then
\li   \Return $(sum \isequal K)$
    \End
\zi
\li \If $sum > K$ \Then
\li   \Return $false$
    \End
\zi
\li $option1 \gets \proc{solve}(i+1, sum + a_i)$ \Comment Pilih elemen $a_i$
\li $option2 \gets \proc{solve}(i+1, sum)$ \Comment Tidak pilih elemen $a_i$
\li \Return $option1$ or $option2$
\end{codebox}
\end{frame}

\begin{frame}
\frametitle{Pruning}
Hal ini biasa disebut sebagai \newTerm{pruning} (pemangkasan).
\begin{block}{Pruning}
  Merupakan optimisasi dengan mengurangi ruang pencarian dengan cara menghindari pencarian yang sudah pasti salah.
\end{block}
\end{frame}

\begin{frame}
\frametitle{Pruning (lanj.)}
\begin{itemize}
  \item Meskipun mengurangi ruang pencarian, \fpruning umumnya tidak mengurangi kompleksitas solusi.
  \item Sebab, biasanya terdapat kasus yang mana \fpruning tidak mengurangi ruang pencarian secara signifikan.
  \item Pada kasus ini, solusi dapat dianggap tetap bekerja dalam $O(2^N)$.
\end{itemize}
\end{frame}

\begin{frame}
\frametitle{Soal: Mengatur Persamaan}
\begin{itemize}
  \item Diberikan sebuah persamaan: $p + q + r = 0$.
  \item Masing-masing dari $p$, $q$, dan $r$ harus merupakan anggota dari himpunan bilangan yang unik $\{ a_1, a_2, ..., a_N \}$
  \item Berapa banyak triplet ($p, q, r$) berbeda yang memenuhi persamaan tersebut?
\end{itemize}

Batasan:
\begin{itemize} 
  \item $1 \leq N \leq 2.000$
  \item $-10^5 \leq a_i \leq 10^5$
\end{itemize}
\end{frame}

\begin{frame}
\frametitle{Solusi Sederhana}

\begin{codebox}
\Procname{$\proc{countTriplets}()$}
\li $count \gets 0$
\li \For $i \gets 1$ \To $N$ \Do
\li   \For $j \gets 1$ \To $N$ \Do
\li     \For $k \gets 1$ \To $N$ \Do
\li       $p \gets a_i$
\li       $q \gets a_j$
\li       $r \gets a_k$
\li       \If $(p + q + r) \isequal 0$ \Then
\li         $count \gets count + 1$
          \End
        \End
      \End
    \End
\li \Return $count$
\end{codebox}
\end{frame}

\begin{frame}
\frametitle{Solusi Sederhana (lanj.)}
\begin{itemize}
  \item Kompleksitas waktu solusi ini adalah $O(N^3)$.
  \item Tentunya terlalu besar untuk nilai $N$ mencapai 2.000.
  \item Ada solusi yang lebih baik?
\end{itemize}
\end{frame}

\begin{frame}
\frametitle{Observasi}
\begin{itemize}
  \item Jika kita sudah menentukan nilai $p$ dan $q$, maka nilai $r$ haruslah $-(p + q)$.
  \item Jadi cukup tentukan nilai $p$ dan $q$, lalu periksa apakah nilai $-(p + q)$ ada pada bilangan-bilangan yang diberikan.
  \item Pemeriksaan ini dapat dilakukan dengan \fBinarySearch.
  \item Kompleksitas solusi menjadi $O(N^2 \log{N})$
\end{itemize}
\end{frame}

\begin{frame}
\frametitle{Solusi Lebih Baik}
\begin{codebox}
\Procname{$\proc{countTripletsFast}()$}
\li $count \gets 0$
\li \For $i \gets 1$ \To $N$ \Do
\li   \For $j \gets 1$ \To $N$ \Do
\li       $p \gets a_i$
\li       $q \gets a_j$
\li       $r \gets -(p + q)$
\li       \If $\proc{exists}(r)$ \Then
\li         $count \gets count + 1$
          \End
        \End
      \End
    \End
\li \Return $count$
\end{codebox}

dengan $\proc{exists}(r)$ adalah algoritma \fBinarySearch untuk memeriksa keberadaan $r$ di $\{ a_1, a_2, ..., a_N \}$ (tentunya setelah diurutkan).
\end{frame}

\begin{frame}
\frametitle{Solusi Lebih Baik (lanj.)}
\begin{itemize}
  \item Kompleksitas $O(N^2 \log{N})$ sudah cukup untuk $N$ yang mencapai 2.000.
  \item Dari sini kita belajar bahwa optimisasi pada pencarian kadang diperlukan, meskipun ide dasarnya adalah \fbruteForce.
\end{itemize}
\end{frame}

\begin{frame}
\frametitle{Penutup}
\begin{itemize}
  \item Ide dari \fbruteForce biasanya sederhana: Anda hanya perlu menjelajahi seluruh kemungkinan solusi.
  \item Biasanya merupakan ide pertama yang didapatkan saat menghadapi masalah.
  \item Lakukan analisis algoritma, jika kompleksitasnya cukup, maka \fbruteForce saja :)
  \item Bila tidak cukup cepat, coba lakukan observasi. 
  \item Bisa jadi kita dapat melakukan \fbruteForce dari "sudut pandang yang lain" dan lebih cepat.
  \item Bila tidak berhasil juga, baru coba pikirkan strategi lainnya.
\end{itemize}
\end{frame}

\end{document}
