\documentclass{beamer}
\usetheme{tokitex}

\usepackage{tikz}
\usepackage{graphics}
\usepackage{multirow}
\usepackage{tabto}
\usepackage{xspace}
\usepackage{amsmath}
\usepackage{hyperref}

\usepackage{tikz}
\usepackage{clrscode3e}

\usepackage[english,bahasa]{babel}
\newtranslation[to=bahasa]{Section}{Bagian}
\newtranslation[to=bahasa]{Subsection}{Subbagian}

\usepackage{listings, lstautogobble}
\usepackage{color}

\definecolor{dkgreen}{rgb}{0,0.6,0}
\definecolor{gray}{rgb}{0.5,0.5,0.5}
\definecolor{mauve}{rgb}{0.58,0,0.82}

\lstset{frame=tb,
  language=pascal,
  aboveskip=1mm,
  belowskip=1mm,
  showstringspaces=false,
  columns=fullflexible,
  keepspaces=true,
  basicstyle={\small\ttfamily},
  numbers=none,
  numberstyle=\tiny\color{gray},
  keywordstyle=\color{blue},
  commentstyle=\color{dkgreen},
  stringstyle=\color{mauve},
  breaklines=true,
  breakatwhitespace=true,
  autogobble=true
}

\usepackage{caption}
\captionsetup[figure]{labelformat=empty}

\newcommand{\progTerm}[1]{\textbf{#1}}
\newcommand{\foreignTerm}[1]{\textit{#1}}
\newcommand{\newTerm}[1]{\alert{\textbf{#1}}}
\newcommand{\emp}[1]{\alert{#1}}
\newcommand{\statement}[1]{"#1"}

% Getting tired of writing \foreignTerm all the time
\newcommand{\farray}{\foreignTerm{array}\xspace}
\newcommand{\fArray}{\foreignTerm{Array}\xspace}
\newcommand{\foverhead}{\foreignTerm{overhead}\xspace}
\newcommand{\fOverhead}{\foreignTerm{Overhead}\xspace}
\newcommand{\fsubarray}{\foreignTerm{subarray}\xspace}
\newcommand{\fSubarray}{\foreignTerm{Subarray}\xspace}
\newcommand{\fbasecase}{\foreignTerm{base case}\xspace}
\newcommand{\fBasecase}{\foreignTerm{Base case}\xspace}
\newcommand{\ftopdown}{\foreignTerm{top down}\xspace}
\newcommand{\fTopdown}{\foreignTerm{Top down}\xspace}
\newcommand{\fbottomup}{\foreignTerm{bottom up}\xspace}
\newcommand{\fBottomup}{\foreignTerm{Bottom up}\xspace}
\newcommand{\fpruning}{\foreignTerm{pruning}\xspace}
\newcommand{\fPruning}{\foreignTerm{Pruning}\xspace}

\newcommand{\fgraph}{\foreignTerm{graph}\xspace}
\newcommand{\fGraph}{\foreignTerm{Graph}\xspace}
\newcommand{\fnode}{\foreignTerm{node}\xspace}
\newcommand{\fNode}{\foreignTerm{Node}\xspace}
\newcommand{\fedge}{\foreignTerm{edge}\xspace}
\newcommand{\fEdge}{\foreignTerm{Edge}\xspace}
\newcommand{\fdegree}{\foreignTerm{degree}\xspace}
\newcommand{\fDegree}{\foreignTerm{Degree}\xspace}
\newcommand{\fadjacencylist}{\foreignTerm{adjacency list}\xspace}
\newcommand{\fAdjacencylist}{\foreignTerm{Adjacency list}\xspace}
\newcommand{\fadjacencymatrix}{\foreignTerm{adjacency matrix}\xspace}
\newcommand{\fAdjacencymatrix}{\foreignTerm{Adjacency matrix}\xspace}
\newcommand{\fedgelist}{\foreignTerm{edge list}\xspace}
\newcommand{\fEdgelist}{\foreignTerm{Edge list}\xspace}
\newcommand{\flist}{\foreignTerm{list}\xspace}
\newcommand{\fList}{\foreignTerm{List}\xspace}
\newcommand{\fgraphtraversal}{\foreignTerm{graph traversal}\xspace}
\newcommand{\fGraphtraversal}{\foreignTerm{Graph traversal}\xspace}
\newcommand{\ftree}{\foreignTerm{tree}\xspace}
\newcommand{\fTree}{\foreignTerm{Tree}\xspace}
\newcommand{\fsubtree}{\foreignTerm{subtree}\xspace}
\newcommand{\fSubtree}{\foreignTerm{Subtree}\xspace}

\newcommand{\fDivideAndConquer}{\foreignTerm{Divide and conquer}\xspace}
\newcommand{\fdivideAndConquer}{\foreignTerm{divide and conquer}\xspace}
\newcommand{\fMergeSort}{\foreignTerm{Merge sort}\xspace}
\newcommand{\fmergeSort}{\foreignTerm{merge sort}\xspace}
\newcommand{\fQuickSort}{\foreignTerm{Quicksort}\xspace}
\newcommand{\fquickSort}{\foreignTerm{quicksort}\xspace}
\newcommand{\fpivot}{\foreignTerm{pivot}\xspace}
\newcommand{\fPivot}{\foreignTerm{Pivot}\xspace}
\newcommand{\fbruteForce}{\foreignTerm{brute force}\xspace}
\newcommand{\fBruteForce}{\foreignTerm{Brute force}\xspace}
\newcommand{\fCompleteSearch}{\foreignTerm{complete search}\xspace}
\newcommand{\fExhaustiveSearch}{\foreignTerm{exhaustive search}\xspace}
\newcommand{\fBinarySearch}{\foreignTerm{binary search}\xspace}
\newcommand{\fGreedy}{\foreignTerm{greedy}\xspace}
\newcommand{\fGreedyChoice}{\foreignTerm{greedy choice}\xspace}

\newcommand{\pheap}{\foreignTerm{heap}\xspace}
\newcommand{\pHeap}{\foreignTerm{Heap}\xspace}
\newcommand{\pBinaryHeap}{\foreignTerm{Binary Heap}\xspace}
\newcommand{\pbinaryHeap}{\foreignTerm{binary heap}\xspace}
\newcommand{\pHeapSort}{\foreignTerm{Heap Sort}\xspace}
\newcommand{\pdjs}{\foreignTerm{disjoint set}\xspace}
\newcommand{\pDjs}{\foreignTerm{Disjoint set}\xspace}

\title{Brute Force}
\author{Tim Olimpiade Komputer Indonesia}
\date{}

\begin{document}

\begin{frame}
\titlepage
\end{frame}

\begin{frame}
\frametitle{Pendahuluan}
Melalui dokumen ini, kalian akan:
\begin{itemize}
  \item Mempelajari konsep pencarian \foreignTerm{brute force}
  \item Mampu mengerjakan persoalan dengan pendekatan \foreignTerm{brute force}
\end{itemize}

\end{frame}

\begin{frame}
\frametitle{Soal : Subset Sum}
Deskripsi:
\begin{itemize}
  \item Diberikan $N$ buah integer $(a_1, a_2, ..., a_n)$ dan integer $K$
  \item Apakah terdapat subset dari integer-integer tersebut sehingga jumlahan dari elemen subset tersebut sama dengan $K$? 
  \item Bila iya, maka keluarkan "YA". Selain itu keluarkan "TIDAK"
\end{itemize}

Batasan:
\begin{itemize} 
  \item $1 \leq N \leq 15$
  \item $1 \leq K \leq 10^9$
  \item $1 \leq a_i \leq 10^9$
\end{itemize}

\end{frame}

\begin{frame}
\frametitle{Solusi}
\begin{itemize}
  \item Untuk setiap elemen, kita memiliki 2 pilihan yaitu memilih elemen tersebut atau tidak memilihnya.
  \item Jika jumlahan dari elemen-elemen yang dipilih sama dengan $K$, maka terdapat solusi.
\end{itemize}
\end{frame}

\begin{frame}[fragile]
\frametitle{Solusi (Lanj.)}
\begin{lstlisting}
solve(pos, sum):
  if pos > n then
    if sum == K then
      return true
    else
      return false

  return solve(pos + 1, sum) 
      or solve(pos + 1, sum + a[pos])

if solve(1,0) == true then
  print "YA"
else
  print "TIDAK"
\end{lstlisting}
\end{frame}

\begin{frame}
\frametitle{Optimisasi}
Bisakah solusi tersebut menjadi lebih cepat?
gunakan \newTerm{pruning}!

\begin{block}{Pruning}
  Merupakan optimisasi dengan mengurangi \foreignTerm{search space} dengan cara melakukan eksklusi pada kemungkinan jawaban yang pasti salah.
\end{block}
\end{frame}

\begin{frame}[fragile]
\frametitle{Solusi Optimasi}
\begin{lstlisting}
solve(pos, sum):
  if sum > K then
    return false

  if pos > n then
    if sum == K then
      return true
    else
      return false

  return solve(pos + 1, sum) 
      or solve(pos + 1, sum + a[pos])

if solve(1,0) == true then
  print "YA"
else
  print "TIDAK"
\end{lstlisting}
\end{frame}

\begin{frame}
\frametitle{Penjelasan}
\begin{itemize}
  \item Karena semua $a_i$ bernilai positif, maka \id{sum} tidak akan menjadi lebih kecil.
  \item Karena itu, bila \id{sum} sudah lebih besar dari \id{K} bisa dipastikan tidak akan tercapai sebuah solusi
\end{itemize}
\end{frame}

\begin{frame}
\frametitle{Backtracking}
  \begin{block}{Backtracking}
    Merupakan metode untuk mengetahui solusi dari \foreignTerm{brute force}
  \end{block}
\end{frame}

\begin{frame}[fragile]
\frametitle{Solusi Backtracking}
\begin{lstlisting}
solve(pos, sum):
  if sum > K then
    return false

  if pos > n then
    if sum == K then
      print sequence
      return true
    else
      return false

  ret = solve(pos + 1, sum)
  sequence.push_back(a[pos])
  ret = res or solve(pos + 1, sum + a[pos])
  sequence.pop_back()

  return ret

\end{lstlisting}
\end{frame}

\begin{frame}[fragile]
\frametitle{Solusi Backtracking}
\begin{lstlisting}
if solve(1,0) == true then
  print "YA"
else
  print "TIDAK"
\end{lstlisting}
\end{frame}

\begin{frame}
\frametitle{TODO Konten}
\begin{itemize}
  \item Berikan contoh lagi yang lebih sulit, dan tunjukkan bahwa kadang2 kita harus pintar dalam menentukan apa yang mau di-brute force
  \item Ajarkan untuk hitung kompleksitasnya. Kalau cukup untuk AC, langsung coding aja
\end{itemize}
\end{frame}

\end{document}
