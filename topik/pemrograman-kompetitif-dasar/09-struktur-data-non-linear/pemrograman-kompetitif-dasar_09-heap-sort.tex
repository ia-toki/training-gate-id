\documentclass{beamer}
\usetheme{tokitex}

\usepackage{tikz}
\usepackage{graphics}
\usepackage{multirow}
\usepackage{tabto}
\usepackage{xspace}
\usepackage{amsmath}
\usepackage{hyperref}

\usepackage{tikz}
\usepackage{clrscode3e}

\usepackage[english,bahasa]{babel}
\newtranslation[to=bahasa]{Section}{Bagian}
\newtranslation[to=bahasa]{Subsection}{Subbagian}

\usepackage{listings, lstautogobble}
\usepackage{color}

\definecolor{dkgreen}{rgb}{0,0.6,0}
\definecolor{gray}{rgb}{0.5,0.5,0.5}
\definecolor{mauve}{rgb}{0.58,0,0.82}

\lstset{frame=tb,
  language=pascal,
  aboveskip=1mm,
  belowskip=1mm,
  showstringspaces=false,
  columns=fullflexible,
  keepspaces=true,
  basicstyle={\small\ttfamily},
  numbers=none,
  numberstyle=\tiny\color{gray},
  keywordstyle=\color{blue},
  commentstyle=\color{dkgreen},
  stringstyle=\color{mauve},
  breaklines=true,
  breakatwhitespace=true,
  autogobble=true
}

\usepackage{caption}
\captionsetup[figure]{labelformat=empty}

\newcommand{\progTerm}[1]{\textbf{#1}}
\newcommand{\foreignTerm}[1]{\textit{#1}}
\newcommand{\newTerm}[1]{\alert{\textbf{#1}}}
\newcommand{\emp}[1]{\alert{#1}}
\newcommand{\statement}[1]{"#1"}

% Getting tired of writing \foreignTerm all the time
\newcommand{\farray}{\foreignTerm{array}\xspace}
\newcommand{\fArray}{\foreignTerm{Array}\xspace}
\newcommand{\foverhead}{\foreignTerm{overhead}\xspace}
\newcommand{\fOverhead}{\foreignTerm{Overhead}\xspace}
\newcommand{\fsubarray}{\foreignTerm{subarray}\xspace}
\newcommand{\fSubarray}{\foreignTerm{Subarray}\xspace}
\newcommand{\fbasecase}{\foreignTerm{base case}\xspace}
\newcommand{\fBasecase}{\foreignTerm{Base case}\xspace}
\newcommand{\ftopdown}{\foreignTerm{top down}\xspace}
\newcommand{\fTopdown}{\foreignTerm{Top down}\xspace}
\newcommand{\fbottomup}{\foreignTerm{bottom up}\xspace}
\newcommand{\fBottomup}{\foreignTerm{Bottom up}\xspace}
\newcommand{\fpruning}{\foreignTerm{pruning}\xspace}
\newcommand{\fPruning}{\foreignTerm{Pruning}\xspace}

\newcommand{\fgraph}{\foreignTerm{graph}\xspace}
\newcommand{\fGraph}{\foreignTerm{Graph}\xspace}
\newcommand{\fnode}{\foreignTerm{node}\xspace}
\newcommand{\fNode}{\foreignTerm{Node}\xspace}
\newcommand{\fedge}{\foreignTerm{edge}\xspace}
\newcommand{\fEdge}{\foreignTerm{Edge}\xspace}
\newcommand{\fdegree}{\foreignTerm{degree}\xspace}
\newcommand{\fDegree}{\foreignTerm{Degree}\xspace}
\newcommand{\fadjacencylist}{\foreignTerm{adjacency list}\xspace}
\newcommand{\fAdjacencylist}{\foreignTerm{Adjacency list}\xspace}
\newcommand{\fadjacencymatrix}{\foreignTerm{adjacency matrix}\xspace}
\newcommand{\fAdjacencymatrix}{\foreignTerm{Adjacency matrix}\xspace}
\newcommand{\fedgelist}{\foreignTerm{edge list}\xspace}
\newcommand{\fEdgelist}{\foreignTerm{Edge list}\xspace}
\newcommand{\flist}{\foreignTerm{list}\xspace}
\newcommand{\fList}{\foreignTerm{List}\xspace}
\newcommand{\fgraphtraversal}{\foreignTerm{graph traversal}\xspace}
\newcommand{\fGraphtraversal}{\foreignTerm{Graph traversal}\xspace}
\newcommand{\ftree}{\foreignTerm{tree}\xspace}
\newcommand{\fTree}{\foreignTerm{Tree}\xspace}
\newcommand{\fsubtree}{\foreignTerm{subtree}\xspace}
\newcommand{\fSubtree}{\foreignTerm{Subtree}\xspace}

\newcommand{\fDivideAndConquer}{\foreignTerm{Divide and conquer}\xspace}
\newcommand{\fdivideAndConquer}{\foreignTerm{divide and conquer}\xspace}
\newcommand{\fMergeSort}{\foreignTerm{Merge sort}\xspace}
\newcommand{\fmergeSort}{\foreignTerm{merge sort}\xspace}
\newcommand{\fQuickSort}{\foreignTerm{Quicksort}\xspace}
\newcommand{\fquickSort}{\foreignTerm{quicksort}\xspace}
\newcommand{\fpivot}{\foreignTerm{pivot}\xspace}
\newcommand{\fPivot}{\foreignTerm{Pivot}\xspace}
\newcommand{\fbruteForce}{\foreignTerm{brute force}\xspace}
\newcommand{\fBruteForce}{\foreignTerm{Brute force}\xspace}
\newcommand{\fCompleteSearch}{\foreignTerm{complete search}\xspace}
\newcommand{\fExhaustiveSearch}{\foreignTerm{exhaustive search}\xspace}
\newcommand{\fBinarySearch}{\foreignTerm{binary search}\xspace}
\newcommand{\fGreedy}{\foreignTerm{greedy}\xspace}
\newcommand{\fGreedyChoice}{\foreignTerm{greedy choice}\xspace}

\newcommand{\pheap}{\foreignTerm{heap}\xspace}
\newcommand{\pHeap}{\foreignTerm{Heap}\xspace}
\newcommand{\pBinaryHeap}{\foreignTerm{Binary Heap}\xspace}
\newcommand{\pbinaryHeap}{\foreignTerm{binary heap}\xspace}
\newcommand{\pHeapSort}{\foreignTerm{Heap Sort}\xspace}
\newcommand{\pdjs}{\foreignTerm{disjoint set}\xspace}
\newcommand{\pDjs}{\foreignTerm{Disjoint set}\xspace}

\title{Aplikasi Struktur Data Nonlinear: \newline Heapsort}
\author{Tim Olimpiade Komputer Indonesia}
\date{}

\begin{document}

\begin{frame}
\titlepage
\end{frame}

\begin{frame}
\frametitle{Pendahuluan}
Melalui dokumen ini, kalian akan:
\begin{itemize}
  \item Menggunakan \pheap untuk \pheapsort.
  \item Mengamati bagaimana struktur data diaplikasikan dalam penyelesaian masalah.
\end{itemize}
\end{frame}

\begin{frame}
\frametitle{Kilas Balik}
Ingat dengan \foreignTerm{selection sort}?\newline

Berikut adalah algoritmanya:
\begin{itemize}
  \item Pilih elemen terkecil, tempatkan di paling awal data.
  \item Ulangi hingga seluruh data terurut menaik.
\end{itemize}
\end{frame}

\begin{frame}
\frametitle{Kilas Balik (lanj.)}
\begin{itemize}
  \item Pencarian elemen terkecil pada \foreignTerm{selection sort} dilakukan dengan \foreignTerm{linear search}.
  \item Sekarang kita telah mengetahui strategi pencarian elemen terbesar/terkecil dari sekumpulan data secara efisien.
  \item Bagian pencarian elemen terkecil pada \foreignTerm{selection sort} dapat dioptimisasi menggunakan \pheap, dan algoritma pengurutan ini dinamakan \newTerm{\pheapsort}.
\end{itemize}
\end{frame}

\begin{frame}
\frametitle{Heapsort}
Misalkan kita memiliki \farray $A$ berisi $N$ elemen yang hendak diurutkan menaik:
\begin{enumerate}
  \item Muat seluruh elemen \farray $A$ pada \textbf{min-heap}, menggunakan \proc{makeHeap}.
  \item Untuk $i$ dari $0$ sampai $N-1$, lakukan:
  \begin{enumerate}
    \item Dapatkan elemen terkecil dari \pheap dengan \proc{pop}.
    \item Tempatkan elemen ini pada $A[i]$.
  \end{enumerate}
\end{enumerate}
\end{frame}

\begin{frame}
\frametitle{Analisis Heapsort}
\begin{itemize}
  \item Operasi \proc{makeHeap} bekerja dalam $O(N)$.
  \item Kemudian dilakukan $N$ kali \proc{pop}, sehingga kompleksitasnya $O(N\log{N})$.
  \item Jadi kompleksitas akhir \pheapsort adalah $O(N\log{N})$.
\end{itemize}
\end{frame}

\begin{frame}
\frametitle{Keuntungan Heapsort}
\begin{itemize}
  \item \pHeapsort memiliki keuntungan yang sama dengan \foreignTerm{selection sort}, yaitu dapat melakukan \foreignTerm{partial sort}.
  \item Artinya, bila Anda hanya membutuhkan $K$ elemen terkecil, Anda dapat berhenti setelah melakukan \proc{pop} sebanyak $K$ kali.
  \item Kompleksitasnya menjadi $O(N + K \log{N})$, yang lebih efisien ketika $K$ jauh lebih kecil dari $N$.
\end{itemize}
\end{frame}

\begin{frame}
\frametitle{Penutup}
\begin{itemize}
  \item Kini Anda telah memahami 3 pengurutan dengan kompleksitas $O(N \log{N})$, setelah \foreignTerm{quicksort} dan \foreignTerm{merge sort}.
  \item Meskipun demikian, \pheapsort lebih jarang digunakan untuk pengurutan $N$ data, karena implementasinya yang cukup rumit.
  \item \pHeapsort hanya digunakan untuk kebutuhan tertentu, seperti \foreignTerm{partial sort}.
\end{itemize}
\end{frame}

\end{document}


