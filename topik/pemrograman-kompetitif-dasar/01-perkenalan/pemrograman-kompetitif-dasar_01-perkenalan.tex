\documentclass{beamer}
\usetheme{tokitex}

\usepackage{tikz}
\usepackage{graphics}
\usepackage{multirow}
\usepackage{tabto}
\usepackage{xspace}
\usepackage{amsmath}
\usepackage{hyperref}

\usepackage{tikz}
\usepackage{clrscode3e}

\usepackage[english,bahasa]{babel}
\newtranslation[to=bahasa]{Section}{Bagian}
\newtranslation[to=bahasa]{Subsection}{Subbagian}

\usepackage{listings, lstautogobble}
\usepackage{color}

\definecolor{dkgreen}{rgb}{0,0.6,0}
\definecolor{gray}{rgb}{0.5,0.5,0.5}
\definecolor{mauve}{rgb}{0.58,0,0.82}

\lstset{frame=tb,
  language=pascal,
  aboveskip=1mm,
  belowskip=1mm,
  showstringspaces=false,
  columns=fullflexible,
  keepspaces=true,
  basicstyle={\small\ttfamily},
  numbers=none,
  numberstyle=\tiny\color{gray},
  keywordstyle=\color{blue},
  commentstyle=\color{dkgreen},
  stringstyle=\color{mauve},
  breaklines=true,
  breakatwhitespace=true,
  autogobble=true
}

\usepackage{caption}
\captionsetup[figure]{labelformat=empty}

\newcommand{\progTerm}[1]{\textbf{#1}}
\newcommand{\foreignTerm}[1]{\textit{#1}}
\newcommand{\newTerm}[1]{\alert{\textbf{#1}}}
\newcommand{\emp}[1]{\alert{#1}}
\newcommand{\statement}[1]{"#1"}

% Getting tired of writing \foreignTerm all the time
\newcommand{\farray}{\foreignTerm{array}\xspace}
\newcommand{\fArray}{\foreignTerm{Array}\xspace}
\newcommand{\foverhead}{\foreignTerm{overhead}\xspace}
\newcommand{\fOverhead}{\foreignTerm{Overhead}\xspace}
\newcommand{\fsubarray}{\foreignTerm{subarray}\xspace}
\newcommand{\fSubarray}{\foreignTerm{Subarray}\xspace}
\newcommand{\fbasecase}{\foreignTerm{base case}\xspace}
\newcommand{\fBasecase}{\foreignTerm{Base case}\xspace}
\newcommand{\ftopdown}{\foreignTerm{top down}\xspace}
\newcommand{\fTopdown}{\foreignTerm{Top down}\xspace}
\newcommand{\fbottomup}{\foreignTerm{bottom up}\xspace}
\newcommand{\fBottomup}{\foreignTerm{Bottom up}\xspace}
\newcommand{\fpruning}{\foreignTerm{pruning}\xspace}
\newcommand{\fPruning}{\foreignTerm{Pruning}\xspace}

\newcommand{\fgraph}{\foreignTerm{graph}\xspace}
\newcommand{\fGraph}{\foreignTerm{Graph}\xspace}
\newcommand{\fnode}{\foreignTerm{node}\xspace}
\newcommand{\fNode}{\foreignTerm{Node}\xspace}
\newcommand{\fedge}{\foreignTerm{edge}\xspace}
\newcommand{\fEdge}{\foreignTerm{Edge}\xspace}
\newcommand{\fdegree}{\foreignTerm{degree}\xspace}
\newcommand{\fDegree}{\foreignTerm{Degree}\xspace}
\newcommand{\fadjacencylist}{\foreignTerm{adjacency list}\xspace}
\newcommand{\fAdjacencylist}{\foreignTerm{Adjacency list}\xspace}
\newcommand{\fadjacencymatrix}{\foreignTerm{adjacency matrix}\xspace}
\newcommand{\fAdjacencymatrix}{\foreignTerm{Adjacency matrix}\xspace}
\newcommand{\fedgelist}{\foreignTerm{edge list}\xspace}
\newcommand{\fEdgelist}{\foreignTerm{Edge list}\xspace}
\newcommand{\flist}{\foreignTerm{list}\xspace}
\newcommand{\fList}{\foreignTerm{List}\xspace}
\newcommand{\fgraphtraversal}{\foreignTerm{graph traversal}\xspace}
\newcommand{\fGraphtraversal}{\foreignTerm{Graph traversal}\xspace}
\newcommand{\ftree}{\foreignTerm{tree}\xspace}
\newcommand{\fTree}{\foreignTerm{Tree}\xspace}
\newcommand{\fsubtree}{\foreignTerm{subtree}\xspace}
\newcommand{\fSubtree}{\foreignTerm{Subtree}\xspace}

\newcommand{\fDivideAndConquer}{\foreignTerm{Divide and conquer}\xspace}
\newcommand{\fdivideAndConquer}{\foreignTerm{divide and conquer}\xspace}
\newcommand{\fMergeSort}{\foreignTerm{Merge sort}\xspace}
\newcommand{\fmergeSort}{\foreignTerm{merge sort}\xspace}
\newcommand{\fQuickSort}{\foreignTerm{Quicksort}\xspace}
\newcommand{\fquickSort}{\foreignTerm{quicksort}\xspace}
\newcommand{\fpivot}{\foreignTerm{pivot}\xspace}
\newcommand{\fPivot}{\foreignTerm{Pivot}\xspace}
\newcommand{\fbruteForce}{\foreignTerm{brute force}\xspace}
\newcommand{\fBruteForce}{\foreignTerm{Brute force}\xspace}
\newcommand{\fCompleteSearch}{\foreignTerm{complete search}\xspace}
\newcommand{\fExhaustiveSearch}{\foreignTerm{exhaustive search}\xspace}
\newcommand{\fBinarySearch}{\foreignTerm{binary search}\xspace}
\newcommand{\fGreedy}{\foreignTerm{greedy}\xspace}
\newcommand{\fGreedyChoice}{\foreignTerm{greedy choice}\xspace}

\newcommand{\pheap}{\foreignTerm{heap}\xspace}
\newcommand{\pHeap}{\foreignTerm{Heap}\xspace}
\newcommand{\pBinaryHeap}{\foreignTerm{Binary Heap}\xspace}
\newcommand{\pbinaryHeap}{\foreignTerm{binary heap}\xspace}
\newcommand{\pHeapSort}{\foreignTerm{Heap Sort}\xspace}
\newcommand{\pdjs}{\foreignTerm{disjoint set}\xspace}
\newcommand{\pDjs}{\foreignTerm{Disjoint set}\xspace}

\title{Perkenalan Pemrograman Kompetitif Dasar}
\author{Tim Olimpiade Komputer Indonesia}
\date{}

\begin{document}

\begin{frame}
\titlepage
\end{frame}

\begin{frame}
\frametitle{Perkenalan}
Selamat datang di topik \newTerm{Pemrograman Kompetitif Dasar}!
\newline
\newline
Anda diharapkan telah menguasai materi pemrograman dasar, dan:
\begin{itemize}
  \item Mengetahui setidaknya satu bahasa pemrograman.
  \item Mampu membaca dan menulis program.
\end{itemize}
~\newline
Mulai dari bab ini, seluruh kode program akan dituliskan dalam \newTerm{pseudocode}.
\end{frame}

\begin{frame}
\frametitle{Pseudocode}
\begin{itemize}
  \item Merupakan \emp{bahasa informal} yang mirip dengan bahasa pemrograman untuk mendeskripsikan program.
  \item Biasa digunakan pada materi pembelajaran algoritma.
  \item \textit{Pseudocode} sendiri bukanlah bahasa pemrograman sungguhan.
\end{itemize}
\end{frame}


\begin{frame}
\frametitle{Contoh Pseudocode}
\begin{codebox}
\Procname{$\proc{insertionSort}(A)$}
\li \For $i \gets 2$ \To $\attrib{A}{length}$
    \Do
\li   $j \gets i$
\li   \While $(j > 1)$ and $(A[j] < A[j-1])$
      \Do
\li     $\proc{swap}(A[j], A[j-1])$
\li     $j \gets j-1$
      \End
    \End
\end{codebox}
\end{frame}

\begin{frame}
\frametitle{Pseudocode (lanj.)}
\begin{itemize}
  \item Memahami \textit{pseudocode} pada awalnya mungkin sulit.
  \item Seiring berjalannya waktu, Anda akan terbiasa memahami dan menggunakan \textit{pseudocode} dengan mudah.
\end{itemize}
\end{frame}

\begin{frame}
\frametitle{Tentang Pemrograman Kompetitif}
\begin{block}{Pemrograman Kompetitif}
Adalah penyelesaian soal yang terdefinisi dengan jelas dengan menulis program komputer dalam batasan-batasan tertentu (memori dan waktu).
\end{block}
\end{frame}

\begin{frame}
\frametitle{Tentang Pemrograman Kompetitif (lanj.)}
Peserta ditantang untuk:
\begin{itemize}
  \item Menganalisis permasalahan
  \item Merancang algoritma solusi
  \item Menuliskannya dalam bentuk program
\end{itemize}
\end{frame}

\begin{frame}
\frametitle{Tentang Pemrograman Kompetitif (lanj.)}
\begin{itemize}
  \item Masalah yang diberikan selalu terdefinisi dengan jelas (\foreignTerm{well-defined}).
  \item Contohnya, batasan masalah dan asumsi yang diperlukan diberikan dengan akurat dan tidak ambigu.
  \item Solusi ditulis dalam bentuk program program dan memenuhi batas-batas yang ditentukan.
  \item Batas yang ditentukan: waktu, memori, dan lainnya.
\end{itemize}
\end{frame}

\begin{frame}
\frametitle{Contoh Soal Pemrograman Kompetitif}
\begin{itemize}
  \item Terdapat $N$ ruangan yang dinomori dari 1 hingga $N$.
  \item Ruangan ke-$i$ memiliki sebuah lampu dan sebuah tombol.
  \item Bila tombol itu ditekan, keadaan lampu pada seluruh ruangan ke-$x$ akan berubah (dari mati menjadi menyala, atau sebaliknya), yang mana $x$ habis dibagi $i$.
  \item Contoh, bila $N$ = 10 dan tombol di ruangan ke-2 ditekan, maka keadaan lampu pada ruangan 2, 4, 6, 8, dan 10 akan berubah.
  \item Bila pada awalnya seluruh lampu berada pada keadaan mati, dan tombol pada setiap ruangan ditekan tepat sekali, bagaimanakah keadaan lampu pada ruangan ke-$N$?
\end{itemize}
\end{frame}

\begin{frame}
\frametitle{Contoh Soal Pemrograman Kompetitif (lanj.)}
\begin{itemize}
  \item Batas waktu: 1 detik.
  \item Batas memori: 32 MB.
  \item Nilai $N$ dijamin selalu memenuhi $1 \leq N \leq 10^{14}$.
\end{itemize}
\end{frame}

\begin{frame}
\frametitle{Solusi Naif}
Salah satu cara penyelesaiannya adalah dengan mensimulasikan skenario pada deskripsi soal:
\begin{itemize}
  \item Kita cukup memperhatikan lampu pada ruangan ke-$N$ saja.
  \item Mulai dari ruangan ke-1, dipastikan keadaan lampu pada ruangan ke-$N$ akan berubah ($N$ habis dibagi 1).
  \item Lanjut ke ruangan ke-2, periksa apakah 2 habis membagi $N$. Bila ya, ubah keadaan lampu ke-$N$.
  \item Lanjut ke ruangan ke-3, periksa apakah 3 habis membagi $N$. Bila ya, ubah keadaan lampu ke-$N$.
  \item ... dan seterusnya hingga ruangan ke-$N$.
\end{itemize}
\end{frame}

\begin{frame}
\frametitle{Solusi Naif (lanj.)}
\begin{itemize}
  \item Setelah selesai disimulasikan, periksa keadaan lampu ruangan ke-$N$ dan cetak jawabannya.
  \item Kompleksitas solusi ini adalah $O(N)$, dan hanya akan bekerja untuk nilai $N$ yang kecil.
  \item Untuk $N$ yang lebih besar, misalnya $N = 10^9$, kemungkinan besar diperlukan waktu lebih dari 1 detik.
  \item Solusi ini tidak akan mendapatkan nilai penuh, atau bahkan 0, tergantung skema penilaian yang digunakan.
\end{itemize}
\end{frame}

\begin{frame}
\frametitle{Solusi yang Lebih Baik}
\begin{itemize}
  \item Dengan sedikit observasi, yang sebenarnya perlu dilakukan adalah menghitung banyaknya pembagi dari $N$.
  \item Apabila banyaknya pembagi ganjil, berarti pada akhirnya lampu di ruangan ke-$N$ akan menyala (mengapa?).
  \item Bila genap, berarti lampu di ruangan ke-$N$ akan mati.
\end{itemize}
\end{frame}

\begin{frame}
\frametitle{Solusi yang Lebih Baik (lanj.)}
\begin{itemize}
  \item Untuk menghitung banyaknya pembagi dari $N$ dengan efisien, lakukan faktorisasi prima terlebih dahulu.
  \item Misalkan $N$ = 12, maka faktorisasi primanya adalah $2^2 \times 3$.
  \item Untuk menjadi pembagi dari 12, suatu bilangan hanya boleh:
  \begin{itemize}
    \item Memiliki faktor 2 maksimal sebanyak 2.
    \item Memiliki faktor 3 maksimal sebanyak 1.
    \item Tidak boleh memiliki faktor lainnya.
  \end{itemize}
\end{itemize}
\end{frame}

\begin{frame}
\frametitle{Solusi yang Lebih Baik (lanj.)}
Sebagai contoh, berikut daftar seluruh pembagi dari 12:
\begin{itemize}
  \item $1 = 2^0 \times 3^0$
  \item $2 = 2^1 \times 3^0$
  \item $3 = 2^0 \times 3^1$
  \item $4 = 2^2 \times 3^0$
  \item $6 = 2^1 \times 3^1$
  \item $12 = 2^2 \times 3^1$
\end{itemize}
\end{frame}

\begin{frame}
\frametitle{Solusi yang Lebih Baik (lanj.)}
\begin{itemize}
  \item Banyaknya pembagi dari 12 sebenarnya sama saja dengan banyaknya kombinasi yang bisa dipilih dari $\lbrace 2^0$, $2^1$, $2^2\rbrace $ dan $\lbrace 3^0$, $3^1\rbrace$.
  \item Banyaknya kombinasi sama dengan mengalikan banyaknya elemen pada tiap-tiap himpunan.
  \item Sehingga, banyaknya cara adalah $3 \times 2 = 6$ cara.
  \item Dengan demikian, banyaknya pembagi dari 12 adalah 6.
\end{itemize}
\end{frame}

\begin{frame}
\frametitle{Solusi yang Lebih Baik (lanj.)}
\begin{itemize}
  \item Cara ini juga berlaku untuk nilai $N$ yang lain.
  \item Misalnya $N = 172.872 = 2^3 \times 3^2 \times 7^4$.
  \item Berarti banyak pembaginya adalah $4 \times 3 \times 5 = 60$.
\end{itemize}
\end{frame}

\begin{frame}
\frametitle{Solusi yang Lebih Baik (lanj.)}
\begin{itemize}
  \item Secara umum, banyaknya pembagi dari:
  \newline
  \begin{center}
    $N = a_1^{p_1} \times a_2^{p_2} \times a_3^{p_3} \times ... \times a_k^{p_k}$
  \end{center}
  adalah:
  \newline
  \begin{center}
    $(1+p_1) \times (1+p_2) \times (1+p_3) \times ... \times (1+p_k)$
    \newline
  \end{center}
  \item Sehingga, cukup faktorkan $N$, lalu hitung banyak pembaginya dengan rumus tersebut.
  \item Faktorisasi bilangan dapat diimplementasikan dalam $O(\sqrt{N})$.
  \item 1 detik cukup untuk $N$ sampai $10^{14}$.
\end{itemize}
\end{frame}

\begin{frame}
\frametitle{Ajang Pemrograman Kompetitif}
\begin{itemize}
  \item Olimpiade Sains Nasional (\newTerm{OSN}) bidang komputer/informatika merupakan kompetisi tingkat nasional di Indonesia.
  \item International Olympiad in Informatics (\newTerm{IOI}) merupakan kompetisi tingkat internasional bagi siswa SMA dari seluruh dunia.
  \item Untuk mahasiswa, terdapat ACM International Collegiate Programming Contest (\newTerm{ACM-ICPC}) yang pesertanya terdiri dari tim-tim beranggotakan tiga orang.
\end{itemize}
\end{frame}

\begin{frame}
\frametitle{Manfaat Pemrograman Kompetitif}
\begin{itemize}
  \item Mengasah kemampuan memecahkan masalah.
  \item Bertemu dengan teman-teman sehobi.
  \item Biasanya, wawancara untuk masuk ke perusahaan teknologi terkemuka menggunakan soal-soal pemrograman kompetitif.
\end{itemize}
\end{frame}

\begin{frame}
\frametitle{Dan Tentunya...}
\begin{center}
  {\LARGE Menantang dan menyenangkan!}
\end{center}
\end{frame}

\begin{frame}
\frametitle{Penutup}
\begin{itemize}
  \item Sebagai pemanasan, silakan mengerjakan soal-soal latihan yang diberikan.
\end{itemize}
\end{frame}

\end{document}
