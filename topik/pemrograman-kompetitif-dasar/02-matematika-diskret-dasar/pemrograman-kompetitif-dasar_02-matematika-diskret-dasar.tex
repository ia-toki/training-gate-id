\documentclass{beamer}
\usetheme{tokitex}

\usepackage{tikz}
\usepackage{graphics}
\usepackage{multirow}
\usepackage{tabto}
\usepackage{xspace}
\usepackage{amsmath}
\usepackage{hyperref}

\usepackage{tikz}
\usepackage{clrscode3e}

\usepackage[english,bahasa]{babel}
\newtranslation[to=bahasa]{Section}{Bagian}
\newtranslation[to=bahasa]{Subsection}{Subbagian}

\usepackage{listings, lstautogobble}
\usepackage{color}

\definecolor{dkgreen}{rgb}{0,0.6,0}
\definecolor{gray}{rgb}{0.5,0.5,0.5}
\definecolor{mauve}{rgb}{0.58,0,0.82}

\lstset{frame=tb,
  language=pascal,
  aboveskip=1mm,
  belowskip=1mm,
  showstringspaces=false,
  columns=fullflexible,
  keepspaces=true,
  basicstyle={\small\ttfamily},
  numbers=none,
  numberstyle=\tiny\color{gray},
  keywordstyle=\color{blue},
  commentstyle=\color{dkgreen},
  stringstyle=\color{mauve},
  breaklines=true,
  breakatwhitespace=true,
  autogobble=true
}

\usepackage{caption}
\captionsetup[figure]{labelformat=empty}

\newcommand{\progTerm}[1]{\textbf{#1}}
\newcommand{\foreignTerm}[1]{\textit{#1}}
\newcommand{\newTerm}[1]{\alert{\textbf{#1}}}
\newcommand{\emp}[1]{\alert{#1}}
\newcommand{\statement}[1]{"#1"}

% Getting tired of writing \foreignTerm all the time
\newcommand{\farray}{\foreignTerm{array}\xspace}
\newcommand{\fArray}{\foreignTerm{Array}\xspace}
\newcommand{\foverhead}{\foreignTerm{overhead}\xspace}
\newcommand{\fOverhead}{\foreignTerm{Overhead}\xspace}
\newcommand{\fsubarray}{\foreignTerm{subarray}\xspace}
\newcommand{\fSubarray}{\foreignTerm{Subarray}\xspace}
\newcommand{\fbasecase}{\foreignTerm{base case}\xspace}
\newcommand{\fBasecase}{\foreignTerm{Base case}\xspace}
\newcommand{\ftopdown}{\foreignTerm{top down}\xspace}
\newcommand{\fTopdown}{\foreignTerm{Top down}\xspace}
\newcommand{\fbottomup}{\foreignTerm{bottom up}\xspace}
\newcommand{\fBottomup}{\foreignTerm{Bottom up}\xspace}
\newcommand{\fpruning}{\foreignTerm{pruning}\xspace}
\newcommand{\fPruning}{\foreignTerm{Pruning}\xspace}

\newcommand{\fgraph}{\foreignTerm{graph}\xspace}
\newcommand{\fGraph}{\foreignTerm{Graph}\xspace}
\newcommand{\fnode}{\foreignTerm{node}\xspace}
\newcommand{\fNode}{\foreignTerm{Node}\xspace}
\newcommand{\fedge}{\foreignTerm{edge}\xspace}
\newcommand{\fEdge}{\foreignTerm{Edge}\xspace}
\newcommand{\fdegree}{\foreignTerm{degree}\xspace}
\newcommand{\fDegree}{\foreignTerm{Degree}\xspace}
\newcommand{\fadjacencylist}{\foreignTerm{adjacency list}\xspace}
\newcommand{\fAdjacencylist}{\foreignTerm{Adjacency list}\xspace}
\newcommand{\fadjacencymatrix}{\foreignTerm{adjacency matrix}\xspace}
\newcommand{\fAdjacencymatrix}{\foreignTerm{Adjacency matrix}\xspace}
\newcommand{\fedgelist}{\foreignTerm{edge list}\xspace}
\newcommand{\fEdgelist}{\foreignTerm{Edge list}\xspace}
\newcommand{\flist}{\foreignTerm{list}\xspace}
\newcommand{\fList}{\foreignTerm{List}\xspace}
\newcommand{\fgraphtraversal}{\foreignTerm{graph traversal}\xspace}
\newcommand{\fGraphtraversal}{\foreignTerm{Graph traversal}\xspace}
\newcommand{\ftree}{\foreignTerm{tree}\xspace}
\newcommand{\fTree}{\foreignTerm{Tree}\xspace}
\newcommand{\fsubtree}{\foreignTerm{subtree}\xspace}
\newcommand{\fSubtree}{\foreignTerm{Subtree}\xspace}

\newcommand{\fDivideAndConquer}{\foreignTerm{Divide and conquer}\xspace}
\newcommand{\fdivideAndConquer}{\foreignTerm{divide and conquer}\xspace}
\newcommand{\fMergeSort}{\foreignTerm{Merge sort}\xspace}
\newcommand{\fmergeSort}{\foreignTerm{merge sort}\xspace}
\newcommand{\fQuickSort}{\foreignTerm{Quicksort}\xspace}
\newcommand{\fquickSort}{\foreignTerm{quicksort}\xspace}
\newcommand{\fpivot}{\foreignTerm{pivot}\xspace}
\newcommand{\fPivot}{\foreignTerm{Pivot}\xspace}
\newcommand{\fbruteForce}{\foreignTerm{brute force}\xspace}
\newcommand{\fBruteForce}{\foreignTerm{Brute force}\xspace}
\newcommand{\fCompleteSearch}{\foreignTerm{complete search}\xspace}
\newcommand{\fExhaustiveSearch}{\foreignTerm{exhaustive search}\xspace}
\newcommand{\fBinarySearch}{\foreignTerm{binary search}\xspace}
\newcommand{\fGreedy}{\foreignTerm{greedy}\xspace}
\newcommand{\fGreedyChoice}{\foreignTerm{greedy choice}\xspace}

\newcommand{\pheap}{\foreignTerm{heap}\xspace}
\newcommand{\pHeap}{\foreignTerm{Heap}\xspace}
\newcommand{\pBinaryHeap}{\foreignTerm{Binary Heap}\xspace}
\newcommand{\pbinaryHeap}{\foreignTerm{binary heap}\xspace}
\newcommand{\pHeapSort}{\foreignTerm{Heap Sort}\xspace}
\newcommand{\pdjs}{\foreignTerm{disjoint set}\xspace}
\newcommand{\pDjs}{\foreignTerm{Disjoint set}\xspace}

\title{Matematika Diskret Dasar}
\author{Tim Olimpiade Komputer Indonesia}
\date{}

\begin{document}

\begin{frame}
\titlepage
\end{frame}

\begin{frame}
\frametitle{Pendahuluan}
Melalui dokumen ini, kalian akan:
\begin{itemize}
  \item Mempelajari \newTerm{aritmetika modular}.
  \item Mempelajari bilangan prima dan \newTerm{uji keprimaan}.
  \item Mempelajari algoritma \newTerm{prime generation}.
  \item Memahami FPB dan KPK.
  \item Mempelajari dan memanfaatkan \newTerm{pigeonhole principle}.
\end{itemize}
\end{frame}

\begin{frame}
\frametitle{Matematika Diskret}
\begin{itemize}
  \item Merupakan cabang matematika yang mempelajari tentang sifat bilangan bulat, logika, kombinatorika, dan graf.
\end{itemize}
\end{frame}

\section{Arimetika Modular}
\frame{\sectionpage}

\begin{frame}
\frametitle{Konsep Modulo}
\begin{itemize}
  \item Operasi $a \bmod m$ biasa disebut "$a$ modulo $m$".
  \item Operasi modulo ini akan memberikan sisa hasil bagi $a$ oleh $m$.
  \item Contoh: 
  \begin{itemize}
    \item $5 \bmod 3 = 2$ 
    \item $10 \bmod 2 = 0$
    \item $21 \bmod 6 = 3$
  \end{itemize}  
\end{itemize}
\end{frame}

\begin{frame}
\frametitle{Sifat-Sifat Modulo}
\begin{itemize}
  \item $(A + B) \bmod M = ((A \bmod M) + (B \bmod M)) \bmod M$
  \item $(A - B) \bmod M = ((A \bmod M) - (B \bmod M)) \bmod M$
  \item $(A \times B) \bmod M = ((A \bmod M) \times (B \bmod M)) \bmod M$  
  \item $(A^{B}) \bmod M = ((A \bmod M)^{B}) \bmod M$
\end{itemize}
\end{frame} 

\begin{frame}
\frametitle{Aplikasi Modulo}
\begin{itemize}
  \item Pada pemrograman kompetitif, tidak jarang kita harus menghitung $n! \bmod k$ (terutama dalam kombinatorika). 
  \item Seandainya kita menghitung $n!$, kemudian menggunakan operasi $\bmod$, kemungkinan besar kita akan mendapatkan \textit{integer overflow}.
  \item Untungnya, kita dapat memanfaatkan sifat modulo.
\end{itemize}
\end{frame}

\begin{frame}[fragile]
\frametitle{Aplikasi Modulo (lanj.)}
Solusi: 

\begin{codebox}
\Procname{$\proc{modularFactorial}(n, k)$}
\li $result \gets 1$
\li \For $i \gets 1$ \To $n$
    \Do
\li   $result \gets (result \times i) \bmod k$
    \End
\li \Return $result$    
\end{codebox}

\end{frame}

\begin{frame}
\frametitle{Hati-Hati!}
\begin{itemize}
  \item Aritmetika modular tidak serta-merta bekerja pada pembagian.
  \item $\frac{a}{b} \; \bmod\; n \; \neq \; \left( \frac{a \bmod n}{b \bmod n} \right) \; \bmod\; n$.
  \item Contoh: $\frac{12}{4} \; \bmod\; 6 \; \neq \; \left( \frac{12 \bmod 6}{4 \bmod 6} \right) \; \bmod\; 6$.
  \newline
  \item Pada aritmetika modular, $\frac{a}{b} \; \bmod\; n$ biasa ditulis sebagai:
  \newline
  \begin{center}
    $a \times b^{-1}\; \bmod\; n$
    \newline
  \end{center}
  dengan $b^{-1}$ adalah \newTerm{modular multiplicative inverse} dari $b$.
  \item Jika tertarik, Anda bisa mempelajari \foreignTerm{modular multiplicative inverse} melalui \textcolor{blue}{\href{https://en.wikipedia.org/wiki/  Modular_multiplicative_inverse}{tautan Wikipedia ini}.} 
\end{itemize}
\end{frame}

\section{Bilangan Prima}
\frame{\sectionpage}

\begin{frame}
\frametitle{Konsep Bilangan Prima}
\begin{block}{Bilangan Prima}
Bilangan bulat positif yang tepat memiliki dua faktor (pembagi), yaitu 1 dan dirinya sendiri.
\newline
Contoh: $2, 3, 5, 13, 97$.
\end{block}

\begin{block}{Bilangan Komposit}
Bilangan yang memiliki lebih dari dua faktor.
\newline
Contoh: $6, 14, 20, 25$.
\end{block}
\end{frame}

\begin{frame}
\frametitle{Uji Keprimaan}
\begin{itemize}
  \item \newTerm{Uji keprimaan} (\foreignTerm{primality testing}) adalah algoritma untuk mengecek apakah suatu bilangan bulat $N$ adalah bilangan prima.
  \item Kita dapat memanfaatkan sifat bilangan prima yang disebutkan pada slide sebelumnya untuk mengecek apakah suatu bilangan merupakan suatu bilangan prima. 
\end{itemize}
\end{frame}

\begin{frame}
\frametitle{Uji Keprimaan (lanj.)}
\begin{itemize}
  \item Solusi yang mudah untuk mengecek apakah $N$ prima atau tidak tentu dengan mengecek apakah ada bilangan selain 1 dan $N$ yang habis membagi $N$.
  \item Maka, kita dapat melakukan iterasi dari 2 hingga $N-1$ untuk mengetahui apakah ada bilangan selain 1 dan $N$ yang habis membagi $N$.
  \item Kompleksitas: $O(N)$.
\end{itemize}
\end{frame}

\begin{frame}
\frametitle{Uji Keprimaan (lanj.)}

\begin{codebox}
\Procname{$\proc{isPrimeNaive}(n)$}
\li \If $n \le 1$
    \Then
\li   \Return $false$
    \End
\zi
\li \For $i \gets 2$ \To $n-1$
    \Do
\li   \If $n \bmod i \isequal 0$
      \Then
\li     \Return $false$
      \End    
    \End
\zi
\li \Return $true$
\end{codebox}
\end{frame}

\begin{frame}
\frametitle{Uji Keprimaan (lanj.)}
\begin{itemize}
  \item Ada solusi yang lebih cepat dari $O(N)$.
  \item Manfaatkan observasi bahwa jika $N = a \times b$, dan $a \leq b$, maka $a \leq \sqrt{N}$ dan $b \geq \sqrt{N}$.
  \item Kita tidak perlu memeriksa $b$; seandainya $N$ habis dibagi $b$, tentu $N$ habis dibagi $a$.
  \item Jadi kita hanya perlu memeriksa hingga $\sqrt{N}$.
  \item Kompleksitas: $O(\sqrt{N})$
\end{itemize}
\end{frame}

\begin{frame}
\frametitle{Uji Keprimaan (lanj.)}
\begin{codebox}
\Procname{$\proc{isPrimeSqrt}(n)$}
\li \If $n \le 1$
    \Then
\li   \Return $false$
    \End
\zi
\li $i \gets 2$
\li \While $i \times i \leq n$
    \Do
\li   \If $n \bmod i \isequal 0$
      \Then
\li     \Return $false$
      \End
\li   $i \gets i + 1$          
    \End
\zi
\li \Return $true$
\end{codebox}
\end{frame}

\section{Prime Generation \newline (Pembangkitan Bilangan Prima)}
\frame{\sectionpage}

\begin{frame}
\frametitle{Solusi Naif}
\begin{itemize}
  \item Misalkan kita ingin mengetahui himpunan bilangan prima yang tidak lebih dari $N$.
  \item Kita dapat membangkitkan bilangan prima dengan iterasi dan uji keprimaan.
  \item Solusi: 
  \begin{enumerate}
    \item Lakukan iterasi dari 2 sampai N
    \item Untuk tiap bilangan, cek apakah dia bilangan prima. Jika ya, kita bisa memasukkannya ke daftar bilangan prima.
  \end{enumerate}
\end{itemize}
\end{frame}

\begin{frame}
\frametitle{Solusi Naif (lanj.)}
\begin{codebox}
\Procname{$\proc{simplePrimeGeneration}(N)$}
\li $primeList \gets \{\}$
\li \For $i \gets 2$ \To $N$
    \Do
\li   \If $\proc{isPrimeSqrt}(i)$
      \Then
\li     $primeList \gets primeList \cup \{i\}$
      \End    
    \End
\li \Return $primeList$
\end{codebox}
\end{frame}

\begin{frame}
\frametitle{Sieve of Erathostenes}
\begin{itemize}
  \item Terdapat solusi yang lebih cepat untuk membangkitkan bilangan prima, yaitu \newTerm{Sieve of Erathostenes}.
  \item Ide utama utama dari algoritma ini adalah mengeliminasi bilangan-bilangan dari calon bilangan prima.
  \item Yang akan kita eliminasi adalah bilangan komposit.
\end{itemize}
\end{frame}

\begin{frame}
\frametitle{Sieve of Erathostenes (lanj.)}
\begin{itemize}
  \item Kita tahu bahwa suatu bilangan komposit $c$ dapat dinyatakan sebagai $c = p \times q$, dengan $p$ suatu bilangan prima.
  \item Seandainya kita mengetahui suatu bilangan prima, kita dapat mengeliminasi kelipatan-kelipatan bilangan tersebut dari calon bilangan prima.
  \item Contoh: jika diketahui 7 adalah bilangan prima, maka 14, 21, 28, ... dieliminasi dari calon bilangan prima.
\end{itemize}
\end{frame}

\begin{frame}
\frametitle{Prosedur Sieve of Erathostenes}
\begin{enumerate}
  \item Awalnya seluruh bilangan bulat dari 2 hingga N belum dieliminasi.
  \item Lakukan iterasi dari 2 hingga N:
  \begin{enumerate}
    \item Jika bilangan ini belum dieliminasi, artinya bilangan ini merupakan bilangan prima.
    \item Lakukan iterasi untuk mengeliminasi kelipatan bilangan tersebut.  
  \end{enumerate}
\end{enumerate}
\end{frame}

\begin{frame}
\frametitle{Implementasi Sieve of Erathostenes}
\begin{itemize}
  \item Kita dapat menggunakan \textit{array} \textit{boolean} untuk menyimpan informasi apakah suatu bilangan telah tereliminasi.
  \item Jika kita ingin mencari bilangan prima yang $\leq N$, maka diperlukan memori sebesar $O(N)$.
  \item Melalui perhitungan matematis, kompleksitas waktu solusi ini adalah $O(N\;log\;log\;N)$.
\end{itemize}
\end{frame}

\begin{frame}[fragile]
\frametitle{Implementasi Sieve of Erathostenes (lanj.)}
\begin{codebox}
\Procname{$\proc{sieveOfErathostenes}(N)$}
\li \Comment Siapkan \textit{array boolean} $eleminated$ berukuran N
\li \Comment Inisialisasi \textit{array} $eleminated$ dengan $false$
\li $primeList \gets \{\}$
\li \For $i \gets 2$ \To $n$
    \Do
\li   \If $not$ $eleminated[i]$
      \Then
\li     $primeList \gets primeList \cup \{i\}$
\li     $j \gets i \times i$ 

\li     \While $j \leq n$
        \Do
\li       $eleminated[j] \gets true$
\li       $j \gets j + i$       
        \End
      \End      
    \End
\li \Return $primeList$
\end{codebox}
\end{frame}

\begin{frame}
\frametitle{Implementasi Sieve of Erathostenes (lanj.)}
\begin{itemize}
  \item Perhatikan baris ke-7 yang berisi $j = i \times i$.
  \item Di sini, $j$ menyatakan kelipatan $i$ yang akan dieliminasi.
  \item Perhatikan bahwa $j$ dimulai dari $i \times i$, bukan $2i$.
  \item Alasannya adalah $2i, 3i, 4i, ..., (i-1)i$ pasti sudah tereliminasi pada iterasi-iterasi sebelumnya.
\end{itemize}
\end{frame}

\section{FPB dan KPK}
\frame{\sectionpage}

\begin{frame}
\frametitle{Faktorisasi Prima}
\begin{itemize}
  \item Ketika masih SD, kita pernah belajar memfaktorkan bilangan dengan pohon faktor.
  \item Melalui faktorisasi prima, kita dapat menyatakan suatu bilangan sebagai hasil perkalian faktor primanya.
  \item Contoh: $7875 = 3^{2} \times 5^{3} \times 7$ .
\end{itemize}
\end{frame}

\begin{frame}
\frametitle{FPB dan KPK}
\begin{itemize}
  \item FPB dan KPK dapat dicari melalui faktorisasi prima. 
  \item Untuk setiap bilangan prima, kita menggunakan pangkat terkecil untuk FPB dan pangkat terbesar untuk KPK.
  \item Contoh:
  \begin{itemize}
    \item $4725 = 3^{3} \times 5^{2} \times 7$
    \item $7875 = 3^{2} \times 5^{3} \times 7$
  \end{itemize}
  \item Maka: 
  \begin{itemize}
    \item $FPB(4725,7875) = 3^{2} \times 5^{2} \times 7 = 1525$
    \item $KPK(4725,7875) = 3^{3} \times 5^{3} \times 7 = 23625$
    \newline
  \end{itemize}
  \item Terdapat pula sifat $KPK(a,b) = \frac{a \times b}{FPB(a,b)}$. 
\end{itemize}
\end{frame}

\begin{frame}
\frametitle{Algoritma Euclid}
\begin{itemize}
  \item Untuk mencari FPB suatu bilangan, menggunakan pohon faktor cukup merepotkan.
  \item Kita perlu mencari faktor prima bilangan tersebut, dan jika faktor primanya besar, tentu akan menghabiskan banyak waktu.
  \item Terdapat algoritma yang dapat mencari $FPB(a,b)$ dalam $O(\log{(\min{(a,b)})})$.
  \item Algoritma ini bernama \newTerm{Algoritma Euclid}.
\end{itemize}
\end{frame}

\begin{frame}[fragile]
\frametitle{Algoritma Euclid (lanj.)}
\begin{codebox}
\Procname{$\proc{euclid}(a, b)$}
\li \If $b \isequal 0$
    \Then
\li   \Return $a$
\li \Else
\li   \Return $\proc{euclid}(b, a \bmod b)$ 
    \End
\end{codebox}
\end{frame}

\begin{frame}
\frametitle{Algoritma Euclid (lanj.)}
\begin{itemize}
  \item Sangat pendek!
  \item Anda dapat menghemat waktu pengetikan kode dalam melakukan pencarian FPB dengan algoritma ini.
  \item Jika Anda tertarik dengan pembuktiannya, baca lebih lanjut \textcolor{blue}{\href{https://en.wikipedia.org/wiki/Euclidean_algorithm}{di tautan Wikipedia ini}.}
\end{itemize}   
\end{frame}

\section{Pigeonhole Principle (PHP)}
\frame{\sectionpage}

\begin{frame}
\frametitle{Pigeonhole Principle}
\begin{itemize}
  \item Konsep PHP adalah \statement{Jika ada $N$ ekor burung dan $M$ sangkar, yang memenuhi $N > M$ , maka pasti ada sangkar yang berisi setidaknya 2 ekor burung}.
  \item Secara matematis, jika ada $N$ ekor burung dan $M$ sangkar, maka pasti ada sangkar yang berisi setidaknya $\big\lceil \frac{N}{M} \big\rceil$ ekor burung.
\end{itemize}
\end{frame}

\begin{frame}
\frametitle{Pigeonhole Principle (lanj.)}
\begin{itemize}
  \item Terkesan sederhana?
  \item Simak contoh aplikasi prinsip ini.
\end{itemize}
\end{frame}

\begin{frame}
\frametitle{Contoh Soal PHP}
\begin{itemize}
  \item Pak Dengklek memiliki sebuah \textit{array} $A$ berisi $N$ bilangan bulat non-negatif.
  \item Anda ditantang untuk memilih angka-angka dari \textit{array}-nya yang jika dijumlahkan habis dibagi $N$.
  \item Angka di suatu indeks \textit{array} tidak boleh dipilih lebih dari sekali.
  \item Apabila mungkin, cetak indeks angka-angka yang Anda ambil.
  \item Apabila tidak mungkin, cetak "\texttt{tidak mungkin}".
  \item Batasan
  \begin{itemize}
    \item $1 \leq N \leq 10^{5}$
    \item \textit{Array} $A$ hanya berisi bilangan bulat non-negatif.
  \end{itemize}
\end{itemize}
\end{frame}

\begin{frame}
\frametitle{Analisis Contoh Soal PHP}
\begin{itemize}
  \item Inti soal ini adalah mencari apakah pada \textit{array} berukuran $N$, terdapat subhimpunan tidak kosong yang jumlahan elemen-elemennya habis dibagi $N$.
\end{itemize}
\end{frame}

\begin{frame}
\frametitle{Analisis Contoh Soal PHP (lanj.)}
\begin{itemize}
  \item Mari kita coba mengerjakan versi lebih mudah dari soal ini:
  Bagaimana jika yang diminta subbarisan, bukan subhimpunan?
  \item Anggap \textit{array} $A$ dimulai dari indeks 1 (\foreignTerm{one-based}).
  \item Misalkan kita memiliki fungsi $sum(k) = \sum\limits_{i=1}^k A[i]$.
  \item Untuk $sum(0)$, sesuai definisi nilainya adalah $0$.
\end{itemize}
\end{frame}

\begin{frame}
\frametitle{Analisis Contoh Soal PHP (lanj.)}
\begin{itemize}
  \item Perhatikan bahwa $\sum\limits_{i=l}^r A[i] = sum(r) - sum(l - 1)$.
  \newline
  \item Jika subbarisan $A[l..r]$ habis dibagi $N$, maka $(sum(r) - sum(l - 1)) \bmod N = 0$.
  \newline
  \item Ini dapat kita tuliskan sebagai $sum(r) \bmod N = sum(l - 1) \bmod N$.
\end{itemize}
\end{frame}

\begin{frame}
\frametitle{Analisis Contoh Soal PHP (lanj.)}
\begin{itemize}
  \item Observasi 1: \newline
  Ada $N$ kemungkinan nilai $(sum(x) \bmod N)$, yaitu $[0..N-1]$.
  \item Observasi 2: \newline
  Ada $N+1$ nilai x untuk $(sum(x) \bmod N)$, yaitu untuk $x \in [0..N]$. \newline \newline
  Ingat bahwa $sum(0)$ ada agar jumlahan subbarisan $A[1..k]$ untuk tiap $k$ dapat kita nyatakan dalam bentuk: $(sum(k) - sum(0))$.
\end{itemize}
\end{frame}

\begin{frame}
\frametitle{Analisis Contoh Soal PHP (lanj.)}
Observasi 3:
\begin{itemize}
  \item Ada $N+1$ kemungkinan nilai $x$
  \item Ada $N$ kemungkinan nilai $sum(x) \bmod N$
  \item \emp{Pasti ada} $a$ dan $b$, sehingga \newline $sum(b) \bmod N = sum(a) \bmod N$
  \item Subbarisan yang menjadi solusi adalah $A[a+1..b]$.
\end{itemize}
\end{frame}

\begin{frame}
\frametitle{Analisis Contoh Soal PHP (lanj.)}
\begin{itemize}
  \item Dengan menyelesaikan versi mudah dari soal awal kita, ternyata kita justru dapat menyelesaikan soal tersebut.
  \item Suatu subbarisan dari $A$ pasti juga merupakan subhimpunan dari $A$.
  \item Ternyata, \emp{selalu} ada cara untuk menjawab pertanyaan Pak Dengklek.
  \item Yakni, keluaran "\texttt{tidak mungkin}" tidak akan pernah terjadi.
\end{itemize}
\end{frame}

\begin{frame}
\frametitle{Implementasi}
\begin{codebox}
\Procname{$\proc{findDivisibleSubsequence}(A, N)$}
\li \Comment Inisialisasi array $sum[0..N]$ dengan $0$
\li \Comment Isikan nilai $sum[i]$ dengan $(A[1] + A[2] + ... + A[i])$
\li \Comment Inisialisasi array $seenInIndex[0..N-1]$ dengan $-1$
\zi 
\li \For $i \gets 0$ \To $N$
    \Do
\li   \If $seenInIndex[sum[i] \bmod N] \isequal -1$
      \Then
\li     $seenInIndex[sum[i] \bmod N] \gets i$    
\li   \Else
\li     $a \gets seenInIndex[sum[i] \bmod N]$
\li     $b \gets i$
\li     \Return $[a+1, a+2, ..., b]$
      \End
    \End
\end{codebox}
\end{frame}

\begin{frame}
\frametitle{Penutup}
\begin{itemize}
  \item Matematika diskret merupakan topik yang sangat luas.
  \item Topik ini berisikan banyak konsep dasar yang umum digunakan pada pemrograman kompetitif.
\end{itemize}
\end{frame}

\end{document}
