\documentclass{beamer}
\usetheme{tokitex}

\usepackage{tikz}
\usepackage{graphics}
\usepackage{multirow}
\usepackage{tabto}
\usepackage{xspace}
\usepackage{amsmath}
\usepackage{hyperref}
\usepackage{wrapfig}
\usepackage{mathtools}

\usepackage{tikz}
\usepackage{clrscode3e}
\usepackage{gensymb}

\usepackage[english,bahasa]{babel}
\newtranslation[to=bahasa]{Section}{Bagian}
\newtranslation[to=bahasa]{Subsection}{Subbagian}

\usepackage{listings, lstautogobble}
\usepackage{color}

\definecolor{dkgreen}{rgb}{0,0.6,0}
\definecolor{gray}{rgb}{0.5,0.5,0.5}
\definecolor{mauve}{rgb}{0.58,0,0.82}

\lstset{frame=tb,
  language=c++,
  aboveskip=0mm,
  belowskip=0mm,
  showstringspaces=false,
  columns=fullflexible,
  keepspaces=true,
  basicstyle={\small\ttfamily},
  numbers=none,
  numberstyle=\tiny\color{gray},
  keywordstyle=\color{blue},
  commentstyle=\color{dkgreen},
  stringstyle=\color{mauve},
  breaklines=true,
  breakatwhitespace=true,
  lineskip={-3pt}
}

\usepackage{caption}
\captionsetup[figure]{labelformat=empty}

\newcommand{\progTerm}[1]{\textbf{#1}}
\newcommand{\foreignTerm}[1]{\textit{#1}}
\newcommand{\newTerm}[1]{\alert{\textbf{#1}}}
\newcommand{\emp}[1]{\alert{#1}}
\newcommand{\statement}[1]{"#1"}

\newcommand{\floor}[1]{\lfloor #1 \rfloor}
\newcommand{\ceil}[1]{\lceil #1 \rceil}
\newcommand{\abs}[1]{\left\lvert#1\right\rvert}
\newcommand{\norm}[1]{\left\lVert#1\right\rVert}

% Getting tired of writing \foreignTerm all the time
\newcommand{\farray}{\foreignTerm{array}\xspace}
\newcommand{\fArray}{\foreignTerm{Array}\xspace}
\newcommand{\foverhead}{\foreignTerm{overhead}\xspace}
\newcommand{\fOverhead}{\foreignTerm{Overhead}\xspace}
\newcommand{\fsubarray}{\foreignTerm{subarray}\xspace}
\newcommand{\fSubarray}{\foreignTerm{Subarray}\xspace}
\newcommand{\fbasecase}{\foreignTerm{base case}\xspace}
\newcommand{\fBasecase}{\foreignTerm{Base case}\xspace}
\newcommand{\ftopdown}{\foreignTerm{top-down}\xspace}
\newcommand{\fTopdown}{\foreignTerm{Top-down}\xspace}
\newcommand{\fbottomup}{\foreignTerm{bottom-up}\xspace}
\newcommand{\fBottomup}{\foreignTerm{Bottom-up}\xspace}
\newcommand{\fpruning}{\foreignTerm{pruning}\xspace}
\newcommand{\fPruning}{\foreignTerm{Pruning}\xspace}

\newcommand{\fgraph}{\foreignTerm{graph}\xspace}
\newcommand{\fGraph}{\foreignTerm{Graph}\xspace}
\newcommand{\froot}{\foreignTerm{root}\xspace}
\newcommand{\fRoot}{\foreignTerm{Root}\xspace}
\newcommand{\fnode}{\foreignTerm{node}\xspace}
\newcommand{\fNode}{\foreignTerm{Node}\xspace}
\newcommand{\fedge}{\foreignTerm{edge}\xspace}
\newcommand{\fEdge}{\foreignTerm{Edge}\xspace}
\newcommand{\fcycle}{\foreignTerm{cycle}\xspace}
\newcommand{\fCycle}{\foreignTerm{Cycle}\xspace}
\newcommand{\fdegree}{\foreignTerm{degree}\xspace}
\newcommand{\fDegree}{\foreignTerm{Degree}\xspace}
\newcommand{\fadjacencylist}{\foreignTerm{adjacency list}\xspace}
\newcommand{\fAdjacencylist}{\foreignTerm{Adjacency list}\xspace}
\newcommand{\fadjacencymatrix}{\foreignTerm{adjacency matrix}\xspace}
\newcommand{\fAdjacencymatrix}{\foreignTerm{Adjacency matrix}\xspace}
\newcommand{\fedgelist}{\foreignTerm{edge list}\xspace}
\newcommand{\fEdgelist}{\foreignTerm{Edge list}\xspace}
\newcommand{\flist}{\foreignTerm{list}\xspace}
\newcommand{\fList}{\foreignTerm{List}\xspace}
\newcommand{\fgraphtraversal}{\foreignTerm{graph traversal}\xspace}
\newcommand{\fGraphtraversal}{\foreignTerm{Graph traversal}\xspace}
\newcommand{\ftree}{\foreignTerm{tree}\xspace}
\newcommand{\fTree}{\foreignTerm{Tree}\xspace}
\newcommand{\fsubtree}{\foreignTerm{subtree}\xspace}
\newcommand{\fSubtree}{\foreignTerm{Subtree}\xspace}
\newcommand{\fparent}{\foreignTerm{parent}\xspace}
\newcommand{\fParent}{\foreignTerm{Parent}\xspace}
\newcommand{\fsibling}{\foreignTerm{sibling}\xspace}
\newcommand{\fSibling}{\foreignTerm{Sibling}\xspace}
\newcommand{\fpath}{\foreignTerm{path}\xspace}
\newcommand{\fPath}{\foreignTerm{Path}\xspace}
\newcommand{\fconnectedcomponent}{\foreignTerm{connected component}\xspace}
\newcommand{\fConnectedcomponent}{\foreignTerm{Connected component}\xspace}
\newcommand{\fbridge}{\foreignTerm{bridge}\xspace}
\newcommand{\fBridge}{\foreignTerm{Bridge}\xspace}
\newcommand{\farticulationpoint}{\foreignTerm{articulation point}\xspace}
\newcommand{\fArticulationpoint}{\foreignTerm{Articulation point}\xspace}
\newcommand{\ftreeedge}{\foreignTerm{tree edge}\xspace}
\newcommand{\fTreeedge}{\foreignTerm{Tree edge}\xspace}
\newcommand{\fbackedge}{\foreignTerm{back edge}\xspace}
\newcommand{\fBackedge}{\foreignTerm{Back edge}\xspace}
\newcommand{\fforwardedge}{\foreignTerm{forward edge}\xspace}
\newcommand{\fForwardedge}{\foreignTerm{Forward edge}\xspace}
\newcommand{\fcrossedge}{\foreignTerm{cross edge}\xspace}
\newcommand{\fCrossedge}{\foreignTerm{Cross edge}\xspace}
\newcommand{\fdiscoverytime}{\foreignTerm{discovery time}\xspace}
\newcommand{\fDiscoverytime}{\foreignTerm{Discovery time}\xspace}
\newcommand{\flowlink}{\foreignTerm{low link}\xspace}
\newcommand{\fLowlink}{\foreignTerm{Low link}\xspace}
\newcommand{\fstack}{\foreignTerm{stack}\xspace}
\newcommand{\fStack}{\foreignTerm{Stack}\xspace}
\newcommand{\for}{\foreignTerm{or}\xspace}
\newcommand{\fOr}{\foreignTerm{Or}\xspace}
\newcommand{\fand}{\foreignTerm{and}\xspace}
\newcommand{\fAnd}{\foreignTerm{And}\xspace}
\newcommand{\fcentroid}{\foreignTerm{centroid}\xspace}
\newcommand{\fCentroid}{\foreignTerm{Centroid}\xspace}

\newcommand{\fDivideAndConquer}{\foreignTerm{Divide and conquer}\xspace}
\newcommand{\fdivideAndConquer}{\foreignTerm{divide and conquer}\xspace}
\newcommand{\fMergeSort}{\foreignTerm{Merge sort}\xspace}
\newcommand{\fmergeSort}{\foreignTerm{merge sort}\xspace}
\newcommand{\fQuickSort}{\foreignTerm{Quicksort}\xspace}
\newcommand{\fquickSort}{\foreignTerm{quicksort}\xspace}
\newcommand{\fpivot}{\foreignTerm{pivot}\xspace}
\newcommand{\fPivot}{\foreignTerm{Pivot}\xspace}
\newcommand{\fbruteForce}{\foreignTerm{brute force}\xspace}
\newcommand{\fBruteForce}{\foreignTerm{Brute force}\xspace}
\newcommand{\fCompleteSearch}{\foreignTerm{complete search}\xspace}
\newcommand{\fExhaustiveSearch}{\foreignTerm{exhaustive search}\xspace}
\newcommand{\fbinarySearch}{\foreignTerm{binary search}\xspace}
\newcommand{\fBinarySearch}{\foreignTerm{Binary search}\xspace}
\newcommand{\fternarySearch}{\foreignTerm{ternary search}\xspace}
\newcommand{\fTernarySearch}{\foreignTerm{Ternary search}\xspace}
\newcommand{\funimodal}{\foreignTerm{unimodal}\xspace}
\newcommand{\fUnimodal}{\foreignTerm{Unimodal}\xspace}
\newcommand{\fGreedy}{\foreignTerm{Greedy}\xspace}
\newcommand{\fgreedy}{\foreignTerm{greedy}\xspace}
\newcommand{\fgreedyChoice}{\foreignTerm{greedy choice}\xspace}
\newcommand{\fGreedyChoice}{\foreignTerm{Greedy choice}\xspace}

\newcommand{\fdp}{\foreignTerm{dynamic programming}\xspace}
\newcommand{\fDp}{\foreignTerm{Dynamic programming}\xspace}
\newcommand{\fbitmask}{\foreignTerm{bitmask}\xspace}
\newcommand{\fBitmask}{\foreignTerm{Bitmask}\xspace}
\newcommand{\fstate}{\foreignTerm{state}\xspace}
\newcommand{\fState}{\foreignTerm{State}\xspace}
\newcommand{\fsubmask}{\foreignTerm{submask}\xspace}
\newcommand{\fSubmask}{\foreignTerm{Submask}\xspace}

\newcommand{\pheap}{\foreignTerm{heap}\xspace}
\newcommand{\pHeap}{\foreignTerm{Heap}\xspace}
\newcommand{\pBinaryHeap}{\foreignTerm{Binary Heap}\xspace}
\newcommand{\pbinaryHeap}{\foreignTerm{binary heap}\xspace}
\newcommand{\pHeapsort}{\foreignTerm{Heapsort}\xspace}
\newcommand{\pheapsort}{\foreignTerm{heapsort}\xspace}
\newcommand{\pdjs}{\foreignTerm{disjoint set}\xspace}
\newcommand{\pDjs}{\foreignTerm{Disjoint set}\xspace}

\newcommand{\fdotProduct}{\foreignTerm{dot product}\xspace}
\newcommand{\fDotProduct}{\foreignTerm{Dot product}\xspace}
\newcommand{\fcrossProduct}{\foreignTerm{cross product}\xspace}
\newcommand{\fCrossProduct}{\foreignTerm{Cross product}\xspace}
\newcommand{\fconvexHull}{\foreignTerm{convex hull}\xspace}
\newcommand{\fConvexHull}{\foreignTerm{Convex hull}\xspace}
\newcommand{\fgrahamScan}{\foreignTerm{graham scan}\xspace}
\newcommand{\fGrahamScan}{\foreignTerm{Graham scan}\xspace}
\newcommand{\flineSweep}{\foreignTerm{line sweep}\xspace}
\newcommand{\fLineSweep}{\foreignTerm{Line sweep}\xspace}

\newcommand{\fset}{\foreignTerm{set}\xspace}
\newcommand{\fSet}{\foreignTerm{Set}\xspace}
\newcommand{\fprefixSum}{\foreignTerm{prefix sum}\xspace}
\newcommand{\fPrefixSum}{\foreignTerm{Prefix sum}\xspace}
\newcommand{\ffenwickTree}{\foreignTerm{fenwick tree}\xspace}
\newcommand{\fFenwickTree}{\foreignTerm{Fenwick tree}\xspace}
\newcommand{\frangeSumQuery}{\foreignTerm{range sum query}\xspace}
\newcommand{\fRangeSumQuery}{\foreignTerm{Range sum query}\xspace}
\newcommand{\fquery}{\foreignTerm{query}\xspace}
\newcommand{\fQuery}{\foreignTerm{Query}\xspace}
\newcommand{\fsegmentTree}{\foreignTerm{segment tree}\xspace}
\newcommand{\fSegmentTree}{\foreignTerm{Segment tree}\xspace}
\newcommand{\fbinaryTree}{\foreignTerm{binary tree}\xspace}
\newcommand{\fBinaryTree}{\foreignTerm{Binary tree}\xspace}
\newcommand{\flazyPropagation}{\foreignTerm{lazy propagation}\xspace}
\newcommand{\fLazyPropagation}{\foreignTerm{Lazy propagation}\xspace}
\newcommand{\fsparseTable}{\foreignTerm{sparse table}\xspace}
\newcommand{\fSparseTable}{\foreignTerm{Sparse table}\xspace}

\newcommand{\ftrail}{\foreignTerm{trail}\xspace}
\newcommand{\fTrail}{\foreignTerm{Trail}\xspace}
\newcommand{\feulerTour}{\foreignTerm{euler tour}\xspace}
\newcommand{\fEulerTour}{\foreignTerm{Euler tour}\xspace}
\newcommand{\feulerTourTree}{\foreignTerm{euler tour tree}\xspace}
\newcommand{\fEulerTourTree}{\foreignTerm{Euler tour tree}\xspace}

\newcommand{\fmaxflow}{\foreignTerm{maximum flow}\xspace}
\newcommand{\fMaxflow}{\foreignTerm{Maximum flow}\xspace}
\newcommand{\fmincut}{\foreignTerm{minimum cut}\xspace}
\newcommand{\fMincut}{\foreignTerm{Minimum cut}\xspace}
\newcommand{\fflow}{\foreignTerm{flow}\xspace}
\newcommand{\fFlow}{\foreignTerm{Flow}\xspace}
\newcommand{\fsource}{\foreignTerm{source}\xspace}
\newcommand{\fSource}{\foreignTerm{Source}\xspace}
\newcommand{\fsink}{\foreignTerm{sink}\xspace}
\newcommand{\fSink}{\foreignTerm{Sink}\xspace}
\newcommand{\fbackEdge}{\foreignTerm{back-edge}\xspace}
\newcommand{\fBackEdge}{\foreignTerm{Back-edge}\xspace}
\newcommand{\fresidualCapacity}{\foreignTerm{residual capacity}\xspace}
\newcommand{\fResidualCapacity}{\foreignTerm{Residual capacity}\xspace}
\newcommand{\fbottleneck}{\foreignTerm{bottleneck}\xspace}
\newcommand{\fBottleneck}{\foreignTerm{Bottleneck}\xspace}
\newcommand{\faugmentingPath}{\foreignTerm{augmenting path}\xspace}
\newcommand{\fAugmentingPath}{\foreignTerm{Augmenting path}\xspace}


\title{Matematika Diskret Dasar}
\author{Tim Olimpiade Komputer Indonesia}
\date{}

\begin{document}

\begin{frame}
\titlepage
\end{frame}

\begin{frame}
\frametitle{Pendahuluan}
Melalui dokumen ini, kalian akan:
\begin{itemize}
  \item Mempelajari \newTerm{aritmetika modular}.
  \item Mempelajari bilangan prima.
  \item Memahami FPB dan KPK.
  \item Mempelajari algoritma \newTerm{prime generation}.
  \item Mempelajari dan memanfaatkan \newTerm{pigeon hole principle}.
\end{itemize}
\end{frame}

\section{Arimetika Modular}
\frame{\sectionpage}

\begin{frame}
\frametitle{Konsep Modulo}
\begin{itemize}
  \item Operasi $A \mod M$ biasa disebut "A modulo M".
  \item Operasi modulo ini akan memberikan sisa hasil bagi A oleh M.
  \item Contoh: 
  \begin{itemize}
    \item $5 \mod 3 = 2$ 
    \item $10 \mod 2 = 0$
    \item $21 \mod 6 = 3$
  \end{itemize}  
\end{itemize}
\end{frame}

\begin{frame}
\frametitle{Sifat-Sifat Modulo}
\begin{itemize}
  \item $(A + B) \mod M = ((A \mod M) + (B \mod M)) \mod M$
  \item $(A - B) \mod M = ((A \mod M) - (B \mod M)) \mod M$
  \item $(AB) \mod M = ((A \mod M) \times (B \mod M)) \mod M$  
  \item $(A^{B}) \mod M = ((A \mod M)^{B}) \mod M$
  \item $(A \times B) \mod (A \times M) = A \times (B \mod M)$
\end{itemize}
\end{frame} 

\begin{frame}
\frametitle{Aplikasi Modulo}
\begin{itemize}
  \item Pada \foreignTerm{Competitive Programming}, tidak jarang kita harus menghitung $n! \mod k$ (terutama dalam kombinatorika). 
  \item Seandainya kita menghitung $n!$ , kemudian menggunakan operasi $\mod$, kemungkinan besar kita akan mendapatkan \emp{overflow!}
  \item Untungnya, kita bisa menggunakan sifat modulo.
\end{itemize}
\end{frame}

\begin{frame}[fragile]
\frametitle{Aplikasi Modulo (lanj.)}
Solusi: 
\begin{lstlisting}
  function modularFactorial(n, k: longint): longint;
  var
    i, result: longint;
  begin
    result := 1;
    for i := 1 to n do begin
      result := (result * i) mod k;
    end
    modularFactorial := result;
  end;  
\end{lstlisting}
\end{frame}

\begin{frame}
\frametitle{Hati-Hati!}
\begin{itemize}
  \item Aritmetika modular tidak langsung bekerja pada pembagian.
  \item $\frac{a}{b} \; (\bmod\; n) \; \neq \; \frac{a \mod n}{b \mod n} \; (\bmod\; n)$.
  \item Contohnya, $\frac{12}{4} \; (\bmod\; 6) \; \neq \; \frac{12 \mod 6}{4 \mod 6} \; (\bmod\; 6)$.
  \item Pada aritmetika modular, $\frac{a}{b} \; (\bmod\; n)$ biasa kita tulis sebagai $a \times b^{-1}\;(\bmod\; n)$, dimana $b^{-1}$ adalah 
      \newTerm{Modular Multiplicative Inverse} dari $b$.
  \item Jika tertarik, Anda bisa mempelajari \foreignTerm{Modular Multiplicative Inverse} melalui \textcolor{blue}{\href{https://en.wikipedia.org/wiki/  Modular_multiplicative_inverse}{link wikipedia ini}.} 
\end{itemize}
\end{frame}

\section{Bilangan Prima}
\frame{\sectionpage}

\begin{frame}
\frametitle{Konsep Bilangan Prima}
\begin{itemize}
  \item Bilangan prima adalah bilangan bulat positif yang hanya habis dibagi oleh 1 dan dirinya sendiri.
  \item Contoh: $2, 3, 5, 13, 97$.
  \item Bilangan komposit adalah bilangan yang memiliki faktor lebih dari 2 buah.
  \item Contoh: $6, 14, 20, 25$.
\end{itemize}
\end{frame}

\begin{frame}
\frametitle{Primality Testing}
\begin{itemize}
  \item \newTerm{Primality Testing} adalah algoritma untuk mengecek apakah suatu bilangan bulat $N$ adalah bilangan prima.
  \item Kita dapat memanfaatkan sifat bilangan prima yang disebut di slide sebelumnya untuk mengecek apakah suatu bilangan merupakan suatu bilangan prima. 
\end{itemize}
\end{frame}

\begin{frame}
\frametitle{Primality Testing (lanj.)}
\begin{itemize}
  \item Solusi yang mudah untuk mengecek apakah $N$ prima atau tidak tentu dengan mengecek apakah ada bilangan selain 1 dan $N$ yang habis membagi $N$.
  \item Maka, kita dapat melakukan iterasi dari 2 sampai $N-1$ untuk mengetahui apakah ada bilangan selain 1 dan $N$ yang habis membagi $N$.
  \item Kompleksitas: $O(N)$.
\end{itemize}
\end{frame}

\begin{frame}[fragile]
\frametitle{Primality Testing (lanj.)}
\begin{lstlisting}
  function isPrimeNaive(n: longint): boolean;
  var
    i: longint;
    prime: boolean;
  begin
    prime := true;
    for i := 2 to n - 1 do begin
      if (n mod i = 0) then begin
        prime := false;
      end;
    end;
    if(n < 2) then begin
      prime := false;
    end;
    isPrimeNaive := prime;
  end;  
\end{lstlisting}
\end{frame}

\begin{frame}
\frametitle{Primality Testing (lanj.)}
\begin{itemize}
  \item Sebenarnya, ada solusi yang lebih cepat dari $O(N)$.
  \item Kita dapat memanfaatkan observasi bahwa jika $N = A \times B$, dan $A \leq B$, maka $A \leq \sqrt{N}$ dan $B \geq \sqrt{N}$.
  \item Kita tidak perlu mengecek $B$, seandainya $N$ habis dibagi $B$, tentu $N$ habis dibagi $A$.
  \item Maka, kita hanya perlu mengecek hingga $\sqrt{N}$.
  \item Kompleksitas : $O(\sqrt{N})$
\end{itemize}
\end{frame}

\begin{frame}[fragile]
\frametitle{Primality Testing (lanj.)}
\begin{lstlisting}
  function isPrimeSqrt(n: longint): boolean;
  var
    i: longint;
    prime: boolean;
  begin
    prime := true;
    i := 2;
    while (i * i <= n) do begin
      if (n mod i = 0) begin
        prime := false;
      end;
      inc(i);
    end;
    if(n < 2) then begin
      prime := false;
    end;
    isPrimeSqrt := prime;
  end;  
\end{lstlisting}
\end{frame}

\begin{frame}
\frametitle{Primality Testing (lanj.)}
\begin{itemize}
  \item \foreignTerm{Primality Testing}, walaupun topiknya mirip dengan faktorisasi, sesungguhnya jauh lebih mudah dibanding faktorisasi.
  \item Jika tertarik, Anda dapat membacanya lebih lanjut di \textcolor{blue}{\href{https://en.wikipedia.org/wiki/Primality_test}{link wikipedia ini}.} 
\end{itemize}
\end{frame}

\section{Generating Prime}
\frame{\sectionpage}

\begin{frame}
\frametitle{Solusi Awal}
\begin{itemize}
  \item Kita dapat membangkitkan bilangan prima dengan iterasi dan \foreignTerm{Primality Testing}.
  \item Solusi: 
  \begin{enumerate}
    \item Lakukan iterasi dari 2 sampai N
    \item Untuk tiap bilangan, cek apakah dia bilangan prima atau bukan. Jika iya, kita bisa memasukkannya ke daftar bilangan prima.
  \end{enumerate}
\end{itemize}
\end{frame}

\begin{frame}[fragile]
\frametitle{Solusi Awal (lanj.)}
  \begin{lstlisting}
  procedure sqrtGeneration(n: longint);
  var
    i, j: longint;
    prime: boolean;
  begin
    for i := 2 to n do begin
        prime := true;
        j := 2;
        while (j * j <= n) do begin
          if (i mod j = 0) then begin
            prime := false;
          end;
          inc(j);
        end;
        if (prime) then
          writeln(i, ' adalah bilangan prima');
    end;
  end;  
  \end{lstlisting}
\end{frame}

\begin{frame}
\frametitle{Sieve of Erathostenes}
\begin{itemize}
  \item Terdapat solusi yang lebih optimal dari segi waktu untuk membangkitkan bilangan prima, yaitu \newTerm{Sieve of Erathostenes}.
  \item Ide utama utama dari algoritma ini adalah mengeliminasi bilangan-bilangan dari calon bilangan prima.
  \item Yang akan kita eliminasi tentu bilangan komposit. Pertama, kita tahu bahwa suatu bilangan bulat $X$ dapat dinyatakan sebagai $X = P \times Q$, dimana P suatu bilangan prima.
  \item Seandainya kita sudah mendapatkan suatu bilangan prima, kita dapat mengeliminasi kelipatan bilangan tersebut dari calon bilangan prima.
\end{itemize}
\end{frame}

\begin{frame}
\frametitle{Sieve of Erathostenes (lanj.)}
\begin{itemize}
  \item Prosedurnya:
  \begin{enumerate}
    \item Lakukan iterasi dari 2 sampai N .
    \item Ketika bertemu dengan bilangan yang belum dieliminasi, artinya bilangan ini merupakan bilangan prima. Kemudian, lakukan iterasi untuk mengeliminasi kelipatan bilangan tersebut.
  \end{enumerate}
  \item Kita memerlukan \foreignTerm{array} untuk mengetahui bilangan yang sudah tereliminasi. Maka, kita memerlukan memori sebesar $O(N)$.
  \item Kompleksitas waktu solusi ini $O(N\;log\;log\;N)$.
\end{itemize}
\end{frame}

\begin{frame}[fragile]
\frametitle{Sieve of Erathostenes (lanj.)}
Solusi:
  \begin{lstlisting}
    procedure sieveOfErathostenes(n: longint);
    var
      i, j: longint;
      flag: array[1..100000] of boolean;
    begin
      for i := 2 to n do 
        if (not flag[i]) then begin
          writeln(i, ' adalah bilangan prima').
          j := i;
          while(i * j <= n) do begin
            flag[i * j] = true;
            inc(j);
          end;
        end;
    end;  
  \end{lstlisting}
\end{frame}

\section{FPB dan KPK}
\frame{\sectionpage}

\begin{frame}
\frametitle{Faktorisasi Prima}
\begin{itemize}
  \item Ketika masih SD, tentunya kita pernah belajar memfaktorkan bilangan dengan pohon faktor.
  \item Melalui faktorisasi prima, kita dapat menyatakan suatu bilangan sebagai hasil kali perkalian faktor-faktornya.
  \item Contoh: $7875 = 3^{2} \times 5^{3} \times 7$ .
  \item Faktorisasi prima ini dapat bermanfaat pada soal-soal matematika di \foreignTerm{Competitive Programming}.
\end{itemize}
\end{frame}

\begin{frame}
\frametitle{FPB dan KPK}
\begin{itemize}
  \item Kita pernah diajari cara mencari FPB dan KPK melalui faktorisasi prima. Singkatnya, untuk setiap bilangan prima, kita menggunakan pangkat terkecil untuk FPB dan pangkat terbesar untuk KPK.
  \item Contoh: $4725 = 3^{3} \times 5^{2} \times 7$ dan $7875 = 3^{2} \times 5^{3} \times 7$.
  \item Maka, $FPB(4725,7875) = 3^{2} \times 5^{2} \times 7 = 1525$ dan $KPK(4725,7875) = 3^{3} \times 5^{3} \times 7 = 23625$.
  \item Jika kita observasi, $KPK(A,B) = \frac{A \times B}{FPB(A,B)}$. 
\end{itemize}
\end{frame}

\begin{frame}
\frametitle{Algoritma Euclid}
\begin{itemize}
  \item Untuk mencari FPB suatu bilangan, menggunakan pohon faktor \emp{sangat rumit!}
  \item Kita harus mencari faktor prima bilangan tersebut, dan jika faktor primanya besar, tentu akan menghabiskan banyak waktu.
  \item Terdapat algoritma yang dapat mencari $FPB(A,B)$ dalam $O(log(min(A,B))$.
\end{itemize}
\end{frame}

\begin{frame}[fragile]
\frametitle{Algoritma Euclid (lanj.)}
\begin{itemize}
  \item Solusi:
  \begin{lstlisting}
  function euclid(a,b: longint): longint;
  begin
    if (b = 0) then begin
      euclid := a;
    end
    else begin
      euclid := euclid(b,a mod b);
    end;
  end;
  \end{lstlisting}
  \item Jika Anda tertarik, Anda dapat membaca lebih lanjut \textcolor{blue}{\href{https://en.wikipedia.org/wiki/Euclidean_algorithm}{di link wikipedia ini}.}   
\end{itemize}
\end{frame}

\section{Pigeon Hole Principle}
\frame{\sectionpage}

\begin{frame}
\frametitle{Motivasi}
Deskripsi: \newline
Pak Dengklek memiliki sebuah array berisi N bilangan bulat non-negatif. Pak Dengklek pun menantang Anda untuk memilih angka-angka dari arraynya yang jika dijumlahkan habis dibagi N. Tentu saja angka di suatu indeks tidak boleh dipilih lebih dari sekali. Apabila hal ini mungkin, beritahu Pak Dengklek berapa banyak angka yang Anda ambil dan apa saja angka-angkanya. Apabila hal ini tidak mungkin, katakan "Tidak mungkin". \newline\newline
Batasan:
\begin{itemize}
  \item $1 \leq N \leq 10^{5}$
  \item Array berisi bilangan bulat non-negatif.
\end{itemize}
\end{frame}

\begin{frame}
\frametitle{Motivasi (lanj.)}
\begin{itemize}
  \item Inti permasalahan ini adalah mencari apakah pada \foreignTerm{array} berukuran $N$, terdapat sub-himpunan tidak kosong yang jumlahan elemennya habis dibagi $N$.
  \item Jika ada, outputkan indeks-indeks yang terdapat di sub-himpunan tersebut. Jika tidak ada, keluarkan "Tidak Mungkin".
\end{itemize}
\end{frame}

\begin{frame}
\frametitle{Motivasi (lanj.)}
\begin{itemize}
  \item Mari kita coba mengerjakan versi lebih mudah dari soal ini. Bagaimana jika yang diminta sub-barisan, bukan sub-himpunan?
  \item Misal kita memiliki fungsi $sum(k) = \sum\limits_{i=1}^k array[i]$.
  \item Bagaimana hubungan sub-barisan yang habis dibagi $N$ seandainya kita kaitkan dengan $sum(k)$ dan sifat modulo?
\end{itemize}
\end{frame}

\begin{frame}
\frametitle{Motivasi (lanj.)}
\begin{itemize}
  \item Kita dapat menyimpulkan bahwa $\sum\limits_{i=l}^r array[i] = sum(r) - sum(l - 1)$.
  \item Jika habis dibagi $N$, maka $(sum(r) - sum(l - 1)) \mod N = 0 \mod N$.
  \item Ini dapat kita tuliskan sebagai $sum(r) \mod N = sum(l - 1) \mod N$.
\end{itemize}
\end{frame}

\begin{frame}
\frametitle{Motivasi (lanj.)}
\begin{itemize}
  \item Observasi 1: ada N buah kemungkinan nilai sum(x), yaitu [0...N-1].
  \item Observasi 2: ada N+1 buah kemungkinan nilai x untuk sum(x), yaitu [0..N]. Nilai 0 harus ada agar sub-barisan dalam bentuk (1,k) untuk tiap k dapat kita nyatakan dalam bentuk dari slide sebelumnya.
  \item Observasi 3: Karena ada N+1 kemungkinan nilai x, dan ada N buah kemungkinan nilai sum(x), maka \emp{pasti ada 2 buah indeks, a dan b, sehingga $sum(b) \mod N = sum(a) \mod N$}
  \item Maka, kita bisa menggunakan sub-barisan [a+1..b] sebagai solusi.
\end{itemize}
\end{frame}

\begin{frame}
\frametitle{Motivasi (lanj.)}
\begin{itemize}
  \item Dengan menyelesaikan versi mudah dari soal awal kita, ternyata kita justru dapat menyelesaikan soal tersebut.
\end{itemize}
\end{frame}

\begin{frame}
\frametitle{Pigeon Hole Principle}
\begin{itemize}
  \item Konsep PHP adalah \statement{Jika ada $N$ burung dan $M$ sangkar, dimana $M > N$ , maka ada sangkar yang berisi setidaknya 2 ekor burung}.
  \item Secara matematis, jika ada $N$ burung dan $M$ sangkar, maka ada sangkar yang berisi setidaknya $\big\lceil \frac{M}{N} \big\rceil$ ekor burung.
\end{itemize}
\end{frame}

\end{document}
