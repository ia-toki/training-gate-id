\documentclass{beamer}
\usetheme{tokitex}

\usepackage{tikz}
\usepackage{graphics}
\usepackage{multirow}
\usepackage{tabto}
\usepackage{xspace}
\usepackage{amsmath}
\usepackage{hyperref}
\usepackage{wrapfig}
\usepackage{mathtools}

\usepackage{tikz}
\usepackage{clrscode3e}
\usepackage{gensymb}

\usepackage[english,bahasa]{babel}
\newtranslation[to=bahasa]{Section}{Bagian}
\newtranslation[to=bahasa]{Subsection}{Subbagian}

\usepackage{listings, lstautogobble}
\usepackage{color}

\definecolor{dkgreen}{rgb}{0,0.6,0}
\definecolor{gray}{rgb}{0.5,0.5,0.5}
\definecolor{mauve}{rgb}{0.58,0,0.82}

\lstset{frame=tb,
  language=c++,
  aboveskip=0mm,
  belowskip=0mm,
  showstringspaces=false,
  columns=fullflexible,
  keepspaces=true,
  basicstyle={\small\ttfamily},
  numbers=none,
  numberstyle=\tiny\color{gray},
  keywordstyle=\color{blue},
  commentstyle=\color{dkgreen},
  stringstyle=\color{mauve},
  breaklines=true,
  breakatwhitespace=true,
  lineskip={-3pt}
}

\usepackage{caption}
\captionsetup[figure]{labelformat=empty}

\newcommand{\progTerm}[1]{\textbf{#1}}
\newcommand{\foreignTerm}[1]{\textit{#1}}
\newcommand{\newTerm}[1]{\alert{\textbf{#1}}}
\newcommand{\emp}[1]{\alert{#1}}
\newcommand{\statement}[1]{"#1"}

\newcommand{\floor}[1]{\lfloor #1 \rfloor}
\newcommand{\ceil}[1]{\lceil #1 \rceil}
\newcommand{\abs}[1]{\left\lvert#1\right\rvert}
\newcommand{\norm}[1]{\left\lVert#1\right\rVert}

% Getting tired of writing \foreignTerm all the time
\newcommand{\farray}{\foreignTerm{array}\xspace}
\newcommand{\fArray}{\foreignTerm{Array}\xspace}
\newcommand{\foverhead}{\foreignTerm{overhead}\xspace}
\newcommand{\fOverhead}{\foreignTerm{Overhead}\xspace}
\newcommand{\fsubarray}{\foreignTerm{subarray}\xspace}
\newcommand{\fSubarray}{\foreignTerm{Subarray}\xspace}
\newcommand{\fbasecase}{\foreignTerm{base case}\xspace}
\newcommand{\fBasecase}{\foreignTerm{Base case}\xspace}
\newcommand{\ftopdown}{\foreignTerm{top-down}\xspace}
\newcommand{\fTopdown}{\foreignTerm{Top-down}\xspace}
\newcommand{\fbottomup}{\foreignTerm{bottom-up}\xspace}
\newcommand{\fBottomup}{\foreignTerm{Bottom-up}\xspace}
\newcommand{\fpruning}{\foreignTerm{pruning}\xspace}
\newcommand{\fPruning}{\foreignTerm{Pruning}\xspace}

\newcommand{\fgraph}{\foreignTerm{graph}\xspace}
\newcommand{\fGraph}{\foreignTerm{Graph}\xspace}
\newcommand{\froot}{\foreignTerm{root}\xspace}
\newcommand{\fRoot}{\foreignTerm{Root}\xspace}
\newcommand{\fnode}{\foreignTerm{node}\xspace}
\newcommand{\fNode}{\foreignTerm{Node}\xspace}
\newcommand{\fedge}{\foreignTerm{edge}\xspace}
\newcommand{\fEdge}{\foreignTerm{Edge}\xspace}
\newcommand{\fcycle}{\foreignTerm{cycle}\xspace}
\newcommand{\fCycle}{\foreignTerm{Cycle}\xspace}
\newcommand{\fdegree}{\foreignTerm{degree}\xspace}
\newcommand{\fDegree}{\foreignTerm{Degree}\xspace}
\newcommand{\fadjacencylist}{\foreignTerm{adjacency list}\xspace}
\newcommand{\fAdjacencylist}{\foreignTerm{Adjacency list}\xspace}
\newcommand{\fadjacencymatrix}{\foreignTerm{adjacency matrix}\xspace}
\newcommand{\fAdjacencymatrix}{\foreignTerm{Adjacency matrix}\xspace}
\newcommand{\fedgelist}{\foreignTerm{edge list}\xspace}
\newcommand{\fEdgelist}{\foreignTerm{Edge list}\xspace}
\newcommand{\flist}{\foreignTerm{list}\xspace}
\newcommand{\fList}{\foreignTerm{List}\xspace}
\newcommand{\fgraphtraversal}{\foreignTerm{graph traversal}\xspace}
\newcommand{\fGraphtraversal}{\foreignTerm{Graph traversal}\xspace}
\newcommand{\ftree}{\foreignTerm{tree}\xspace}
\newcommand{\fTree}{\foreignTerm{Tree}\xspace}
\newcommand{\fsubtree}{\foreignTerm{subtree}\xspace}
\newcommand{\fSubtree}{\foreignTerm{Subtree}\xspace}
\newcommand{\fparent}{\foreignTerm{parent}\xspace}
\newcommand{\fParent}{\foreignTerm{Parent}\xspace}
\newcommand{\fsibling}{\foreignTerm{sibling}\xspace}
\newcommand{\fSibling}{\foreignTerm{Sibling}\xspace}
\newcommand{\fpath}{\foreignTerm{path}\xspace}
\newcommand{\fPath}{\foreignTerm{Path}\xspace}
\newcommand{\fconnectedcomponent}{\foreignTerm{connected component}\xspace}
\newcommand{\fConnectedcomponent}{\foreignTerm{Connected component}\xspace}
\newcommand{\fbridge}{\foreignTerm{bridge}\xspace}
\newcommand{\fBridge}{\foreignTerm{Bridge}\xspace}
\newcommand{\farticulationpoint}{\foreignTerm{articulation point}\xspace}
\newcommand{\fArticulationpoint}{\foreignTerm{Articulation point}\xspace}
\newcommand{\ftreeedge}{\foreignTerm{tree edge}\xspace}
\newcommand{\fTreeedge}{\foreignTerm{Tree edge}\xspace}
\newcommand{\fbackedge}{\foreignTerm{back edge}\xspace}
\newcommand{\fBackedge}{\foreignTerm{Back edge}\xspace}
\newcommand{\fforwardedge}{\foreignTerm{forward edge}\xspace}
\newcommand{\fForwardedge}{\foreignTerm{Forward edge}\xspace}
\newcommand{\fcrossedge}{\foreignTerm{cross edge}\xspace}
\newcommand{\fCrossedge}{\foreignTerm{Cross edge}\xspace}
\newcommand{\fdiscoverytime}{\foreignTerm{discovery time}\xspace}
\newcommand{\fDiscoverytime}{\foreignTerm{Discovery time}\xspace}
\newcommand{\flowlink}{\foreignTerm{low link}\xspace}
\newcommand{\fLowlink}{\foreignTerm{Low link}\xspace}
\newcommand{\fstack}{\foreignTerm{stack}\xspace}
\newcommand{\fStack}{\foreignTerm{Stack}\xspace}
\newcommand{\for}{\foreignTerm{or}\xspace}
\newcommand{\fOr}{\foreignTerm{Or}\xspace}
\newcommand{\fand}{\foreignTerm{and}\xspace}
\newcommand{\fAnd}{\foreignTerm{And}\xspace}
\newcommand{\fcentroid}{\foreignTerm{centroid}\xspace}
\newcommand{\fCentroid}{\foreignTerm{Centroid}\xspace}

\newcommand{\fDivideAndConquer}{\foreignTerm{Divide and conquer}\xspace}
\newcommand{\fdivideAndConquer}{\foreignTerm{divide and conquer}\xspace}
\newcommand{\fMergeSort}{\foreignTerm{Merge sort}\xspace}
\newcommand{\fmergeSort}{\foreignTerm{merge sort}\xspace}
\newcommand{\fQuickSort}{\foreignTerm{Quicksort}\xspace}
\newcommand{\fquickSort}{\foreignTerm{quicksort}\xspace}
\newcommand{\fpivot}{\foreignTerm{pivot}\xspace}
\newcommand{\fPivot}{\foreignTerm{Pivot}\xspace}
\newcommand{\fbruteForce}{\foreignTerm{brute force}\xspace}
\newcommand{\fBruteForce}{\foreignTerm{Brute force}\xspace}
\newcommand{\fCompleteSearch}{\foreignTerm{complete search}\xspace}
\newcommand{\fExhaustiveSearch}{\foreignTerm{exhaustive search}\xspace}
\newcommand{\fbinarySearch}{\foreignTerm{binary search}\xspace}
\newcommand{\fBinarySearch}{\foreignTerm{Binary search}\xspace}
\newcommand{\fternarySearch}{\foreignTerm{ternary search}\xspace}
\newcommand{\fTernarySearch}{\foreignTerm{Ternary search}\xspace}
\newcommand{\funimodal}{\foreignTerm{unimodal}\xspace}
\newcommand{\fUnimodal}{\foreignTerm{Unimodal}\xspace}
\newcommand{\fGreedy}{\foreignTerm{Greedy}\xspace}
\newcommand{\fgreedy}{\foreignTerm{greedy}\xspace}
\newcommand{\fgreedyChoice}{\foreignTerm{greedy choice}\xspace}
\newcommand{\fGreedyChoice}{\foreignTerm{Greedy choice}\xspace}

\newcommand{\fdp}{\foreignTerm{dynamic programming}\xspace}
\newcommand{\fDp}{\foreignTerm{Dynamic programming}\xspace}
\newcommand{\fbitmask}{\foreignTerm{bitmask}\xspace}
\newcommand{\fBitmask}{\foreignTerm{Bitmask}\xspace}
\newcommand{\fstate}{\foreignTerm{state}\xspace}
\newcommand{\fState}{\foreignTerm{State}\xspace}
\newcommand{\fsubmask}{\foreignTerm{submask}\xspace}
\newcommand{\fSubmask}{\foreignTerm{Submask}\xspace}

\newcommand{\pheap}{\foreignTerm{heap}\xspace}
\newcommand{\pHeap}{\foreignTerm{Heap}\xspace}
\newcommand{\pBinaryHeap}{\foreignTerm{Binary Heap}\xspace}
\newcommand{\pbinaryHeap}{\foreignTerm{binary heap}\xspace}
\newcommand{\pHeapsort}{\foreignTerm{Heapsort}\xspace}
\newcommand{\pheapsort}{\foreignTerm{heapsort}\xspace}
\newcommand{\pdjs}{\foreignTerm{disjoint set}\xspace}
\newcommand{\pDjs}{\foreignTerm{Disjoint set}\xspace}

\newcommand{\fdotProduct}{\foreignTerm{dot product}\xspace}
\newcommand{\fDotProduct}{\foreignTerm{Dot product}\xspace}
\newcommand{\fcrossProduct}{\foreignTerm{cross product}\xspace}
\newcommand{\fCrossProduct}{\foreignTerm{Cross product}\xspace}
\newcommand{\fconvexHull}{\foreignTerm{convex hull}\xspace}
\newcommand{\fConvexHull}{\foreignTerm{Convex hull}\xspace}
\newcommand{\fgrahamScan}{\foreignTerm{graham scan}\xspace}
\newcommand{\fGrahamScan}{\foreignTerm{Graham scan}\xspace}
\newcommand{\flineSweep}{\foreignTerm{line sweep}\xspace}
\newcommand{\fLineSweep}{\foreignTerm{Line sweep}\xspace}

\newcommand{\fset}{\foreignTerm{set}\xspace}
\newcommand{\fSet}{\foreignTerm{Set}\xspace}
\newcommand{\fprefixSum}{\foreignTerm{prefix sum}\xspace}
\newcommand{\fPrefixSum}{\foreignTerm{Prefix sum}\xspace}
\newcommand{\ffenwickTree}{\foreignTerm{fenwick tree}\xspace}
\newcommand{\fFenwickTree}{\foreignTerm{Fenwick tree}\xspace}
\newcommand{\frangeSumQuery}{\foreignTerm{range sum query}\xspace}
\newcommand{\fRangeSumQuery}{\foreignTerm{Range sum query}\xspace}
\newcommand{\fquery}{\foreignTerm{query}\xspace}
\newcommand{\fQuery}{\foreignTerm{Query}\xspace}
\newcommand{\fsegmentTree}{\foreignTerm{segment tree}\xspace}
\newcommand{\fSegmentTree}{\foreignTerm{Segment tree}\xspace}
\newcommand{\fbinaryTree}{\foreignTerm{binary tree}\xspace}
\newcommand{\fBinaryTree}{\foreignTerm{Binary tree}\xspace}
\newcommand{\flazyPropagation}{\foreignTerm{lazy propagation}\xspace}
\newcommand{\fLazyPropagation}{\foreignTerm{Lazy propagation}\xspace}
\newcommand{\fsparseTable}{\foreignTerm{sparse table}\xspace}
\newcommand{\fSparseTable}{\foreignTerm{Sparse table}\xspace}

\newcommand{\ftrail}{\foreignTerm{trail}\xspace}
\newcommand{\fTrail}{\foreignTerm{Trail}\xspace}
\newcommand{\feulerTour}{\foreignTerm{euler tour}\xspace}
\newcommand{\fEulerTour}{\foreignTerm{Euler tour}\xspace}
\newcommand{\feulerTourTree}{\foreignTerm{euler tour tree}\xspace}
\newcommand{\fEulerTourTree}{\foreignTerm{Euler tour tree}\xspace}

\newcommand{\fmaxflow}{\foreignTerm{maximum flow}\xspace}
\newcommand{\fMaxflow}{\foreignTerm{Maximum flow}\xspace}
\newcommand{\fmincut}{\foreignTerm{minimum cut}\xspace}
\newcommand{\fMincut}{\foreignTerm{Minimum cut}\xspace}
\newcommand{\fflow}{\foreignTerm{flow}\xspace}
\newcommand{\fFlow}{\foreignTerm{Flow}\xspace}
\newcommand{\fsource}{\foreignTerm{source}\xspace}
\newcommand{\fSource}{\foreignTerm{Source}\xspace}
\newcommand{\fsink}{\foreignTerm{sink}\xspace}
\newcommand{\fSink}{\foreignTerm{Sink}\xspace}
\newcommand{\fbackEdge}{\foreignTerm{back-edge}\xspace}
\newcommand{\fBackEdge}{\foreignTerm{Back-edge}\xspace}
\newcommand{\fresidualCapacity}{\foreignTerm{residual capacity}\xspace}
\newcommand{\fResidualCapacity}{\foreignTerm{Residual capacity}\xspace}
\newcommand{\fbottleneck}{\foreignTerm{bottleneck}\xspace}
\newcommand{\fBottleneck}{\foreignTerm{Bottleneck}\xspace}
\newcommand{\faugmentingPath}{\foreignTerm{augmenting path}\xspace}
\newcommand{\fAugmentingPath}{\foreignTerm{Augmenting path}\xspace}

\usepackage{tikz}

\title{Kombinatorik}
\author{Tim Olimpiade Komputer Indonesia}
\date{}

\begin{document}

\begin{frame}
\titlepage
\end{frame}

\begin{frame}
\frametitle{Pendahuluan}
Melalui dokumen ini, kalian akan:
\begin{itemize}
  \item Mempelajari \newTerm{aturan perkalian} dan \newTerm{aturan penjumlahan}.
  \item Mempelajari \newTerm{permutasi}.
  \item Mempelajari \newTerm{kombinasi}.
  \item Memahami \newTerm{segitiga pascal}.
\end{itemize}
\end{frame}

\section{Aturan Perkalian dan Aturan Penjumlahan}
\frame{\sectionpage}

\begin{frame}
\frametitle{Permasalahan 1}
\begin{itemize}
  \item Terdapat 3 buah kota yaitu A, B, dan C.
  \item Kota A dan kota B terhubung oleh 3 jalur berbeda yaitu $e_{1}$, $e_{2}$, dan $e_{3}$.
  \item Sedangkan kota B dan kota C terhubung oleh 2 jalur berbeda yaitu $e_{4}$ dan $e_{5}$.
  \item Berapa banyak cara berbeda untuk menuju kota C dari kota A?
  \item Ilustrasi : 
  \newline
  \begin{tikzpicture}[auto, node distance=3cm, every loop/.style={},
                      thick,main node/.style={circle,draw,font=\sffamily\Large\bfseries}]
  
    \node[main node] (1) {A};
    \node[main node] (2) [right of=1] {B};
    \node[main node] (3) [right of=2] {C};
  
    \path[every node/.style={font=\sffamily\small}]
      (1) edge[bend left] node {$e_{1}$} (2)
      	  edge node {$e_{2}$} (2)
          edge[bend right] node {$e_{3}$} (2)
      (2) edge[bend left] node {$e_{4}$} (3)
          edge[bend right] node {$e_{5}$} (3);
  \end{tikzpicture}
\end{itemize}
\end{frame}

\begin{frame}
\frametitle{Solusi Awal}
\begin{itemize}
  \item Apabila kita hitung satu per satu, maka cara yang berbeda untuk menuju kota C dari kota A adalah sebagai berikut :
  \begin{itemize}
    \item Melalui \foreignTerm{jalur $e_{1}$} kemudian \foreignTerm{jalur $e_{4}$}
    \item Melalui \foreignTerm{jalur $e_{1}$} kemudian \foreignTerm{jalur $e_{5}$}
    \item Melalui \foreignTerm{jalur $e_{2}$} kemudian \foreignTerm{jalur $e_{4}$}
    \item Melalui \foreignTerm{jalur $e_{2}$} kemudian \foreignTerm{jalur $e_{5}$}
    \item Melalui \foreignTerm{jalur $e_{3}$} kemudian \foreignTerm{jalur $e_{4}$}
    \item Melalui \foreignTerm{jalur $e_{3}$} kemudian \foreignTerm{jalur $e_{5}$}
  \end{itemize}
  \item Dengan kata lain, terdapat 6 cara berbeda untuk menuju kota C dari kota A.
  \item Tetapi, apabila jumlah kota dan jalur yang ada sangatlah banyak, kita tidak mungkin menulis satu per satu cara yang berbeda. Karena itulah kita gunakan \newTerm{Aturan Perkalian}.
\end{itemize}
\end{frame} 

\begin{frame}
\frametitle{Aturan Perkalian}
\begin{itemize}
  \item Apabila suatu proses dapat dibagi menjadi \foreignTerm{N sub-proses berbeda} dimana terdapat $a_{i}$ cara untuk menyelesaikan sub-proses ke-i.
  \item Banyak cara yang berbeda untuk menyelesaikan proses tersebut adalah $a_{1} \times a_{2} \times a_{3} \times ... \times a_{i}$.	
\end{itemize}
\end{frame}

\begin{frame}
\frametitle{Solusi dengan Aturan Perkalian}
\begin{itemize}
  \item Jika kita anggap bahwa banyak cara dari kota A menuju kota B merupakan sub-proses pertama, dimana terdapat 3 cara untuk menyelesaikan sub-proses tersebut.
  \item Kita anggap pula bahwa banyak cara dari kota B menuju kota C merupakan sub-proses kedua, dimana terdapat 2 cara untuk menyelesaikan sub-proses tersebut.
  \item Karena perjalanan dari kota A menuju kota B dan dari kota B menuju kota C merupakan 2 sub-proses yang berbeda, maka kita dapat gunakan aturan perkalian.
  \item Banyak cara berbeda dari kota A menuju kota C adalah $3 \times 2 = 6$.
\end{itemize}
\end{frame}

\begin{frame}
\frametitle{Permasalahan 2}
\begin{itemize}
  \item Permasalahan ini merupakan lanjutan dari \newTerm{Permasalahan 1}. Deskripsi permasalahan, jumlah kota dan jalur serta susunan jalur yang ada sama persis dengan permasalahan tersebut.
  \item Apabila kita tambahkan 1 jalur lagi, yaitu $e_{6}$ yang menghubungkan kota A dan C.
  \item Berapa banyak cara berbeda untuk menuju kota C dari kota A?
  \item Ilustrasi :
  \newline
  \begin{tikzpicture}[auto, node distance=3cm, every loop/.style={},
                        thick,main node/.style={circle,draw,font=\sffamily\Large\bfseries}]
    
      \node[main node] (1) {A};
      \node[main node] (2) [right of=1] {B};
      \node[main node] (3) [right of=2] {C};
    
      \path[every node/.style={font=\sffamily\small}]
        (1) edge[bend left] node {$e_{1}$} (2)
        	edge node {$e_{2}$} (2)
            edge[bend right] node {$e_{3}$} (2)
        (2) edge[bend left] node {$e_{4}$} (3)
            edge[bend right] node {$e_{5}$} (3)
        (3) edge[bend left] node {$e_{6}$} (1);
    \end{tikzpicture}
\end{itemize}
\end{frame}

\begin{frame}
\frametitle{Analisa Permasalahan}
\begin{itemize}
  \item Dengan menulis satu per satu setiap cara, maka terdapat 7 cara yang berbeda, yaitu 6 cara sesuai dengan permasalahan sebelumnya, ditambah dengan \foreignTerm{menggunakan jalur $e_{6}$}.
  \item Apabila kita menggunakan Aturan Perkalian, maka didapatkan banyak cara yang berbeda adalah $3 \times 2 \times 1 = 6$ dimana solusi tersebut tidaklah tepat.
  \item Kita tidak dapat menggunakan Aturan Perkalian dalam permasalahan ini, dikarenakan antara perjalanan dari kota A menuju kota C melalui kota B dengan tanpa melalui kota B merupakan 2 proses yang berbeda.
  \item Oleh karena itu, kita dapat menggunakan \newTerm{Aturan Penjumlahan}.
\end{itemize}
\end{frame}

\begin{frame}
\frametitle{Aturan Penjumlahan}
\begin{itemize}
  \item Apabila suatu proses dapat dibagi menjadi \foreignTerm{N himpunan proses berbeda} yaitu $H_{1}, H_{2}, H_{3}, ... , H_{N}$ dimana setiap himpunannya \foreignTerm{saling lepas} (tidak beririsan).
  \item Banyak cara yang berbeda untuk menyelesaikan proses tersebut adalah $|H_{1}| + |H_{2}| + |H_{3}| + ... + |H_{i}|$ dimana $|H_{i}|$ merupakan banyaknya cara berbeda untuk menyelesaikan proses ke-i.
\end{itemize}
\end{frame}

\begin{frame}
\frametitle{Solusi dengan Aturan Penjumlahan}
\begin{itemize}
  \item Proses perjalanan dari kota A menuju kota C dapat kita bagi menjadi 2 himpunan proses yang berbeda, yaitu dari kota A menuju kota C melalui kota B, dan dari kota A langsung menuju kota C.
  \item Banyak cara dari kota A menuju kota C melalui kota B dapat kita dapatkan dengan Aturan Perkalian seperti yang dibahas pada permasalahan sebelumnya, yaitu 6 cara berbeda.
  \item Banyak cara dari kota A langsung menuju kota C adalah 1 cara, yaitu melalui jalur $e_{6}$.
  \item Dengan \foreignTerm{Aturan Penjumlahan}, banyak cara berbeda dari kota A menuju kota C adalah $6 + 1 = 7$ cara berbeda.
\end{itemize}
\end{frame}

\begin{frame}
\frametitle{Hati-Hati!}
\begin{itemize}
  \item Apabila terdapat irisan dari himpunan proses tersebut, maka solusi yang kita dapatkan dengan Aturan Penjumlahan menjadi tidak tepat, dikarenakan solusi dalam irisan tersebut terhitung 2 kali.
  \item Agar solusi tersebut menjadi tepat, gunakan \textcolor{blue}{\href{https://en.wikipedia.org/wiki/Inclusion-exclusion\_principle}{Prinsip Inklusi-Eksklusi}} pada \textcolor{blue}{\href{https://id.wikipedia.org/wiki/Himpunan}{Teori Himpunan}}.
\end{itemize}
\end{frame}

\section{Permutasi}
\frame{\sectionpage}

\begin{frame}
\frametitle{Notasi Faktorial}
\begin{itemize}
  \item Perlu diperhatikan bahwa faktorial dari N ($N!$) merupakan hasil perkalian dari semua bilangan asli kurang dari atau sama dengan N.
  \item $N! = N \times (N-1) \times (N-2) \times ... \times 3 \times 2 \times 1$, dimana $O! = 1$.
  \item Contoh : $5! = 5 \times 4 \times 3 \times 2 \times 1 = 120$.
\end{itemize}
\end{frame}

\begin{frame}
\frametitle{Redundansi}
\begin{itemize}
  \item Redundansi sering disebut juga dengan \foreignTerm{Aturan Pembagian}.
  \item Apabila terdapat K susunan cara yang kita anggap berbeda dan seharusnya merupakan 1 cara yang sama, maka kita dapat membagi total keseluruhan cara dengan K, sehingga K cara tersebut dianggap sama sebagai 1 cara.
\end{itemize}
\end{frame}

\begin{frame}
\frametitle{Contoh Redundansi}
\begin{itemize}
  \item Banyak kata berbeda yang disusun dari huruf-huruf penyusun "TOKI" adalah $4!$ (menggunakan aturan perkalian).
  \item Apabila kita ganti soal tersebut, yaitu kata berbeda yang disusun dari huruf-huruf penyusun "BACA", solusi $4!$ merupakan solusi yang salah.
  \item Hal ini disebabkan karena terdapat 2 buah huruf 'A'. Sebagai contoh : $BA_{1}CA_{2}$ dan $BA_{2}CA_{1}$ pada dasarnya merupakan kata yang sama.
\end{itemize}
\end{frame}

\begin{frame}
\frametitle{Contoh Redundansi (lanj.)}
\begin{itemize}
  \item Terdapat 2! cara yang kita anggap berbeda tetapi seharusnya merupakan cara yang sama, yaitu penggunaan $A_{1}A_{2}$ dan $A_{2}A_{1}$.
  \item Sehingga banyak kata berbeda yang dapat kita bentuk dari huruf-huruf penyusun kata "BACA" adalah $\frac{4!}{2!} = \frac{24}{2} = 12$ kata berbeda.
\end{itemize}
\end{frame}

\begin{frame}
\frametitle{Permasalahan 1}
\begin{itemize}
  \item Terdapat 5 anak (sebut saja A, B, C, D, dan E) yang sedang mengikuti sebuah kompetisi.
  \item Dalam kompetisi tersebut akan diambil 3 peserta sebagai pemenang.
  \item Berapa banyak susunan pemenang yang berbeda dari kelima orang tersebut?
\end{itemize}
\end{frame}

\begin{frame}
\frametitle{Solusi Awal}
\begin{itemize}
  \item Anggap bahwa kita mengambil semua anak sebagai pemenang, sehingga terdapat $5! = 120$ susunan pemenang yang berbeda (ABCDE, ABCED, ABDCE, ..., EDCBA).
  \item Apabila kita hanya mengambil 3 peserta saja, perhatikan bahwa terdapat 2 cara yang kita anggap berbeda dimana seharusnya merupakan cara yang sama. Contoh : (ABC)DE dan (ABC)ED merupakan cara yang sama, karena 3 peserta yang menang adalah A, B, dan C.
  \item Dengan menggunakan prinsip \foreignTerm{Redundansi}, maka banyak susunan pemenang yang berbeda adalah $\frac{5!}{2!} = \frac{120}{2} = 60$ susunan berbeda.
\end{itemize}
\end{frame}

\begin{frame}
\frametitle{Solusi Secara Umum}
\begin{itemize}
  \item Apabila terdapat N anak, dan kita mengambil semua anak sebagai pemenang, maka terdapat $N!$ susunan cara berbeda.
  \item Tetapi apabila kita hanya mengambil R anak saja, maka akan terdapat $(N-R)!$ susunan yang kita anggap berbeda dimana seharusnya merupakan satu cara yang sama.
  \item Secara umum dengan prinsip Redundansi, banyak susunan berbeda adalah $\frac{N!}{(N-R)!}$.
  \item Rumus itulah yang kita kenal dengan istilah \newTerm{Permutasi}.
\end{itemize}
\end{frame}


\begin{frame}
\frametitle{Permutasi}
\begin{itemize}
  \item Apabila terdapat \foreignTerm{n objek} dan kita akan mengambil \foreignTerm{r objek} dari n objek tersebut dimana $r<n$ dan \foreignTerm{urutan pengambilan diperhitungkan}.
  \item Banyak cara pengambilan yang berbeda adalah Permutasi r terhadap n : 
  \item $P(n,r) = _{n}P_{r} = P^{n}_{r} = \frac{n!}{(n-r)!}$.
\end{itemize}
\end{frame}

\begin{frame}
\frametitle{Algoritma next\_permutation}
\begin{itemize}
  \item Apabila kita ingin melakukan permutasi r terhadap n, maka kita dapat mencari semua kemungkinan yang ada dengan menggunakan \foreignTerm{Algoritma next\_permutation}.
  \item Contoh : apabila kita memiliki array $A={1,2,3,4}$ dan kita ingin melakukan permutasi dengan mengambil 3 angka dari array tersebut, maka semua kemungkinan yang ada adalah $(1,2,3), (1,2,4), (1,3,2), ..., (4,3,2)$.
\end{itemize}
\end{frame}

\begin{frame}
\frametitle{Prosedur dari next\_permutation}
\begin{itemize}
  \item Jika jumlah objek dalam permutasi kita belum sama dengan r
  \begin{enumerate}
    \item Iterasi dari 1 hingga n.
    \item Apabila bilangan tersebut belum kita gunakan, maka kita dapat mencoba menggunakan bilangan tersebut.
    \item Tetapi jika bilangan tersebut sudah kita gunakan, maka abaikan bilangan tersebut.
  \end{enumerate}
  \item Jika jumlah objek dalam permutasi kita sudah sama dengan r
  \begin{enumerate}
    \item Artinya sudah kita dapatkan 1 cara dari permutasi tersebut.
    \item Kita bisa mencetaknya/menyimpannya dalam array.
  \end{enumerate}
\end{itemize}
\end{frame}

\begin{frame}
\frametitle{Implementasi dari next\_permutation}
\begin{itemize}
  \item Implementasi dari algoritma tersebut menggunakan teknik \newTerm{Recursive Backtracking}.
  \item Kita dapat menyimpan objek dalam permutasi kita menggunakan struktur data \foreignTerm{stack}.
  \item Kita dapat menyimpan angka yang belum/sudah kita gunakan dengan \foreignTerm{Array of boolean}.
  \item Apabila kita menemukan angka yang belum kita gunakan, maka rekursi ke state selanjutnya dimana jumlah objek dalam stack kita telah bertambah 1.
\end{itemize}
\end{frame}

\begin{frame}[fragile]
\frametitle{Implementasi next\_permutation (lanj.)}
\begin{codebox}
\Procname{$\proc{next\_permutation}(current)$}
\li \Comment Input berupa array $data$ berukuran n
\li \Comment Sediakan array of boolean $used$ berukuran n
\li \Comment Sediakan array $object$ berukuran n
\li \Comment Inisialisasi array $used$ dengan $false$
\li \If $current \neq r$
    \Then
\li   \For $i \gets 1$ \To $n$
      \Do
\li     \If $not$ $used[i]$
        \Then
\li       $used[i] \gets true$
\li       $object[current] \gets data[i]$
\li       $next\_permutation(current+1)$
\li       $used[i] \gets false$
        \End
      \End
    \End
\li \Else
    \Then
\li   \Return $object$
\end{codebox}
\end{frame}

\begin{frame}
\frametitle{Kompleksitas dari next\_permutation}
\begin{itemize}
  \item Apabila kita tidak menyimpan hasil dari permutasi kita, maka diperlukan memori sebesar O(N) untuk array of boolean.
  \item Banyak permutasi yang ada adalah $P^{n}_{r}$ dimana setiap permutasinya kita mengiterasi sebanyak N kali.
  \item Sehingga, kompleksitas waktu dari algoritma ini adalah $O(N \times P^{n}_{r})$.
  \item Apabila r = n, maka kompleksitas waktu dari algoritma ini adalah $O(N \times N!)$.
\end{itemize}
\end{frame}

\begin{frame}
\frametitle{Permasalahan 2}
\begin{itemize}
  \item Permasalahan ini sejenis dengan contoh pada Redundansi.
  \item Berapa banyak kata berbeda yang disusun dari huruf-huruf penyusun kata "MEGAGIGA"?
\end{itemize}
\end{frame}

\begin{frame}
\frametitle{Solusi Awal}
\begin{itemize}
  \item Terdapat 8 huruf, sehingga banyak kata yang dapat disusun adalah $8!$.
  \item Terdapat 3 huruf 'G' sehingga terdapat 6 kata yang kita anggap berbeda ($G_{1}G_{2}G_{3}, G_{1}G_{3}G_{2}, ...,G_{3}G_{2}G_{1}$) dimana seharusnya keenam kata tersebut merupakan kata yang sama.
  \item Dengan prinsip Redundansi, maka banyak kata yang dapat disusun mengingat kesamaan kata pada huruf G adalah $\frac{8!}{3!}$.
  \item Perlu kita perhatikan pula bahwa terdapat 2 huruf A, sehingga dengan cara yang sama akan didapatkan banyak kata yang berbeda adalah $\frac{8!}{(3! \times 2!)}$.
\end{itemize}
\end{frame}

\begin{frame}
\frametitle{Solusi Secara Umum}
\begin{itemize}
  \item Terdapat N huruf, sehingga banyak kata yang dapat kita susun adalah $N!$.
  \item Apabila terdapat K huruf dimana setiap hurufnya memiliki $R_{i}$ huruf yang sama, maka dengan prinsip Redundansi banyak kata berbeda yang dapat disusun adalah $\frac{N!}{(R_{1}! \times R_{2}! \times R_{3}! \times ... \times R_{K}!)}$.
  \item Rumus itulah yang kita kenal dengan \newTerm{Permutasi Elemen Berulang}.
\end{itemize}
\end{frame}

\begin{frame}
\frametitle{Permutasi Elemen Berulang}
\begin{itemize}
  \item Apabila terdapat n objek dan terdapat k objek dimana setiap objeknya memiliki $r_{i}$ elemen yang berulang, maka banyaknya cara berbeda dalam menyusun objek tersebut adalah :
  \item $P^{n}_{r_{1},r_{2},r_{3},...,r{k}} = \frac{n!}{(r_{1}! \times r_{2}! \times r_{3}! \times ... \times r_{k}!)}$ 
  \item Syarat untuk menggunakan rumus tersebut adalah jika $r_{1} + r_{2} + r_{3} + ... + r_{k} \leq n$.
\end{itemize}
\end{frame}

\begin{frame}
\frametitle{Permasalahan 3}
\begin{itemize}
  \item Terdapat 4 anak, sebut saja A, B, C, dan D.
  \item Berapa banyak susunan posisi duduk yang berbeda apabila mereka duduk melingkar?
\end{itemize}
\end{frame}

\begin{frame}
\frametitle{Solusi Awal}
\begin{itemize}
  \item Banyak susunan posisi duduk yang berbeda apabila mereka duduk seperti biasa (tidak melingkar) adalah $4! = 120$.
  \item Perhatikan bahwa posisi duduk ABCD, BCDA, CDAB, dan DABC merupakan susunan yang sama apabila mereka duduk melingkar, dikarenakan susunan tersebut merupakan rotasi dari susunan yang lainnya.
  \item Dengan kata lain terdapat 4 cara yang kita anggap berbeda dimana seharusnya hanya dihitung sebagai 1 cara yang sama.
  \item Dengan prinsip Redundansi, banyak susunan posisi duduk yang berbeda adalah $\frac{120}{4} = 30$ susunan berbeda.
\end{itemize}
\end{frame}


\begin{frame}
\frametitle{Solusi Secara Umum}
\begin{itemize}
  \item Banyak susunan posisi duduk yang berbeda apabila N anak seperti biasa (tidak melingkar) adalah $N!$.
  \item Dengan analisa yang sama, akan terdapat N cara yang kita anggap berbeda (pada posisi duduk tidak melingkar) dimana seharusnya dihitung sebagai 1 cara yang sama (pada posisi duduk melingkar).
  \item Dengan prinsip Redundansi, banyak susunan posisi duduk yang berbeda adalah $\frac{N!}{N} = (N-1)!$ susunan berbeda.
  \item Rumus itulah yang kita kenal dengan istilah \newTerm{Permutasi Siklis}.
\end{itemize}
\end{frame}

\begin{frame}
\frametitle{Permutasi Siklis}
\begin{itemize}
  \item Permutasi Siklis adalah Permutasi yang disusun melingkar.
  \item Banyaknya susunan yang berbeda dari Permutasi siklis terhadap n objek adalah :
  \item $P^{n}_{(siklis)} = (n-1)!$
\end{itemize}
\end{frame}

\section{Kombinasi}
\frame{\sectionpage}

\begin{frame}
\frametitle{Permasalahan 1}
\begin{itemize}
  \item Terdapat 5 siswa (sebut saja A, B, C, D, dan E) dimana akan dipilih 3 siswa untuk mengikuti kompetisi.
  \item Berapa banyak susunan tim berbeda yang dapat dibentuk?
\end{itemize}
\end{frame}

\begin{frame}
\frametitle{Solusi Awal}
\begin{itemize}
  \item Solusi ini berbeda dengan Permutasi, karena susunan ABC dan BAC merupakan susunan yang sama, yaitu 1 tim terdiri dari A, B, dan C.
  \item Apabila kita anggap bahwa mereka merupakan susunan yang berbeda, maka banyak susunan tim adalah $P^{5}_{2} = \frac{120}{2} = 60$.
  \item Untuk setiap susunan yang terdiri dari anggota yang sama akan terhitung 6 susunan berbeda dimana seharusnya hanya dihitung sebagai 1 susunan yang sama.
  \item Contoh : ABC, ACB, BAC, BCA, CAB, CBA merupakan 1 susunan yang sama.
  \item Oleh karena itu dengan prinsip Redundansi, banyaknya susunan tim yang berbeda adalah $\frac{60}{6} = 10$ susunan berbeda.
\end{itemize}
\end{frame}

\begin{frame}
\frametitle{Solusi Secara Umum}
\begin{itemize}
  \item Apabila terdapat N siswa dan akan kita ambil R siswa untuk dibentuk sebagai 1 tim, maka banyak susunan berbeda adalah sebagai berikut :
  \item Apabila kita anggap bahwa mereka merupakan susunan yang berbeda, maka banyak susunan tim adalah $P^{N}_{R}$.
  \item Untuk setiap susunan yang terdiri dari anggota yang sama akan terhitung R! susunan berbeda dimana seharusnya hanya dihitung sebagai 1 susunan yang sama.
  \item Oleh karena itu dengan prinsip Redundansi, banyaknya susunan tim yang berbeda adalah $\frac{P^{N}_{R}}{R!} = \frac{N!}{(N-R)! \times R!}$ susunan berbeda.
  \item Rumus itulah yang kita kenal dengan istilah \newTerm{Kombinasi}.
\end{itemize}
\end{frame}

\begin{frame}
\frametitle{Kombinasi}
\begin{itemize}
  \item Apabila terdapat \foreignTerm{n objek} dan kita akan mengambil \foreignTerm{r objek} dari n objek tersebut dimana $r<n$ dan \foreignTerm{urutan pengambilan tidak diperhitungkan}.
  \item Banyak cara pengambilan yang berbeda adalah Kombinasi r terhadap n : 
  \item $C(n,r) = _{n}C_{r} = C^{n}_{r} = \frac{n!}{(n-r)! \times r!}$.
\end{itemize}
\end{frame}

\begin{frame}
\frametitle{Permasalahan 2}
\begin{itemize}
  \item Pak Dengklek ingin membeli kue pada toko kue, dimana toko tersebut menjual 3 jenis kue, yaitu rasa coklat, strawberry, dan kopi.
  \item Apabila Pak Dengklek ingin membeli 4 buah kue, maka berapa banyak susunan kue berbeda yang Pak Dengklek dapat beli?
\end{itemize}
\end{frame}

\begin{frame}
\frametitle{Analisa Permasalahan}
\begin{itemize}
  \item Dapat kita lihat bahwa jika kita membeli coklat-strawberry dengan strawberry-coklat akan menghasilkan susunan yang sama, dengan kata lain permasalahan ini merupakan permasalahan \foreignTerm{Kombinasi}.
  \item Akan tetapi, kita dapat membeli suatu jenis kue beberapa kali atau bahkan tidak membeli sama sekali.
  \item Apabila kita sederhanakan permasalahan tersebut dalam model matematika, maka permasalahan tersebut menjadi :
  \item Banyaknya kemungkinan nilai A, B, dan C dimana $A + B + C = 4$ dan $A,B,C \geq 0$.
\end{itemize}
\end{frame}

\begin{frame}
\frametitle{Solusi}
\begin{itemize}
  \item Kita dapat membagi 4 kue tersebut menjadi 3 bagian. Untuk mempermudah ilustrasi tersebut, kita gunakan lambang o yang berarti kue, dan $|$ yang berarti pembatas.
  \item Bagian kiri merupakan kue A, bagian tengah merupakan kue B, dan bagian kanan merupakan kue C.
  \item Contoh : $(o|o|oo)$ melambangkan 1 kue A, 1 kue B, dan 2 kue C.
  \item Contoh Lain : $(oo|oo|)$ melambangkan 2 kue A, 2 kue B, dan 0 kue C.
  \item Dengan kata lain, semua susunan yang mungkin adalah $(oooo||)$, $(ooo|o|)$, $(oo|oo|)$, ..., $(||oooo)$ yang tidak lain merupakan $C^{6}_{2} = \frac{6!}{4! \times 2!} = 15$ susunan berbeda.
\end{itemize}
\end{frame}

\begin{frame}
\frametitle{Solusi secara umum}
\begin{itemize}
  \item Berapa banyak susunan nilai berbeda dari $X_{1}, X_{2}, X_{3}, ..., X_{n}$ dimana $X_{1} + X_{2} + X_{3} + ... + X_{n} = r$.
  \item Untuk membagi n kue tersebut menjadi r bagian, maka akan dibutuhkan r-1 buah pembatas, sehingga akan terdapat n+r-1 buah objek, dimana kita akan memilih n-1 objek.
  \item Dengan kata lain, banyaknya susunan nilai yang berbeda adalah $C^{n+r-1}_{n-1}$.
  \item Rumus itulah yang kita kenal dengan istilah \newTerm{Kombinasi dengan Pengulangan}.
\end{itemize}
\end{frame}

\begin{frame}
\frametitle{Kombinasi dengan Pengulangan}
\begin{itemize}
  \item Terdapat n jenis objek dimana kita akan mengambil r objek, dan \foreignTerm{tiap objeknya dapat tidak diambil / diambil lebih dari 1 kali}.
  \item Banyak cara berbeda yang memenuhi syarat tersebut adalah sebagai berikut :
  \item $C^{n+r-1}_{n-1} = C^{n+r-1}_{r} = \frac{(n+r-1)!}{(n-1)! \times r!}$.
  \item Keterangan : dikarenakan setiap objek dapat diambil lebih dari 1 kali, maka r tidak harus kurang dari atau sama dengan n.
\end{itemize}
\end{frame}

\section{Segitiga Pascal}
\frame{\sectionpage}

\begin{frame}
\frametitle{Segitiga Pascal}
\begin{itemize}
  \item Segitiga Pascal merupakan susunan dari Koefisien Binomial dalam bentuk segitiga.
  \item Nilai dari baris ke-n suku ke-r adalah :
  \item $A_{ij} = C^{n}_{r}$.
  \item Contoh Segitiga Pascal :
  \begin{itemize} 
    \item Baris ke-1 : 1
    \item Baris ke-2 : 1 1
    \item Baris ke-3 : 1 2 1
    \item Baris ke-4 : 1 3 3 1
    \item Baris ke-5 : 1 4 6 4 1
  \end{itemize}
\end{itemize}
\end{frame}

\begin{frame}
\frametitle{Analisa}
\begin{itemize}
  \item Diberikan suatu himpunan semesta $S = {X_{1},X_{2},...,X_{n}}$. Berapa banyak cara berbeda apabila kita mengambil r objek dari S?
  \item Kita bagi 2 kasus, yaitu :
  \begin{itemize}
    \item Kasus 1 : Apabila kita menggunakan $X_{n}$, sehingga kita memilih r-1 objek dari himpunan $A = {X_{1},X_{2},X_{3},...,X_{n-1}}$, 1 objek sisanya merupakan $X_{n}$. Banyak cara berbeda dari kasus ini adalah $C^{n-1}_{r-1}$.
    \item Kasus 2 : Apabila kita tidak menggunakan $X_{n}$, sehingga kita memilih r objek dari himpunan $A = {X_{1},X_{2},X_{3},...,X_{n}}$, 1 objek sisanya merupakan $X_{n}$. Banyak cara berbeda dari kasus ini adalah $C^{n-1}_{r}$.
  \end{itemize}
\end{itemize}
\end{frame}

\begin{frame}
\frametitle{Analisa (lanj.)}
\begin{itemize}
  \item Pada dasarnya apabila kita menggunakan rumus kombinasi, maka kita akan memilih r objek dari n objek, sehingga banyak cara berbeda secara keseluruhan adalah $C^{n}_{r}$.
  \item Dengan aturan penjumlahan dari kasus 1 dan kasus 2, kita dapatkan $C^{n}_{r} = C^{n-1}_{r-1} + C^{n-1}_{r}$.
  \item Dengan kata lain, dalam koefisien binomial pada segitiga pascal, akan berlaku :
  \item $A_{(n)(r)} = A_{(n-1)(r-1)} + A_{(n-1)(r)}$.
  \item Persamaan itulah yang sering kita gunakan dalam membuat segitiga pascal.
\end{itemize}
\end{frame}

\begin{frame}
\frametitle{Analisa (lanj.)}
\begin{itemize}
  \item Dalam dunia pemrograman, persamaan tersebut dapat kita gunakan untuk pre-komputasi semua nilai dari $C^{n}_{r}$ yang nantinya akan kita butuhkan.
  \item Untuk mencari nilai dari $C^{n}_{r}$, karena kita perlu mencari nilai faktorial dari n, sehingga kompleksitasinya O(N).
  \item Apabila kita gunakan iterasi untuk mencari nilai dari $C^{n}_{r}$ untuk pre-komputasi semua nilai kombinasi yang mungkin (dimana kurang lebihnya akan ada $N^{2}$ perhitungan), kompleksitasnya akan menjadi $O(N^{3})$.
  \item Dengan menggunakan persamaan $A_{(n)(r)} = A_{(n-1)(r-1)} + A_{(n-1)(r)}$ dimana kompleksitasnya adalah O(1), maka secara keseluruhan kompleksitasnya akan menjadi $O(N^{2})$ menjadikan solusi ini jauh lebih cepat.
\end{itemize}
\end{frame}

\begin{frame}
\frametitle{Implementasi Segitiga Pascal}
\begin{enumerate}
  \item Kita gunakan iterasi untuk mencari semua nilai dari kombinasi r terhadap n.
  \item Untuk menyimpan semua nilai dari kombinasi tersebut akan dibutuhkan array 2-dimensi dimana A[n][r] menyimpan nilai kombinasi r terhadap n.
  \item Nilai dari A[0][0] adalah 1.
  \item Iterasi i dari 1 hingga n
  \begin{enumerate}
    \item Iterasi j dari 0 hingga n
    \begin{enumerate}
      \item Jika $j = 0$ atau $j = n$, maka $A[i][j] = 1$
      \item Selain itu, $A[i][j] = A[i-1][j-1] + A[i-1][j]$
    \end{enumerate}
  \end{enumerate}
\end{enumerate}
\end{frame}

\begin{frame}[fragile]
\frametitle{Implementasi Segitiga Pascal (lanj.)}
\begin{codebox}
\Procname{$\proc{segitiga\_pascal}(baris)$}
\li \Comment Sediakan array 2-dimensi $C$ berukuran $baris \times baris$
\li \Comment Perhatikan bahwa index array $C$ pada contoh ini dimulai dari 0.
\li $C[0][0] \gets 1$
\li \For $i \gets 1$ \To $baris$
    \Do
\li   \For $j \gets 0$ \To $baris$
      \Do
\li     \If $j = 0$ $or$ $j = baris$
        \Then
\li       $C[i][j] \gets 1$
\li     \Else
\li       $C[n][r] = C[n-1][r-1] + C[n-1][r]$
        \End
      \End
    \End
\end{codebox}
\end{frame}

\begin{frame}
\frametitle{Penggunaan Segitiga Pascal}
\begin{itemize}
  \item Dalam bidang matematika, segitiga pascal merupakan kumpulan dari koefisien binomial yang dapat digunakan dalam Binomial Newton $(x+y)^{n} = \sum\limits_{r=0}^{n} a_{r} x^{n-r} y^{r}$ dimana $a_{r}$ merupakan bilangan dalam segitiga pascal baris ke-n suku ke-r.
  \item Dalam bidang programming, algoritma dari segitiga pascal dapat digunakan untuk mencari semua nilai dari kombinasi r terhadap n dengan kompleksitas waktu $O(N^{2})$ dan memori yang dibutuhkan adalah $O(N^{2})$.
\end{itemize}
\end{frame}

\begin{frame}
\frametitle{Penutup}
\begin{itemize}
  \item Materi ini berisi mengenai Kombinatorik Dasar dan implementasinya yang umum digunakan dalam pemrograman kompetitif.
  \item Dengan materi ini, Anda diharapkan dapat menggunakan konsep kombinatorik yang ada.
  \item Selamat berlatih dan semoga sukses!
\end{itemize}
\end{frame}

\end{document}
