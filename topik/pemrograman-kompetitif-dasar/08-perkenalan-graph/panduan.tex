\documentclass{beamer}
\usetheme{tokitex}

\usepackage{tikz}
\usepackage{graphics}
\usepackage{multirow}
\usepackage{tabto}
\usepackage{xspace}
\usepackage{amsmath}
\usepackage{hyperref}

\usepackage{tikz}
\usepackage{clrscode3e}

\usepackage[english,bahasa]{babel}
\newtranslation[to=bahasa]{Section}{Bagian}
\newtranslation[to=bahasa]{Subsection}{Subbagian}

\usepackage{listings, lstautogobble}
\usepackage{color}

\definecolor{dkgreen}{rgb}{0,0.6,0}
\definecolor{gray}{rgb}{0.5,0.5,0.5}
\definecolor{mauve}{rgb}{0.58,0,0.82}

\lstset{frame=tb,
  language=pascal,
  aboveskip=1mm,
  belowskip=1mm,
  showstringspaces=false,
  columns=fullflexible,
  keepspaces=true,
  basicstyle={\small\ttfamily},
  numbers=none,
  numberstyle=\tiny\color{gray},
  keywordstyle=\color{blue},
  commentstyle=\color{dkgreen},
  stringstyle=\color{mauve},
  breaklines=true,
  breakatwhitespace=true,
  autogobble=true
}

\usepackage{caption}
\captionsetup[figure]{labelformat=empty}

\newcommand{\progTerm}[1]{\textbf{#1}}
\newcommand{\foreignTerm}[1]{\textit{#1}}
\newcommand{\newTerm}[1]{\alert{\textbf{#1}}}
\newcommand{\emp}[1]{\alert{#1}}
\newcommand{\statement}[1]{"#1"}

% Getting tired of writing \foreignTerm all the time
\newcommand{\farray}{\foreignTerm{array}\xspace}
\newcommand{\fArray}{\foreignTerm{Array}\xspace}
\newcommand{\foverhead}{\foreignTerm{overhead}\xspace}
\newcommand{\fOverhead}{\foreignTerm{Overhead}\xspace}
\newcommand{\fsubarray}{\foreignTerm{subarray}\xspace}
\newcommand{\fSubarray}{\foreignTerm{Subarray}\xspace}
\newcommand{\fbasecase}{\foreignTerm{base case}\xspace}
\newcommand{\fBasecase}{\foreignTerm{Base case}\xspace}
\newcommand{\ftopdown}{\foreignTerm{top down}\xspace}
\newcommand{\fTopdown}{\foreignTerm{Top down}\xspace}
\newcommand{\fbottomup}{\foreignTerm{bottom up}\xspace}
\newcommand{\fBottomup}{\foreignTerm{Bottom up}\xspace}
\newcommand{\fpruning}{\foreignTerm{pruning}\xspace}
\newcommand{\fPruning}{\foreignTerm{Pruning}\xspace}

\newcommand{\fgraph}{\foreignTerm{graph}\xspace}
\newcommand{\fGraph}{\foreignTerm{Graph}\xspace}
\newcommand{\fnode}{\foreignTerm{node}\xspace}
\newcommand{\fNode}{\foreignTerm{Node}\xspace}
\newcommand{\fedge}{\foreignTerm{edge}\xspace}
\newcommand{\fEdge}{\foreignTerm{Edge}\xspace}
\newcommand{\fdegree}{\foreignTerm{degree}\xspace}
\newcommand{\fDegree}{\foreignTerm{Degree}\xspace}
\newcommand{\fadjacencylist}{\foreignTerm{adjacency list}\xspace}
\newcommand{\fAdjacencylist}{\foreignTerm{Adjacency list}\xspace}
\newcommand{\fadjacencymatrix}{\foreignTerm{adjacency matrix}\xspace}
\newcommand{\fAdjacencymatrix}{\foreignTerm{Adjacency matrix}\xspace}
\newcommand{\fedgelist}{\foreignTerm{edge list}\xspace}
\newcommand{\fEdgelist}{\foreignTerm{Edge list}\xspace}
\newcommand{\flist}{\foreignTerm{list}\xspace}
\newcommand{\fList}{\foreignTerm{List}\xspace}
\newcommand{\fgraphtraversal}{\foreignTerm{graph traversal}\xspace}
\newcommand{\fGraphtraversal}{\foreignTerm{Graph traversal}\xspace}
\newcommand{\ftree}{\foreignTerm{tree}\xspace}
\newcommand{\fTree}{\foreignTerm{Tree}\xspace}
\newcommand{\fsubtree}{\foreignTerm{subtree}\xspace}
\newcommand{\fSubtree}{\foreignTerm{Subtree}\xspace}

\newcommand{\fDivideAndConquer}{\foreignTerm{Divide and conquer}\xspace}
\newcommand{\fdivideAndConquer}{\foreignTerm{divide and conquer}\xspace}
\newcommand{\fMergeSort}{\foreignTerm{Merge sort}\xspace}
\newcommand{\fmergeSort}{\foreignTerm{merge sort}\xspace}
\newcommand{\fQuickSort}{\foreignTerm{Quicksort}\xspace}
\newcommand{\fquickSort}{\foreignTerm{quicksort}\xspace}
\newcommand{\fpivot}{\foreignTerm{pivot}\xspace}
\newcommand{\fPivot}{\foreignTerm{Pivot}\xspace}
\newcommand{\fbruteForce}{\foreignTerm{brute force}\xspace}
\newcommand{\fBruteForce}{\foreignTerm{Brute force}\xspace}
\newcommand{\fCompleteSearch}{\foreignTerm{complete search}\xspace}
\newcommand{\fExhaustiveSearch}{\foreignTerm{exhaustive search}\xspace}
\newcommand{\fBinarySearch}{\foreignTerm{binary search}\xspace}
\newcommand{\fGreedy}{\foreignTerm{greedy}\xspace}
\newcommand{\fGreedyChoice}{\foreignTerm{greedy choice}\xspace}

\newcommand{\pheap}{\foreignTerm{heap}\xspace}
\newcommand{\pHeap}{\foreignTerm{Heap}\xspace}
\newcommand{\pBinaryHeap}{\foreignTerm{Binary Heap}\xspace}
\newcommand{\pbinaryHeap}{\foreignTerm{binary heap}\xspace}
\newcommand{\pHeapSort}{\foreignTerm{Heap Sort}\xspace}
\newcommand{\pdjs}{\foreignTerm{disjoint set}\xspace}
\newcommand{\pDjs}{\foreignTerm{Disjoint set}\xspace}

\title{Perkenalan Graph}
\author{Tim Olimpiade Komputer Indonesia}
\date{}

\begin{document}



\begin{frame}
\titlepage
\end{frame}

\begin{frame}
\frametitle{Panduan Konten}
\begin{itemize}
  \item Jelaskan konsep graph, terutama terminologi seperti apa itu node/vertex, edge, indegree, outdegree, degree, cycle
  \item Jelaskan jenis2 graph (unweighted/weighted, undirected/directed)
  \item Jelaskan representasi graph pada pemrograman: adjacency matrix, adjacency list, edge list
  \item Untuk setiap representasi, jelaskan keuntungan dan kerugiannya. Sekalian kompleksitas untuk operasi cek ketetanggaan, tambah edge, buang edge, dapetin daftar tetangga, konsumsi memori, dsb
\end{itemize}
\end{frame}

\begin{frame}
\frametitle{Panduan Konten}
\begin{itemize}
  \item Bahas graph traversal (penelusuran graph): BFS dan DFS
  \item Disarankan untuk mulai dengan persoalan (misalnya diberikan grid maze, cari shortest path dari pintu masuk ke pintu keluar)
  \item Jelaskan bahwa BFS selalu menjamin didapatkan shortest path
  \item Jelaskan kompleksitas dan konsumsi memori dari masing2 pendekatan
\end{itemize}
\end{frame}

\begin{frame}
\frametitle{Panduan Konten}
\begin{itemize}
  \item Tuliskan secara singkat tentang graph khusus: tree, directed acyclic graph, bipartite graph
  \item Boleh juga ditambahkan jenis directed graph yang setiap nodenya punya tepat 1 outdegree, yang pada akhirnya selalu memiliki cycle maksimal O(V) (V = vertex)
\end{itemize}
\end{frame}

\begin{frame}
\frametitle{Tips}
Demi keseragaman dan kemudahan, gunakan command yang sudah didefinisikan di ../config.tex:
% Komentar: meskipun ada yang menghasilkan format yang sama, tetapi menggunakan command ini mempermudah kalau ke depannya mau mengubah format, tidak perlu find and replace 1 per 1. 
\begin{itemize}
  \item Gunakan \textbackslash id\{\} untuk menulis identifier (nama fungsi, variabel), contoh: \textbackslash id\{primeCount\} akan mencetak \id{primeCount}.
  \item Gunakan \textbackslash progTerm\{\} untuk menulis istilah dalam pemrograman (tipe data, jenis error, for), contoh: \textbackslash progTerm\{integer\} akan mencetak \progTerm{integer}.
  \item Gunakan \textbackslash foreignTerm\{\} untuk menulis istilah asing lainnya, contoh: \textbackslash foreignTerm\{default\} akan mencetak \foreignTerm{default}.
\end{itemize}
\end{frame}

\begin{frame}
\frametitle{Tips (lanj.)}
\begin{itemize}
  \item Jika diperlukan, gunakan \textbackslash newTerm\{\} untuk menulis istilah yang baru diperkanalkan, contoh: \textbackslash newTerm\{rekursi\} akan mencetak \newTerm{rekursi}.
  \item Gunakan \textbackslash emp\{\} untuk memberi penekanan, contoh: \textbackslash emp\{sangat lambat\} akan mencetak \emp{sangat lambat}.
  \item Gunakan \textbackslash statement\{\} untuk merujuk pada statement program, contoh: \textbackslash statement\{count := count + 1\} akan mencetak \statement{count := count + 1}.
\end{itemize}
\end{frame}

\end{document}