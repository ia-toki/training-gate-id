\documentclass{beamer}
\usetheme{tokitex}

\usepackage{tikz}
\usepackage{graphics}
\usepackage{multirow}
\usepackage{tabto}
\usepackage{xspace}
\usepackage{amsmath}
\usepackage{hyperref}
\usepackage{wrapfig}
\usepackage{mathtools}

\usepackage{tikz}
\usepackage{clrscode3e}
\usepackage{gensymb}

\usepackage[english,bahasa]{babel}
\newtranslation[to=bahasa]{Section}{Bagian}
\newtranslation[to=bahasa]{Subsection}{Subbagian}

\usepackage{listings, lstautogobble}
\usepackage{color}

\definecolor{dkgreen}{rgb}{0,0.6,0}
\definecolor{gray}{rgb}{0.5,0.5,0.5}
\definecolor{mauve}{rgb}{0.58,0,0.82}

\lstset{frame=tb,
  language=c++,
  aboveskip=0mm,
  belowskip=0mm,
  showstringspaces=false,
  columns=fullflexible,
  keepspaces=true,
  basicstyle={\small\ttfamily},
  numbers=none,
  numberstyle=\tiny\color{gray},
  keywordstyle=\color{blue},
  commentstyle=\color{dkgreen},
  stringstyle=\color{mauve},
  breaklines=true,
  breakatwhitespace=true,
  lineskip={-3pt}
}

\usepackage{caption}
\captionsetup[figure]{labelformat=empty}

\newcommand{\progTerm}[1]{\textbf{#1}}
\newcommand{\foreignTerm}[1]{\textit{#1}}
\newcommand{\newTerm}[1]{\alert{\textbf{#1}}}
\newcommand{\emp}[1]{\alert{#1}}
\newcommand{\statement}[1]{"#1"}

\newcommand{\floor}[1]{\lfloor #1 \rfloor}
\newcommand{\ceil}[1]{\lceil #1 \rceil}
\newcommand{\abs}[1]{\left\lvert#1\right\rvert}
\newcommand{\norm}[1]{\left\lVert#1\right\rVert}

% Getting tired of writing \foreignTerm all the time
\newcommand{\farray}{\foreignTerm{array}\xspace}
\newcommand{\fArray}{\foreignTerm{Array}\xspace}
\newcommand{\foverhead}{\foreignTerm{overhead}\xspace}
\newcommand{\fOverhead}{\foreignTerm{Overhead}\xspace}
\newcommand{\fsubarray}{\foreignTerm{subarray}\xspace}
\newcommand{\fSubarray}{\foreignTerm{Subarray}\xspace}
\newcommand{\fbasecase}{\foreignTerm{base case}\xspace}
\newcommand{\fBasecase}{\foreignTerm{Base case}\xspace}
\newcommand{\ftopdown}{\foreignTerm{top-down}\xspace}
\newcommand{\fTopdown}{\foreignTerm{Top-down}\xspace}
\newcommand{\fbottomup}{\foreignTerm{bottom-up}\xspace}
\newcommand{\fBottomup}{\foreignTerm{Bottom-up}\xspace}
\newcommand{\fpruning}{\foreignTerm{pruning}\xspace}
\newcommand{\fPruning}{\foreignTerm{Pruning}\xspace}

\newcommand{\fgraph}{\foreignTerm{graph}\xspace}
\newcommand{\fGraph}{\foreignTerm{Graph}\xspace}
\newcommand{\froot}{\foreignTerm{root}\xspace}
\newcommand{\fRoot}{\foreignTerm{Root}\xspace}
\newcommand{\fnode}{\foreignTerm{node}\xspace}
\newcommand{\fNode}{\foreignTerm{Node}\xspace}
\newcommand{\fedge}{\foreignTerm{edge}\xspace}
\newcommand{\fEdge}{\foreignTerm{Edge}\xspace}
\newcommand{\fcycle}{\foreignTerm{cycle}\xspace}
\newcommand{\fCycle}{\foreignTerm{Cycle}\xspace}
\newcommand{\fdegree}{\foreignTerm{degree}\xspace}
\newcommand{\fDegree}{\foreignTerm{Degree}\xspace}
\newcommand{\fadjacencylist}{\foreignTerm{adjacency list}\xspace}
\newcommand{\fAdjacencylist}{\foreignTerm{Adjacency list}\xspace}
\newcommand{\fadjacencymatrix}{\foreignTerm{adjacency matrix}\xspace}
\newcommand{\fAdjacencymatrix}{\foreignTerm{Adjacency matrix}\xspace}
\newcommand{\fedgelist}{\foreignTerm{edge list}\xspace}
\newcommand{\fEdgelist}{\foreignTerm{Edge list}\xspace}
\newcommand{\flist}{\foreignTerm{list}\xspace}
\newcommand{\fList}{\foreignTerm{List}\xspace}
\newcommand{\fgraphtraversal}{\foreignTerm{graph traversal}\xspace}
\newcommand{\fGraphtraversal}{\foreignTerm{Graph traversal}\xspace}
\newcommand{\ftree}{\foreignTerm{tree}\xspace}
\newcommand{\fTree}{\foreignTerm{Tree}\xspace}
\newcommand{\fsubtree}{\foreignTerm{subtree}\xspace}
\newcommand{\fSubtree}{\foreignTerm{Subtree}\xspace}
\newcommand{\fparent}{\foreignTerm{parent}\xspace}
\newcommand{\fParent}{\foreignTerm{Parent}\xspace}
\newcommand{\fsibling}{\foreignTerm{sibling}\xspace}
\newcommand{\fSibling}{\foreignTerm{Sibling}\xspace}
\newcommand{\fpath}{\foreignTerm{path}\xspace}
\newcommand{\fPath}{\foreignTerm{Path}\xspace}
\newcommand{\fconnectedcomponent}{\foreignTerm{connected component}\xspace}
\newcommand{\fConnectedcomponent}{\foreignTerm{Connected component}\xspace}
\newcommand{\fbridge}{\foreignTerm{bridge}\xspace}
\newcommand{\fBridge}{\foreignTerm{Bridge}\xspace}
\newcommand{\farticulationpoint}{\foreignTerm{articulation point}\xspace}
\newcommand{\fArticulationpoint}{\foreignTerm{Articulation point}\xspace}
\newcommand{\ftreeedge}{\foreignTerm{tree edge}\xspace}
\newcommand{\fTreeedge}{\foreignTerm{Tree edge}\xspace}
\newcommand{\fbackedge}{\foreignTerm{back edge}\xspace}
\newcommand{\fBackedge}{\foreignTerm{Back edge}\xspace}
\newcommand{\fforwardedge}{\foreignTerm{forward edge}\xspace}
\newcommand{\fForwardedge}{\foreignTerm{Forward edge}\xspace}
\newcommand{\fcrossedge}{\foreignTerm{cross edge}\xspace}
\newcommand{\fCrossedge}{\foreignTerm{Cross edge}\xspace}
\newcommand{\fdiscoverytime}{\foreignTerm{discovery time}\xspace}
\newcommand{\fDiscoverytime}{\foreignTerm{Discovery time}\xspace}
\newcommand{\flowlink}{\foreignTerm{low link}\xspace}
\newcommand{\fLowlink}{\foreignTerm{Low link}\xspace}
\newcommand{\fstack}{\foreignTerm{stack}\xspace}
\newcommand{\fStack}{\foreignTerm{Stack}\xspace}
\newcommand{\for}{\foreignTerm{or}\xspace}
\newcommand{\fOr}{\foreignTerm{Or}\xspace}
\newcommand{\fand}{\foreignTerm{and}\xspace}
\newcommand{\fAnd}{\foreignTerm{And}\xspace}
\newcommand{\fcentroid}{\foreignTerm{centroid}\xspace}
\newcommand{\fCentroid}{\foreignTerm{Centroid}\xspace}

\newcommand{\fDivideAndConquer}{\foreignTerm{Divide and conquer}\xspace}
\newcommand{\fdivideAndConquer}{\foreignTerm{divide and conquer}\xspace}
\newcommand{\fMergeSort}{\foreignTerm{Merge sort}\xspace}
\newcommand{\fmergeSort}{\foreignTerm{merge sort}\xspace}
\newcommand{\fQuickSort}{\foreignTerm{Quicksort}\xspace}
\newcommand{\fquickSort}{\foreignTerm{quicksort}\xspace}
\newcommand{\fpivot}{\foreignTerm{pivot}\xspace}
\newcommand{\fPivot}{\foreignTerm{Pivot}\xspace}
\newcommand{\fbruteForce}{\foreignTerm{brute force}\xspace}
\newcommand{\fBruteForce}{\foreignTerm{Brute force}\xspace}
\newcommand{\fCompleteSearch}{\foreignTerm{complete search}\xspace}
\newcommand{\fExhaustiveSearch}{\foreignTerm{exhaustive search}\xspace}
\newcommand{\fbinarySearch}{\foreignTerm{binary search}\xspace}
\newcommand{\fBinarySearch}{\foreignTerm{Binary search}\xspace}
\newcommand{\fternarySearch}{\foreignTerm{ternary search}\xspace}
\newcommand{\fTernarySearch}{\foreignTerm{Ternary search}\xspace}
\newcommand{\funimodal}{\foreignTerm{unimodal}\xspace}
\newcommand{\fUnimodal}{\foreignTerm{Unimodal}\xspace}
\newcommand{\fGreedy}{\foreignTerm{Greedy}\xspace}
\newcommand{\fgreedy}{\foreignTerm{greedy}\xspace}
\newcommand{\fgreedyChoice}{\foreignTerm{greedy choice}\xspace}
\newcommand{\fGreedyChoice}{\foreignTerm{Greedy choice}\xspace}

\newcommand{\fdp}{\foreignTerm{dynamic programming}\xspace}
\newcommand{\fDp}{\foreignTerm{Dynamic programming}\xspace}
\newcommand{\fbitmask}{\foreignTerm{bitmask}\xspace}
\newcommand{\fBitmask}{\foreignTerm{Bitmask}\xspace}
\newcommand{\fstate}{\foreignTerm{state}\xspace}
\newcommand{\fState}{\foreignTerm{State}\xspace}
\newcommand{\fsubmask}{\foreignTerm{submask}\xspace}
\newcommand{\fSubmask}{\foreignTerm{Submask}\xspace}

\newcommand{\pheap}{\foreignTerm{heap}\xspace}
\newcommand{\pHeap}{\foreignTerm{Heap}\xspace}
\newcommand{\pBinaryHeap}{\foreignTerm{Binary Heap}\xspace}
\newcommand{\pbinaryHeap}{\foreignTerm{binary heap}\xspace}
\newcommand{\pHeapsort}{\foreignTerm{Heapsort}\xspace}
\newcommand{\pheapsort}{\foreignTerm{heapsort}\xspace}
\newcommand{\pdjs}{\foreignTerm{disjoint set}\xspace}
\newcommand{\pDjs}{\foreignTerm{Disjoint set}\xspace}

\newcommand{\fdotProduct}{\foreignTerm{dot product}\xspace}
\newcommand{\fDotProduct}{\foreignTerm{Dot product}\xspace}
\newcommand{\fcrossProduct}{\foreignTerm{cross product}\xspace}
\newcommand{\fCrossProduct}{\foreignTerm{Cross product}\xspace}
\newcommand{\fconvexHull}{\foreignTerm{convex hull}\xspace}
\newcommand{\fConvexHull}{\foreignTerm{Convex hull}\xspace}
\newcommand{\fgrahamScan}{\foreignTerm{graham scan}\xspace}
\newcommand{\fGrahamScan}{\foreignTerm{Graham scan}\xspace}
\newcommand{\flineSweep}{\foreignTerm{line sweep}\xspace}
\newcommand{\fLineSweep}{\foreignTerm{Line sweep}\xspace}

\newcommand{\fset}{\foreignTerm{set}\xspace}
\newcommand{\fSet}{\foreignTerm{Set}\xspace}
\newcommand{\fprefixSum}{\foreignTerm{prefix sum}\xspace}
\newcommand{\fPrefixSum}{\foreignTerm{Prefix sum}\xspace}
\newcommand{\ffenwickTree}{\foreignTerm{fenwick tree}\xspace}
\newcommand{\fFenwickTree}{\foreignTerm{Fenwick tree}\xspace}
\newcommand{\frangeSumQuery}{\foreignTerm{range sum query}\xspace}
\newcommand{\fRangeSumQuery}{\foreignTerm{Range sum query}\xspace}
\newcommand{\fquery}{\foreignTerm{query}\xspace}
\newcommand{\fQuery}{\foreignTerm{Query}\xspace}
\newcommand{\fsegmentTree}{\foreignTerm{segment tree}\xspace}
\newcommand{\fSegmentTree}{\foreignTerm{Segment tree}\xspace}
\newcommand{\fbinaryTree}{\foreignTerm{binary tree}\xspace}
\newcommand{\fBinaryTree}{\foreignTerm{Binary tree}\xspace}
\newcommand{\flazyPropagation}{\foreignTerm{lazy propagation}\xspace}
\newcommand{\fLazyPropagation}{\foreignTerm{Lazy propagation}\xspace}
\newcommand{\fsparseTable}{\foreignTerm{sparse table}\xspace}
\newcommand{\fSparseTable}{\foreignTerm{Sparse table}\xspace}

\newcommand{\ftrail}{\foreignTerm{trail}\xspace}
\newcommand{\fTrail}{\foreignTerm{Trail}\xspace}
\newcommand{\feulerTour}{\foreignTerm{euler tour}\xspace}
\newcommand{\fEulerTour}{\foreignTerm{Euler tour}\xspace}
\newcommand{\feulerTourTree}{\foreignTerm{euler tour tree}\xspace}
\newcommand{\fEulerTourTree}{\foreignTerm{Euler tour tree}\xspace}

\newcommand{\fmaxflow}{\foreignTerm{maximum flow}\xspace}
\newcommand{\fMaxflow}{\foreignTerm{Maximum flow}\xspace}
\newcommand{\fmincut}{\foreignTerm{minimum cut}\xspace}
\newcommand{\fMincut}{\foreignTerm{Minimum cut}\xspace}
\newcommand{\fflow}{\foreignTerm{flow}\xspace}
\newcommand{\fFlow}{\foreignTerm{Flow}\xspace}
\newcommand{\fsource}{\foreignTerm{source}\xspace}
\newcommand{\fSource}{\foreignTerm{Source}\xspace}
\newcommand{\fsink}{\foreignTerm{sink}\xspace}
\newcommand{\fSink}{\foreignTerm{Sink}\xspace}
\newcommand{\fbackEdge}{\foreignTerm{back-edge}\xspace}
\newcommand{\fBackEdge}{\foreignTerm{Back-edge}\xspace}
\newcommand{\fresidualCapacity}{\foreignTerm{residual capacity}\xspace}
\newcommand{\fResidualCapacity}{\foreignTerm{Residual capacity}\xspace}
\newcommand{\fbottleneck}{\foreignTerm{bottleneck}\xspace}
\newcommand{\fBottleneck}{\foreignTerm{Bottleneck}\xspace}
\newcommand{\faugmentingPath}{\foreignTerm{augmenting path}\xspace}
\newcommand{\fAugmentingPath}{\foreignTerm{Augmenting path}\xspace}


\title{Variabel dan Tipe Data}
\author{Tim Olimpiade Komputer Indonesia}
\date{}

\begin{document}

\begin{frame}
\titlepage
\end{frame}

\begin{frame}
\frametitle{Pendahuluan}
Melalui dokumen ini, kalian akan:
\begin{itemize}
  \item Mengenal konsep variabel.
  \item Mempelajari berbagai tipe data.
  \item Mempelajari cara deklarasi variabel.
  \item Mengenal operasi assignment.
\end{itemize}
\end{frame}

\begin{frame}[fragile]
\frametitle{Kilas Balik}
\begin{itemize}
  \item Mari kita lihat kembali program halo.cpp.
  \begin{lstlisting}
#include <cstdio>

int main() {
  printf("halo dunia\n");
}\end{lstlisting}
  \item Pada program tersebut, terdapat kata kunci "\texttt{int main() \{}" dan "\texttt{\}}".
  \item Kedua kata kunci tersebut blok program utama.
  \item Ketika halo.cpp dieksekusi, seluruh perintah di blok program utama akan dieksekusi secara berurutan.
\end{itemize}
\end{frame}

\begin{frame}
\frametitle{Baris Perintah Program}
\begin{itemize}
  \item Pada halo.cpp, satu-satunya perintah yang ada adalah \texttt{printf("halo dunia\textbackslash n");}
  \item Pada C++, \texttt{printf(x)} merupakan fungsi untuk mencetak \texttt{x} ke layar. 
  \item Dalam program ini, \texttt{x} = 'halo dunia\textbackslash n'.
  \item '\textbackslash n' merupakan karakter "baris baru" atau "enter".
\end{itemize}
\end{frame}

\section{Konsep Variabel}
\frame{\sectionpage}

\begin{frame}
\frametitle{Perkenalan Variabel}
\begin{block}{Variabel}
   Merupakan istilah yang diadopsi dari dunia matematika, yang memetakan sebuah nama ke suatu nilai.
\end{block}
\end{frame}

\begin{frame}
\frametitle{Perkenalan Variabel (lanj.)}
\begin{itemize}
  \item Setiap kali suatu variabel digunakan dalam ekspresi matematika, yang diacu sebenarnya adalah nilai yang dipetakan oleh nama variabel tersebut.
  \item Contoh: jika kita menyatakan $x=5$, maka hasil dari $3x^2 + x$ adalah $80$.
  \item Dalam pemrograman, kita bisa membuat variabel, mengisikan nilai pada variabel, dan mengacu nilai yang dipetakan variabel tersebut.
\end{itemize}
\end{frame}

\begin{frame}
\frametitle{Aturan Penamaan Variabel}
\begin{itemize}
  \item Variabel bebas diberi nama apapun, tetapi terbatas pada beberapa aturan berikut:
  \begin{itemize}
    \item Terdiri dari kombinasi karakter huruf, angka, dan underscore (\_).
    \item Tidak boleh dimulai dengan angka.
    \item Huruf kapital dan huruf kecil dianggap berbeda. Artinya "a1" dan "A1" dianggap merupakan dua variabel yang berbeda.
    \item Tidak boleh merupakan \alert{\textit{reserved word}}. Contoh \textit{reserved word} pada C++: \textbf{int}, \textbf{if}, \textbf{while}, \textbf{for}, atau \textbf{switch}.
  \end{itemize}
  \item Contoh penulisan variabel yang tepat: \texttt{nilai}, \texttt{xKecil}, \texttt{y1}, \texttt{tambahan\_string}.
  \item Contoh penulisan variabel yang salah: \texttt{2kar}, \texttt{wow!?}, \texttt{while}.
\end{itemize}
\end{frame}

\begin{frame}
\frametitle{Aturan Penamaan Variabel (lanj.)}
\begin{itemize}
  \item Lebih jauh lagi, aturan ini berlaku pada seluruh penamaan \alert{\textit{identifier}}, yaitu nama variabel dan fungsi yang akan dipelajari selanjutnya.
\end{itemize}
\end{frame}

\begin{frame}
\frametitle{Assignment}
\begin{block}{Assignment}
  Pengisian nilai yang diacu oleh variabel dengan suatu nilai disebut \alert{\textit{assignment}.}
\end{block}
\begin{itemize}
  \item Operator untuk \textit{assignment} adalah =
  \item Isikan ruas kiri dengan nama suatu variabel, dan ruas kanan dengan nilai yang ingin diisikan ke variabel tersebut.
  \item Tipe data dari variabel dan nilai yang diacu \alert{harus sesuai}.
\end{itemize}
\end{frame}


\begin{frame}[fragile]
\frametitle{Contoh Program: assign.cpp}
\begin{itemize}
  \item Perhatikan contoh program assign.cpp berikut. Tuliskan, lalu jalankan program ini.
  \begin{lstlisting}
#include <cstdio>

int x;

int main() {
  x = 12;
  printf("Nilai = %d\n", x);
}\end{lstlisting}
\end{itemize}
\end{frame}

\begin{frame}[fragile]
\frametitle{Penjelasan Program: assign.cpp}
\begin{itemize}
  \item Keluaran yang dihasilkan dari program itu adalah sebuah baris berisikan:
\begin{lstlisting}
Nilai = 12
\end{lstlisting}
  \item Pada program tersebut, \texttt{x} merupakan suatu variabel.
  \item Variabel \texttt{x} didaftarkan terlebih dahulu dengan menuliskan \texttt{int x} di luar blok program utama.
  \item Pada blok program utama, \texttt{x} diisi dengan nilai 12, lalu perintah \texttt{printf} dieksekusi.
\end{itemize}
\end{frame}

\begin{frame}[fragile]
\frametitle{Sekilas Tentang printf}
\begin{itemize}
  \item Untuk pencetakan, digunakan perintah berikut:
\begin{lstlisting}
printf("Nilai = %d\n", x);
\end{lstlisting}
  \item Untuk mencetak nilai dari variabel, diperlukan simbol sementara yang akan digantikan dengan nilai variabel.
  \item Simbol sementara untuk variabel bertipe bilangan bulat seperti \texttt{x} adalah "\%d".
  \item Variabel-variabel untuk menggantikan simbol sementara perlu dituliskan sesudah pola cetakan.
\end{itemize}
\end{frame}

\begin{frame}[fragile]
\frametitle{Contoh Program: assign2.cpp}
\begin{itemize}
  \item Berikut adalah contoh program yang melibatkan beberapa variabel.
\begin{lstlisting}
#include <cstdio>

int x;
int y;

int main() {
  x = 12;
  y = 123456;
  printf("Nilai x = %d\n", x);
  printf("Nilai y = %d\n", y);

  x = 15;
  printf("Sekarang nilai x = %d\n", x);
}
\end{lstlisting}
\end{itemize}
\end{frame}

\begin{frame}[fragile]
\frametitle{Penjelasan Program: assign2.cpp}
\begin{itemize}
  \item Keluaran yang dihasilkan dari program itu adalah:
  \begin{lstlisting}
    Nilai x = 12
    Nilai y = 123456
    Sekarang nilai x = 15
  \end{lstlisting}
  \item Apa maksud dari kata kunci \textbf{int}? Dijelaskan pada bagian selanjutnya.
\end{itemize}
\end{frame}

\section{Tipe Data Variabel}
\frame{\sectionpage}

\begin{frame}
\frametitle{Tipe Data Variabel}
\begin{itemize}
  \item Setiap variabel pada C++ memiliki \alert{tipe data}.
  \item Jenis tipe data dasar dari suatu variabel pada:
  \begin{itemize}
    \item Bilangan bulat.
    \item Bilangan riil (bilangan bulat dan pecahan).
    \item Karakter (merepresentasikan karakter, seperti 'a', 'b', '3', atau '?').
    \item Nilai kebenaran, yaitu benar (\textbf{TRUE}) atau salah (\textbf{FALSE}).
  \end{itemize}
\end{itemize}
\end{frame}

\begin{frame}
\frametitle{Tipe Data: Bilangan Bulat}
\begin{table}[ht]
  \begin{tabular}{|c|c|c|}
    \hline Nama  & Jangkauan  & Ukuran \\
    \hline short & $-2^{15} .. 2^{15}-1$ & 2 byte\\
    \hline unsigned short & $0 .. 2^{16}-1$ & 2 byte\\
    \hline int & $-2^{31} .. 2^{31}-1$ & 4 byte\\
    \hline unsigned int & $0 .. 2^{32}-1$ & 4 byte\\
    \hline long long & $-2^{63} .. 2^{63}-1$ & 8 byte\\
    \hline unsigned long long & $0 .. 2^{64}-1$ & 8 byte\\
    \hline
  \end{tabular}
\end{table}
\begin{itemize}
  \item C++ menawarkan beberapa tipe data bilangan bulat yang variasinya terletak pada jangkauan nilai yang bisa direpresentasikan dan ukurannya pada memori.
  \item Dalam memprogram, yang umum digunakan adalah \alert{\textbf{int}} dan \alert{\textbf{long long}}.
\end{itemize}
\end{frame}

\begin{frame}
\frametitle{Tipe Data: Bilangan Riil}
\begin{table}[ht]
  \begin{tabular}{|c|c|c|c|}
    \hline Nama  & Jangkauan (magnitudo) & Akurasi & Ukuran \\
    \hline float & $1.5\times10^{-45} .. 3.4\times10^{38}$ & 7-8 digit & 4 byte\\
    \hline double & $5.0\times10^{-324} .. 1.7\times10^{308}$ & 15-16 digit & 8 byte \\
    \hline
  \end{tabular}
\end{table}
\begin{itemize}
  \item Biasa disebut dengan \textit{floating point}.
  \item Tipe data \textit{floating point} bisa merepresentasikan negatif atau positif dari magnitudonya.
  \item Pada pemrograman, umumnya tipe data \textit{floating point} dihindari karena kurang akurat. Representasi 3 pada \textit{floating point} bisa jadi 2.99999999999999 atau 3.000000000000001 karena keterbatasan pada struktur penyimpanan bilangan pecahan pada komputer.
  \item Tipe yang umum digunakan adalah \alert{\textbf{double}}.
\end{itemize}
\end{frame}

\begin{frame}
\frametitle{Tipe Data: Karakter}
\begin{itemize}
  \item Merupakan tipe data untuk merepresentasikan karakter menurut ASCII (\textit{American Standart Code for Information Interchange}).
  \item Dalam ASCII, terdapat 128 karakter yang direpresentasikan dengan angka dari 0 sampai 127.
  \item Misalnya, kode ASCII untuk karakter spasi (' ') adalah 32, huruf 'A' adalah 65, 'B' adalah 66, huruf 'a' adalah 97, dan huruf 'b' adalah 98.
  \item Pada C++, tipe data ini dinyatakan sebagai \alert{\textbf{char}}, dengan ukuran 1 byte.
\end{itemize}
\end{frame}

\begin{frame}
\frametitle{Tipe Data: Boolean}
\begin{itemize}
  \item Merupakan tipe data yang menyimpan nilai kebenaran, yaitu hanya \textbf{TRUE} atau \textbf{FALSE}.
  \item Tipe data ini akan lebih terasa kebermanfaatannya ketika kita sudah mempelajari struktur percabangan dan \textbf{array}.
  \item Pada C++, kalian dapat menggunakan tipe data \alert{\textbf{boolean}}.
\end{itemize}
\end{frame}


\begin{frame}
\frametitle{Deklarasi Variabel}
\begin{itemize}
  \item Deklarasi variabel adakah aktivitas mendaftarkan nama-nama dan tipe variabel yang akan digunakan.
  \item Pada saat dideklarasi, setiap variabel perlu disertakan tipe datanya.
\end{itemize}
\end{frame}

\begin{frame}
\frametitle{Deklarasi Variabel (lanj.)}
\begin{itemize}
  \item Pada C++, variabel dapat dideklarasikan di luar atau di dalam blok program.
  \item Apabila variabel dideklarasikan di luar blok program, artinya variabel tersebut bersifat \foreignTerm{global}.
  \item Tipe data dituliskan sebelum nama variabel, dipisahkan oleh spasi.

  Contoh: "\texttt{int nilai}" atau "\texttt{double rerata}".
  \item Beberapa variabel juga bisa dideklarasikan secara bersamaan jika memiliki tipe data yang sama. Contoh: "\texttt{double x, y}".
\end{itemize}
\end{frame}


\begin{frame}[fragile]
\frametitle{Contoh Program: tipedasar.cpp}
\begin{itemize}
  \item Pahami program berikut ini dan coba jalankan!
\begin{lstlisting}
#include <cstdio>

int p1, p2;
double x, y;

int main() {
  p1 = 100;
  p2 = p1;
  printf("p1: %d, p2: %d\n", p1, p2);

  x = 3.1418;
  y = 234.432;
  printf("x %lf\n", x);
  printf("y %lf\n", y);
}
\end{lstlisting}
\end{itemize}
\end{frame}

\begin{frame}[fragile]
\frametitle{Penjelasan Program: tipedasar.cpp}
\begin{itemize}
  \item Berikut adalah keluaran dari program tipedasar.cpp:
\begin{lstlisting}
p1: 100, p2: 100
x 3.141800
y 234.432000
\end{lstlisting}
  \item Perhatikan bahwa perintah \texttt{p2 = p1} sama artinya dengan \\ \texttt{p2 = 100}, karena \texttt{p1} sendiri mengacu pada nilai 100.
  \item Untuk mencetak variabel bertipe \texttt{double}, gunakan simbol "\texttt{\%lf}" (seperti "long float").
\end{itemize}
\end{frame}

\begin{frame}[fragile]
\frametitle{Simbol Variabel pada printf}
\begin{itemize}
  \item Sejauh ini, kita mengenal bahwa "\texttt{\%d}" digunakan untuk mencetak \texttt{int}, dan "\texttt{\%lf}" untuk \texttt{double}.
  \item Berikut tabel variabel beserta simbolnya:
\end{itemize}
\begin{table}[ht]
  \begin{tabular}{|c|c|c|}
    \hline Variabel  & Simbol \\
    \hline short & \texttt{\%d} \\
    \hline unsigned short & \texttt{\%u} \\
    \hline int & \texttt{\%d} \\
    \hline unsigned int & \texttt{\%u} \\
    \hline long long & \texttt{\%lld} atau \texttt{\%I64d} \\
    \hline unsigned long long & \texttt{\%llu} atau \texttt{\%I64u} \\
    \hline float & \texttt{\%f} \\
    \hline double & \texttt{\%lf} \\
    \hline char &  \texttt{\%c} \\
    \hline
  \end{tabular}
\end{table}
\end{frame}

\begin{frame}[fragile]
\frametitle{Simbol Variabel pada printf (lanj.)}
\begin{itemize}
  \item Untuk \texttt{boolean}, Anda dapat menggunakan \texttt{\%d} yang akan mencetak 1 apabila \textbf{TRUE} atau 0 apabila \textbf{FALSE}.
  \item Khusus untuk \texttt{long long}, simbolnya bergantung pada sistem operasi yang digunakan.
  \item Untuk sistem operasi berbasis UNIX (Linux dan Mac), gunakan \texttt{\%lld} dan \texttt{\%llu}.
  \item Untuk sistem operasi Windows, gunakan \texttt{\%I64d} dan \texttt{\%I64u}.
\end{itemize}
\end{frame}

\begin{frame}
\frametitle{Tipe Data Komposit: Struct}
\begin{itemize}
  \item Kadang-kadang, kita membutuhkan suatu tipe data yang sifatnya komposit; terdiri dari beberapa data lainnya.
  \item Contoh kasusnya adalah ketika kita butuh suatu representasi dari titik. Setiap titik pada bidang memiliki dua komponen, yaitu \textbf{x} dan \textbf{y}.
\end{itemize}
\end{frame}

\begin{frame}
\frametitle{Tipe Data Komposit: Struct (lanj.)}
\begin{itemize}
  \item Memang bisa saja kita mendeklarasi dua variabel, yaitu \texttt{x} dan \texttt{y}. Namun bagaimana jika kita hendak membuat beberapa titik? Apakah kita harus membuat \texttt{x1}, \texttt{y1}, \texttt{x2}, \texttt{y2}, ...? Sungguh melelahkan!
  \item Karena itulah C++ menyajikan suatu tipe data komposit, yaitu \alert{\texttt{struct}}.
\end{itemize}
\end{frame}

\begin{frame}[fragile]
\frametitle{Tipe Data Komposit: Struct (lanj.)}
\begin{itemize}
  \item Struct dapat dideklarasikan di luar blok program utama.
\begin{lstlisting}
struct <nama_struct> {
  <tipe_1> <variabel_1>;
  <tipe_2> <variabel_2>;
  ...
};
\end{lstlisting}
  \item Setelah dideklarasikan, sebuah tipe data \texttt{$<$nama\_struct$>$} sudah bisa digunakan.
  \item Untuk mengakses nilai dari \texttt{$<$variabel 1$>$} dari suatu variabel bertipe \texttt{struct}, gunakan tanda titik (.).
\end{itemize}
\end{frame}

\begin{frame}[fragile]
\frametitle{Tipe Data Komposit: Struct (lanj.)}
\begin{itemize}
  \item Sebagai contoh, perhatikan contoh program titik.cpp berikut:
\begin{lstlisting}
#include <cstdio>

struct titik {
  int x, y;
};

titik a, b;

int main() {
  a.x = 5;
  a.y = 3;
  
  b.x = 1;
  b.y = 2;
  printf("%d %d\n", a.x, a.y);
  printf("%d %d\n", b.x, b.y);
}
\end{lstlisting}
\end{itemize}
\end{frame}

\begin{frame}
\frametitle{Konsumsi Memori Struct}
\begin{itemize}
  \item Memori yang dibutuhkan bagi sebuah tipe data \texttt{struct} bisa dianggap sama dengan jumlah memori variabel-variabel yang menyusunnya.
  \item Artinya, \texttt{struct} bernama \texttt{titik} pada contoh titik.cpp mengkonsumsi memori yang sama dengan dua buah longint, yaitu 8 byte.
  \item Perhitungan ini hanya perkiraan saja, sebab konsumsi memori yang sesungguhnya sulit dilakukan.
\end{itemize}
\end{frame}

\begin{frame}
\frametitle{Ordinalitas}
\begin{itemize}
  \item Menurut keberurutannya, tipe data dapat dibedakan menjadi tipe data \alert{ordinal} atau \alert{non-ordinal}.
  \item Suatu tipe data memiliki sifat ordinal jika untuk suatu elemennya, kita bisa mengetahui secara pasti apa elemen sebelum atau selanjutnya. Contoh:
  \begin{itemize}
    \item Diberikan bilangan bulat 6, kita tahu pasti sebelumnya adalah angka 5 dan sesudahnya adalah angka 7.
    \item Diberikan karakter 'y', kita tahu pasti sebelumnya adalah karakter 'x' dan sesudahnya adalah karakter 'z'.
  \end{itemize}
  \item Dengan demikian, seluruh tipe data bilangan bulat dan karakter adalah tipe data ordinal.
\end{itemize}
\end{frame}

\begin{frame}
\frametitle{Ordinalitas (lanj.)}
\begin{itemize}
  \item Kebalikannya, suatu tipe data dinyatakan memiliki sifat non-ordinal jika kita tidak bisa menentukan elemen sebelum dan sesudahnya. Contohnya:
  \begin{itemize}
    \item Diberikan bilangan riil 6, apakah elemen sesudahnya 7, atau 6.1, atau 6.01, atau 6.001, atau 6.00000000001?
  \end{itemize}
  \item Bilangan \textit{floating point} termasuk dalam tipe data non-ordinal.
\end{itemize}
\end{frame}

\begin{frame}
\frametitle{Yang Sudah Kita Pelajari...}
\begin{itemize}
  \item Mengenal konsep variabel.
  \item Mempelajari berbagai tipe data.
  \item Mempelajari cara deklarasi variabel.
  \item Mengenal operasi assignment.
\end{itemize}
\end{frame}

\end{document}
