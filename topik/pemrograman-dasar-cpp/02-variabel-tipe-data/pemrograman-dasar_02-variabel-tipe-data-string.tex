\documentclass{beamer}
\usetheme{tokitex}

\usepackage{tikz}
\usepackage{graphics}
\usepackage{multirow}
\usepackage{tabto}
\usepackage{xspace}
\usepackage{amsmath}
\usepackage{hyperref}

\usepackage{tikz}
\usepackage{clrscode3e}

\usepackage[english,bahasa]{babel}
\newtranslation[to=bahasa]{Section}{Bagian}
\newtranslation[to=bahasa]{Subsection}{Subbagian}

\usepackage{listings, lstautogobble}
\usepackage{color}

\definecolor{dkgreen}{rgb}{0,0.6,0}
\definecolor{gray}{rgb}{0.5,0.5,0.5}
\definecolor{mauve}{rgb}{0.58,0,0.82}

\lstset{frame=tb,
  language=pascal,
  aboveskip=1mm,
  belowskip=1mm,
  showstringspaces=false,
  columns=fullflexible,
  keepspaces=true,
  basicstyle={\small\ttfamily},
  numbers=none,
  numberstyle=\tiny\color{gray},
  keywordstyle=\color{blue},
  commentstyle=\color{dkgreen},
  stringstyle=\color{mauve},
  breaklines=true,
  breakatwhitespace=true,
  autogobble=true
}

\usepackage{caption}
\captionsetup[figure]{labelformat=empty}

\newcommand{\progTerm}[1]{\textbf{#1}}
\newcommand{\foreignTerm}[1]{\textit{#1}}
\newcommand{\newTerm}[1]{\alert{\textbf{#1}}}
\newcommand{\emp}[1]{\alert{#1}}
\newcommand{\statement}[1]{"#1"}

% Getting tired of writing \foreignTerm all the time
\newcommand{\farray}{\foreignTerm{array}\xspace}
\newcommand{\fArray}{\foreignTerm{Array}\xspace}
\newcommand{\foverhead}{\foreignTerm{overhead}\xspace}
\newcommand{\fOverhead}{\foreignTerm{Overhead}\xspace}
\newcommand{\fsubarray}{\foreignTerm{subarray}\xspace}
\newcommand{\fSubarray}{\foreignTerm{Subarray}\xspace}
\newcommand{\fbasecase}{\foreignTerm{base case}\xspace}
\newcommand{\fBasecase}{\foreignTerm{Base case}\xspace}
\newcommand{\ftopdown}{\foreignTerm{top down}\xspace}
\newcommand{\fTopdown}{\foreignTerm{Top down}\xspace}
\newcommand{\fbottomup}{\foreignTerm{bottom up}\xspace}
\newcommand{\fBottomup}{\foreignTerm{Bottom up}\xspace}
\newcommand{\fpruning}{\foreignTerm{pruning}\xspace}
\newcommand{\fPruning}{\foreignTerm{Pruning}\xspace}

\newcommand{\fgraph}{\foreignTerm{graph}\xspace}
\newcommand{\fGraph}{\foreignTerm{Graph}\xspace}
\newcommand{\fnode}{\foreignTerm{node}\xspace}
\newcommand{\fNode}{\foreignTerm{Node}\xspace}
\newcommand{\fedge}{\foreignTerm{edge}\xspace}
\newcommand{\fEdge}{\foreignTerm{Edge}\xspace}
\newcommand{\fdegree}{\foreignTerm{degree}\xspace}
\newcommand{\fDegree}{\foreignTerm{Degree}\xspace}
\newcommand{\fadjacencylist}{\foreignTerm{adjacency list}\xspace}
\newcommand{\fAdjacencylist}{\foreignTerm{Adjacency list}\xspace}
\newcommand{\fadjacencymatrix}{\foreignTerm{adjacency matrix}\xspace}
\newcommand{\fAdjacencymatrix}{\foreignTerm{Adjacency matrix}\xspace}
\newcommand{\fedgelist}{\foreignTerm{edge list}\xspace}
\newcommand{\fEdgelist}{\foreignTerm{Edge list}\xspace}
\newcommand{\flist}{\foreignTerm{list}\xspace}
\newcommand{\fList}{\foreignTerm{List}\xspace}
\newcommand{\fgraphtraversal}{\foreignTerm{graph traversal}\xspace}
\newcommand{\fGraphtraversal}{\foreignTerm{Graph traversal}\xspace}
\newcommand{\ftree}{\foreignTerm{tree}\xspace}
\newcommand{\fTree}{\foreignTerm{Tree}\xspace}
\newcommand{\fsubtree}{\foreignTerm{subtree}\xspace}
\newcommand{\fSubtree}{\foreignTerm{Subtree}\xspace}

\newcommand{\fDivideAndConquer}{\foreignTerm{Divide and conquer}\xspace}
\newcommand{\fdivideAndConquer}{\foreignTerm{divide and conquer}\xspace}
\newcommand{\fMergeSort}{\foreignTerm{Merge sort}\xspace}
\newcommand{\fmergeSort}{\foreignTerm{merge sort}\xspace}
\newcommand{\fQuickSort}{\foreignTerm{Quicksort}\xspace}
\newcommand{\fquickSort}{\foreignTerm{quicksort}\xspace}
\newcommand{\fpivot}{\foreignTerm{pivot}\xspace}
\newcommand{\fPivot}{\foreignTerm{Pivot}\xspace}
\newcommand{\fbruteForce}{\foreignTerm{brute force}\xspace}
\newcommand{\fBruteForce}{\foreignTerm{Brute force}\xspace}
\newcommand{\fCompleteSearch}{\foreignTerm{complete search}\xspace}
\newcommand{\fExhaustiveSearch}{\foreignTerm{exhaustive search}\xspace}
\newcommand{\fBinarySearch}{\foreignTerm{binary search}\xspace}
\newcommand{\fGreedy}{\foreignTerm{greedy}\xspace}
\newcommand{\fGreedyChoice}{\foreignTerm{greedy choice}\xspace}

\newcommand{\pheap}{\foreignTerm{heap}\xspace}
\newcommand{\pHeap}{\foreignTerm{Heap}\xspace}
\newcommand{\pBinaryHeap}{\foreignTerm{Binary Heap}\xspace}
\newcommand{\pbinaryHeap}{\foreignTerm{binary heap}\xspace}
\newcommand{\pHeapSort}{\foreignTerm{Heap Sort}\xspace}
\newcommand{\pdjs}{\foreignTerm{disjoint set}\xspace}
\newcommand{\pDjs}{\foreignTerm{Disjoint set}\xspace}

\title{Variabel dan Tipe Data: String}
\author{Tim Olimpiade Komputer Indonesia}
\date{}

\begin{document}

\begin{frame}
\titlepage
\end{frame}

\begin{frame}
\frametitle{Pendahuluan}
Melalui dokumen ini, kalian akan:
\begin{itemize}
  \item Mengenal konsep string pada C dan C++.
  \item Mengenal sedikit tentang STL.
\end{itemize}
\end{frame}

\begin{frame}[fragile]
\frametitle{Tipe Data String}
\begin{itemize}
  \item Merupakan tipe data untuk merepresentasikan untaian karakter.
  \item Contohnya adalah untuk penyimpanan data berupa teks.
  \item Tipe data string berperilaku berbeda dengan tipe data dasar pada C++.
  \item Sebab, string bukan tipe data primitif, melainkan suatu tipe dari \foreignTerm{Standard Template Library} (STL).
\end{itemize}
\end{frame}

\begin{frame}[fragile]
\frametitle{STL}
\begin{itemize}
  \item STL merupakan kumpulan \foreignTerm{library} yang disediakan C++.
  \item Sebuah \foreignTerm{library} dapat dianggap sebagai komponen program yang sering digunakan, sehingga sudah disediakan dan siap pakai.
  \item Contoh komponen program yang akan sering digunakan adalah pembacaan data, pencetakan data, pengurutan, pencarian, dan pencarian akar kuadrat.
\end{itemize}
\end{frame}

\begin{frame}[fragile]
\frametitle{Contoh Program String}
\begin{itemize}
  \item Sebagai contoh, perhatikan contoh program berikut:
\end{itemize}
\begin{lstlisting}
#include <cstdio>
#include <string>

int main() {
  std::string s = "ini adalah string";
  printf("%s\n", s.c_str());
}\end{lstlisting}
\end{frame}

\begin{frame}
\frametitle{Penjelasan Contoh Program String}
\begin{itemize}
  \item Pertama, perhatikan penambahan "\texttt{\#include <string>}".
  \item Sebuah perintah "\texttt{\#include <nama\_library>}" digunakan untuk memberitahu C++ bahwa kita hendak menggunakan komponen STL "nama\_library".
  \item Pada kasus ini, STL yang kita akan gunakan adalah \emp{string}.
  \item Hal ini juga berlaku dengan yang sejauh ini telah dilakukan, yaitu "\texttt{\#include <cstdio>}".
  \item STL cstdio merupakan STL yang memberikan kemampuan untuk membaca dan mencetak keluaran. Komponen yang telah kita gunakan adalah "\texttt{printf}".
\end{itemize}
\end{frame}

\begin{frame}
\frametitle{Penjelasan Contoh Program String (lanj.)}
\begin{itemize}
  \item Kedua, perhatikan bahwa tipe data string ditulis dengan gaya yang berbeda, yaitu "\texttt{std::string}".
  \item Sebenarnya, nama tipe datanya hanya "string".
  \item Nilai yang tertera sebelum "::" disebut dengan \emp{namespace}.
  \item Namespace merupakan mekanisme C++ untuk menjamin tidak adanya pertubrukan antara nama \foreignTerm{identifier} yang didefinisikan, terutama antar \foreignTerm{library}.
  \item Namespace yang digunakan oleh STL string adalah "std".
\end{itemize}
\end{frame}

\begin{frame}[fragile]
\frametitle{Program String Tanpa std}
\begin{itemize}
  \item Berhubung terus menerus menulis "std::" cukup merepotkan, kita dapat menggunakan perintah "\texttt{using namespace std}" untuk memberitahu C++ bahwa kita hendak menggunakan namespace tersebut secara standar.
\end{itemize}
\begin{lstlisting}
#include <cstdio>
#include <string>

using namespace std;

int main() {
  string s = "ini adalah string";
  printf("%s\n", s.c_str());
}\end{lstlisting}
\end{frame}

\begin{frame}
\frametitle{Penjelasan Contoh Program String (lanj.)}
\begin{itemize}
  \item Ketiga, perhatikan bahwa pencetakan string menggunakan simbol "\texttt{\%s}".
  \item Variabel yang dicantumkan juga diberikan "\texttt{.c\_str()}".
  \item Alasannya adalah string dari STL merupakan kepunyaan C++, sementara \texttt{printf} merupakan kepunyaan bahasa C.
  \item Karena bahasa C lebih tua daripada C++, \texttt{printf} tidak paham cara mencetak tipe data string.
  \item Kita perlu mengubah string C++ menjadi string bahasa C dengan \texttt{.c\_str()}".
\end{itemize}
\end{frame}

\begin{frame}
\frametitle{String pada C++ dan C}
\begin{itemize}
  \item Lalu tipe data string seperti apa yang ada pada bahasa C?
  \item Pada bahasa C, string diwujudkan dengan membuat \foreignTerm{array of char}, atau biasa disebut dengan \emp{cstring}.
  \item Pembelajaran tentang cstring akan diperdalam pada bab yang akan datang.
\end{itemize}
\end{frame}

\begin{frame}
\frametitle{Sifat String}
\begin{itemize}
  \item Memori yang digunakan adalah 1 byte dikali banyak karakternya.
  \item String bukan tipe data ordinal.
  \item Penulisannya pada program perlu diapit oleh tanda petik (").
\end{itemize}
\end{frame}

\begin{frame}
\frametitle{Yang Sudah Kita Pelajari...}
\begin{itemize}
  \item Mengenal cara membuat variabel string.
  \item Mengenal sedikit tentang STL.
  \item Mempelajari sifat-sifat string.
\end{itemize}
\end{frame}

\end{document}
