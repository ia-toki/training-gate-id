\documentclass{beamer}
\usetheme{tokitex}

\usepackage{tikz}
\usepackage{graphics}
\usepackage{multirow}
\usepackage{tabto}
\usepackage{xspace}
\usepackage{amsmath}
\usepackage{hyperref}
\usepackage{wrapfig}
\usepackage{mathtools}

\usepackage{tikz}
\usepackage{clrscode3e}
\usepackage{gensymb}

\usepackage[english,bahasa]{babel}
\newtranslation[to=bahasa]{Section}{Bagian}
\newtranslation[to=bahasa]{Subsection}{Subbagian}

\usepackage{listings, lstautogobble}
\usepackage{color}

\definecolor{dkgreen}{rgb}{0,0.6,0}
\definecolor{gray}{rgb}{0.5,0.5,0.5}
\definecolor{mauve}{rgb}{0.58,0,0.82}

\lstset{frame=tb,
  language=c++,
  aboveskip=0mm,
  belowskip=0mm,
  showstringspaces=false,
  columns=fullflexible,
  keepspaces=true,
  basicstyle={\small\ttfamily},
  numbers=none,
  numberstyle=\tiny\color{gray},
  keywordstyle=\color{blue},
  commentstyle=\color{dkgreen},
  stringstyle=\color{mauve},
  breaklines=true,
  breakatwhitespace=true,
  lineskip={-3pt}
}

\usepackage{caption}
\captionsetup[figure]{labelformat=empty}

\newcommand{\progTerm}[1]{\textbf{#1}}
\newcommand{\foreignTerm}[1]{\textit{#1}}
\newcommand{\newTerm}[1]{\alert{\textbf{#1}}}
\newcommand{\emp}[1]{\alert{#1}}
\newcommand{\statement}[1]{"#1"}

\newcommand{\floor}[1]{\lfloor #1 \rfloor}
\newcommand{\ceil}[1]{\lceil #1 \rceil}
\newcommand{\abs}[1]{\left\lvert#1\right\rvert}
\newcommand{\norm}[1]{\left\lVert#1\right\rVert}

% Getting tired of writing \foreignTerm all the time
\newcommand{\farray}{\foreignTerm{array}\xspace}
\newcommand{\fArray}{\foreignTerm{Array}\xspace}
\newcommand{\foverhead}{\foreignTerm{overhead}\xspace}
\newcommand{\fOverhead}{\foreignTerm{Overhead}\xspace}
\newcommand{\fsubarray}{\foreignTerm{subarray}\xspace}
\newcommand{\fSubarray}{\foreignTerm{Subarray}\xspace}
\newcommand{\fbasecase}{\foreignTerm{base case}\xspace}
\newcommand{\fBasecase}{\foreignTerm{Base case}\xspace}
\newcommand{\ftopdown}{\foreignTerm{top-down}\xspace}
\newcommand{\fTopdown}{\foreignTerm{Top-down}\xspace}
\newcommand{\fbottomup}{\foreignTerm{bottom-up}\xspace}
\newcommand{\fBottomup}{\foreignTerm{Bottom-up}\xspace}
\newcommand{\fpruning}{\foreignTerm{pruning}\xspace}
\newcommand{\fPruning}{\foreignTerm{Pruning}\xspace}

\newcommand{\fgraph}{\foreignTerm{graph}\xspace}
\newcommand{\fGraph}{\foreignTerm{Graph}\xspace}
\newcommand{\froot}{\foreignTerm{root}\xspace}
\newcommand{\fRoot}{\foreignTerm{Root}\xspace}
\newcommand{\fnode}{\foreignTerm{node}\xspace}
\newcommand{\fNode}{\foreignTerm{Node}\xspace}
\newcommand{\fedge}{\foreignTerm{edge}\xspace}
\newcommand{\fEdge}{\foreignTerm{Edge}\xspace}
\newcommand{\fcycle}{\foreignTerm{cycle}\xspace}
\newcommand{\fCycle}{\foreignTerm{Cycle}\xspace}
\newcommand{\fdegree}{\foreignTerm{degree}\xspace}
\newcommand{\fDegree}{\foreignTerm{Degree}\xspace}
\newcommand{\fadjacencylist}{\foreignTerm{adjacency list}\xspace}
\newcommand{\fAdjacencylist}{\foreignTerm{Adjacency list}\xspace}
\newcommand{\fadjacencymatrix}{\foreignTerm{adjacency matrix}\xspace}
\newcommand{\fAdjacencymatrix}{\foreignTerm{Adjacency matrix}\xspace}
\newcommand{\fedgelist}{\foreignTerm{edge list}\xspace}
\newcommand{\fEdgelist}{\foreignTerm{Edge list}\xspace}
\newcommand{\flist}{\foreignTerm{list}\xspace}
\newcommand{\fList}{\foreignTerm{List}\xspace}
\newcommand{\fgraphtraversal}{\foreignTerm{graph traversal}\xspace}
\newcommand{\fGraphtraversal}{\foreignTerm{Graph traversal}\xspace}
\newcommand{\ftree}{\foreignTerm{tree}\xspace}
\newcommand{\fTree}{\foreignTerm{Tree}\xspace}
\newcommand{\fsubtree}{\foreignTerm{subtree}\xspace}
\newcommand{\fSubtree}{\foreignTerm{Subtree}\xspace}
\newcommand{\fparent}{\foreignTerm{parent}\xspace}
\newcommand{\fParent}{\foreignTerm{Parent}\xspace}
\newcommand{\fsibling}{\foreignTerm{sibling}\xspace}
\newcommand{\fSibling}{\foreignTerm{Sibling}\xspace}
\newcommand{\fpath}{\foreignTerm{path}\xspace}
\newcommand{\fPath}{\foreignTerm{Path}\xspace}
\newcommand{\fconnectedcomponent}{\foreignTerm{connected component}\xspace}
\newcommand{\fConnectedcomponent}{\foreignTerm{Connected component}\xspace}
\newcommand{\fbridge}{\foreignTerm{bridge}\xspace}
\newcommand{\fBridge}{\foreignTerm{Bridge}\xspace}
\newcommand{\farticulationpoint}{\foreignTerm{articulation point}\xspace}
\newcommand{\fArticulationpoint}{\foreignTerm{Articulation point}\xspace}
\newcommand{\ftreeedge}{\foreignTerm{tree edge}\xspace}
\newcommand{\fTreeedge}{\foreignTerm{Tree edge}\xspace}
\newcommand{\fbackedge}{\foreignTerm{back edge}\xspace}
\newcommand{\fBackedge}{\foreignTerm{Back edge}\xspace}
\newcommand{\fforwardedge}{\foreignTerm{forward edge}\xspace}
\newcommand{\fForwardedge}{\foreignTerm{Forward edge}\xspace}
\newcommand{\fcrossedge}{\foreignTerm{cross edge}\xspace}
\newcommand{\fCrossedge}{\foreignTerm{Cross edge}\xspace}
\newcommand{\fdiscoverytime}{\foreignTerm{discovery time}\xspace}
\newcommand{\fDiscoverytime}{\foreignTerm{Discovery time}\xspace}
\newcommand{\flowlink}{\foreignTerm{low link}\xspace}
\newcommand{\fLowlink}{\foreignTerm{Low link}\xspace}
\newcommand{\fstack}{\foreignTerm{stack}\xspace}
\newcommand{\fStack}{\foreignTerm{Stack}\xspace}
\newcommand{\for}{\foreignTerm{or}\xspace}
\newcommand{\fOr}{\foreignTerm{Or}\xspace}
\newcommand{\fand}{\foreignTerm{and}\xspace}
\newcommand{\fAnd}{\foreignTerm{And}\xspace}
\newcommand{\fcentroid}{\foreignTerm{centroid}\xspace}
\newcommand{\fCentroid}{\foreignTerm{Centroid}\xspace}

\newcommand{\fDivideAndConquer}{\foreignTerm{Divide and conquer}\xspace}
\newcommand{\fdivideAndConquer}{\foreignTerm{divide and conquer}\xspace}
\newcommand{\fMergeSort}{\foreignTerm{Merge sort}\xspace}
\newcommand{\fmergeSort}{\foreignTerm{merge sort}\xspace}
\newcommand{\fQuickSort}{\foreignTerm{Quicksort}\xspace}
\newcommand{\fquickSort}{\foreignTerm{quicksort}\xspace}
\newcommand{\fpivot}{\foreignTerm{pivot}\xspace}
\newcommand{\fPivot}{\foreignTerm{Pivot}\xspace}
\newcommand{\fbruteForce}{\foreignTerm{brute force}\xspace}
\newcommand{\fBruteForce}{\foreignTerm{Brute force}\xspace}
\newcommand{\fCompleteSearch}{\foreignTerm{complete search}\xspace}
\newcommand{\fExhaustiveSearch}{\foreignTerm{exhaustive search}\xspace}
\newcommand{\fbinarySearch}{\foreignTerm{binary search}\xspace}
\newcommand{\fBinarySearch}{\foreignTerm{Binary search}\xspace}
\newcommand{\fternarySearch}{\foreignTerm{ternary search}\xspace}
\newcommand{\fTernarySearch}{\foreignTerm{Ternary search}\xspace}
\newcommand{\funimodal}{\foreignTerm{unimodal}\xspace}
\newcommand{\fUnimodal}{\foreignTerm{Unimodal}\xspace}
\newcommand{\fGreedy}{\foreignTerm{Greedy}\xspace}
\newcommand{\fgreedy}{\foreignTerm{greedy}\xspace}
\newcommand{\fgreedyChoice}{\foreignTerm{greedy choice}\xspace}
\newcommand{\fGreedyChoice}{\foreignTerm{Greedy choice}\xspace}

\newcommand{\fdp}{\foreignTerm{dynamic programming}\xspace}
\newcommand{\fDp}{\foreignTerm{Dynamic programming}\xspace}
\newcommand{\fbitmask}{\foreignTerm{bitmask}\xspace}
\newcommand{\fBitmask}{\foreignTerm{Bitmask}\xspace}
\newcommand{\fstate}{\foreignTerm{state}\xspace}
\newcommand{\fState}{\foreignTerm{State}\xspace}
\newcommand{\fsubmask}{\foreignTerm{submask}\xspace}
\newcommand{\fSubmask}{\foreignTerm{Submask}\xspace}

\newcommand{\pheap}{\foreignTerm{heap}\xspace}
\newcommand{\pHeap}{\foreignTerm{Heap}\xspace}
\newcommand{\pBinaryHeap}{\foreignTerm{Binary Heap}\xspace}
\newcommand{\pbinaryHeap}{\foreignTerm{binary heap}\xspace}
\newcommand{\pHeapsort}{\foreignTerm{Heapsort}\xspace}
\newcommand{\pheapsort}{\foreignTerm{heapsort}\xspace}
\newcommand{\pdjs}{\foreignTerm{disjoint set}\xspace}
\newcommand{\pDjs}{\foreignTerm{Disjoint set}\xspace}

\newcommand{\fdotProduct}{\foreignTerm{dot product}\xspace}
\newcommand{\fDotProduct}{\foreignTerm{Dot product}\xspace}
\newcommand{\fcrossProduct}{\foreignTerm{cross product}\xspace}
\newcommand{\fCrossProduct}{\foreignTerm{Cross product}\xspace}
\newcommand{\fconvexHull}{\foreignTerm{convex hull}\xspace}
\newcommand{\fConvexHull}{\foreignTerm{Convex hull}\xspace}
\newcommand{\fgrahamScan}{\foreignTerm{graham scan}\xspace}
\newcommand{\fGrahamScan}{\foreignTerm{Graham scan}\xspace}
\newcommand{\flineSweep}{\foreignTerm{line sweep}\xspace}
\newcommand{\fLineSweep}{\foreignTerm{Line sweep}\xspace}

\newcommand{\fset}{\foreignTerm{set}\xspace}
\newcommand{\fSet}{\foreignTerm{Set}\xspace}
\newcommand{\fprefixSum}{\foreignTerm{prefix sum}\xspace}
\newcommand{\fPrefixSum}{\foreignTerm{Prefix sum}\xspace}
\newcommand{\ffenwickTree}{\foreignTerm{fenwick tree}\xspace}
\newcommand{\fFenwickTree}{\foreignTerm{Fenwick tree}\xspace}
\newcommand{\frangeSumQuery}{\foreignTerm{range sum query}\xspace}
\newcommand{\fRangeSumQuery}{\foreignTerm{Range sum query}\xspace}
\newcommand{\fquery}{\foreignTerm{query}\xspace}
\newcommand{\fQuery}{\foreignTerm{Query}\xspace}
\newcommand{\fsegmentTree}{\foreignTerm{segment tree}\xspace}
\newcommand{\fSegmentTree}{\foreignTerm{Segment tree}\xspace}
\newcommand{\fbinaryTree}{\foreignTerm{binary tree}\xspace}
\newcommand{\fBinaryTree}{\foreignTerm{Binary tree}\xspace}
\newcommand{\flazyPropagation}{\foreignTerm{lazy propagation}\xspace}
\newcommand{\fLazyPropagation}{\foreignTerm{Lazy propagation}\xspace}
\newcommand{\fsparseTable}{\foreignTerm{sparse table}\xspace}
\newcommand{\fSparseTable}{\foreignTerm{Sparse table}\xspace}

\newcommand{\ftrail}{\foreignTerm{trail}\xspace}
\newcommand{\fTrail}{\foreignTerm{Trail}\xspace}
\newcommand{\feulerTour}{\foreignTerm{euler tour}\xspace}
\newcommand{\fEulerTour}{\foreignTerm{Euler tour}\xspace}
\newcommand{\feulerTourTree}{\foreignTerm{euler tour tree}\xspace}
\newcommand{\fEulerTourTree}{\foreignTerm{Euler tour tree}\xspace}

\newcommand{\fmaxflow}{\foreignTerm{maximum flow}\xspace}
\newcommand{\fMaxflow}{\foreignTerm{Maximum flow}\xspace}
\newcommand{\fmincut}{\foreignTerm{minimum cut}\xspace}
\newcommand{\fMincut}{\foreignTerm{Minimum cut}\xspace}
\newcommand{\fflow}{\foreignTerm{flow}\xspace}
\newcommand{\fFlow}{\foreignTerm{Flow}\xspace}
\newcommand{\fsource}{\foreignTerm{source}\xspace}
\newcommand{\fSource}{\foreignTerm{Source}\xspace}
\newcommand{\fsink}{\foreignTerm{sink}\xspace}
\newcommand{\fSink}{\foreignTerm{Sink}\xspace}
\newcommand{\fbackEdge}{\foreignTerm{back-edge}\xspace}
\newcommand{\fBackEdge}{\foreignTerm{Back-edge}\xspace}
\newcommand{\fresidualCapacity}{\foreignTerm{residual capacity}\xspace}
\newcommand{\fResidualCapacity}{\foreignTerm{Residual capacity}\xspace}
\newcommand{\fbottleneck}{\foreignTerm{bottleneck}\xspace}
\newcommand{\fBottleneck}{\foreignTerm{Bottleneck}\xspace}
\newcommand{\faugmentingPath}{\foreignTerm{augmenting path}\xspace}
\newcommand{\fAugmentingPath}{\foreignTerm{Augmenting path}\xspace}


\title{Perulangan}
\author{Tim Olimpiade Komputer Indonesia}
\date{}

\begin{document}

\begin{frame}
\titlepage
\end{frame}

\begin{frame}
\frametitle{Pendahuluan}
Melalui dokumen ini, kalian akan:
\begin{itemize}
  \item Memahami konsep perulangan.
  \item Mempelajari struktur \textbf{for} dan \textbf{while} pada C++.
\end{itemize}
\end{frame}

\begin{frame}[fragile]
\frametitle{Motivasi}
\begin{itemize}
  \item Hari ini, Pak Dengklek ingin menyambut N ekor bebeknya yang baru lahir dari telur.
  \item Diberikan N, cetak tulisan "halo dunia!" sebanyak N kali!
  \item Contoh untuk N = 3:
  \begin{lstlisting}
    halo dunia!
    halo dunia!
    halo dunia!
  \end{lstlisting}
\end{itemize}
\end{frame}

\begin{frame}
\frametitle{Motivasi (lanj.)}
\begin{itemize}
  \item Solusi "if (N = 1) $<$cetak satu kali$>$, else if (N = 2) $<$cetak dua kali$>$, ..." tidak mungkin digunakan, karena N bisa jadi sangat besar.
  \item Kita membutuhkan suatu struktur yang memungkinkan untuk mengulangi serangkaian pekerjaan!
  \item C++ menyediakan struktur perulangan berupa \emp{for} dan \emp{while}.
\end{itemize}
\end{frame}


\begin{frame}[fragile]
\frametitle{Perulangan: for}
\begin{itemize}
  \item Biasanya digunakan ketika kita tahu berapa kali perulangan perlu dilakukan.
  \item Pada C++, strukturnya:
\begin{lstlisting}
for (<kondisi_awal>; <kondisi_ulang>; <perubahan>) {
  <perintah 1>;
  <perintah 2>;
  ...
}
\end{lstlisting}
  \item $<$kondisi\_awal$>$ dapat diisi dengan inisialisasi variabel untuk perulangan.
  \item $<$kondisi\_ulang$>$ biasanya berupa ekspresi yang menghasilkan \texttt{boolean}, untuk menandakan apakah perulangan sudah patut diberhentikan.
  \item $<$perubahan$>$ merupakan bagian yang dieksekusi pada akhir setiap siklus perulangan.
  \item Penjelasan berikutnya dengan contoh akan meningkatkan pemahaman kalian.
\end{itemize}
\end{frame}

\begin{frame}[fragile]
\frametitle{Contoh Program: for.cpp}
\begin{itemize}
  \item Ketikkan program berikut dan coba jalankan:
\begin{lstlisting}
#include <cstdio>

int main() {
  int N;
  scanf("%d", &N);

  for (int i = 0; i < N; i++) {
    printf("tulisan ini dicetak saat i = %d\n", i);
  }
  printf("akhir dari program\n");
}
\end{lstlisting}
  \item Masukkan berbagai nilai N, misalnya 1, 2, 10, dan 0.
\end{itemize}
\end{frame}

\begin{frame}
\frametitle{Penjelasan Program: for.cpp}
\begin{itemize}
  \item Misalnya kita memasukkan \identifier{N} = 5.
  \item Pertama kali dijalankan, \identifier{i} dibuat dan diisi nilai 0.
  \item Kedua, C++ memeriksa apakah kondisi ulang tercapai. Berhubung \identifier{i} kurang dari \identifier{N}, maka bagian dalam for dilaksanakan dan tulisan dicetak saat \identifier{i} = 0.
  \item Setelah itu, akhir dari struktur for ditemukan. C++ akan mengeksekusi bagian perubahan, yakni menambah \identifier{i} dengan 1, lalu kembali ke awal dari for.
  \item Jika \identifier{i} masih kurang dari \identifier{N}, maka perintah di dalamnya akan kembali dilaksanakan.
  \item Dengan demikian, tercetaklah tulisan saat \identifier{i} = 1, 2, dan seterusnya hingga \identifier{N}-1.
\end{itemize}
\end{frame}

\begin{frame}
\frametitle{Penjelasan Program: for.cpp (lanj.)}
\begin{itemize}
  \item Jika \identifier{i} sudah lebih dari \identifier{N}, perulangan akan berhenti dan C++ akan menjalankan perintah sesudah for tersebut.
  \item Pada contoh ini, mencetak tulisan "akhir dari program".
\end{itemize}
\end{frame}

\begin{frame}
\frametitle{Masa Hidup Variabel}
\begin{itemize}
  \item Variabel \identifier{i} hanya memiliki masa hidup di dalam lingkungan for.
  \item Di luar for berikut \{ dan \}, variabel tersebut sudah tidak ada.
  \item Anda boleh saja mendeklarasikan variabel \identifier{i} lagi, tanpa mendapatkan \foreignTerm{error} bahwa nama variabel telah terdefinisi.
  \item Variabel yang hanya hidup dalam suatu cakupan \{ \} disebut dengan variabel lokal.
\end{itemize}
\end{frame}


\begin{frame}[fragile]
\frametitle{Contoh Program: fordownto.cpp}
\begin{itemize}
  \item Struktur for cukup luwes untuk kasus-kasus lainnya.
  \item Berikut ini contoh dari penggunaan for yang bekerja secara terbalik:
\begin{lstlisting}
#include <cstdio>

int main() {
  int N;
  scanf("%d", &N);

  for (int i = N-1; i >= 0; i--) {
    printf("tulisan ini dicetak saat i = %d\n", i);
  }
  printf("akhir dari program\n");
}
\end{lstlisting}
\end{itemize}
\end{frame}

\begin{frame}[fragile]
\frametitle{Contoh Program: forskip.cpp}
\begin{itemize}
  \item Sementara berikut contoh dari penggunaan for yang lompatannya 2:
\begin{lstlisting}
#include <cstdio>

int main() {
  int N;
  scanf("%d", &N);

  for (int i = 0; i < N; i += 2) {
    printf("tulisan ini dicetak saat i = %d\n", i);
  }
  printf("akhir dari program\n");
}
\end{lstlisting}
  \item Ekspresi \texttt{i += 2} setara dengan \texttt{i = i + 2}, yang mengisikan \identifier{i} dengan dirinya ditambah 2.
\end{itemize}
\end{frame}

\begin{frame}[fragile]
\frametitle{Contoh Program: forskip2.cpp}
\begin{itemize}
  \item Tidak terbatas pada penjumlahan, hal semacam ini pun bisa dilakukan:
\begin{lstlisting}
#include <cstdio>

int main() {
  int N;
  scanf("%d", &N);

  for (int i = 1; i <= N; i *= 2) {
    printf("tulisan ini dicetak saat i = %d\n", i);
  }
  printf("akhir dari program\n");
}
\end{lstlisting}
  \item Ekspresi \texttt{i *= 2} setara dengan \texttt{i = i * 2}, yang mengisikan \identifier{i} dengan dirinya dikali 2.
\end{itemize}
\end{frame}


\begin{frame}[fragile]
\frametitle{Contoh Program: forsum.cpp}
\begin{itemize}
  \item Berikut contoh program untuk menjumlahkan seluruh bilangan di antara dua bilangan:
\begin{lstlisting}
#include <cstdio>

int main() {
  int awal, akhir;
  scanf("%d %d", &awal, &akhir);

  int jumlah = 0;
  for (int i = awal; i <= akhir; i++) {
    jumlah += i;
  }
  printf("jumlah bilangan bulat di antara %d dan %d (inklusif) adalah %d\n", awal, akhir, jumlah);
}
\end{lstlisting}
\end{itemize}
\end{frame}

\begin{frame}[fragile]
\frametitle{Perulangan: while}
\begin{itemize}
  \item Selain for, terdapat pula struktur while.
  \item Biasa digunakan ketika tidak diketahui harus berapa kali serangkaian perintah dilaksanakan, tetapi diketahui perintah-perintah itu perlu dilaksanakan selama suatu kondisi terpenuhi.
  \item Pada C++, strukturnya:
\begin{lstlisting}
while (<kondisi>) {
  <perintah 1>;
  <perintah 2>;
  ...
}
\end{lstlisting}
  \item Seperti pada if, $<$kondisi$>$ adalah suatu nilai boolean. Selama nilainya \textbf{TRUE}, seluruh $<$perintah x$>$ di dalamnya akan dieksekusi secara berurutan.
\end{itemize}
\end{frame}

\begin{frame}[fragile]
\frametitle{Contoh Program: while.cpp}
\begin{itemize}
  \item Berikut adalah contoh penggunaan \textbf{while} untuk kasus yang sama dengan for.cpp:
\begin{lstlisting}
#include <cstdio>

int main() {
  int N;
  scanf("%d", &N);

  int i = 0;
  while (i < N) {
    printf("tulisan ini dicetak saat i = %d\n", i);
    i++;
  }
  printf("akhir dari program\n");
}
\end{lstlisting}
\end{itemize}
\end{frame}

\begin{frame}[fragile]
\frametitle{Penjelasan Program: while.cpp}
\begin{itemize}
  \item Alur programmnya sangat mirip dengan for.
  \item Perbedaannya hanya pada gaya penulisan.
\end{itemize}
\begin{lstlisting}
// Bentuk for
for (<kondisi_awal>; <kondisi_ulang>; <perubahan>) {
  <perintah 1>;
  <perintah 2>;
  ...
}
// Bentuk while
<kondisi_awal>;
while (<kondisi_ulang>) {
  <perintah 1>;
  <perintah 2>;
  ...
  <perubahan>;
}
\end{lstlisting}

\end{frame}

\begin{frame}
\frametitle{Perulangan: while (lanj.)}
\begin{itemize}
  \item Perhatikan bahwa perintah "\texttt{i++}" diperlukan, supaya suatu saat nanti $<$kondisi$>$ pada while akan tidak dipenuhi.
  \item Sekarang coba hapus perintah "\texttt{i++}" pada while.cpp, dan jalankan kembali programnya.
  \item Apa yang terjadi? Program akan terjebak dalam \alert{\textit{infinite loop}}, atau \alert{perulangan yang tidak akan pernah berhenti}! Gunakan tombol CTRL+C pada \textit{keyboard} untuk memberhentikan program secara paksa.
  \item Dengan demikian, pastikan suatu saat "kondisi pada while" tidak dipenuhi, atau program tidak akan pernah berhenti $:)$
\end{itemize}
\end{frame}

\begin{frame}[fragile]
\frametitle{Contoh Program: whilesum.cpp}
\begin{itemize}
  \item Berikut ini contoh program dengan while yang melakukan hal serupa dengan forsum.cpp:
\begin{lstlisting}
#include <cstdio>

int main() {
  int awal, akhir;
  scanf("%d %d", &awal, &akhir);

  int jumlah = 0;
  int i = awal;
  while (i <= akhir) {
    jumlah += i;
    i++;
  }
  printf("jumlah bilangan bulat di antara %d dan %d (inklusif) adalah %d\n", awal, akhir, jumlah);
}
\end{lstlisting}
\end{itemize}
\end{frame}

\begin{frame}[fragile]
\frametitle{Perulangan: while (lanj.)}
\begin{itemize}
  \item Terdapat variasi lain dari while, yang biasa disebut "do ... while".
  \item Pada C++, strukturnya:
\begin{lstlisting}
do {
  <perintah 1>;
  <perintah 2>;
  ...
} while (<kondisi>);
\end{lstlisting}
  \item Perbedaannya adalah, seluruh perintah akan dilakukan dulu, baru diperiksa apakah kondisi masih terpenuhi. Bila ya, maka seluruh perintah akan diulang.
  \item Hal ini menjamin seluruh perintah dijalankan paling sedikit satu kali.  
\end{itemize}
\end{frame}

\begin{frame}[fragile]
\frametitle{Contoh Program: dowhile.cpp}
\begin{itemize}
  \item Berikut ini adalah contoh program dengan \textbf{do ... while} yang menjalankan tugas serupa dengan for.cpp dan while.cpp.
\begin{lstlisting}
#include <cstdio>

int main() {
  int N;
  scanf("%d", &N);

  int i = 0;
  do {
    printf("tulisan ini dicetak saat i = %d\n", i);
    i++;
  } while (i < N);
  printf("akhir dari program\n");
}
\end{lstlisting}
\end{itemize}
\end{frame}

\begin{frame}
\frametitle{Sejauh ini...}
Kalian sudah belajar tentang:
\begin{itemize}
  \item Konsep perulangan pada pemrograman.
  \item Struktur for dan while, beserta kegunaan dan perbedaannya.
\end{itemize}
Selanjutnya kita akan memasuki tentang:
\begin{itemize}
  \item Penggunaan perulangan yang lebih kompleks, yaitu perulangan bersarang.
  \item Membuat program dengan apa yang telah dipelajari sejauh ini.
\end{itemize}
\end{frame}

\end{document}
