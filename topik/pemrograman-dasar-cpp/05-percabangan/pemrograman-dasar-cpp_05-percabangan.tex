\documentclass{beamer}
\usetheme{tokitex}

\usepackage{tikz}
\usepackage{graphics}
\usepackage{multirow}
\usepackage{tabto}
\usepackage{xspace}
\usepackage{amsmath}
\usepackage{hyperref}
\usepackage{wrapfig}
\usepackage{mathtools}

\usepackage{tikz}
\usepackage{clrscode3e}
\usepackage{gensymb}

\usepackage[english,bahasa]{babel}
\newtranslation[to=bahasa]{Section}{Bagian}
\newtranslation[to=bahasa]{Subsection}{Subbagian}

\usepackage{listings, lstautogobble}
\usepackage{color}

\definecolor{dkgreen}{rgb}{0,0.6,0}
\definecolor{gray}{rgb}{0.5,0.5,0.5}
\definecolor{mauve}{rgb}{0.58,0,0.82}

\lstset{frame=tb,
  language=c++,
  aboveskip=0mm,
  belowskip=0mm,
  showstringspaces=false,
  columns=fullflexible,
  keepspaces=true,
  basicstyle={\small\ttfamily},
  numbers=none,
  numberstyle=\tiny\color{gray},
  keywordstyle=\color{blue},
  commentstyle=\color{dkgreen},
  stringstyle=\color{mauve},
  breaklines=true,
  breakatwhitespace=true,
  lineskip={-3pt}
}

\usepackage{caption}
\captionsetup[figure]{labelformat=empty}

\newcommand{\progTerm}[1]{\textbf{#1}}
\newcommand{\foreignTerm}[1]{\textit{#1}}
\newcommand{\newTerm}[1]{\alert{\textbf{#1}}}
\newcommand{\emp}[1]{\alert{#1}}
\newcommand{\statement}[1]{"#1"}

\newcommand{\floor}[1]{\lfloor #1 \rfloor}
\newcommand{\ceil}[1]{\lceil #1 \rceil}
\newcommand{\abs}[1]{\left\lvert#1\right\rvert}
\newcommand{\norm}[1]{\left\lVert#1\right\rVert}

% Getting tired of writing \foreignTerm all the time
\newcommand{\farray}{\foreignTerm{array}\xspace}
\newcommand{\fArray}{\foreignTerm{Array}\xspace}
\newcommand{\foverhead}{\foreignTerm{overhead}\xspace}
\newcommand{\fOverhead}{\foreignTerm{Overhead}\xspace}
\newcommand{\fsubarray}{\foreignTerm{subarray}\xspace}
\newcommand{\fSubarray}{\foreignTerm{Subarray}\xspace}
\newcommand{\fbasecase}{\foreignTerm{base case}\xspace}
\newcommand{\fBasecase}{\foreignTerm{Base case}\xspace}
\newcommand{\ftopdown}{\foreignTerm{top-down}\xspace}
\newcommand{\fTopdown}{\foreignTerm{Top-down}\xspace}
\newcommand{\fbottomup}{\foreignTerm{bottom-up}\xspace}
\newcommand{\fBottomup}{\foreignTerm{Bottom-up}\xspace}
\newcommand{\fpruning}{\foreignTerm{pruning}\xspace}
\newcommand{\fPruning}{\foreignTerm{Pruning}\xspace}

\newcommand{\fgraph}{\foreignTerm{graph}\xspace}
\newcommand{\fGraph}{\foreignTerm{Graph}\xspace}
\newcommand{\froot}{\foreignTerm{root}\xspace}
\newcommand{\fRoot}{\foreignTerm{Root}\xspace}
\newcommand{\fnode}{\foreignTerm{node}\xspace}
\newcommand{\fNode}{\foreignTerm{Node}\xspace}
\newcommand{\fedge}{\foreignTerm{edge}\xspace}
\newcommand{\fEdge}{\foreignTerm{Edge}\xspace}
\newcommand{\fcycle}{\foreignTerm{cycle}\xspace}
\newcommand{\fCycle}{\foreignTerm{Cycle}\xspace}
\newcommand{\fdegree}{\foreignTerm{degree}\xspace}
\newcommand{\fDegree}{\foreignTerm{Degree}\xspace}
\newcommand{\fadjacencylist}{\foreignTerm{adjacency list}\xspace}
\newcommand{\fAdjacencylist}{\foreignTerm{Adjacency list}\xspace}
\newcommand{\fadjacencymatrix}{\foreignTerm{adjacency matrix}\xspace}
\newcommand{\fAdjacencymatrix}{\foreignTerm{Adjacency matrix}\xspace}
\newcommand{\fedgelist}{\foreignTerm{edge list}\xspace}
\newcommand{\fEdgelist}{\foreignTerm{Edge list}\xspace}
\newcommand{\flist}{\foreignTerm{list}\xspace}
\newcommand{\fList}{\foreignTerm{List}\xspace}
\newcommand{\fgraphtraversal}{\foreignTerm{graph traversal}\xspace}
\newcommand{\fGraphtraversal}{\foreignTerm{Graph traversal}\xspace}
\newcommand{\ftree}{\foreignTerm{tree}\xspace}
\newcommand{\fTree}{\foreignTerm{Tree}\xspace}
\newcommand{\fsubtree}{\foreignTerm{subtree}\xspace}
\newcommand{\fSubtree}{\foreignTerm{Subtree}\xspace}
\newcommand{\fparent}{\foreignTerm{parent}\xspace}
\newcommand{\fParent}{\foreignTerm{Parent}\xspace}
\newcommand{\fsibling}{\foreignTerm{sibling}\xspace}
\newcommand{\fSibling}{\foreignTerm{Sibling}\xspace}
\newcommand{\fpath}{\foreignTerm{path}\xspace}
\newcommand{\fPath}{\foreignTerm{Path}\xspace}
\newcommand{\fconnectedcomponent}{\foreignTerm{connected component}\xspace}
\newcommand{\fConnectedcomponent}{\foreignTerm{Connected component}\xspace}
\newcommand{\fbridge}{\foreignTerm{bridge}\xspace}
\newcommand{\fBridge}{\foreignTerm{Bridge}\xspace}
\newcommand{\farticulationpoint}{\foreignTerm{articulation point}\xspace}
\newcommand{\fArticulationpoint}{\foreignTerm{Articulation point}\xspace}
\newcommand{\ftreeedge}{\foreignTerm{tree edge}\xspace}
\newcommand{\fTreeedge}{\foreignTerm{Tree edge}\xspace}
\newcommand{\fbackedge}{\foreignTerm{back edge}\xspace}
\newcommand{\fBackedge}{\foreignTerm{Back edge}\xspace}
\newcommand{\fforwardedge}{\foreignTerm{forward edge}\xspace}
\newcommand{\fForwardedge}{\foreignTerm{Forward edge}\xspace}
\newcommand{\fcrossedge}{\foreignTerm{cross edge}\xspace}
\newcommand{\fCrossedge}{\foreignTerm{Cross edge}\xspace}
\newcommand{\fdiscoverytime}{\foreignTerm{discovery time}\xspace}
\newcommand{\fDiscoverytime}{\foreignTerm{Discovery time}\xspace}
\newcommand{\flowlink}{\foreignTerm{low link}\xspace}
\newcommand{\fLowlink}{\foreignTerm{Low link}\xspace}
\newcommand{\fstack}{\foreignTerm{stack}\xspace}
\newcommand{\fStack}{\foreignTerm{Stack}\xspace}
\newcommand{\for}{\foreignTerm{or}\xspace}
\newcommand{\fOr}{\foreignTerm{Or}\xspace}
\newcommand{\fand}{\foreignTerm{and}\xspace}
\newcommand{\fAnd}{\foreignTerm{And}\xspace}
\newcommand{\fcentroid}{\foreignTerm{centroid}\xspace}
\newcommand{\fCentroid}{\foreignTerm{Centroid}\xspace}

\newcommand{\fDivideAndConquer}{\foreignTerm{Divide and conquer}\xspace}
\newcommand{\fdivideAndConquer}{\foreignTerm{divide and conquer}\xspace}
\newcommand{\fMergeSort}{\foreignTerm{Merge sort}\xspace}
\newcommand{\fmergeSort}{\foreignTerm{merge sort}\xspace}
\newcommand{\fQuickSort}{\foreignTerm{Quicksort}\xspace}
\newcommand{\fquickSort}{\foreignTerm{quicksort}\xspace}
\newcommand{\fpivot}{\foreignTerm{pivot}\xspace}
\newcommand{\fPivot}{\foreignTerm{Pivot}\xspace}
\newcommand{\fbruteForce}{\foreignTerm{brute force}\xspace}
\newcommand{\fBruteForce}{\foreignTerm{Brute force}\xspace}
\newcommand{\fCompleteSearch}{\foreignTerm{complete search}\xspace}
\newcommand{\fExhaustiveSearch}{\foreignTerm{exhaustive search}\xspace}
\newcommand{\fbinarySearch}{\foreignTerm{binary search}\xspace}
\newcommand{\fBinarySearch}{\foreignTerm{Binary search}\xspace}
\newcommand{\fternarySearch}{\foreignTerm{ternary search}\xspace}
\newcommand{\fTernarySearch}{\foreignTerm{Ternary search}\xspace}
\newcommand{\funimodal}{\foreignTerm{unimodal}\xspace}
\newcommand{\fUnimodal}{\foreignTerm{Unimodal}\xspace}
\newcommand{\fGreedy}{\foreignTerm{Greedy}\xspace}
\newcommand{\fgreedy}{\foreignTerm{greedy}\xspace}
\newcommand{\fgreedyChoice}{\foreignTerm{greedy choice}\xspace}
\newcommand{\fGreedyChoice}{\foreignTerm{Greedy choice}\xspace}

\newcommand{\fdp}{\foreignTerm{dynamic programming}\xspace}
\newcommand{\fDp}{\foreignTerm{Dynamic programming}\xspace}
\newcommand{\fbitmask}{\foreignTerm{bitmask}\xspace}
\newcommand{\fBitmask}{\foreignTerm{Bitmask}\xspace}
\newcommand{\fstate}{\foreignTerm{state}\xspace}
\newcommand{\fState}{\foreignTerm{State}\xspace}
\newcommand{\fsubmask}{\foreignTerm{submask}\xspace}
\newcommand{\fSubmask}{\foreignTerm{Submask}\xspace}

\newcommand{\pheap}{\foreignTerm{heap}\xspace}
\newcommand{\pHeap}{\foreignTerm{Heap}\xspace}
\newcommand{\pBinaryHeap}{\foreignTerm{Binary Heap}\xspace}
\newcommand{\pbinaryHeap}{\foreignTerm{binary heap}\xspace}
\newcommand{\pHeapsort}{\foreignTerm{Heapsort}\xspace}
\newcommand{\pheapsort}{\foreignTerm{heapsort}\xspace}
\newcommand{\pdjs}{\foreignTerm{disjoint set}\xspace}
\newcommand{\pDjs}{\foreignTerm{Disjoint set}\xspace}

\newcommand{\fdotProduct}{\foreignTerm{dot product}\xspace}
\newcommand{\fDotProduct}{\foreignTerm{Dot product}\xspace}
\newcommand{\fcrossProduct}{\foreignTerm{cross product}\xspace}
\newcommand{\fCrossProduct}{\foreignTerm{Cross product}\xspace}
\newcommand{\fconvexHull}{\foreignTerm{convex hull}\xspace}
\newcommand{\fConvexHull}{\foreignTerm{Convex hull}\xspace}
\newcommand{\fgrahamScan}{\foreignTerm{graham scan}\xspace}
\newcommand{\fGrahamScan}{\foreignTerm{Graham scan}\xspace}
\newcommand{\flineSweep}{\foreignTerm{line sweep}\xspace}
\newcommand{\fLineSweep}{\foreignTerm{Line sweep}\xspace}

\newcommand{\fset}{\foreignTerm{set}\xspace}
\newcommand{\fSet}{\foreignTerm{Set}\xspace}
\newcommand{\fprefixSum}{\foreignTerm{prefix sum}\xspace}
\newcommand{\fPrefixSum}{\foreignTerm{Prefix sum}\xspace}
\newcommand{\ffenwickTree}{\foreignTerm{fenwick tree}\xspace}
\newcommand{\fFenwickTree}{\foreignTerm{Fenwick tree}\xspace}
\newcommand{\frangeSumQuery}{\foreignTerm{range sum query}\xspace}
\newcommand{\fRangeSumQuery}{\foreignTerm{Range sum query}\xspace}
\newcommand{\fquery}{\foreignTerm{query}\xspace}
\newcommand{\fQuery}{\foreignTerm{Query}\xspace}
\newcommand{\fsegmentTree}{\foreignTerm{segment tree}\xspace}
\newcommand{\fSegmentTree}{\foreignTerm{Segment tree}\xspace}
\newcommand{\fbinaryTree}{\foreignTerm{binary tree}\xspace}
\newcommand{\fBinaryTree}{\foreignTerm{Binary tree}\xspace}
\newcommand{\flazyPropagation}{\foreignTerm{lazy propagation}\xspace}
\newcommand{\fLazyPropagation}{\foreignTerm{Lazy propagation}\xspace}
\newcommand{\fsparseTable}{\foreignTerm{sparse table}\xspace}
\newcommand{\fSparseTable}{\foreignTerm{Sparse table}\xspace}

\newcommand{\ftrail}{\foreignTerm{trail}\xspace}
\newcommand{\fTrail}{\foreignTerm{Trail}\xspace}
\newcommand{\feulerTour}{\foreignTerm{euler tour}\xspace}
\newcommand{\fEulerTour}{\foreignTerm{Euler tour}\xspace}
\newcommand{\feulerTourTree}{\foreignTerm{euler tour tree}\xspace}
\newcommand{\fEulerTourTree}{\foreignTerm{Euler tour tree}\xspace}

\newcommand{\fmaxflow}{\foreignTerm{maximum flow}\xspace}
\newcommand{\fMaxflow}{\foreignTerm{Maximum flow}\xspace}
\newcommand{\fmincut}{\foreignTerm{minimum cut}\xspace}
\newcommand{\fMincut}{\foreignTerm{Minimum cut}\xspace}
\newcommand{\fflow}{\foreignTerm{flow}\xspace}
\newcommand{\fFlow}{\foreignTerm{Flow}\xspace}
\newcommand{\fsource}{\foreignTerm{source}\xspace}
\newcommand{\fSource}{\foreignTerm{Source}\xspace}
\newcommand{\fsink}{\foreignTerm{sink}\xspace}
\newcommand{\fSink}{\foreignTerm{Sink}\xspace}
\newcommand{\fbackEdge}{\foreignTerm{back-edge}\xspace}
\newcommand{\fBackEdge}{\foreignTerm{Back-edge}\xspace}
\newcommand{\fresidualCapacity}{\foreignTerm{residual capacity}\xspace}
\newcommand{\fResidualCapacity}{\foreignTerm{Residual capacity}\xspace}
\newcommand{\fbottleneck}{\foreignTerm{bottleneck}\xspace}
\newcommand{\fBottleneck}{\foreignTerm{Bottleneck}\xspace}
\newcommand{\faugmentingPath}{\foreignTerm{augmenting path}\xspace}
\newcommand{\fAugmentingPath}{\foreignTerm{Augmenting path}\xspace}


\title{Percabangan}
\author{Tim Olimpiade Komputer Indonesia}
\date{}

\begin{document}

\begin{frame}
\titlepage
\end{frame}

\begin{frame}
\frametitle{Pendahuluan}
Melalui dokumen ini, kalian akan:
\begin{itemize}
  \item Mengenal percabangan.
  \item Analisa kasus dan mengimplementasikannya pada C++.
\end{itemize}
\end{frame}

\begin{frame}
\frametitle{Motivasi}
\begin{itemize}
  \item Bebek-bebek Pak Dengklek sedang belajar tentang membedakan bilangan positif, nol, atau negatif.
  \item Karena bebek-bebek kebingungan, mereka memberikan kalian sebuah bilangan dan meminta kalian menentukan apakah bilangan itu positif atau bukan positif!
  \item Jika positif, cetak "positif". Jika tidak, jangan cetak apa-apa.
\end{itemize}
\end{frame}

\begin{frame}
\frametitle{Motivasi (lanj.)}
\begin{itemize}
  \item Sebuah bilangan dinyatakan positif apabila bilangan tersebut lebih dari nol.
  \item Dengan begitu, kita memerlukan suatu struktur yang memungkinkan "\alert{jika} bilangan itu lebih dari 0, \alert{maka} cetak positif".
  \item Pada C++, hal ini bisa diwujudkan dengan struktur kondisional \alert{\textbf{if}}.
\end{itemize}
\end{frame}

\begin{frame}[fragile]
\frametitle{Struktur "if ... then ..."}
\begin{itemize}
  \item Struktur dari penulisan "if ... then ..." adalah:
\begin{lstlisting}
if (<kondisi>) {
  <perintah 1>;
  <perintah 2>;
  ...
}
\end{lstlisting}

  \item Dengan $<$kondisi$>$ adalah suatu boolean.
  \item Jika nilai $<$kondisi$>$ adalah \textbf{TRUE}, seluruh perintah yang ada di antara blok "\{" dan "\}" akan dilaksanakan.
  \item Jika \textbf{FALSE}, seluruh perintah yang ada di antara blok "\{" dan "\}" akan dilewati.
\end{itemize}
\end{frame}

\begin{frame}[fragile]
\frametitle{Blok "\{ ... \}"}
\begin{itemize}
  \item Struktur yang sebenarnya dari penulisan "if ... then ..." adalah:
\begin{lstlisting}
if (<kondisi>)
  <perintah>;
\end{lstlisting}
  \item Jika nilai $<$kondisi$>$ adalah \textbf{TRUE}, $<$perintah$>$ akan dilaksanakan.
  \item Jika nilai $<$kondisi$>$ adalah \textbf{FALSE}, $<$perintah$>$ tidak dilaksanakan.
  \item Lalu di mana bedanya?
\end{itemize}
\end{frame}

\begin{frame}[fragile]
\frametitle{Blok "\{ ... \}" (lanj.)}
\begin{itemize}
  \item Setelah kondisi dari if, sebenarnya hanya \emp{satu} perintah yang dapat dieksekusi jika $<$kondisi$>$ bernilai \textbf{TRUE}.
  \item Perhatikan contoh berikut:
\begin{lstlisting}
if (nilai == 10)
   printf("masuk\n");
   printf("lagi\n");
\end{lstlisting}
\end{itemize}
\end{frame}

\begin{frame}[fragile]
\frametitle{Blok "\{ ... \}" (lanj.)}
\begin{itemize}
  \item Meskipun perintah \texttt{printf("masuk$\backslash$n")} hanya dieksekusi ketika nilai sama dengan 10, \texttt{printf("lagi$\backslash$n")} akan selalu dilaksanakan tanpa peduli isi variabel nilai.
  \item Blok "\{ ... \}", akan berperan sebagai "pembungkus" beberapa perintah menjadi "satu" perintah, sehingga berapapun perintah di dalam blok tersebut, akan dilihat oleh C++ sebagai "satu" perintah.
\end{itemize}
\end{frame}

\begin{frame}[fragile]
\frametitle{Blok "\{ ... \}" (lanj.)}
\begin{itemize}
  \item Menulis "\{ ... \}" setiap sesudah if merupakan kebiasaan yang bagus, meskipun isi dari if itu hanya satu perintah.
  \item Dengan cara ini, ketika ada tambahan perintah yang perlu dimasukkan ke dalam if, kalian tidak perlu menuliskan lagi "\{ ... \}".
  \item Konsisten dengan selalu menulis "\{ ... \}" juga menjaga program tetap rapi.
\end{itemize}
\end{frame}

\begin{frame}[fragile]
\frametitle{Contoh Program: kondisi.cpp}
\begin{itemize}
  \item Ketikkan dan jalankan program berikut:
\begin{lstlisting}
#include <cstdio>

int main() {
  int x;
  scanf("%d", &x);

  if (x > 0) {
    printf("positif\n");
  }
}
\end{lstlisting}
  \item Perhatikan bahwa ekspresi "x $>$ 0" akan merupakan operasi relasional yang menghasilkan nilai boolean. Sehingga tepat untuk digunakan pada if.
  \item Bagaimana jika ingin dibuat jika bilangan itu bukan positif, cetak "non-positif"?
\end{itemize}
\end{frame}

\begin{frame}[fragile]
\frametitle{Struktur "if ... then ... else ..."}
\begin{itemize}
  \item Kita juga bisa membuat percabangan jika nilai pada $<$kondisi$>$ adalah \textbf{FALSE}, yaitu dengan kata kunci \textbf{else}.
  \item Struktur dari penulisan "if ... then ... else ..." adalah:
\begin{lstlisting}
if (<kondisi>) {
  <perintah 1>;
  <perintah 2>;
  ...
} else {
  <perintah a>;
  <perintah b>;
  ...
}
\end{lstlisting}

  \item Jika nilai $<$kondisi$>$ adalah \textbf{TRUE}, $<$perintah 1$>$, $<$perintah 2$>$, ..., akan dilaksanakan.
  \item Jika \textbf{FALSE}, $<$perintah a$>$, $<$perintah b$>$, ..., akan dilaksanakan.
\end{itemize}
\end{frame}

\begin{frame}[fragile]
\frametitle{Contoh Program: kondisi2.cpp}
\begin{itemize}
  \item Dengan "if ... then ... else ...", kita bisa memodifikasi kondisi.cpp menjadi kondisi2.cpp:
\begin{lstlisting}
#include <cstdio>

int main() {
  int x;
  scanf("%d", &x);

  if (x > 0) {
    printf("positif\n");
  } else {
    printf("non-positif\n");
  }
}
\end{lstlisting}
\end{itemize}
\end{frame}

\begin{frame}
\frametitle{Persoalan Sebenarnya}
Ketika bebek-bebek memberikan kalian sebuah bilangan, sebut saja \textbf{x}, mereka ingin tahu:
\begin{itemize}
  \item Jika \textbf{x} positif, cetak "positif".
  \item Jika \textbf{x} sama dengan nol, cetak "nol".
  \item Jika \textbf{x} negatif, cetak "negatif".
\end{itemize}

Pada kasus ini, diperlukan struktur if yang lebih dari dua cabang!
\end{frame}

\begin{frame}[fragile]
\frametitle{Struktur "if ... then ... else if ..."}
\begin{itemize}
  \item C++ menyediakan struktur yang memungkinkan kita memilah-milah untuk cabang yang lebih dari dua, yaitu dengan struktur "if ... then ... else if ...".
  \item Struktur dari penulisan "if ... then ... else if ..." adalah:
\begin{lstlisting}
if (<kondisi 1>) {
  <perintah 1>;
  <perintah 2>;
  ...
} else if (<kondisi 2>) {
  <perintah a>;
  <perintah b>;
  ...
} else if (<kondisi 3>) {
  ...
}
\end{lstlisting}

\end{itemize}
\end{frame}

\begin{frame}
\frametitle{Struktur "if ... then ... else if ..." (lanj.)}
\begin{itemize}
  \item Jika nilai $<$kondisi 1$>$ \textbf{TRUE}, $<$perintah 1$>$, $<$perintah 2$>$, ..., akan dilaksanakan.
  \item Jika nilai $<$kondisi 1$>$ \textbf{FALSE}, diperiksa apakah $<$kondisi 2$>$ bernilai \textbf{TRUE}. Jika ya, $<$perintah a$>$, $<$perintah b$>$, ..., akan dilaksanakan.
  \item Jika nilai $<$kondisi 2$>$ \textbf{FALSE}, diperiksa apakah $<$kondisi 3$>$ bernilai \textbf{TRUE}. Hal ini akan terus diulang sampai seluruh percabangan habis.
  \item Kalian juga bisa mengakhiri struktur ini dengan "else ...", yaitu ketika seluruh kondisi yang diberikan tidak terpenuhi, maka perintah-perintah di bawah \textbf{else} ini yang akan dilaksanakan.
\end{itemize}
\end{frame}

\begin{frame}[fragile]
\frametitle{Contoh Program: kondisi3.cpp}
\begin{itemize}
  \item Dengan "if ... then ... else if ...", kita bisa memodifikasi kondisi2.cpp menjadi kondisi3.cpp:
\begin{lstlisting}
#include <cstdio>

int main() {
  int x;
  scanf("%d", &x);

  if (x > 0) {
    printf("positif\n");
  } else if (x == 0) {
    printf("nol\n");
  } else if (x < 0) {
    printf("negatif\n");
  }
}
\end{lstlisting}
\end{itemize}
\end{frame}

\begin{frame}[fragile]
\frametitle{Contoh Program: kondisi4.cpp}
\begin{itemize}
  \item Pada kondisi3.cpp, sebenarnya "else if ..." yang terakhir tidak diperlukan.
  \item Ketika suatu bilangan bukan positif dan bukan nol, sudah pasti bilangan itu negatif. Sehingga bisa didapatkan kondisi4.cpp:
\begin{lstlisting}
#include <cstdio>

int main() {
  int x;
  scanf("%d", &x);

  if (x > 0) {
    printf("positif\n");
  } else if (x == 0) {
    printf("nol\n");
  } else {
    printf("negatif\n");
  }
}
\end{lstlisting}
\end{itemize}
\end{frame}

\begin{frame}[fragile]
\frametitle{Kombinasi dengan Ekspresi Boolean}
\begin{itemize}
  \item Kalian juga bisa menggabungkan struktur if dengan ekspresi \textbf{boolean}:
  \begin{lstlisting}
    if ((x > 0) && (x % 2 == 1)) {
      printf("positif dan ganjil\n");

    } else if ((x > 0) && (x % 2 == 0)) {
      printf("positif dan genap\n");

    } else if ((x < 0) && (x % 2 == 1)) {
      printf("negatif dan ganjil\n");

    } else if ((x < 0) && (x % 2 == 0)) {
      ...
  \end{lstlisting}
\end{itemize}
\end{frame}

\begin{frame}[fragile]
\frametitle{If Bersarang}
\begin{itemize}
  \item Solusi yang lebih rapi dicapai dengan menggunakan if secara bersarang:
  \begin{lstlisting}
    if (x > 0) {
      if (x % 2 == 1) {
        printf("positif dan ganjil\n");
      } else {
        printf("positif dan genap\n");
      }
    } else if (x < 0) {
      ...
  \end{lstlisting}
\end{itemize}
\end{frame}

\begin{frame}
\frametitle{Selanjutnya...}
\begin{itemize}
  \item Ke bagian yang lebih menarik lagi, yaitu perulangan!
  \item Pastikan kalian menguasai materi percabangan terlebih dahulu.
\end{itemize}
\end{frame}

\end{document}
