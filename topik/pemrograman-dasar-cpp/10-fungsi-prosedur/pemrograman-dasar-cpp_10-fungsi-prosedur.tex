\documentclass{beamer}
\usetheme{tokitex}

\usepackage{tikz}
\usepackage{graphics}
\usepackage{multirow}
\usepackage{tabto}
\usepackage{xspace}
\usepackage{amsmath}
\usepackage{hyperref}

\usepackage{tikz}
\usepackage{clrscode3e}

\usepackage[english,bahasa]{babel}
\newtranslation[to=bahasa]{Section}{Bagian}
\newtranslation[to=bahasa]{Subsection}{Subbagian}

\usepackage{listings, lstautogobble}
\usepackage{color}

\definecolor{dkgreen}{rgb}{0,0.6,0}
\definecolor{gray}{rgb}{0.5,0.5,0.5}
\definecolor{mauve}{rgb}{0.58,0,0.82}

\lstset{frame=tb,
  language=pascal,
  aboveskip=1mm,
  belowskip=1mm,
  showstringspaces=false,
  columns=fullflexible,
  keepspaces=true,
  basicstyle={\small\ttfamily},
  numbers=none,
  numberstyle=\tiny\color{gray},
  keywordstyle=\color{blue},
  commentstyle=\color{dkgreen},
  stringstyle=\color{mauve},
  breaklines=true,
  breakatwhitespace=true,
  autogobble=true
}

\usepackage{caption}
\captionsetup[figure]{labelformat=empty}

\newcommand{\progTerm}[1]{\textbf{#1}}
\newcommand{\foreignTerm}[1]{\textit{#1}}
\newcommand{\newTerm}[1]{\alert{\textbf{#1}}}
\newcommand{\emp}[1]{\alert{#1}}
\newcommand{\statement}[1]{"#1"}

% Getting tired of writing \foreignTerm all the time
\newcommand{\farray}{\foreignTerm{array}\xspace}
\newcommand{\fArray}{\foreignTerm{Array}\xspace}
\newcommand{\foverhead}{\foreignTerm{overhead}\xspace}
\newcommand{\fOverhead}{\foreignTerm{Overhead}\xspace}
\newcommand{\fsubarray}{\foreignTerm{subarray}\xspace}
\newcommand{\fSubarray}{\foreignTerm{Subarray}\xspace}
\newcommand{\fbasecase}{\foreignTerm{base case}\xspace}
\newcommand{\fBasecase}{\foreignTerm{Base case}\xspace}
\newcommand{\ftopdown}{\foreignTerm{top down}\xspace}
\newcommand{\fTopdown}{\foreignTerm{Top down}\xspace}
\newcommand{\fbottomup}{\foreignTerm{bottom up}\xspace}
\newcommand{\fBottomup}{\foreignTerm{Bottom up}\xspace}
\newcommand{\fpruning}{\foreignTerm{pruning}\xspace}
\newcommand{\fPruning}{\foreignTerm{Pruning}\xspace}

\newcommand{\fgraph}{\foreignTerm{graph}\xspace}
\newcommand{\fGraph}{\foreignTerm{Graph}\xspace}
\newcommand{\fnode}{\foreignTerm{node}\xspace}
\newcommand{\fNode}{\foreignTerm{Node}\xspace}
\newcommand{\fedge}{\foreignTerm{edge}\xspace}
\newcommand{\fEdge}{\foreignTerm{Edge}\xspace}
\newcommand{\fdegree}{\foreignTerm{degree}\xspace}
\newcommand{\fDegree}{\foreignTerm{Degree}\xspace}
\newcommand{\fadjacencylist}{\foreignTerm{adjacency list}\xspace}
\newcommand{\fAdjacencylist}{\foreignTerm{Adjacency list}\xspace}
\newcommand{\fadjacencymatrix}{\foreignTerm{adjacency matrix}\xspace}
\newcommand{\fAdjacencymatrix}{\foreignTerm{Adjacency matrix}\xspace}
\newcommand{\fedgelist}{\foreignTerm{edge list}\xspace}
\newcommand{\fEdgelist}{\foreignTerm{Edge list}\xspace}
\newcommand{\flist}{\foreignTerm{list}\xspace}
\newcommand{\fList}{\foreignTerm{List}\xspace}
\newcommand{\fgraphtraversal}{\foreignTerm{graph traversal}\xspace}
\newcommand{\fGraphtraversal}{\foreignTerm{Graph traversal}\xspace}
\newcommand{\ftree}{\foreignTerm{tree}\xspace}
\newcommand{\fTree}{\foreignTerm{Tree}\xspace}
\newcommand{\fsubtree}{\foreignTerm{subtree}\xspace}
\newcommand{\fSubtree}{\foreignTerm{Subtree}\xspace}

\newcommand{\fDivideAndConquer}{\foreignTerm{Divide and conquer}\xspace}
\newcommand{\fdivideAndConquer}{\foreignTerm{divide and conquer}\xspace}
\newcommand{\fMergeSort}{\foreignTerm{Merge sort}\xspace}
\newcommand{\fmergeSort}{\foreignTerm{merge sort}\xspace}
\newcommand{\fQuickSort}{\foreignTerm{Quicksort}\xspace}
\newcommand{\fquickSort}{\foreignTerm{quicksort}\xspace}
\newcommand{\fpivot}{\foreignTerm{pivot}\xspace}
\newcommand{\fPivot}{\foreignTerm{Pivot}\xspace}
\newcommand{\fbruteForce}{\foreignTerm{brute force}\xspace}
\newcommand{\fBruteForce}{\foreignTerm{Brute force}\xspace}
\newcommand{\fCompleteSearch}{\foreignTerm{complete search}\xspace}
\newcommand{\fExhaustiveSearch}{\foreignTerm{exhaustive search}\xspace}
\newcommand{\fBinarySearch}{\foreignTerm{binary search}\xspace}
\newcommand{\fGreedy}{\foreignTerm{greedy}\xspace}
\newcommand{\fGreedyChoice}{\foreignTerm{greedy choice}\xspace}

\newcommand{\pheap}{\foreignTerm{heap}\xspace}
\newcommand{\pHeap}{\foreignTerm{Heap}\xspace}
\newcommand{\pBinaryHeap}{\foreignTerm{Binary Heap}\xspace}
\newcommand{\pbinaryHeap}{\foreignTerm{binary heap}\xspace}
\newcommand{\pHeapSort}{\foreignTerm{Heap Sort}\xspace}
\newcommand{\pdjs}{\foreignTerm{disjoint set}\xspace}
\newcommand{\pDjs}{\foreignTerm{Disjoint set}\xspace}

\title{Subprogram}
\author{Tim Olimpiade Komputer Indonesia}
\date{}

\begin{document}

\begin{frame}
\titlepage
\end{frame}

\begin{frame}
\frametitle{Motivasi-1}
\begin{itemize}
  \item Ketika menulis program, kadang-kadang kita memerlukan suatu rutinitas yang sama di beberapa tempat.
  \item Sebagai gambaran, perhatikan contoh soal berikut:
  \begin{itemize}
    \item Pak Dengklek menancapkan tiga buah tiang pancang di halaman rumahnya untuk membangun sebuah kandang bebek.
    \item Setiap tiang pancang bisa dianggap terletak di suatu sistem koordinat Kartesius, yaitu di $(A_x, A_y)$, $(B_x, B_y)$, dan $(C_x, C_y)$.
    \item Pagar akan dibentangkan menurut garis lurus antar setiap tiang pancang.
    \item Sekarang Pak Dengklek ingin tahu berapa luas kandang bebeknya.
  \end{itemize}
  \item Persoalan yang kita hadapi adalah menghitung luas dari segitiga, jika hanya diberikan titik-titik sudutnya.
\end{itemize}
\end{frame}

\begin{frame}
\frametitle{Motivasi-1 (lanj.)}
Salah satu penyelesaian untuk soal tersebut adalah menggunakan rumus Heron:
\newline \newline
Jika sebuah segitiga memiliki panjang sisi sebesar $a$, $b$, dan $c$, maka luas segitiga tersebut adalah:
\newline
$L = \sqrt{S(S-a)(S-b)(S-c)}$, dengan $S = \frac{a+b+c}{2}$
\vfill
Bagaimana implementasinya pada program?
\end{frame}

\begin{frame}[fragile]
\frametitle{Motivasi-1 (lanj.)}
\begin{itemize}
  \item Kita perlu menghitung jarak antar titik terlebih dahulu, baru bisa menghitung luasnya.
  \item Kita dapat menuliskannya:
\begin{lstlisting}
// tA, tB, tC merupakan ketiga titik yang diberikan
a = sqrt((tA.x - tB.x)*(tA.x - tB.x) + (tA.y - tB.y)*(tA.y - tB.y));
b = sqrt((tA.x - tC.x)*(tA.x - tC.x) + (tA.y - tC.y)*(tA.y - tC.y));
c = sqrt((tB.x - tC.x)*(tB.x - tC.x) + (tB.y - tC.y)*(tB.y - tC.y));
\end{lstlisting}
  \item Perhatikan bahwa hal yang sama, yaitu menghitung jarak titik, dituliskan secara berulang-ulang.
  \item Bagaimana jika pada salah satunya terdapat kesalahan pengetikan? Atau suatu ketika rumusnya perlu diubah? Sungguh merepotkan!
\end{itemize}
\end{frame}

\begin{frame}[fragile]
\frametitle{Motivasi-2}
\begin{itemize}
  \item Seringkali ketika kita menulis program yang panjang, program menjadi lebih sulit dipahami, meskipun telah ditulis komentar sekalipun.
  \item Alangkah baiknya jika kita bisa membuat subprogram untuk suatu rutinitas tertentu dan menyatukannya di akhir, seperti:
\begin{lstlisting}
bacaMasukan(N);
cariPrimaSampai(N);
printf("faktorisasi:\n");
temp = N;
while (!cekPrima(temp)) {
  d = cariPembagiTerkecil(temp);
  temp = temp / d;
  printf("%d\n", d);
}
if (temp > 1) {
  printf("%d\n", temp);
}
\end{lstlisting}
\end{itemize}
\end{frame}

\section{Konsep Subprogram}
\frame{\sectionpage}

\begin{frame}
\frametitle{Konsep Subprogram}
\begin{itemize}
  \item Sesuai dengan namanya, subprogram adalah bagian dari program.
  \item Jika program merupakan serangkaian instruksi untuk mencapai suatu tujuan tertentu, maka subprogram bisa dianggap sebagai serangkaian instruksi untuk mencapai suatu tujuan tertentu, \alert{sebagai bagian dari tujuan program}.
  \item Subprogram bisa dipanggil di bagian manapun pada program.
  \item Pada C++, subprogram disebut dengan \emp{fungsi}.
\end{itemize}
\end{frame}

\begin{frame}[fragile]
\frametitle{Contoh Subprogram}
\begin{itemize}
  \item Kita bisa memindahkan serangkaian kode menjadi sebuah fungsi, lalu memanggilnya pada program utama.
  \item Perhatikan contoh pesan.cpp berikut!
\begin{lstlisting}
#include <cstdio>
#include <string>
using namespace std;

char buff[1001];
string pesan;

// Subprogram
void bacaPesan() {
  printf("masukkan pesan: \n");
  scanf("%s", buff);
  pesan = buff;
}

// Program utama
int main() {
  bacaPesan();
  printf("pesan = %s\n", pesan.c_str());
}
\end{lstlisting}
\end{itemize}
\end{frame}

\begin{frame}[fragile]
\frametitle{Penjelasan}
\begin{itemize}
  \item Pada pesan.cpp, terdapat sebuah fungsi bernama \texttt{bacaPesan} yang melakukan perintah untuk membaca masukan.
  \item Ketika \texttt{bacaPesan} dipanggil pada program utama, bisa dianggap seluruh instruksi yang ada di dalam fungsi tersebut dipindahkan ke program utama yang memanggilnya.
  \item Sehingga, program utama seakan-akan menjadi:
\begin{lstlisting}
int main() {
  printf("masukkan pesan: \n");
  scanf("%s", buff);
  pesan = buff;
  printf("pesan = %s\n", pesan.c_str());
}
\end{lstlisting}
\end{itemize}
\end{frame}

\begin{frame}[fragile]
\frametitle{Penjelasan (lanj.)}
\begin{itemize}
  \item Tentu saja, sebuah fungsi bisa dipanggil berkali-kali, dan hal yang dilakukan tetap sama.
  \item Coba modifikasi blok progam utama pesan.cpp menjadi:
\begin{lstlisting}
int main() {
  bacaPesan();
  printf("pesan = %s\n", pesan.c_str());

  bacaPesan();
  printf("sekarang pesan berisi = %s\n", pesan.c_str());
}
\end{lstlisting}
\end{itemize}
\end{frame}

\section{Implementasi pada C++}
\frame{\sectionpage}

\begin{frame}[fragile]
\frametitle{Fungsi}
Pada C++, fungsi bisa ditulis dengan format berikut:
\begin{lstlisting}
<tipe> <nama>(<parameter...>) {
  <instruksi...>
}
\end{lstlisting}
\begin{itemize}
  \item $<$tipe$>$: nilai kembalian yang dihasilkan fungsi. Untuk fungsi yang tidak menghasilkan nilai kembalian, kita gunakan tipe \texttt{void}. Fungsi yang menghasilkan nilai kembalian akan dipelajari di bagian selanjutnya.
  \item $<$nama$>$: nama dari fungsi.
  \item $<$parameter$>$: informasi yang hendak diberikan ke fungsi.
\end{itemize}
\end{frame}

\begin{frame}[fragile]
\frametitle{Konsep Parameter}
Parameter merupakan tempat untuk "memberi masukan" bagi fungsi, sehingga fungsi bisa berperilaku berdasarkan masukan yang diterima.
\vfill
Perhatikan contoh berikut:
\begin{lstlisting}
void gambar(int x) {
  for (int i = 0; i < x; i++) {
    printf("*");
  }
  printf("\n");
}
\end{lstlisting}
\end{frame}

\begin{frame}[fragile]
\frametitle{Konsep Parameter (lanj.)}
\begin{itemize}
  \item Fungsi \texttt{gambar} berfungsi untuk menuliskan karakter '*' pada sebuah baris sebanyak $x$ kali.
  \item Lalu apa $x$ pada fungsi tersebut?
  \item Untuk menjawabnya, perhatikan blok program utama berikut:
\begin{lstlisting}
// Program utama
int main() {
  gambar(3);
  gambar(5);
}
\end{lstlisting}
  \item Yang akan tercetak adalah:
\begin{lstlisting}
***
*****
\end{lstlisting}
\end{itemize}
\end{frame}

\begin{frame}[fragile]
\frametitle{Konsep Parameter (lanj.)}
\begin{itemize}
  \item Pada pemanggilan pertama, angka 3 pada \texttt{gambar(3)} mengakibatkan nilai $x$ untuk fungsi \texttt{gambar} bernilai 3. Sehingga tercetak 3 karakter '*'.
  \item Pada pemanggilan kedua, nilai $x$ yang diterima adalah 5. Sehingga tercetak 5 karakter '*'.
  \item Variabel $x$ pada fungsi \texttt{gambar} disebut sebagai \emp{parameter}.
  \item Melalui contoh ini, kalian dapat memahami bahwa nilai parameter dapat digunakan untuk mengatur perilaku fungsi.
\end{itemize}
\end{frame}

\begin{frame}[fragile]
\frametitle{Konsep Parameter (lanj.)}
\begin{itemize}
\item Tentu saja, pemanggilan juga bisa dilakukan dengan variabel seperti contoh berikut:
\begin{lstlisting}
int main() {
  int n;
  scanf("%d", &n);
  gambar(n);
}
\end{lstlisting}
\end{itemize}
\end{frame}

\begin{frame}[fragile]
\frametitle{Parameter}
\begin{itemize}
  \item Parameter dituliskan dengan format tipe dan namanya.
  \item Suatu fungsi boleh memiliki beberapa parameter.
  \item Contoh:
\begin{lstlisting}
// Tanpa parameter
void baca()

// Satu parameter
void tes(int x)

// Dua parameter
void sama(int x, int y)

// Dua parameter, berbeda tipe data
void berbeda(int x, string y)
\end{lstlisting}
\end{itemize}
\end{frame}

\begin{frame}[fragile]
\frametitle{Lingkup Variabel}
\begin{itemize}
  \item Perhatikan kembali fungsi \texttt{gambar} berikut:
\begin{lstlisting}
void gambar(int x) {
  for (int i = 0; i < x; i++) {
    printf("*");
  }
  printf("\n");
}

\end{lstlisting}
  \item Variabel \texttt{i} dan \texttt{x} pada fungsi tersebut hanya terdefinisi di antara blok \texttt{\{} dan \texttt{\}} fungsi \texttt{gambar} saja.
  \item Artinya jika pada program utama terdapat pula variabel bernama \texttt{x} atau \texttt{i}, maka variabel tersebut \alert{bukan} mengacu pada \texttt{x} dan \texttt{i} pada fungsi \texttt{gambar}.
\end{itemize}
\end{frame}

\begin{frame}[fragile]
\frametitle{Lingkup Variabel (lanj.)}
\begin{itemize}
  \item Variabel yang dideklarasikan di dalam fungsi biasa disebut sebagai \emp{variabel lokal}.
  \item Sementara variabel yang dideklarasikan di luar fungsi atau program utama disebut sebagai \emp{variabel global}.
  \item Variabel global dapat diakses di mana saja, bahkan di dalam subprogram sekalipun.
  \item Variabel lokal hanya bisa diakses pada subprogram yang mendeklarasikannya.
\end{itemize}
\end{frame}

\begin{frame}[fragile]
\frametitle{Fungsi dengan Nilai Kembalian}
\begin{itemize}
  \item Setelah kalian memahami tentang fungsi, mari kita bahas tentang nilai kembalian.
  \item Perhatikan fungsi berikut:
\begin{lstlisting}
int kubik(int x) {
  return x*x*x;
}
\end{lstlisting}
\end{itemize}
\end{frame}

\begin{frame}[fragile]
\frametitle{Penjelasan}
\begin{itemize}
  \item Pada program utama, kita bisa melakukan:
\begin{lstlisting}
int main() {
  int volume = kubik(3);
  int selisih = volume - kubik(2);
  printf("4 kubik adalah %d\n", kubik(4));
}
\end{lstlisting}
  \item Teringat dengan sesuatu?
  \item Fungsi \texttt{kubik} kini terlihat seperti fungsi yang biasa kalian gunakan, seperti \texttt{sqrt}, \texttt{round}, atau \texttt{abs}!
\end{itemize}
\end{frame}

\begin{frame}[fragile]
\frametitle{Nilai Kembalian}
\begin{itemize}
  \item Ketika fungsi mengembalikan nilai, nilai ini bisa dioperasikan ke dalam ekspresi atau assignment.
  \item Sebagai ilustrasi, perhatikan ekspresi berikut:
\begin{lstlisting}
x = 2;
y = 3*kubik(x) - 1;
\end{lstlisting}
  \item Pada saat dijalankan, fungsi kubik(x) akan dieksekusi dan \emp{mengembalikan nilai} 8.
\begin{lstlisting}
y := 3*8 - 1;
\end{lstlisting}
  \item Setelah ekspresi itu dievaluasi, \texttt{y} bernilai 23.
\end{itemize}
\end{frame}

\begin{frame}[fragile]
\frametitle{Nilai Kembalian (lanj.)}
\begin{itemize}
  \item Hal semacam ini tidak berlaku ketika fungsi tidak mengembalikan nilai.
  \item Hal ini juga membedakan cara pemanggilannya:
\begin{lstlisting}
// Tidak mengembalikan nilai
kerja1()

// Mengembalikan nilai
x = kerja2();
\end{lstlisting}
\end{itemize}
\end{frame}

\begin{frame}[fragile]
\frametitle{Fungsi}
\begin{itemize}
  \item Untuk mengembalikan nilai, isikan tipe data nilai kembalian pada deklarasi fungsi.
  \item Selanjutnya, kembalikan nilai dengan cara menuliskan "\texttt{return}" diikuti dengan nilai kembaliannya.
  \item Perintah \texttt{return} akan menghentikan eksekusi, keluar dari fungsi, dan mengembalikan nilai ke pemanggilnya.
  \item Perintah \texttt{return} boleh dipanggil di baris fungsi manapun.
\end{itemize}
\end{frame}

\begin{frame}[fragile]
\frametitle{Contoh Fungsi Lain}
\begin{itemize}
  \item Perhatikan fungsi yang memeriksa keprimaan berikut:
\end{itemize}
\begin{lstlisting}
bool prima(int x) {
  if (x < 2) {
    return false;
  }
  for (int i = 2; i*i <= x; i++) {
    if (x % i == 0) {
      return false;
    }
  }
  return true;
}
\end{lstlisting}
\begin{itemize}
  \item Pertama, periksa apakah bilangan yang diberikan kurang dari dua. Bila ya, langsung kembalikan nilai \texttt{FALSE}.
  \item Kedua, periksa apakah ada angka di antara 2 dan $\sqrt{x}$ yang habis membali x. Bila ada, langsung kembalikan \texttt{FALSE}.
  \item Selain daripada itu, dijamin x prima.
\end{itemize}
\end{frame}

\begin{frame}[fragile]
\frametitle{Return pada Fungsi void}
\begin{itemize}
  \item Sebenarnya, \texttt{return} juga dapat dilakukan pada fungsi yang tidak mengembalikan nilai.
  \item Perintah \texttt{return} akan menghentikan eksekusi dan keluar dari program.  
  \item Pada contoh berikut, gambar '*' tidak akan dicetak apabila x lebih dari 1000.
\end{itemize}
\begin{lstlisting}
void gambar(int x) {
  if (x > 1000) {
    return;
  }
  for (int i = 0; i < x; i++) {
    printf("*");
  }
  printf("\n");
}
\end{lstlisting}
\end{frame}

\section{Passing Parameter}
\frame{\sectionpage}

\begin{frame}
\frametitle{Passing Parameter}
\begin{itemize}
  \item \textit{Passing parameter} merupakan aktivitas menyalurkan nilai pada parameter saat memanggil subprogram.
  \item Umumnya, dikenal dua macam \textit{passing parameter}:
  \begin{itemize}
    \item \textit{By value}, yaitu mengirimkan \alert{nilai} dari setiap parameter yang diberikan.
    \item \textit{By reference}, yaitu mengirimkan \alert{alamat} dari setiap parameter yang diberikan.
  \end{itemize}
\end{itemize}
\end{frame}

\begin{frame}[fragile]
\frametitle{Passing Parameter by Value}
\begin{itemize}
  \item Sebagai penjelasan, perhatikan program berikut:
\begin{lstlisting}
#include <cstdio>

void tukar(int a, int b) {
  int temp = a;
  a = b;
  b = temp;
}

int main() {
  int x = 1;
  int y = 2;
  tukar(x, y);
  printf("x=%d y=%d\n", x, y);
}
\end{lstlisting}
\end{itemize}
\end{frame}

\begin{frame}[fragile]
\frametitle{Passing Parameter by Value (lanj.)}
\begin{itemize}
  \item Jika dijalankan, apa keluaran dari program tersebut? Apakah "x=2 y=1"?
  \item Jawabannya tidak, yang tercetak adalah "x=1 y=2".
  \item Ketika fungsi \texttt{tukar} dipanggil, nilai dari $x$ dan $y$ dikirim ke parameter $a$ dan $b$ pada fungsi \texttt{tukar}.
  \item Jadi hanya dilakukan \foreignTerm{assignment} nilai dari $x$ dan $y$ ke $a$ dan $b$. \item Apapun yang terjadi pada $a$ dan $b$ selanjutnya tidak mempengaruhi $x$ dan $y$ karena mereka \emp{tidak berhubungan}.
\end{itemize}
\end{frame}

\begin{frame}[fragile]
\frametitle{Passing Parameter by Reference}
\begin{itemize}
  \item Lain halnya ketika kita menambahkan lambang \textbf{\&} pada penulisan parameter:
\begin{lstlisting}
void tukar(int &a, int &b) {
  int temp = a;
  a = b;
  b = temp;
}
\end{lstlisting}
\end{itemize}
\end{frame}

\begin{frame}[fragile]
\frametitle{Passing Parameter by Reference (lanj.)}
\begin{itemize}
  \item Dengan cara ini, ketika \texttt{tukar}($x$, $y$) dipanggil, \emp{alamat memori variabel} $x$ dan $y$ dikirimkan ke parameter $a$ dan $b$.
  \item Kini, $x$ dan $a$ mengacu pada alamat memori yang sama.
  \item Apabila dilakukan perintah "$a = 3$", maka nilai $x$ juga ikut menjadi 3. Sebab $x$ dan $a$ mengacu pada alamat memori yang sama.
  \item Demikian pula untuk $y$ dan $b$.
  \item Sehingga keluaran program menjadi "x=2 y=1"!
\end{itemize}
\end{frame}

\begin{frame}[fragile]
\frametitle{Passing Parameter by Reference (lanj.)}
\begin{itemize}
  \item Jika \texttt{tukar} ditulis dengan \foreignTerm{passing parameter by reference}, kita tidak bisa melakukan:
\begin{lstlisting}
tukar(2, 3);
\end{lstlisting}

  \item Mengirimkan alamat memori dari variabel memang bisa dilakukan, tetapi angka 2 atau 3 jelas bukan variabel dan jelas tidak punya alamat memori.
  \item Jika kita mengembalikan kedua parameter fungsi \texttt{tukar} untuk menggunakan \foreignTerm{passing parameter by value}, barulah hal ini bisa dilakukan.
\end{itemize}
\end{frame}

\begin{frame}[fragile]
\frametitle{Penulisan pada C++}
\begin{itemize}
  \item Untuk melakukan \foreignTerm{passing parameter by reference}, cukup tambahkan \& di depan parameter yang akan di-\textit{pass-by-reference}.
  \item Sebuah subprogram bisa juga menerima parameter dengan cara \textit{passing parameter} yang campuran:
\begin{lstlisting}
void bagi(int a, int b, int &hasil, int &sisa) {
  hasil = a / b;
  sisa = a % b;
}
\end{lstlisting}
\end{itemize}
\end{frame}

\section{Studi Kasus Subprogram}
\frame{\sectionpage}

\begin{frame}[fragile]
\frametitle{Fungsi Pangkat (int)}
Berikut ini adalah fungsi untuk menghitung $a^b$:
\begin{lstlisting}
int pangkat(int a, int b) {
  int hasil = 1;
  for (int i = 0; i < b; i++) {
    hasil *= a;
  }
  return hasil;
}
\end{lstlisting}
\end{frame}

\begin{frame}[fragile]
\frametitle{Fungsi Pangkat (void)}
Berikut ini adalah fungsi untuk menghitung $a^b$, hasil ditampung pada variabel $hasil$:
\begin{lstlisting}
void pangkat(int a, int b, int &hasil) {
  hasil = 1;
  for (int i = 0; i < b; i++) {
    hasil *= a;
  }
}
\end{lstlisting}
\end{frame}

\begin{frame}[fragile]
\frametitle{Mengembalikan Nilai atau Tidak}
\begin{itemize}
  \item Baik dengan kedua cara, kita bisa mencapai hal yang sama.
  \item Pertanyaannya adalah: \alert{mana yang lebih tepat}?
\end{itemize}
\end{frame}

\begin{frame}[fragile]
\frametitle{Mengembalikan Nilai atau Tidak (lanj.)}
\begin{itemize}
  \item Dengan mengembalikan nilai, menghitung $y = 3x^5$ bisa dilakukan dengan:
\begin{lstlisting}
y = 3 * pangkat(x, 5);
\end{lstlisting}

  \item Tanpa mengembalikan nilai, sedikit lebih rumit:
\begin{lstlisting}
pangkat(x, 5, y);
y = 3 * y;
\end{lstlisting}

  \item Untuk kasus ini, \alert{mengembalikan nilai lebih tepat}.
\end{itemize}
\end{frame}

\begin{frame}
\frametitle{Kilas Balik: Fungsi tukar}
\begin{itemize}
  \item Sekarang coba ingat kembali fungsi \texttt{tukar} yang kita bahas sebelumnya.
  \item Kurang masuk akal apabila kita menggunakan mengembalikan nilai saat melakukan penukaran.
  \item Sehingga untuk kasus penukaran isi variabel, \alert{tanpa mengembalikan nilai lebih tepat}.
\end{itemize}
\end{frame}

\begin{frame}
\frametitle{Penggunaan Subprogram}
\begin{itemize}
  \item Dari sini kita mempelajari bahwa ada subprogram yang lebih cocok diimplementasikan mengembalikan nilai atau tidak.
  \item Biasanya, yang mengembalikan nilai bersifat:
  \begin{itemize}
    \item Menghasilkan suatu nilai berdasarkan parameter.
    \item Tidak mengakibatkan efek samping, misalnya adanya perubahan nilai pada parameter yang diberikan seperti pada fungsi \texttt{tukar}.
  \end{itemize}
  \item Sementara yang tidak mengembalikan nilai bersifat:
  \begin{itemize}
    \item Tidak menghasilkan suatu nilai berdasarkan parameter.
    \item Boleh jadi mengakibatkan perubahan pada variabel global atau parameter yang dikirimkan.
  \end{itemize}
\end{itemize}
\end{frame}

\begin{frame}
\frametitle{Manfaat Subprogram}
\begin{itemize}
  \item Meningkatkan daya daur ulang kode (\textit{reusability}).
  \newline Satu kali saja kita mendefinisikan subprogram untuk menukar isi variabel, berapa kali pun penukaran isi variabel bisa dilakukan tanpa perlu menuliskan algoritma penukaran lagi.
  \item Memecah program menjadi beberapa subprogram yang lebih kecil.
  \newline Keuntungannya adalah didapatkan kumpulan subprogram yang:
  \begin{itemize}
    \item Fokus pada suatu tujuan tertentu.
    \item Tersusun atas kode yang cenderung pendek.
    \item Karena kedua hal di atas, lebih mudah dibaca dan ditelusuri.
  \end{itemize}
\end{itemize}
\end{frame}


\end{document}
