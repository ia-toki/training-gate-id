\documentclass{beamer}
\usetheme{tokitex}

\usepackage{tikz}
\usepackage{graphics}
\usepackage{multirow}
\usepackage{tabto}
\usepackage{xspace}
\usepackage{amsmath}
\usepackage{hyperref}

\usepackage{tikz}
\usepackage{clrscode3e}

\usepackage[english,bahasa]{babel}
\newtranslation[to=bahasa]{Section}{Bagian}
\newtranslation[to=bahasa]{Subsection}{Subbagian}

\usepackage{listings, lstautogobble}
\usepackage{color}

\definecolor{dkgreen}{rgb}{0,0.6,0}
\definecolor{gray}{rgb}{0.5,0.5,0.5}
\definecolor{mauve}{rgb}{0.58,0,0.82}

\lstset{frame=tb,
  language=pascal,
  aboveskip=1mm,
  belowskip=1mm,
  showstringspaces=false,
  columns=fullflexible,
  keepspaces=true,
  basicstyle={\small\ttfamily},
  numbers=none,
  numberstyle=\tiny\color{gray},
  keywordstyle=\color{blue},
  commentstyle=\color{dkgreen},
  stringstyle=\color{mauve},
  breaklines=true,
  breakatwhitespace=true,
  autogobble=true
}

\usepackage{caption}
\captionsetup[figure]{labelformat=empty}

\newcommand{\progTerm}[1]{\textbf{#1}}
\newcommand{\foreignTerm}[1]{\textit{#1}}
\newcommand{\newTerm}[1]{\alert{\textbf{#1}}}
\newcommand{\emp}[1]{\alert{#1}}
\newcommand{\statement}[1]{"#1"}

% Getting tired of writing \foreignTerm all the time
\newcommand{\farray}{\foreignTerm{array}\xspace}
\newcommand{\fArray}{\foreignTerm{Array}\xspace}
\newcommand{\foverhead}{\foreignTerm{overhead}\xspace}
\newcommand{\fOverhead}{\foreignTerm{Overhead}\xspace}
\newcommand{\fsubarray}{\foreignTerm{subarray}\xspace}
\newcommand{\fSubarray}{\foreignTerm{Subarray}\xspace}
\newcommand{\fbasecase}{\foreignTerm{base case}\xspace}
\newcommand{\fBasecase}{\foreignTerm{Base case}\xspace}
\newcommand{\ftopdown}{\foreignTerm{top down}\xspace}
\newcommand{\fTopdown}{\foreignTerm{Top down}\xspace}
\newcommand{\fbottomup}{\foreignTerm{bottom up}\xspace}
\newcommand{\fBottomup}{\foreignTerm{Bottom up}\xspace}
\newcommand{\fpruning}{\foreignTerm{pruning}\xspace}
\newcommand{\fPruning}{\foreignTerm{Pruning}\xspace}

\newcommand{\fgraph}{\foreignTerm{graph}\xspace}
\newcommand{\fGraph}{\foreignTerm{Graph}\xspace}
\newcommand{\fnode}{\foreignTerm{node}\xspace}
\newcommand{\fNode}{\foreignTerm{Node}\xspace}
\newcommand{\fedge}{\foreignTerm{edge}\xspace}
\newcommand{\fEdge}{\foreignTerm{Edge}\xspace}
\newcommand{\fdegree}{\foreignTerm{degree}\xspace}
\newcommand{\fDegree}{\foreignTerm{Degree}\xspace}
\newcommand{\fadjacencylist}{\foreignTerm{adjacency list}\xspace}
\newcommand{\fAdjacencylist}{\foreignTerm{Adjacency list}\xspace}
\newcommand{\fadjacencymatrix}{\foreignTerm{adjacency matrix}\xspace}
\newcommand{\fAdjacencymatrix}{\foreignTerm{Adjacency matrix}\xspace}
\newcommand{\fedgelist}{\foreignTerm{edge list}\xspace}
\newcommand{\fEdgelist}{\foreignTerm{Edge list}\xspace}
\newcommand{\flist}{\foreignTerm{list}\xspace}
\newcommand{\fList}{\foreignTerm{List}\xspace}
\newcommand{\fgraphtraversal}{\foreignTerm{graph traversal}\xspace}
\newcommand{\fGraphtraversal}{\foreignTerm{Graph traversal}\xspace}
\newcommand{\ftree}{\foreignTerm{tree}\xspace}
\newcommand{\fTree}{\foreignTerm{Tree}\xspace}
\newcommand{\fsubtree}{\foreignTerm{subtree}\xspace}
\newcommand{\fSubtree}{\foreignTerm{Subtree}\xspace}

\newcommand{\fDivideAndConquer}{\foreignTerm{Divide and conquer}\xspace}
\newcommand{\fdivideAndConquer}{\foreignTerm{divide and conquer}\xspace}
\newcommand{\fMergeSort}{\foreignTerm{Merge sort}\xspace}
\newcommand{\fmergeSort}{\foreignTerm{merge sort}\xspace}
\newcommand{\fQuickSort}{\foreignTerm{Quicksort}\xspace}
\newcommand{\fquickSort}{\foreignTerm{quicksort}\xspace}
\newcommand{\fpivot}{\foreignTerm{pivot}\xspace}
\newcommand{\fPivot}{\foreignTerm{Pivot}\xspace}
\newcommand{\fbruteForce}{\foreignTerm{brute force}\xspace}
\newcommand{\fBruteForce}{\foreignTerm{Brute force}\xspace}
\newcommand{\fCompleteSearch}{\foreignTerm{complete search}\xspace}
\newcommand{\fExhaustiveSearch}{\foreignTerm{exhaustive search}\xspace}
\newcommand{\fBinarySearch}{\foreignTerm{binary search}\xspace}
\newcommand{\fGreedy}{\foreignTerm{greedy}\xspace}
\newcommand{\fGreedyChoice}{\foreignTerm{greedy choice}\xspace}

\newcommand{\pheap}{\foreignTerm{heap}\xspace}
\newcommand{\pHeap}{\foreignTerm{Heap}\xspace}
\newcommand{\pBinaryHeap}{\foreignTerm{Binary Heap}\xspace}
\newcommand{\pbinaryHeap}{\foreignTerm{binary heap}\xspace}
\newcommand{\pHeapSort}{\foreignTerm{Heap Sort}\xspace}
\newcommand{\pdjs}{\foreignTerm{disjoint set}\xspace}
\newcommand{\pDjs}{\foreignTerm{Disjoint set}\xspace}

\title{Perulangan Lanjut}
\author{Tim Olimpiade Komputer Indonesia}
\date{}

\begin{document}

\begin{frame}
\titlepage
\end{frame}

\begin{frame}
\frametitle{Pendahuluan}
Melalui dokumen ini, kalian akan:
\begin{itemize}
  \item Memahami penggunaan perulangan yang bersarang.
  \item Memecahkan beberapa persoalan dengan perulangan.
\end{itemize}
\end{frame}

\section{Perulangan Bersarang}
\frame{\sectionpage}

\begin{frame}[fragile]
\frametitle{Motivasi: Pola 0}
\begin{itemize}
  \item Pak Dengklek akan memberikan sebuah bilangan, misalnya \textbf{N}.
  \item Anda diminta untuk mencetak karakter bintang (*) yang tersusun N baris.
  \item Contoh untuk N = 3:
\begin{lstlisting}
*
*
*
\end{lstlisting}
\end{itemize}
\end{frame}

\begin{frame}[fragile]
\frametitle{Motivasi (lanj.)}
\begin{itemize}
  \item Tentu saja solusinya sederhana, cukup gunakan salah satu struktur perulangan yang kalian kuasai.
  \item Misalnya menggunakan for:
\begin{lstlisting}
for (int i = 0; i < N; i++) {
  printf("\n");
}
\end{lstlisting}
\end{itemize}
\end{frame}

\begin{frame}[fragile]
\frametitle{Motivasi: Pola 1}
\begin{itemize}
  \item Kemudian Pak Dengklek memberikan persoalan yang sedikit lebih sulit.
  \item Diberikan dua bilangan, misalnya N dan M.
  \item Cetak karakter bintang (*) yang tersusun N baris dan M kolom!

  \item Contoh untuk N = 3 dan M = 5:
\begin{lstlisting}
*****
*****
*****
\end{lstlisting}

  \item Kali ini, untuk setiap barisnya kita perlu melakukan perulangan untuk mencetak M karakter bintang!
\end{itemize}
\end{frame}

\begin{frame}[fragile]
\frametitle{Contoh Program: pola1\_1.cpp}
\begin{itemize}
  \item Kita bisa membuat "for di dalam for", sehingga membentuk struktur yang bersarang.
\begin{lstlisting}
#include <cstdio>

int main() {
  int N, M;
  scanf("%d %d", &N, &M);

  for (int i = 0; i < N; i++) {
    for (int j = 0; j < M; j++) {
      printf("*");
    }
    printf("\n");
  }
}
\end{lstlisting}
\end{itemize}
\end{frame}

\begin{frame}[fragile]
\frametitle{Contoh Program: pola1\_2.cpp}
\begin{itemize}
  \item Tentu saja kita bisa melakukannya dengan struktur perulangan yang lain:
\begin{lstlisting}
#include <cstdio>

int main() {
  int N, M;
  scanf("%d %d", &N, &M);

  int i = 0;
  while (i < N) {
    int j = 0;
    while (j < M) {
      printf("*");
      j++;
    }
    printf("\n");
    i++;
  }
}
\end{lstlisting}
\end{itemize}
\end{frame}

\begin{frame}[fragile]
\frametitle{Contoh Lain: Pola 2}
\begin{itemize}
  \item Soal "Pola 1" dapat diselesaikan dengan mudah. Dengan demikian Pak Dengklek memberikan soal yang lebih menantang.
  \item Diberikan sebuah bilangan, misalnya N.
  \item Cetak "struktur segitiga rata kiri" yang terdiri dari N baris.
  \item Misalnya untuk N = 5, hasilnya adalah:
\begin{lstlisting}
*
**
***
****
*****
\end{lstlisting}
\end{itemize}
\end{frame}

\begin{frame}[fragile]
\frametitle{Contoh Solusi: pola2.cpp}
\begin{itemize}
  \item Berikut ini adalah contoh solusinya, dimodifikasi dari pola1\_1.cpp:
\begin{lstlisting}
#include <cstdio>

int main() {
  int N;
  scanf("%d", &N);

  for (int i = 0; i < N; i++) {
    for (int j = 0; j <= i; j++) {
      printf("*");
    }
    printf("\n");
  }
}
\end{lstlisting}
\end{itemize}
\end{frame}

\begin{frame}[fragile]
\frametitle{Latihan: Pola 3}
\begin{itemize}
  \item Pak Dengklek kemudian memberikan tugas yang lebih sulit lagi, yang kali ini perlu Anda kerjakan sendiri.
  \item Diberikan sebuah bilangan, misalnya N.
  \item Cetak "struktur segitiga rata kanan" yang terdiri dari N baris.
  \item Misalnya untuk N = 5, hasilnya adalah:
\begin{lstlisting}
    *
   **
  ***
 ****
*****
\end{lstlisting}
  \item Petunjuk: cetak bagian kiri terlebih dahulu!
\end{itemize}
\end{frame}

\section{Break \& Continue}
\frame{\sectionpage}

\begin{frame}
\frametitle{Break \& Continue}
\begin{itemize}
  \item Kadang kala, kita membutuhkan suatu perulangan untuk diberhentikan secara paksa atau lompat ke iterasi berikutnya.
  \item C++ menyediakan kedua fitur tersebut, yaitu dengan kata kunci \alert{break} dan \alert{continue}.
\end{itemize}
\end{frame}

\begin{frame}
\frametitle{Break \& Continue (lanj.)}
\begin{block}{Break}
Penggunaan \textbf{break} akan membuat program keluar dari perulangan yang mengandung kata kunci tersebut.
\end{block}
\begin{block}{Continue}
Penggunaan \textbf{continue} akan membuat program kembali ke baris awal perulangan, yaitu baris "for", atau "while".
\end{block}
\end{frame}

\begin{frame}[fragile]
\frametitle{Contoh Soal: Berhitung 1}
\begin{itemize}
  \item Setelah mahir dalam menggambar pola, kini Pak Dengklek ingin mengajar tentang berhitung.
  \item Pak Dengklek akan memberikan dua bilangan, yaitu N dan M.
  \item Anda diminta untuk menuliskan bilangan dari 1 sampai dengan N. Namun, ketika bilangan yang hendak ditulis adalah M, jangan cetak bilangan itu dan jangan cetak bilangan apapun lagi.
  \item Setelah selesai mencetak bilangan, cetak "selesai".
  \item Contoh untuk N = 10 dan M = 5:
\begin{lstlisting}
1
2
3
4
selesai
\end{lstlisting}
\end{itemize}
\end{frame}

\begin{frame}[fragile]
\frametitle{Contoh Program: break.cpp}
\begin{itemize}
  \item Berikut ini adalah contoh solusi dari soal "Berhitung 1".
\begin{lstlisting}
#include <cstdio>

int main() {
  int N, M;
  scanf("%d %d", &N, &M);

  for (int i = 1; i <= N; i++) {
    if (i == M) {
      break;
    }

    printf("%d\n", i);
  }
  printf("selesai\n");
}
\end{lstlisting}
\end{itemize}
\end{frame}

\begin{frame}
\frametitle{Penjelasan Program: break.cpp}
\begin{itemize}
  \item Ketika \textit{break} ditemui, perulangan "for" akan diberhentikan secara paksa dan lanjut mengeksekusi perintah selanjutnya, yaitu mencetak tulisan "selesai".
\end{itemize}
\end{frame}

\begin{frame}[fragile]
\frametitle{Contoh Soal: Berhitung 2}
\begin{itemize}
  \item Kali ini Pak Dengklek mengubah soalnya: diberikan dua bilangan, yaitu N dan M.
  \item Anda diminta untuk menuliskan bilangan dari 1 sampai dengan N. Namun, ketika bilangan yang hendak ditulis adalah \emp{kelipatan} dari M, jangan cetak bilangan itu.
  \item Setelah selesai mencetak bilangan, cetak "selesai".
  \item Contoh untuk N = 10 dan M = 2:
\begin{lstlisting}
1
3
5
7
9
selesai
\end{lstlisting}
\end{itemize}
\end{frame}

\begin{frame}[fragile]
\frametitle{Contoh Program: continue.cpp}
\begin{itemize}
  \item Berikut ini adalah contoh solusi dari soal "Berhitung 2".
\begin{lstlisting}
#include <cstdio>

int main() {
  int N, M;
  scanf("%d %d", &N, &M);

  for (int i = 1; i <= N; i++) {
    if (i % M == 0) {
      continue;
    }

    printf("%d\n", i);
  }
  printf("selesai\n");
}
\end{lstlisting}
\end{itemize}
\end{frame}

\begin{frame}
\frametitle{Penjelasan Program: continue.cpp}
\begin{itemize}
  \item Ketika \textit{continue} ditemui, eksekusi perintah di dalam "for" untuk \identifier{i} tersebut langsung dilewati dan lanjut ke bagian perubahan.
  \item Artinya, untuk N = 10 dan M = 2, ketika nilai \identifier{i} = 2 dan "continue" ditemui, eksekusi akan dilewati langsung ke bagian perubahan \texttt{i++}. 
  \item Selanjutnya, perulangan dilanjutkan pada \identifier{i} = 3.
\end{itemize}
\end{frame}


\begin{frame}
\frametitle{Contoh Soal: Tes Keprimaan}
\begin{itemize}
  \item Diberikan sebuah bilangan positif yang lebih dari 1, misalnya N.
  \item Suatu bilangan N dikatakan prima apabila N positif dan hanya habis dibagi oleh 1 dan dirinya sendiri.
  \item Jika N prima, cetak "$<$N$>$ adalah bilangan prima" dan jika tidak, cetak "$<$N$>$ bukan bilangan prima".
\end{itemize}
Bagaimanakah kalian akan menyelesaikan persoalan ini?
\end{frame}

\begin{frame}
\frametitle{Solusi 1}
\begin{itemize}
  \item Salah satu solusi yang sederhana adalah: periksa semua bilangan di antara 2 sampai dengan N-1.
  \item Jika ada setidaknya satu bilangan yang habis membagi N, artinya N bukan prima.
\end{itemize}
\end{frame}


\begin{frame}[fragile]
\frametitle{Solusi 1: prima1\_1.cpp}
\begin{lstlisting}
#include <cstdio>

int main() {
  int N;
  scanf("%d", &N);

  bool prima = true;
  for (int i = 2; i <= N-1; i++) {
    if (N % i == 0) {
      prima = false;
    }
  }

  if (prima) {
    printf("%d adalah bilangan prima\n", N);
  } else {
    printf("%d bukan bilangan prima\n", N);
  }
}
\end{lstlisting}
\end{frame}

\begin{frame}
\frametitle{Solusi 2}
\begin{itemize}
  \item Solusi 1 melakukan pemeriksaan dari 2 sampai dengan N-1, artinya dibutuhkan pemeriksaan sebanyak N-2 kali.
  \item Sebetulnya pemeriksaan bisa dihentikan ketika ditemukan setidaknya satu saja bilangan yang habis membagi N.
  \item Dengan demikian bisa digunakan \texttt{break} untuk memberhentikan perulangan begitu ditemukan bilangan yang habis membagi N.
\end{itemize}
\end{frame}

\begin{frame}[fragile]
\frametitle{Solusi 2: prima1\_2.cpp}
\begin{lstlisting}
#include <cstdio>

int main() {
  int N;
  scanf("%d", &N);

  bool prima = true;
  for (int i = 2; i <= N-1; i++) {
    if (N % i == 0) {
      prima = false;
      break;
    }
  }

  if (prima) {
    printf("%d adalah bilangan prima\n", N);
  } else {
    printf("%d bukan bilangan prima\n", N);
  }
}
\end{lstlisting}
\end{frame}

\begin{frame}[fragile]
\frametitle{Contoh Soal: Pembangkit Prima}
\begin{itemize}
  \item Diberikan sebuah bilangan bulat N. Pak Dengklek meminta Anda untuk menuliskan N bilangan prima pertama.
  \item Contoh untuk N = 5:
\begin{lstlisting}
2
3
5
7
11
\end{lstlisting}
\end{itemize}
\end{frame}

\begin{frame}
\frametitle{Solusi: Pembangkit Prima}
\begin{itemize}
  \item Salah satu strategi yang dapat kalian gunakan adalah "selama belum ditemukan N bilangan prima, cari bilangan prima!".
  \item Bagaimana mencari bilangan prima? Coba saja dari 2, 3, 4, dan seterusnya sampai ditemukan N bilangan prima.
\end{itemize}
\end{frame}

\begin{frame}[fragile]
\frametitle{Contoh Solusi: prima2.cpp}
\begin{lstlisting}
#include <cstdio>

int main() {
  int N;
  scanf("%d", &N);

  int count = 0; // Banyaknya prima yang sudah ditemukan
  int cur = 2; // nilai yang akan diperiksa keprimaannya

  while (count < N) {
    bool prima = true;
    for (int i = 2; i <= cur-1; i++) {
      if (cur % i == 0) {
        prima = false;
        break;
      }
    }
\end{lstlisting}
\end{frame}

\begin{frame}[fragile]
\frametitle{Contoh Solusi: prima2.cpp (lanj.)}
\begin{lstlisting}
    if (prima) {
      // Ditemukan prima!
      // Cetak dan tambahkan prima yg sudah ditemukan
      printf("%d\n", cur);
      count++;
    }

    // Entah ini prima atau bukan, lanjut untuk
    // memeriksa bilangan berikutnya
    cur++;
  }
  // Keluar dari while, dipastikan count = N
}
\end{lstlisting}
\end{frame}

\begin{frame}
\frametitle{Penutup}
\begin{itemize}
  \item Percabangan dan perulangan merupakan dua struktur kontrol yang sangat penting pada pemrograman.
  \item Kalian diharapkan berlatih sampai lancar di kedua hal tersebut, baru lanjut untuk mempelajari materi selanjutnya.
\end{itemize}
\end{frame}

\end{document}
