\documentclass{beamer}
\usetheme{tokitex}

\usepackage{tikz}
\usepackage{graphics}
\usepackage{multirow}
\usepackage{tabto}
\usepackage{xspace}
\usepackage{amsmath}
\usepackage{hyperref}
\usepackage{wrapfig}
\usepackage{mathtools}

\usepackage{tikz}
\usepackage{clrscode3e}
\usepackage{gensymb}

\usepackage[english,bahasa]{babel}
\newtranslation[to=bahasa]{Section}{Bagian}
\newtranslation[to=bahasa]{Subsection}{Subbagian}

\usepackage{listings, lstautogobble}
\usepackage{color}

\definecolor{dkgreen}{rgb}{0,0.6,0}
\definecolor{gray}{rgb}{0.5,0.5,0.5}
\definecolor{mauve}{rgb}{0.58,0,0.82}

\lstset{frame=tb,
  language=c++,
  aboveskip=0mm,
  belowskip=0mm,
  showstringspaces=false,
  columns=fullflexible,
  keepspaces=true,
  basicstyle={\small\ttfamily},
  numbers=none,
  numberstyle=\tiny\color{gray},
  keywordstyle=\color{blue},
  commentstyle=\color{dkgreen},
  stringstyle=\color{mauve},
  breaklines=true,
  breakatwhitespace=true,
  lineskip={-3pt}
}

\usepackage{caption}
\captionsetup[figure]{labelformat=empty}

\newcommand{\progTerm}[1]{\textbf{#1}}
\newcommand{\foreignTerm}[1]{\textit{#1}}
\newcommand{\newTerm}[1]{\alert{\textbf{#1}}}
\newcommand{\emp}[1]{\alert{#1}}
\newcommand{\statement}[1]{"#1"}

\newcommand{\floor}[1]{\lfloor #1 \rfloor}
\newcommand{\ceil}[1]{\lceil #1 \rceil}
\newcommand{\abs}[1]{\left\lvert#1\right\rvert}
\newcommand{\norm}[1]{\left\lVert#1\right\rVert}

% Getting tired of writing \foreignTerm all the time
\newcommand{\farray}{\foreignTerm{array}\xspace}
\newcommand{\fArray}{\foreignTerm{Array}\xspace}
\newcommand{\foverhead}{\foreignTerm{overhead}\xspace}
\newcommand{\fOverhead}{\foreignTerm{Overhead}\xspace}
\newcommand{\fsubarray}{\foreignTerm{subarray}\xspace}
\newcommand{\fSubarray}{\foreignTerm{Subarray}\xspace}
\newcommand{\fbasecase}{\foreignTerm{base case}\xspace}
\newcommand{\fBasecase}{\foreignTerm{Base case}\xspace}
\newcommand{\ftopdown}{\foreignTerm{top-down}\xspace}
\newcommand{\fTopdown}{\foreignTerm{Top-down}\xspace}
\newcommand{\fbottomup}{\foreignTerm{bottom-up}\xspace}
\newcommand{\fBottomup}{\foreignTerm{Bottom-up}\xspace}
\newcommand{\fpruning}{\foreignTerm{pruning}\xspace}
\newcommand{\fPruning}{\foreignTerm{Pruning}\xspace}

\newcommand{\fgraph}{\foreignTerm{graph}\xspace}
\newcommand{\fGraph}{\foreignTerm{Graph}\xspace}
\newcommand{\froot}{\foreignTerm{root}\xspace}
\newcommand{\fRoot}{\foreignTerm{Root}\xspace}
\newcommand{\fnode}{\foreignTerm{node}\xspace}
\newcommand{\fNode}{\foreignTerm{Node}\xspace}
\newcommand{\fedge}{\foreignTerm{edge}\xspace}
\newcommand{\fEdge}{\foreignTerm{Edge}\xspace}
\newcommand{\fcycle}{\foreignTerm{cycle}\xspace}
\newcommand{\fCycle}{\foreignTerm{Cycle}\xspace}
\newcommand{\fdegree}{\foreignTerm{degree}\xspace}
\newcommand{\fDegree}{\foreignTerm{Degree}\xspace}
\newcommand{\fadjacencylist}{\foreignTerm{adjacency list}\xspace}
\newcommand{\fAdjacencylist}{\foreignTerm{Adjacency list}\xspace}
\newcommand{\fadjacencymatrix}{\foreignTerm{adjacency matrix}\xspace}
\newcommand{\fAdjacencymatrix}{\foreignTerm{Adjacency matrix}\xspace}
\newcommand{\fedgelist}{\foreignTerm{edge list}\xspace}
\newcommand{\fEdgelist}{\foreignTerm{Edge list}\xspace}
\newcommand{\flist}{\foreignTerm{list}\xspace}
\newcommand{\fList}{\foreignTerm{List}\xspace}
\newcommand{\fgraphtraversal}{\foreignTerm{graph traversal}\xspace}
\newcommand{\fGraphtraversal}{\foreignTerm{Graph traversal}\xspace}
\newcommand{\ftree}{\foreignTerm{tree}\xspace}
\newcommand{\fTree}{\foreignTerm{Tree}\xspace}
\newcommand{\fsubtree}{\foreignTerm{subtree}\xspace}
\newcommand{\fSubtree}{\foreignTerm{Subtree}\xspace}
\newcommand{\fparent}{\foreignTerm{parent}\xspace}
\newcommand{\fParent}{\foreignTerm{Parent}\xspace}
\newcommand{\fsibling}{\foreignTerm{sibling}\xspace}
\newcommand{\fSibling}{\foreignTerm{Sibling}\xspace}
\newcommand{\fpath}{\foreignTerm{path}\xspace}
\newcommand{\fPath}{\foreignTerm{Path}\xspace}
\newcommand{\fconnectedcomponent}{\foreignTerm{connected component}\xspace}
\newcommand{\fConnectedcomponent}{\foreignTerm{Connected component}\xspace}
\newcommand{\fbridge}{\foreignTerm{bridge}\xspace}
\newcommand{\fBridge}{\foreignTerm{Bridge}\xspace}
\newcommand{\farticulationpoint}{\foreignTerm{articulation point}\xspace}
\newcommand{\fArticulationpoint}{\foreignTerm{Articulation point}\xspace}
\newcommand{\ftreeedge}{\foreignTerm{tree edge}\xspace}
\newcommand{\fTreeedge}{\foreignTerm{Tree edge}\xspace}
\newcommand{\fbackedge}{\foreignTerm{back edge}\xspace}
\newcommand{\fBackedge}{\foreignTerm{Back edge}\xspace}
\newcommand{\fforwardedge}{\foreignTerm{forward edge}\xspace}
\newcommand{\fForwardedge}{\foreignTerm{Forward edge}\xspace}
\newcommand{\fcrossedge}{\foreignTerm{cross edge}\xspace}
\newcommand{\fCrossedge}{\foreignTerm{Cross edge}\xspace}
\newcommand{\fdiscoverytime}{\foreignTerm{discovery time}\xspace}
\newcommand{\fDiscoverytime}{\foreignTerm{Discovery time}\xspace}
\newcommand{\flowlink}{\foreignTerm{low link}\xspace}
\newcommand{\fLowlink}{\foreignTerm{Low link}\xspace}
\newcommand{\fstack}{\foreignTerm{stack}\xspace}
\newcommand{\fStack}{\foreignTerm{Stack}\xspace}
\newcommand{\for}{\foreignTerm{or}\xspace}
\newcommand{\fOr}{\foreignTerm{Or}\xspace}
\newcommand{\fand}{\foreignTerm{and}\xspace}
\newcommand{\fAnd}{\foreignTerm{And}\xspace}
\newcommand{\fcentroid}{\foreignTerm{centroid}\xspace}
\newcommand{\fCentroid}{\foreignTerm{Centroid}\xspace}

\newcommand{\fDivideAndConquer}{\foreignTerm{Divide and conquer}\xspace}
\newcommand{\fdivideAndConquer}{\foreignTerm{divide and conquer}\xspace}
\newcommand{\fMergeSort}{\foreignTerm{Merge sort}\xspace}
\newcommand{\fmergeSort}{\foreignTerm{merge sort}\xspace}
\newcommand{\fQuickSort}{\foreignTerm{Quicksort}\xspace}
\newcommand{\fquickSort}{\foreignTerm{quicksort}\xspace}
\newcommand{\fpivot}{\foreignTerm{pivot}\xspace}
\newcommand{\fPivot}{\foreignTerm{Pivot}\xspace}
\newcommand{\fbruteForce}{\foreignTerm{brute force}\xspace}
\newcommand{\fBruteForce}{\foreignTerm{Brute force}\xspace}
\newcommand{\fCompleteSearch}{\foreignTerm{complete search}\xspace}
\newcommand{\fExhaustiveSearch}{\foreignTerm{exhaustive search}\xspace}
\newcommand{\fbinarySearch}{\foreignTerm{binary search}\xspace}
\newcommand{\fBinarySearch}{\foreignTerm{Binary search}\xspace}
\newcommand{\fternarySearch}{\foreignTerm{ternary search}\xspace}
\newcommand{\fTernarySearch}{\foreignTerm{Ternary search}\xspace}
\newcommand{\funimodal}{\foreignTerm{unimodal}\xspace}
\newcommand{\fUnimodal}{\foreignTerm{Unimodal}\xspace}
\newcommand{\fGreedy}{\foreignTerm{Greedy}\xspace}
\newcommand{\fgreedy}{\foreignTerm{greedy}\xspace}
\newcommand{\fgreedyChoice}{\foreignTerm{greedy choice}\xspace}
\newcommand{\fGreedyChoice}{\foreignTerm{Greedy choice}\xspace}

\newcommand{\fdp}{\foreignTerm{dynamic programming}\xspace}
\newcommand{\fDp}{\foreignTerm{Dynamic programming}\xspace}
\newcommand{\fbitmask}{\foreignTerm{bitmask}\xspace}
\newcommand{\fBitmask}{\foreignTerm{Bitmask}\xspace}
\newcommand{\fstate}{\foreignTerm{state}\xspace}
\newcommand{\fState}{\foreignTerm{State}\xspace}
\newcommand{\fsubmask}{\foreignTerm{submask}\xspace}
\newcommand{\fSubmask}{\foreignTerm{Submask}\xspace}

\newcommand{\pheap}{\foreignTerm{heap}\xspace}
\newcommand{\pHeap}{\foreignTerm{Heap}\xspace}
\newcommand{\pBinaryHeap}{\foreignTerm{Binary Heap}\xspace}
\newcommand{\pbinaryHeap}{\foreignTerm{binary heap}\xspace}
\newcommand{\pHeapsort}{\foreignTerm{Heapsort}\xspace}
\newcommand{\pheapsort}{\foreignTerm{heapsort}\xspace}
\newcommand{\pdjs}{\foreignTerm{disjoint set}\xspace}
\newcommand{\pDjs}{\foreignTerm{Disjoint set}\xspace}

\newcommand{\fdotProduct}{\foreignTerm{dot product}\xspace}
\newcommand{\fDotProduct}{\foreignTerm{Dot product}\xspace}
\newcommand{\fcrossProduct}{\foreignTerm{cross product}\xspace}
\newcommand{\fCrossProduct}{\foreignTerm{Cross product}\xspace}
\newcommand{\fconvexHull}{\foreignTerm{convex hull}\xspace}
\newcommand{\fConvexHull}{\foreignTerm{Convex hull}\xspace}
\newcommand{\fgrahamScan}{\foreignTerm{graham scan}\xspace}
\newcommand{\fGrahamScan}{\foreignTerm{Graham scan}\xspace}
\newcommand{\flineSweep}{\foreignTerm{line sweep}\xspace}
\newcommand{\fLineSweep}{\foreignTerm{Line sweep}\xspace}

\newcommand{\fset}{\foreignTerm{set}\xspace}
\newcommand{\fSet}{\foreignTerm{Set}\xspace}
\newcommand{\fprefixSum}{\foreignTerm{prefix sum}\xspace}
\newcommand{\fPrefixSum}{\foreignTerm{Prefix sum}\xspace}
\newcommand{\ffenwickTree}{\foreignTerm{fenwick tree}\xspace}
\newcommand{\fFenwickTree}{\foreignTerm{Fenwick tree}\xspace}
\newcommand{\frangeSumQuery}{\foreignTerm{range sum query}\xspace}
\newcommand{\fRangeSumQuery}{\foreignTerm{Range sum query}\xspace}
\newcommand{\fquery}{\foreignTerm{query}\xspace}
\newcommand{\fQuery}{\foreignTerm{Query}\xspace}
\newcommand{\fsegmentTree}{\foreignTerm{segment tree}\xspace}
\newcommand{\fSegmentTree}{\foreignTerm{Segment tree}\xspace}
\newcommand{\fbinaryTree}{\foreignTerm{binary tree}\xspace}
\newcommand{\fBinaryTree}{\foreignTerm{Binary tree}\xspace}
\newcommand{\flazyPropagation}{\foreignTerm{lazy propagation}\xspace}
\newcommand{\fLazyPropagation}{\foreignTerm{Lazy propagation}\xspace}
\newcommand{\fsparseTable}{\foreignTerm{sparse table}\xspace}
\newcommand{\fSparseTable}{\foreignTerm{Sparse table}\xspace}

\newcommand{\ftrail}{\foreignTerm{trail}\xspace}
\newcommand{\fTrail}{\foreignTerm{Trail}\xspace}
\newcommand{\feulerTour}{\foreignTerm{euler tour}\xspace}
\newcommand{\fEulerTour}{\foreignTerm{Euler tour}\xspace}
\newcommand{\feulerTourTree}{\foreignTerm{euler tour tree}\xspace}
\newcommand{\fEulerTourTree}{\foreignTerm{Euler tour tree}\xspace}

\newcommand{\fmaxflow}{\foreignTerm{maximum flow}\xspace}
\newcommand{\fMaxflow}{\foreignTerm{Maximum flow}\xspace}
\newcommand{\fmincut}{\foreignTerm{minimum cut}\xspace}
\newcommand{\fMincut}{\foreignTerm{Minimum cut}\xspace}
\newcommand{\fflow}{\foreignTerm{flow}\xspace}
\newcommand{\fFlow}{\foreignTerm{Flow}\xspace}
\newcommand{\fsource}{\foreignTerm{source}\xspace}
\newcommand{\fSource}{\foreignTerm{Source}\xspace}
\newcommand{\fsink}{\foreignTerm{sink}\xspace}
\newcommand{\fSink}{\foreignTerm{Sink}\xspace}
\newcommand{\fbackEdge}{\foreignTerm{back-edge}\xspace}
\newcommand{\fBackEdge}{\foreignTerm{Back-edge}\xspace}
\newcommand{\fresidualCapacity}{\foreignTerm{residual capacity}\xspace}
\newcommand{\fResidualCapacity}{\foreignTerm{Residual capacity}\xspace}
\newcommand{\fbottleneck}{\foreignTerm{bottleneck}\xspace}
\newcommand{\fBottleneck}{\foreignTerm{Bottleneck}\xspace}
\newcommand{\faugmentingPath}{\foreignTerm{augmenting path}\xspace}
\newcommand{\fAugmentingPath}{\foreignTerm{Augmenting path}\xspace}


\title{Array}
\author{Tim Olimpiade Komputer Indonesia}
\date{}

\begin{document}

\begin{frame}
\titlepage
\end{frame}

\begin{frame}
\frametitle{Pendahuluan}
Melalui dokumen ini, kalian akan:
\begin{itemize}
  \item Memahami konsep array.
  \item Mengimplementasikan array pada bahasa C++.
  \item Menggunakan array untuk penyelesaian beberapa contoh masalah.
\end{itemize}
\end{frame}

\section{Konsep Array}
\frame{\sectionpage}

\begin{frame}
\frametitle{Motivasi}
\begin{itemize}
  \item Pak Dengklek memiliki sebuah tumpukan berisi $N$ kartu, yang dipenuhi $1 \le N \le 100$.
  \item Setiap kartu bertuliskan suatu bilangan bulat.
  \item Sekarang Pak Dengklek ingin tahu urutan angka-angka pada kartu tersebut bila tumpukan kartu itu dibalik.
  \item Contoh: jika diberikan 5 kartu dengan angka-angka dari atasnya [1, 5, 3, 20, 4], maka setelah dibalik urutannya menjadi: [4, 20, 3, 5, 1].
  \item Bantulah Pak Dengklek menentukan urutan angka-angka tersebut setelah tumpukan kartu dibalik!
\end{itemize}
\end{frame}

\begin{frame}[fragile]
\frametitle{Solusi?}
\begin{itemize}
  \item Sederhana, idenya adalah dengan menampung seluruh bilangan terlebih dahulu, baru dicetak dalam urutan terbalik.
  \item Misalnya jika $N$ selalu 3, kita bisa membuat 3 variabel (misalnya a, b, c), lalu:
\begin{lstlisting}
scanf("%d", &a);
scanf("%d", &b);
scanf("%d", &c);

printf("%d\n", c);
printf("%d\n", b);
printf("%d\n", a);
\end{lstlisting}
  \item Sayangnya nilai $N$ tidak tetap! Dibutuhkan suatu mekanisme lain untuk menggunakan dan mengakses variabel!
\end{itemize}
\end{frame}

\begin{frame}
\frametitle{Pengertian Array}
\begin{block}{Array}
Variabel dengan satu nama, tetapi mengandung banyak nilai.
Akses nilai-nilainya dilakukan dengan indeks.
\end{block}
\vfill
Perhatikan contoh berikut!
\vfill
\begin{tabular}{|c|c|c|c|c|c|c|c|c|c|c|}
\hline indeks & 1 & 2 & 3 & 4 & 5 & 6 & 7 & 8 & 9 & 10 \\
\hline A & 3 & 10 & 11 & 23 & 35 & 12 & 31 & 53 & 0 & 19 \\
\hline
\end{tabular}

\begin{itemize}
  \item A[1] = 3
  \item A[2] = 10
  \item A[5] = 35
\end{itemize}
\end{frame}

\begin{frame}
\frametitle{Penjelasan}
\begin{itemize}
  \item Pada contoh sebelumnya, kita memiliki sebuah variabel bernama A.
  \item A memiliki 10 nilai, yang masing-masing dapat diakses dengan indeks.
  \item Untuk mengakses nilai A yang ke-x, digunakan A[x].
  \item Lebih jauh lagi, sebenarnya A[x] bisa dianggap sebagai sebuah variabel yang berdiri sendiri.
  \item Konsep inilah yang disebut sebagai \emp{array}!
\end{itemize}
\end{frame}

\section{Implementasi Array pada C++}
\frame{\sectionpage}

\begin{frame}[fragile]
\frametitle{Deklarasi}
\begin{itemize}
  \item Karena array merupakan variabel, diperlukan deklarasi seperti variabel lainnya.
  \item Format deklarasi array adalah:
\begin{lstlisting}
<tipe> <nama>[<ukuran>];
\end{lstlisting}
  \item Dengan:
  \begin{itemize}
    \item $<$nama$>$ adalah nama dari array (aturan penamaan sama seperti variabel biasanya)
    \item $<$ukuran$>$ adalah ukuran dari array, yang terdefinisi dari 0 sampai dengan ukuran-1.
    \item $<$tipe$>$ adalah tipe data dari array.
  \end{itemize}
  \item Tentu saja, tipe data di sini bisa berupa int, double, string, bool atau suatu struct.
\end{itemize}
\end{frame}

\begin{frame}[fragile]
\frametitle{Contoh Deklarasi}
Berikut ini adalah contoh deklarasi array pada C++:
\begin{lstlisting}
bool tabel[101];
int frekuensi[1000];
\end{lstlisting}
\begin{itemize}
  \item Untuk contoh array tabel, hanya tabel[0], tabel[1], tabel[2], ..., tabel[100] yang terdefinisi.
  \item Untuk contoh array frekuensi, hanya frekuensi[0], frekuensi[1], frekuensi[2], ..., frekuensi[999] yang terdefinisi.
  \item Mengakses nilai tabel[-1], tabel[-2], atau tabel[500] dapat menyebabkan \emp{runtime error}.
  \item Untuk itu, tentukan rentang indeks yang akan kalian gunakan saat deklarasi dengan tepat (sesuai kebutuhan).
\end{itemize}
\end{frame}

\begin{frame}[fragile]
\frametitle{Array dan Variabel}
\begin{itemize}
  \item Karena suatu elemen dari array juga bisa dianggap variabel, tentu saja kita bisa melakukan perintah scanf padanya.
  \item Sebagai contoh, jika kita memiliki array int bernama tabel yang terdefinisi dari 1 sampai dengan 100, kita bisa melakukan:
\begin{lstlisting}
scanf("%d", &tabel[2]);
\end{lstlisting}
\end{itemize}
\end{frame}

\begin{frame}[fragile]
\frametitle{Array dan Variabel (lanj.)}
\begin{itemize}
  \item Jika diberikan 5 bilangan, dan kita perlu menyimpan masing-masing bilangan di tabel, kita bisa melakukan:
\begin{lstlisting}
scanf("%d", &tabel[0]);
scanf("%d", &tabel[1]);
scanf("%d", &tabel[2]);
scanf("%d", &tabel[3]);
scanf("%d", &tabel[4]);
\end{lstlisting}
  \item Tentu saja hal ini sangat tidak efisien!
  \item Untungnya, kita sudah mempelajari sebuah teknik yang sangat penting, yaitu \textbf{perulangan}.
\end{itemize}
\end{frame}

\begin{frame}[fragile]
\frametitle{Array dan Variabel (lanj.)}
\begin{itemize}
  \item Proses membaca 5 bilangan pada 5 baris kini bisa dilakukan dengan cara:
\begin{lstlisting}
for (int i = 0; i < 5; i++) {
  scanf("%d", &tabel[i]);
}
\end{lstlisting}
  \item Untuk kasus umum, yaitu ketika diberikan $N$ bilangan, cukup ganti angka 5 dengan variabel $N$.
\begin{lstlisting}
for (int i = 0; i < N; i++) {
  scanf("%d", &tabel[i]);
}
\end{lstlisting}
\end{itemize}
\end{frame}

\begin{frame}[fragile]
\frametitle{Array dan Variabel (lanj.)}
\begin{itemize}
  \item Demikian pula untuk pencetakan secara terbalik, kita bisa menggunakan perulangan sebagai berikut:
\begin{lstlisting}
for (int i = N-1; i >= 0; i--) {
  printf("%d\n", tabel[i]);
}
\end{lstlisting}
  \item Sekarang masalah Pak Dengklek terpecahkan!
\end{itemize}
\end{frame}

\begin{frame}[fragile]
\frametitle{Contoh Solusi: balik.cpp}
Berikut contoh solusi lengkap untuk permasalahan motivasi:
\begin{lstlisting}
#include <cstdio>

int main() {
  int N;
  int tabel[100];
  scanf("%d", &N);

  for (int i = 0; i < N; i++) {
    scanf("%d", &tabel[i]);
  }

  for (int i = N-1; i >= 0; i--) {
    printf("%d\n", tabel[i]);
  }
}
\end{lstlisting}
\end{frame}


\begin{frame}
\frametitle{Array dan Memori}
\begin{itemize}
  \item Setiap elemen pada array membutuhkan memori, bergantung pada tipe data yang digunakan.
  \item Total memori yang dibutuhkan untuk sebuah array sama dengan banyaknya elemennya dikali ukuran memori satu elemennya.
  \item Sebagai contoh, array dengan 100 elemen dan memiliki tipe int membutuhkan memori sebesar $100 \times 4$ byte $= 400$ byte,
\end{itemize}
\end{frame}

\begin{frame}[fragile]
\frametitle{Rentang Array}
\begin{itemize}
  \item Pada program membalik array, dideklarasikan array sebesar 100 elemen (dari 0 sampai dengan 99), padahal bisa jadi hanya digunakan sebagian saja.
  \item Cara ini memang "boros" memori, tetapi merupakan cara yang paling mudah adalah mendeklarasikannya sebesar nilai $N$ maksimal yang mungkin.
  \item Bisa juga kita deklarasikan sesudah $N$ diketahui sebagai berikut:
\begin{lstlisting}
...
int N;

scanf("%d", &N);

int tabel[N];
...
\end{lstlisting}
\end{itemize}
\end{frame}

\begin{frame}
\frametitle{Contoh Soal: Ujian Harian}
Deskripsi:
\begin{itemize}
  \item Pak Dengklek menyelenggarakan ujian harian setelah selesai mengajarkan $N$ ekor bebeknya mengenai konsep array.
  \item Setiap bebek ke-i mendapatkan nilai sebesar \textbf{$h_i$}, yang merupakan bilangan bulat.
  \item Tentukan banyaknya bebek yang memiliki nilai tidak kurang dari rata-rata seluruh bebek!
\end{itemize}
Batasan:
\begin{itemize}
  \item $1 \le N \le 100$
  \item $1 \le h_i \le 100$, untuk $1 \le i \le N$
\end{itemize}
\end{frame}

\begin{frame}
\frametitle{Contoh Soal: Ujian Harian (lanj.)}
Format masukan:
\begin{itemize}
  \item Baris pertama berisi sebuah bilangan bulat $N$.
  \item $N$ baris berikutnya berisi nilai ujian bebek. Baris ke-$i$ ini merupakan $h_i$.
\end{itemize}
Format keluaran:
\begin{itemize}
  \item Sebuah baris yang menyatakan banyaknya bebek yang lulus ujian.
\end{itemize}
\end{frame}

\begin{frame}[fragile]
\frametitle{Contoh Soal: Ujian Harian (lanj.)}
Contoh masukan:
\begin{lstlisting}
3
5
6
7
\end{lstlisting}

\hfill

Contoh keluaran:
\begin{lstlisting}
2
\end{lstlisting}
\begin{block}{Penjelasan}
Nilai rata-rata dari seluruh bebek adalah 6, dan terdapat 2 ekor bebek yang nilainya tidak kurang dari 6.
\end{block}
\end{frame}

\begin{frame}
\frametitle{Petunjuk}
\begin{itemize}
  \item Salah satu solusinya adalah melalui dua tahap:
  \begin{enumerate}
    \item Hitung rata-ratanya.
    \item Hitung banyaknya bebek yang nilainya tidak kurang dari rata-rata.
  \end{enumerate}
  \item Sebisa mungkin, hindari penggunaan \emp{floating-point}!
  \begin{itemize}
    \item Ingat bahwa tipe data floating-point kurang bisa menyatakan bilangan secara akurat; nilai 1/3*3 bisa jadi 0.999999999999999 atau 1.0000000000001.
    \item Pengoperasian tipe data bilangan bulat oleh komputer jauh lebih cepat daripada pengoperasian tipe data floating-point!
  \end{itemize}
\end{itemize}
\end{frame}

\begin{frame}[fragile]
\frametitle{Contoh Solusi: lulus.cpp}
\begin{lstlisting}
#include <cstdio>

int main() {
  int N;
  scanf("%d", &N);

  int nilai[N];
  for (int i = 0; i < N; i++) {
    scanf("%d", &nilai[i]);
  }

  int total = 0;
  for (int i = 0; i < N; i++) {
    total += nilai[i];
  }
\end{lstlisting}
\end{frame}

\begin{frame}[fragile]
\frametitle{Contoh Solusi: lulus.cpp (lanj.)}
\begin{lstlisting}
  int lulus = 0;
  for (int i = 0; i < N; i++) {
    // Trik menghindari pembagian
    if (nilai[i]*N >= total) {
      lulus++;
    }
  }

  printf("%d\n", lulus);
}
\end{lstlisting}
\end{frame}

\section{Penggunaan Array Lanjutan}
\frame{\sectionpage}

\begin{frame}[fragile]
\frametitle{Array Dua Dimensi}
\begin{itemize}
  \item Struktur array bisa juga membentuk sebuah tabel dua dimensi.
  \item Perhatikan contoh deklarasi berikut:
\begin{lstlisting}
int matriks[2][5];
\end{lstlisting}
  \item Kini kita mendapatkan variabel bernama $matriks[a][b]$, yang terdefinisi untuk $0 \le a \le 1$ dan $0 \le b \le 4$.
\end{itemize}
\end{frame}

\begin{frame}
\frametitle{Array Dua Dimensi (lanj.)}
\begin{itemize}
  \item Akses suatu elemen dapat dilakukan dengan matriks[a][b].
  \item Tabel berikut menunjukkan struktur dari array matriks:
  \begin{table}[h]
    \begin{tabular}{c|c|c|c|c|c|}
        & 1 & 2 & 3 & 4 & 5\\
      \hline 1 & & & & & \\
      \hline 2 & & & & & \\
      \hline
    \end{tabular}
  \end{table}
  \item Aturan perhitungan memori tetap sama; banyaknya elemen dikali memori per elemennya.\newline Pada kasus ini: $2 \times 5 \times 4$ byte $= 40$ byte.
\end{itemize}
\end{frame}

\begin{frame}
\frametitle{Contoh Soal:\newline Cokelat Bebek}
Deskripsi:
\begin{itemize}
  \item Pak Ganesh datang bertamu ke peternakan bebek Pak Dengklek.
  \item Pada peternakan bebek Pak Dengklek, terdapat kandang bebek yang tersusun atas petak-petak $N$ baris dan $N$ kolom.
  \item Pak Dengklek memberi $d_{i,j}$ gram cokelat* ke kandang di baris ke-$i$ dan kolom ke-$j$.
  \item Pak Ganesh memberi $g_{i,j}$ gram cokelat* ke kandang di baris ke-$i$ dan kolom ke-$j$.
  \item Tentukan berapa gram cokelat yang diperoleh setiap bebek di kandangnya!
\end{itemize}
Batasan:
\begin{itemize}
  \item $1 \le N \le 100$
  \item $0 \le d_{i,j}, h_{i,j} \le 10$, untuk $1 \le i,j \le N$
\end{itemize}

\tiny *Catatan: bebek-bebek suka cokelat!
\end{frame}

\begin{frame}
\frametitle{Contoh Soal:\newline Cokelat Bebek (lanj.)}
\begin{itemize}
  \item Sebagai contoh, misalkan $N = 3$.
  \item Kemudian berikut adalah cokelat yang diberikan Pak Dengklek ($D$) dan Pak Ganesh ($G$):
  \vfill
  \(D =
  \left[\begin{matrix}
  1 & 3 & 0 \\
  6 & 2 & 4 \\
  2 & 1 & 5
  \end{matrix}\right]
  \)
  \hfil
  \(G =
  \left[\begin{matrix}
  2 & 1 & 7 \\
  0 & 0 & 1 \\
  1 & 1 & 2
  \end{matrix}\right]
  \) \centering
  \item Maka total cokelat yang didapatkan setiap kandang adalah:
  \vfill
  \(
  \left[\begin{matrix}
  3 & 4 & 7 \\
  6 & 2 & 5 \\
  3 & 2 & 7
  \end{matrix}\right]
  \) \centering
\end{itemize}
\end{frame}

\begin{frame}
\frametitle{Contoh Soal:\newline Cokelat Bebek (lanj.)}
Format masukan:
\begin{itemize}
  \item Baris pertama berisi sebuah bilangan bulat $N$.
  \item $N$ baris berikutnya berisi $N$ bilangan. Bilangan di baris ke-$i$ dan kolom ke-$j$ ini adalah $d_{i,j}$.
  \item $N$ baris sisanya berisi $N$ bilangan. Bilangan di baris ke-$i$ dan kolom ke-$j$ ini adalah $g_{i,j}$.
\end{itemize}
Format keluaran:
\begin{itemize}
  \item $N$ baris yang berisi $N$ bilangan. Bilangan di baris ke-$i$ dan kolom ke-$j$ ini adalah total makanan yang ada di kandang baris ke-$i$ dan kolom ke-$j$.
\end{itemize}
\end{frame}

\begin{frame}[fragile]
\frametitle{Contoh Soal:\newline Cokelat Bebek (lanj.)}
Contoh masukan:
\begin{lstlisting}
3
1 3 0
6 2 4
2 1 5
2 1 7
0 0 1
1 1 2
\end{lstlisting}

\hfill

Contoh keluaran:
\begin{lstlisting}
3 4 7
6 2 5
3 2 7
\end{lstlisting}
\end{frame}

\begin{frame}
\frametitle{Petunjuk}
\begin{itemize}
  \item Salah satu cara yang mudah adalah membuat tiga array dua dimensi, masing-masing untuk menampung makanan yang diberikan Pak Dengklek ($D$), Pak Ganesh ($G$), dan hasil akhirnya ($hasil$).
  \item Tentu saja hubungannya adalah $hasil[i][j] = D[i][j] + G[i][j]$, untuk $1 \le i,j \le N$.
\end{itemize}
\end{frame}

\begin{frame}[fragile]
\frametitle{Solusi: cokelat.cpp}
Pertama, mari kita deklarasikan variabel dan baca masukan:
\begin{lstlisting}
#include <cstdio>

int main() {
  int N;
  scanf("%d", &N);

  int D[N][N], G[N][N], hasil[N][N];
  for (int i = 0; i < N; i++) {
    for (int j = 0; j < N; j++) {
      scanf("%d", &D[i][j]);
    }
  }

 for (int i = 0; i < N; i++) {
    for (int j = 0; j < N; j++) {
      scanf("%d", &G[i][j]);
    }
  }
\end{lstlisting}
\end{frame}

\begin{frame}[fragile]
\frametitle{Solusi: cokelat.cpp (lanj.)}
Lakukan penjumlahan, lalu cetak hasilnya:
\begin{lstlisting}
 for (int i = 0; i < N; i++) {
    for (int j = 0; j < N; j++) {
      hasil[i][j] = D[i][j] + G[i][j];
    }
  }

  for (int i = 0; i < N; i++) {
    for (int j = 0; j < N; j++) {
      printf("%d", hasil[i][j]);
      if (j+1 < N) {
        printf(" ");
      }
    }
    printf("\n");
  }
}
\end{lstlisting}
\end{frame}

\begin{frame}[fragile]
\frametitle{Solusi: cokelat\_2.cpp}
Nilai array $D$ dan $G$ sebenarnya tidak perlu disimpan, kita bisa menghemat memori dengan langsung menjumlahkannya.
\begin{lstlisting}
#include <cstdio>

int main() {
  int N;
  scanf("%d", &N);

  int hasil[N][N];
  for (int i = 0; i < N; i++) {
    for (int j = 0; j < N; j++) {
      int temp;
      scanf("%d", &temp);
      hasil[i][j] = temp;
    }
  }
\end{lstlisting}
\end{frame}

\begin{frame}[fragile]
\frametitle{Solusi: cokelat\_2.cpp (lanj.)}
\begin{lstlisting}
  for (int i = 0; i < N; i++) {
    for (int j = 0; j < N; j++) {
      int temp;
      scanf("%d", &temp);
      hasil[i][j] += temp;
    }
  }

  for (int i = 0; i < N; i++) {
    for (int j = 0; j < N; j++) {
      printf("%d", hasil[i][j]);
      if (j+1 < N) {
        printf(" ");
      }
    }
    printf("\n");
  }
}

\end{lstlisting}
\end{frame}

\begin{frame}[fragile]
\frametitle{Array Multidimensi}
\begin{itemize}
  \item Tidak hanya sampai dua dimensi, dimensi tiga, empat, atau lebih pun bisa.
  \item Sebagai contoh:
\begin{lstlisting}
int data[2][50][50];
\end{lstlisting}
  \item Kita akan mendapatkan variabel $data[i][j][k]$ yang terdefinisi untuk $0 \le i \le 1$, dan $0 \le j, k \le 49$.
\end{itemize}
\end{frame}

\begin{frame}[fragile]
\frametitle{Catatan}
\begin{itemize}
  \item Pada saat array dideklarasi, nilai yang ada di dalam array bisa jadi tidak tentu.
  \item Sebagai contoh, program berikut akan mencetak angka yang tidak tentu:
\begin{lstlisting}
#include <cstdio>

int main() {
  int arr[10];
  printf("%d\n", arr[0]);
}
\end{lstlisting}
  \item Pastikan Anda melakukan inisialisasi pada array dengan tepat.
\end{itemize}
\end{frame}


\begin{frame}
\frametitle{Selanjutnya...}
\begin{itemize}
  \item Mempelajari tentang \textbf{fungsi} dan \textbf{prosedur}.
\end{itemize}
\end{frame}

\end{document}
