\documentclass{beamer}
\usetheme{tokitex}

\usepackage{tikz}
\usepackage{graphics}
\usepackage{multirow}
\usepackage{tabto}
\usepackage{xspace}
\usepackage{amsmath}
\usepackage{hyperref}
\usepackage{wrapfig}
\usepackage{mathtools}

\usepackage{tikz}
\usepackage{clrscode3e}
\usepackage{gensymb}

\usepackage[english,bahasa]{babel}
\newtranslation[to=bahasa]{Section}{Bagian}
\newtranslation[to=bahasa]{Subsection}{Subbagian}

\usepackage{listings, lstautogobble}
\usepackage{color}

\definecolor{dkgreen}{rgb}{0,0.6,0}
\definecolor{gray}{rgb}{0.5,0.5,0.5}
\definecolor{mauve}{rgb}{0.58,0,0.82}

\lstset{frame=tb,
  language=c++,
  aboveskip=0mm,
  belowskip=0mm,
  showstringspaces=false,
  columns=fullflexible,
  keepspaces=true,
  basicstyle={\small\ttfamily},
  numbers=none,
  numberstyle=\tiny\color{gray},
  keywordstyle=\color{blue},
  commentstyle=\color{dkgreen},
  stringstyle=\color{mauve},
  breaklines=true,
  breakatwhitespace=true,
  lineskip={-3pt}
}

\usepackage{caption}
\captionsetup[figure]{labelformat=empty}

\newcommand{\progTerm}[1]{\textbf{#1}}
\newcommand{\foreignTerm}[1]{\textit{#1}}
\newcommand{\newTerm}[1]{\alert{\textbf{#1}}}
\newcommand{\emp}[1]{\alert{#1}}
\newcommand{\statement}[1]{"#1"}

\newcommand{\floor}[1]{\lfloor #1 \rfloor}
\newcommand{\ceil}[1]{\lceil #1 \rceil}
\newcommand{\abs}[1]{\left\lvert#1\right\rvert}
\newcommand{\norm}[1]{\left\lVert#1\right\rVert}

% Getting tired of writing \foreignTerm all the time
\newcommand{\farray}{\foreignTerm{array}\xspace}
\newcommand{\fArray}{\foreignTerm{Array}\xspace}
\newcommand{\foverhead}{\foreignTerm{overhead}\xspace}
\newcommand{\fOverhead}{\foreignTerm{Overhead}\xspace}
\newcommand{\fsubarray}{\foreignTerm{subarray}\xspace}
\newcommand{\fSubarray}{\foreignTerm{Subarray}\xspace}
\newcommand{\fbasecase}{\foreignTerm{base case}\xspace}
\newcommand{\fBasecase}{\foreignTerm{Base case}\xspace}
\newcommand{\ftopdown}{\foreignTerm{top-down}\xspace}
\newcommand{\fTopdown}{\foreignTerm{Top-down}\xspace}
\newcommand{\fbottomup}{\foreignTerm{bottom-up}\xspace}
\newcommand{\fBottomup}{\foreignTerm{Bottom-up}\xspace}
\newcommand{\fpruning}{\foreignTerm{pruning}\xspace}
\newcommand{\fPruning}{\foreignTerm{Pruning}\xspace}

\newcommand{\fgraph}{\foreignTerm{graph}\xspace}
\newcommand{\fGraph}{\foreignTerm{Graph}\xspace}
\newcommand{\froot}{\foreignTerm{root}\xspace}
\newcommand{\fRoot}{\foreignTerm{Root}\xspace}
\newcommand{\fnode}{\foreignTerm{node}\xspace}
\newcommand{\fNode}{\foreignTerm{Node}\xspace}
\newcommand{\fedge}{\foreignTerm{edge}\xspace}
\newcommand{\fEdge}{\foreignTerm{Edge}\xspace}
\newcommand{\fcycle}{\foreignTerm{cycle}\xspace}
\newcommand{\fCycle}{\foreignTerm{Cycle}\xspace}
\newcommand{\fdegree}{\foreignTerm{degree}\xspace}
\newcommand{\fDegree}{\foreignTerm{Degree}\xspace}
\newcommand{\fadjacencylist}{\foreignTerm{adjacency list}\xspace}
\newcommand{\fAdjacencylist}{\foreignTerm{Adjacency list}\xspace}
\newcommand{\fadjacencymatrix}{\foreignTerm{adjacency matrix}\xspace}
\newcommand{\fAdjacencymatrix}{\foreignTerm{Adjacency matrix}\xspace}
\newcommand{\fedgelist}{\foreignTerm{edge list}\xspace}
\newcommand{\fEdgelist}{\foreignTerm{Edge list}\xspace}
\newcommand{\flist}{\foreignTerm{list}\xspace}
\newcommand{\fList}{\foreignTerm{List}\xspace}
\newcommand{\fgraphtraversal}{\foreignTerm{graph traversal}\xspace}
\newcommand{\fGraphtraversal}{\foreignTerm{Graph traversal}\xspace}
\newcommand{\ftree}{\foreignTerm{tree}\xspace}
\newcommand{\fTree}{\foreignTerm{Tree}\xspace}
\newcommand{\fsubtree}{\foreignTerm{subtree}\xspace}
\newcommand{\fSubtree}{\foreignTerm{Subtree}\xspace}
\newcommand{\fparent}{\foreignTerm{parent}\xspace}
\newcommand{\fParent}{\foreignTerm{Parent}\xspace}
\newcommand{\fsibling}{\foreignTerm{sibling}\xspace}
\newcommand{\fSibling}{\foreignTerm{Sibling}\xspace}
\newcommand{\fpath}{\foreignTerm{path}\xspace}
\newcommand{\fPath}{\foreignTerm{Path}\xspace}
\newcommand{\fconnectedcomponent}{\foreignTerm{connected component}\xspace}
\newcommand{\fConnectedcomponent}{\foreignTerm{Connected component}\xspace}
\newcommand{\fbridge}{\foreignTerm{bridge}\xspace}
\newcommand{\fBridge}{\foreignTerm{Bridge}\xspace}
\newcommand{\farticulationpoint}{\foreignTerm{articulation point}\xspace}
\newcommand{\fArticulationpoint}{\foreignTerm{Articulation point}\xspace}
\newcommand{\ftreeedge}{\foreignTerm{tree edge}\xspace}
\newcommand{\fTreeedge}{\foreignTerm{Tree edge}\xspace}
\newcommand{\fbackedge}{\foreignTerm{back edge}\xspace}
\newcommand{\fBackedge}{\foreignTerm{Back edge}\xspace}
\newcommand{\fforwardedge}{\foreignTerm{forward edge}\xspace}
\newcommand{\fForwardedge}{\foreignTerm{Forward edge}\xspace}
\newcommand{\fcrossedge}{\foreignTerm{cross edge}\xspace}
\newcommand{\fCrossedge}{\foreignTerm{Cross edge}\xspace}
\newcommand{\fdiscoverytime}{\foreignTerm{discovery time}\xspace}
\newcommand{\fDiscoverytime}{\foreignTerm{Discovery time}\xspace}
\newcommand{\flowlink}{\foreignTerm{low link}\xspace}
\newcommand{\fLowlink}{\foreignTerm{Low link}\xspace}
\newcommand{\fstack}{\foreignTerm{stack}\xspace}
\newcommand{\fStack}{\foreignTerm{Stack}\xspace}
\newcommand{\for}{\foreignTerm{or}\xspace}
\newcommand{\fOr}{\foreignTerm{Or}\xspace}
\newcommand{\fand}{\foreignTerm{and}\xspace}
\newcommand{\fAnd}{\foreignTerm{And}\xspace}
\newcommand{\fcentroid}{\foreignTerm{centroid}\xspace}
\newcommand{\fCentroid}{\foreignTerm{Centroid}\xspace}

\newcommand{\fDivideAndConquer}{\foreignTerm{Divide and conquer}\xspace}
\newcommand{\fdivideAndConquer}{\foreignTerm{divide and conquer}\xspace}
\newcommand{\fMergeSort}{\foreignTerm{Merge sort}\xspace}
\newcommand{\fmergeSort}{\foreignTerm{merge sort}\xspace}
\newcommand{\fQuickSort}{\foreignTerm{Quicksort}\xspace}
\newcommand{\fquickSort}{\foreignTerm{quicksort}\xspace}
\newcommand{\fpivot}{\foreignTerm{pivot}\xspace}
\newcommand{\fPivot}{\foreignTerm{Pivot}\xspace}
\newcommand{\fbruteForce}{\foreignTerm{brute force}\xspace}
\newcommand{\fBruteForce}{\foreignTerm{Brute force}\xspace}
\newcommand{\fCompleteSearch}{\foreignTerm{complete search}\xspace}
\newcommand{\fExhaustiveSearch}{\foreignTerm{exhaustive search}\xspace}
\newcommand{\fbinarySearch}{\foreignTerm{binary search}\xspace}
\newcommand{\fBinarySearch}{\foreignTerm{Binary search}\xspace}
\newcommand{\fternarySearch}{\foreignTerm{ternary search}\xspace}
\newcommand{\fTernarySearch}{\foreignTerm{Ternary search}\xspace}
\newcommand{\funimodal}{\foreignTerm{unimodal}\xspace}
\newcommand{\fUnimodal}{\foreignTerm{Unimodal}\xspace}
\newcommand{\fGreedy}{\foreignTerm{Greedy}\xspace}
\newcommand{\fgreedy}{\foreignTerm{greedy}\xspace}
\newcommand{\fgreedyChoice}{\foreignTerm{greedy choice}\xspace}
\newcommand{\fGreedyChoice}{\foreignTerm{Greedy choice}\xspace}

\newcommand{\fdp}{\foreignTerm{dynamic programming}\xspace}
\newcommand{\fDp}{\foreignTerm{Dynamic programming}\xspace}
\newcommand{\fbitmask}{\foreignTerm{bitmask}\xspace}
\newcommand{\fBitmask}{\foreignTerm{Bitmask}\xspace}
\newcommand{\fstate}{\foreignTerm{state}\xspace}
\newcommand{\fState}{\foreignTerm{State}\xspace}
\newcommand{\fsubmask}{\foreignTerm{submask}\xspace}
\newcommand{\fSubmask}{\foreignTerm{Submask}\xspace}

\newcommand{\pheap}{\foreignTerm{heap}\xspace}
\newcommand{\pHeap}{\foreignTerm{Heap}\xspace}
\newcommand{\pBinaryHeap}{\foreignTerm{Binary Heap}\xspace}
\newcommand{\pbinaryHeap}{\foreignTerm{binary heap}\xspace}
\newcommand{\pHeapsort}{\foreignTerm{Heapsort}\xspace}
\newcommand{\pheapsort}{\foreignTerm{heapsort}\xspace}
\newcommand{\pdjs}{\foreignTerm{disjoint set}\xspace}
\newcommand{\pDjs}{\foreignTerm{Disjoint set}\xspace}

\newcommand{\fdotProduct}{\foreignTerm{dot product}\xspace}
\newcommand{\fDotProduct}{\foreignTerm{Dot product}\xspace}
\newcommand{\fcrossProduct}{\foreignTerm{cross product}\xspace}
\newcommand{\fCrossProduct}{\foreignTerm{Cross product}\xspace}
\newcommand{\fconvexHull}{\foreignTerm{convex hull}\xspace}
\newcommand{\fConvexHull}{\foreignTerm{Convex hull}\xspace}
\newcommand{\fgrahamScan}{\foreignTerm{graham scan}\xspace}
\newcommand{\fGrahamScan}{\foreignTerm{Graham scan}\xspace}
\newcommand{\flineSweep}{\foreignTerm{line sweep}\xspace}
\newcommand{\fLineSweep}{\foreignTerm{Line sweep}\xspace}

\newcommand{\fset}{\foreignTerm{set}\xspace}
\newcommand{\fSet}{\foreignTerm{Set}\xspace}
\newcommand{\fprefixSum}{\foreignTerm{prefix sum}\xspace}
\newcommand{\fPrefixSum}{\foreignTerm{Prefix sum}\xspace}
\newcommand{\ffenwickTree}{\foreignTerm{fenwick tree}\xspace}
\newcommand{\fFenwickTree}{\foreignTerm{Fenwick tree}\xspace}
\newcommand{\frangeSumQuery}{\foreignTerm{range sum query}\xspace}
\newcommand{\fRangeSumQuery}{\foreignTerm{Range sum query}\xspace}
\newcommand{\fquery}{\foreignTerm{query}\xspace}
\newcommand{\fQuery}{\foreignTerm{Query}\xspace}
\newcommand{\fsegmentTree}{\foreignTerm{segment tree}\xspace}
\newcommand{\fSegmentTree}{\foreignTerm{Segment tree}\xspace}
\newcommand{\fbinaryTree}{\foreignTerm{binary tree}\xspace}
\newcommand{\fBinaryTree}{\foreignTerm{Binary tree}\xspace}
\newcommand{\flazyPropagation}{\foreignTerm{lazy propagation}\xspace}
\newcommand{\fLazyPropagation}{\foreignTerm{Lazy propagation}\xspace}
\newcommand{\fsparseTable}{\foreignTerm{sparse table}\xspace}
\newcommand{\fSparseTable}{\foreignTerm{Sparse table}\xspace}

\newcommand{\ftrail}{\foreignTerm{trail}\xspace}
\newcommand{\fTrail}{\foreignTerm{Trail}\xspace}
\newcommand{\feulerTour}{\foreignTerm{euler tour}\xspace}
\newcommand{\fEulerTour}{\foreignTerm{Euler tour}\xspace}
\newcommand{\feulerTourTree}{\foreignTerm{euler tour tree}\xspace}
\newcommand{\fEulerTourTree}{\foreignTerm{Euler tour tree}\xspace}

\newcommand{\fmaxflow}{\foreignTerm{maximum flow}\xspace}
\newcommand{\fMaxflow}{\foreignTerm{Maximum flow}\xspace}
\newcommand{\fmincut}{\foreignTerm{minimum cut}\xspace}
\newcommand{\fMincut}{\foreignTerm{Minimum cut}\xspace}
\newcommand{\fflow}{\foreignTerm{flow}\xspace}
\newcommand{\fFlow}{\foreignTerm{Flow}\xspace}
\newcommand{\fsource}{\foreignTerm{source}\xspace}
\newcommand{\fSource}{\foreignTerm{Source}\xspace}
\newcommand{\fsink}{\foreignTerm{sink}\xspace}
\newcommand{\fSink}{\foreignTerm{Sink}\xspace}
\newcommand{\fbackEdge}{\foreignTerm{back-edge}\xspace}
\newcommand{\fBackEdge}{\foreignTerm{Back-edge}\xspace}
\newcommand{\fresidualCapacity}{\foreignTerm{residual capacity}\xspace}
\newcommand{\fResidualCapacity}{\foreignTerm{Residual capacity}\xspace}
\newcommand{\fbottleneck}{\foreignTerm{bottleneck}\xspace}
\newcommand{\fBottleneck}{\foreignTerm{Bottleneck}\xspace}
\newcommand{\faugmentingPath}{\foreignTerm{augmenting path}\xspace}
\newcommand{\fAugmentingPath}{\foreignTerm{Augmenting path}\xspace}


\title{Struktur Data Lanjutan}
\author{Tim Olimpiade Komputer Indonesia}
\date{}

\begin{document}

\begin{frame}
\titlepage
\end{frame}

\begin{frame}
\frametitle{Pendahuluan}
Melalui dokumen ini, kalian akan:
\begin{itemize}
  \item Memahami penggunaan C++ set/map
  \item Memahami konsep Fenwick Tree
  \item Memahami konsep Segment Tree
  \item Memahami konsep Sparse Table
\end{itemize}
\end{frame}

\begin{frame}[fragile]
\frametitle{C++ set}
\begin{itemize}
  \item Pada C++ STL (Standard Template Library), terdapat tipe data \lstinline{std::set<T>} yang dapat digunakan untuk menyimpan sebuah himpunan (\fset) \lstinline{T}.
  \begin{itemize}
    \item Sebagai contoh, \lstinline{set<int>} merupakan himpunan \lstinline{int}.
  \end{itemize}
  \item Untuk menggunakan tipe data ini, kita harus menambahkan \lstinline{include <set>}.
  \item Pada umumnya, \lstinline{set} akan menyimpan objek secara terurut menaik.
  \begin{itemize}
    \item Perilaku ini dapat diubah menggunakan \foreignTerm{custom comparator}. Sebagai contoh, pada C++11:
\begin{lstlisting}
auto cmp = [](int a, int b) { ... };
set<int, decltype(cmp)> s(cmp);
\end{lstlisting}
  \end{itemize}
  \item \lstinline{set} diimplementasikan menggunakan \foreignTerm{self balancing binary tree}\xspace, yang akan dibahas pada beberapa materi selanjutnya.
\end{itemize}
\end{frame}

\begin{frame}
\frametitle{Operasi C++ set}
\begin{itemize}
  \item Beberapa fungsi/operasi umum yang sering digunakan pada C++ \lstinline{set}.
  \begin{itemize}
    \item \lstinline{set::insert(x)} menambahkan elemen \lstinline{x} pada \lstinline{set} dalam waktu $O(\log N)$.
    \item \lstinline{set::erase(x)} menghapus elemen \lstinline{x} pada \lstinline{set} dalam waktu $O(\log N)$.
    \item \lstinline{set::size()} mengembalikan banyaknya elemen pada \lstinline{set} dalam waktu $O(1)$.
    \item \lstinline{set::count(x)} mengembalikan banyaknya elemen $x$ pada \lstinline{set} dalam waktu $O(\log N)$ (antara $0$ atau $1$).
    \item \lstinline{set::clear()} menghapus seluruh elemen \lstinline{set} dalam waktu $O(N)$.
  \end{itemize}
  dengan $N$ adalah banyaknya elemen pada \lstinline{set}.
\end{itemize}
\end{frame}

\begin{frame}
\frametitle{Set Iterator}
\begin{itemize}
  \item \lstinline{set<int>::iterator} adalah tipe data penunjuk sebuah elemen pada \lstinline{set}.
  \begin{itemize}
    \item Sebagai contoh, operasi \lstinline{set<int>::iterator it = s.begin();} akan membuat iterator \lstinline{it} menunjuk pada elemen pertama pada himpunan \lstinline{s} (jika \lstinline{s} tidak kosong).
    \item Kemudian operasi \lstinline{it++} akan membuat iterator \lstinline{it} menunjuk pada elemen kedua pada himpunan \lstinline{s} (jika \lstinline{s} berisi setidaknya dua bilangan).
    \item Untuk mendapatkan elemen yang ditunjuk, kita dapat menggunakan \lstinline{*it}.
  \end{itemize}
\end{itemize}
\end{frame}

\begin{frame}
\frametitle{Set Iterator (lanj.)}
\begin{itemize}
  \item Beberapa fungsi/operasi umum yang sering digunakan pada C++ \lstinline{set} yang mengembalikan \lstinline{set<int>::iterator} dalam waktu $O(\log N)$.
  \begin{itemize}
    \item \lstinline{set::lower_bound(x)} mengembalikan penunjuk elemen terkecil yang tidak lebih kecil dari \lstinline{x} dalam waktu $O(\log N)$. Jika tidak ada elemen yang memenuhi, maka \lstinline{set::lower_bound(x)} akan mengembalikan \lstinline{set::end()}.
    \item \lstinline{set::upper_bound(x)} mengembalikan penunjuk elemen terkecil yang lebih besar dari \lstinline{x} dalam waktu $O(\log N)$. Jika tidak ada elemen yang memenuhi, maka \lstinline{set::upper_bound(x)} akan mengembalikan \lstinline{set::end()}.
    \item \lstinline{set::find(x)} mengembalikan penunjuk elemen yang bernilai \lstinline{x} dalam waktu $O(\log N)$. Jika tidak ada elemen yang memenuhi, maka \lstinline{set::find(x)} akan mengembalikan \lstinline{set::end()}.
  \end{itemize}
\end{itemize}
\end{frame}

\begin{frame}
\frametitle{Operasi C++ multiset}
\begin{itemize}
  \item C++ \lstinline{std::multiset<T>} merupakan tipe data yang mirip dengan \lstinline{set} namun dapat menyimpan beberapa elemen yang sama.
  \item Beberapa perbedaan antara C++ \lstinline{multiset} dan \lstinline{set}
  \begin{itemize}
    \item \lstinline{multiset::count(x)} dapat mengembalikan bilangan bulat lebih besar dari $1$.
    \item \lstinline{multiset::erase(x)} menghapus seluruh elemen \lstinline{x} pada \lstinline{multiset}. Jika kita ingin menghapus hanya satu elemen \lstinline{x}, gunakan \lstinline{multiset::erase(multiset::find(x))}.
  \end{itemize}
\end{itemize}
\end{frame}

\begin{frame}
\frametitle{C++ map}
\begin{itemize}
  \item C++ \lstinline{std::map<K, V>} merupakan tipe data yang digunakan untuk menyimpan pemetaan dari tipe data \lstinline{K} ke tipe data \lstinline{V}.
  \begin{itemize}
    \item Sebagai contoh, \lstinline{map<string, int>} merupakan pemetaan dari \lstinline{string} ke \lstinline{int}.
    \item Contoh penggunaan tipe data ini adalah untuk menyimpan pemetaan dari nama murid ke nilai ujian, dan mengakses nilai ujian seorang murid dalam waktu cepat.
  \end{itemize}
  \item Untuk menggunakan tipe data ini, kita harus menambahkan \lstinline{include <map>}.
  \item \lstinline{map} juga diimplementasikan menggunakan \foreignTerm{self balancing binary tree}\xspace.
\end{itemize}
\end{frame}

\begin{frame}
\frametitle{Operasi C++ map}
\begin{itemize}
  \item Misalkan kita memiliki \lstinline{map<K, V> myMap;}.
  \item Beberapa fungsi/operasi umum yang sering digunakan pada C++ \lstinline{map}.
  \begin{itemize}
    \item \lstinline{myMap[k]} mengakses nilai pemetaan \lstinline{k} dalam waktu $O(\log N)$.
    \item \lstinline{myMap[k] = v} menentukan atau mengganti pemetaan dari \lstinline{k} menjadi ke \lstinline{v} dalam waktu $O(\log N)$.
    \item \lstinline{myMap.erase(k)} menghapus nilai pemetaan $k$ dalam waktu $O(\log N)$.
    \item \lstinline{myMap.count(k)} mengembalikan $1$ jika terdapat nilai pemetaan $k$, atau $0$ jika tidak, dalam waktu $O(\log N)$.
    \item \lstinline{myMap.clear()} menghapus seluruh pemetaan dalam waktu $O(N)$.
  \end{itemize}
  dengan $N$ adalah banyaknya elemen pada \lstinline{myMap}.
\end{itemize}
\end{frame}

\begin{frame}
\frametitle{Persoalan Range Sum Query}
\begin{itemize}
  \item Persoalan Range Sum Query adalah persoalan menghitung jumlah elemen berurutan pada sebuah \farray $A$ berukuran $N$.
  \begin{itemize}
    \item Pada umumnya, diberikan $Q$ pertanyaan yang merepresentasikan sebuah \fsubarray.
  \end{itemize}
  \item Jika \farray yang diberikan tidak dapat berubah, persoalan ini dapat diselesaikan menggunakan \fprefixSum yang menjawab satu pertanyaan dalam waktu $O(1)$.
  \item Jika \farray yang diberikan dapat berubah, persoalan ini dapat diselesaikan menggunakan \ffenwickTree yang menjawab satu pertanyaan dalam waktu $O(\log N)$ dan mengganti satu elemen \farray dalam waktu $O(\log N)$.
\end{itemize}
\end{frame}

\begin{frame}
\frametitle{Fenwick Tree}
\begin{itemize}
  \item \newTerm{Fenwick Tree} (sering disebut juga \newTerm{Binary Indexed Tree}) adalah data struktur yang menyimpan sebuah \farray berukuran $N$ dengan indeks $1$ sampai $N$ dengan setiap elemennya menyimpan jumlah elemen berurutan pada \farray $A$.
  \begin{itemize}
    \item $BIT[j]$ menyimpan jumlah elemen $\sum_{i=j - LSBIT(j) + 1}^{j} A[i]$, dengan $LSBIT(j)$ adalah nilai dari \lstinline{j & (-j)} pada C++. Sebagai contoh,
    \begin{itemize}
      \item $LSBIT(1) = 1, BIT[1] = A[1]$,
      \item $LSBIT(2) = 2, BIT[2] = A[1] + A[2]$,
      \item $LSBIT(3) = 1, BIT[3] = A[3]$,
      \item $LSBIT(4) = 4, BIT[4] = A[1] + A[2] + A[3] + A[4]$, dan
      \item $LSBIT(6) = 2, BIT[6] = A[5] + A[6]$.
    \end{itemize}
  \end{itemize}
\end{itemize}
\end{frame}

\begin{frame}
\frametitle{Fenwick Tree (lanj.)}
\begin{itemize}
  \item Dengan \ffenwickTree, kita dapat menghitung nilai $\sum_{i=1}^{x} A[i]$ dalam $O(\log N)$.
  \begin{itemize}
    \item Sebagai contoh, menghitung nilai $A[1] + A[2] + A[3] + A[4] + A[5] + A[6]$ dapat disederhanakan menjadi $(A[1] + A[2] + A[3] + A[4]) + (A[5] + A[6]) = BIT[4] + BIT[6]$.
  \end{itemize}
  \item Secara umum, $\sum_{i=1}^{x}$ dapat dihitung menggunakan rumus berikut:
  \begin{itemize}
    \item $0$, jika $x = 0$,
    \item $BIT[x] + \sum_{i=1}^{x - LSBIT(x)}$, jika $x > 0$.
  \end{itemize}
\end{itemize}
\end{frame}

\begin{frame}
\frametitle{Fenwick Tree (lanj.)}
\begin{itemize}
  \item Jika nilai $A[i]$ berubah, maka kita harus memperbaharui semua nilai $BIT[j]$ yang memenuhi $j - LSBIT(j) < i \leq j$. Terdapat $O(\log N)$ nilai yang harus diperbaharui.
  \begin{itemize}
    \item Sebagai contoh, jika nilai $A[5]$ berubah, maka kita harus memperbaharui nilai $BIT[5], BIT[6], BIT[8], BIT[16], \cdots$.
  \end{itemize}
  \item Secara umum, jika nilai $A[i]$ berubah menjadi $A[i] + \delta$, kita dapat memanggil fungsi \lstinline{update(i)} yang melakukan hal berikut jika $i \leq N$:
  \begin{itemize}
    \item Perbaharui nilai $BIT[i]$ menjadi $BIT[i] + \delta$.
    \item Panggil fungsi \lstinline{update(i + LSBIT(i)}.
  \end{itemize}
\end{itemize}
\end{frame}

\begin{frame}
\frametitle{Segment Tree}
\begin{itemize}
  \item TBA
\end{itemize}
\end{frame}

\begin{frame}
\frametitle{Sparse Tree}
\begin{itemize}
  \item TBA
\end{itemize}
\end{frame}

\end{document}
