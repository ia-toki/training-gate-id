\documentclass{beamer}
\usetheme{tokitex}

\usepackage{tikz}
\usepackage{graphics}
\usepackage{multirow}
\usepackage{tabto}
\usepackage{xspace}
\usepackage{amsmath}
\usepackage{hyperref}

\usepackage{tikz}
\usepackage{clrscode3e}

\usepackage[english,bahasa]{babel}
\newtranslation[to=bahasa]{Section}{Bagian}
\newtranslation[to=bahasa]{Subsection}{Subbagian}

\usepackage{listings, lstautogobble}
\usepackage{color}

\definecolor{dkgreen}{rgb}{0,0.6,0}
\definecolor{gray}{rgb}{0.5,0.5,0.5}
\definecolor{mauve}{rgb}{0.58,0,0.82}

\lstset{frame=tb,
  language=pascal,
  aboveskip=1mm,
  belowskip=1mm,
  showstringspaces=false,
  columns=fullflexible,
  keepspaces=true,
  basicstyle={\small\ttfamily},
  numbers=none,
  numberstyle=\tiny\color{gray},
  keywordstyle=\color{blue},
  commentstyle=\color{dkgreen},
  stringstyle=\color{mauve},
  breaklines=true,
  breakatwhitespace=true,
  autogobble=true
}

\usepackage{caption}
\captionsetup[figure]{labelformat=empty}

\newcommand{\progTerm}[1]{\textbf{#1}}
\newcommand{\foreignTerm}[1]{\textit{#1}}
\newcommand{\newTerm}[1]{\alert{\textbf{#1}}}
\newcommand{\emp}[1]{\alert{#1}}
\newcommand{\statement}[1]{"#1"}

% Getting tired of writing \foreignTerm all the time
\newcommand{\farray}{\foreignTerm{array}\xspace}
\newcommand{\fArray}{\foreignTerm{Array}\xspace}
\newcommand{\foverhead}{\foreignTerm{overhead}\xspace}
\newcommand{\fOverhead}{\foreignTerm{Overhead}\xspace}
\newcommand{\fsubarray}{\foreignTerm{subarray}\xspace}
\newcommand{\fSubarray}{\foreignTerm{Subarray}\xspace}
\newcommand{\fbasecase}{\foreignTerm{base case}\xspace}
\newcommand{\fBasecase}{\foreignTerm{Base case}\xspace}
\newcommand{\ftopdown}{\foreignTerm{top down}\xspace}
\newcommand{\fTopdown}{\foreignTerm{Top down}\xspace}
\newcommand{\fbottomup}{\foreignTerm{bottom up}\xspace}
\newcommand{\fBottomup}{\foreignTerm{Bottom up}\xspace}
\newcommand{\fpruning}{\foreignTerm{pruning}\xspace}
\newcommand{\fPruning}{\foreignTerm{Pruning}\xspace}

\newcommand{\fgraph}{\foreignTerm{graph}\xspace}
\newcommand{\fGraph}{\foreignTerm{Graph}\xspace}
\newcommand{\fnode}{\foreignTerm{node}\xspace}
\newcommand{\fNode}{\foreignTerm{Node}\xspace}
\newcommand{\fedge}{\foreignTerm{edge}\xspace}
\newcommand{\fEdge}{\foreignTerm{Edge}\xspace}
\newcommand{\fdegree}{\foreignTerm{degree}\xspace}
\newcommand{\fDegree}{\foreignTerm{Degree}\xspace}
\newcommand{\fadjacencylist}{\foreignTerm{adjacency list}\xspace}
\newcommand{\fAdjacencylist}{\foreignTerm{Adjacency list}\xspace}
\newcommand{\fadjacencymatrix}{\foreignTerm{adjacency matrix}\xspace}
\newcommand{\fAdjacencymatrix}{\foreignTerm{Adjacency matrix}\xspace}
\newcommand{\fedgelist}{\foreignTerm{edge list}\xspace}
\newcommand{\fEdgelist}{\foreignTerm{Edge list}\xspace}
\newcommand{\flist}{\foreignTerm{list}\xspace}
\newcommand{\fList}{\foreignTerm{List}\xspace}
\newcommand{\fgraphtraversal}{\foreignTerm{graph traversal}\xspace}
\newcommand{\fGraphtraversal}{\foreignTerm{Graph traversal}\xspace}
\newcommand{\ftree}{\foreignTerm{tree}\xspace}
\newcommand{\fTree}{\foreignTerm{Tree}\xspace}
\newcommand{\fsubtree}{\foreignTerm{subtree}\xspace}
\newcommand{\fSubtree}{\foreignTerm{Subtree}\xspace}

\newcommand{\fDivideAndConquer}{\foreignTerm{Divide and conquer}\xspace}
\newcommand{\fdivideAndConquer}{\foreignTerm{divide and conquer}\xspace}
\newcommand{\fMergeSort}{\foreignTerm{Merge sort}\xspace}
\newcommand{\fmergeSort}{\foreignTerm{merge sort}\xspace}
\newcommand{\fQuickSort}{\foreignTerm{Quicksort}\xspace}
\newcommand{\fquickSort}{\foreignTerm{quicksort}\xspace}
\newcommand{\fpivot}{\foreignTerm{pivot}\xspace}
\newcommand{\fPivot}{\foreignTerm{Pivot}\xspace}
\newcommand{\fbruteForce}{\foreignTerm{brute force}\xspace}
\newcommand{\fBruteForce}{\foreignTerm{Brute force}\xspace}
\newcommand{\fCompleteSearch}{\foreignTerm{complete search}\xspace}
\newcommand{\fExhaustiveSearch}{\foreignTerm{exhaustive search}\xspace}
\newcommand{\fBinarySearch}{\foreignTerm{binary search}\xspace}
\newcommand{\fGreedy}{\foreignTerm{greedy}\xspace}
\newcommand{\fGreedyChoice}{\foreignTerm{greedy choice}\xspace}

\newcommand{\pheap}{\foreignTerm{heap}\xspace}
\newcommand{\pHeap}{\foreignTerm{Heap}\xspace}
\newcommand{\pBinaryHeap}{\foreignTerm{Binary Heap}\xspace}
\newcommand{\pbinaryHeap}{\foreignTerm{binary heap}\xspace}
\newcommand{\pHeapSort}{\foreignTerm{Heap Sort}\xspace}
\newcommand{\pdjs}{\foreignTerm{disjoint set}\xspace}
\newcommand{\pDjs}{\foreignTerm{Disjoint set}\xspace}

\title{Divide and Conquer Lanjutan}
\author{Tim Olimpiade Komputer Indonesia}
\date{}

\begin{document}

\begin{frame}
\titlepage
\end{frame}

\begin{frame}
\frametitle{Pendahuluan}
Melalui dokumen ini, kalian akan:
\begin{itemize}
  \item Memahami konsep meet in the middle
  \item Memahami konsep perpangkatan matriks
\end{itemize}
~\newline
Mulai dari bab ini, seluruh kode program akan dituliskan dalam bahasa pseudo-C++.
\end{frame}

\begin{frame}
\frametitle{Meet in the Middle}
\begin{itemize}
  \item Teknik \newTerm{Meet in the Middle} digunakan dengan membagi masalah menjadi dua submasalah.
  \item Pencarian lengkap akan dilakukan untuk setiap submasalah dan memeriksa apakah terdapat irisan pada kedua pencarian lengkap tersebut.
  \item Contoh soal: Diberikan suatu himpunan $S$ berisi $N$ bilangan. Tentukan apakah terdapat subhimpunan dari $S$ yang berjumlah $0$.
  \item Tentunya terdapat solusi menggunakan pencarian penuh dalam waktu $O(2^N)$, namun kita akan menggunakan solusi yang lebih cepat.
\end{itemize}
\end{frame}

\begin{frame}
\frametitle{Meet in the Middle (lanj.)}
\begin{itemize}
  \item Soal ini dapat diselesaikan dengan membagi himpunan $S$ menjadi dua subhimpunan $A$ dan $B$ sedemikian sehingga $|A| = \floor{\frac{N}{2}}$ dan $|B| = \ceil{\frac{N}{2}}$.
  \item Terdapat subhimpunan dari $S$ yang berjumlah $0$ jika dan hanya jika terdapat subhimpunan $A' \subseteq A$ dan subhimpunan $B' \subseteq B$ sedemikian sehingga jumlah seluruh bilangan pada $A'$ adalah negasi dari jumlah seluruh bilangan pada $B'$.
  \item Untuk setiap subhimpunan $B' \subseteq B$, kita dapat menyimpan jumlah seluruh bilangannya dalam \farray $P(B)$ dan mengurutkannya.
  \item Untuk setiap subhimpunan $A' \subseteq A$, kita dapat memeriksa apakah negasi dari jumlah seluruh bilangannnya terdapat pada \farray$P(B)$ menggunakan \fbinarySearch.
  \item Kompleksitas dari solusi ini adalah $O(2^{\floor{\frac{N}{2}}} \times \log(2^{\ceil{\frac{N}{2}}}))$.
\end{itemize}
\end{frame}

\begin{frame}
\frametitle{Perpangkatan Matriks}
\begin{itemize}
  \item Teknik perpangkatan matriks dapat menghitung nilai dari matriks $M^K$ untuk sebuah matriks $M$ berukuran $N \times N$ dalam waktu $O(N^3 \times \log(K))$.
  \item Teknik ini sebenarnya serupa dengan perpangkatan bilangan bulat.
\end{itemize}
\[
    M^x= 
\begin{dcases}
    M,                                    & \text{jika } x = 1\\
    (M^{\frac{x}{2}})^2,                  & \text{jika } x \text{ bernilai genap}\\
    (M^{\floor{\frac{x}{2}}})^2 \times M, & \text{jika tidak}
\end{dcases}
\]
\end{frame}

\begin{frame}[fragile]
\frametitle{Perpangkatan Matriks (lanj.)}
Kode di bawah ini dapat digunakan untuk menghitung perpangkatan tipe data apapun.
\newline
\begin{lstlisting}
template<typename T>
T exp(T M, int K) {
  if (K == 1) {
    return M;
  }
  T res = exp(M, K >> 1);
  res = res * res;
  if (K & 1) {
    res = res * M;
  }
  return res;
}
\end{lstlisting}
\end{frame}

\begin{frame}[fragile]
\frametitle{Perpangkatan Matriks (lanj.)}
Untuk menghitung perpangkatan matriks menggunakan kode pada halaman sebelumnya, maka kita dapat mendefinisikan tipe data matriks dan fungsi perkalian.
\begin{lstlisting}
template<typename T> struct Matrix {
  vector<vector<T>> M;

  Matrix operator*(const Matrix& other) {
    Matrix res = (Matrix){vector<vector<T>>(
        M.size(), vector<T>(other.M[0].size(), 0))};

    for (int i = 0; i < M.size(); ++i) {
      for (int j = 0; j < other.M[0].size(); ++j) {
        for (int k = 0; k < M[0].size(); ++k) {
          res.M[i][j] += M[i][k] * other.M[k][j];
        }
      }
    }

    return res;
  }
};
\end{lstlisting}
\end{frame}

\begin{frame}
\frametitle{Perpangkatan Matriks: Fibonaci}
\begin{itemize}
  \item Salah satu penggunaan perpangkatan matriks adalah untuk menghitung $f_N$ (bilangan fibonaci ke-$N$) dalam $O(\log(N))$.
  \item Untuk seluruh bilangan bulat positif $n$, perhatikan bahwa $f_{n+2} = f_n + f_{n+1}$, sehingga
\[
  \begin{bmatrix} f_n & f_{n+1} \end{bmatrix} \times
  \begin{bmatrix} 0 & 1 \\ 1 & 1  \end{bmatrix} =
  \begin{bmatrix} f_{n+1} & f_{n+2} \end{bmatrix}
\]
  \item Sehingga, untuk setiap bilangan bulat positif $k$
\[
  \begin{bmatrix} f_n & f_{n+1} \end{bmatrix} \times
  \begin{bmatrix} 0 & 1 \\ 1 & 1  \end{bmatrix}^k =
  \begin{bmatrix} f_{n+k} & f_{n+k} \end{bmatrix}
\]
\end{itemize}
\end{frame}

\begin{frame}
\frametitle{Perpangkatan Matriks: Fibonaci (lanj.)}
\begin{itemize}
  \item, Sehingga, nilai $f_N$ dapat diperoleh dengan menghitung nilai
  \[
    \begin{bmatrix} f_1 & f_2 \end{bmatrix} \times
    \begin{bmatrix} 0 & 1 \\ 1 & 1  \end{bmatrix}^{N - 1} =
    \begin{bmatrix} f_N & f_{N+1} \end{bmatrix}
  \]
  \item Nilai $\begin{bmatrix} 0 & 1 \\ 1 & 1  \end{bmatrix}^{N - 1}$ dapat dihitung menggunakan perpangkatan matriks dalam $O(\log N)$.
\end{itemize}
\end{frame}

\end{document}
