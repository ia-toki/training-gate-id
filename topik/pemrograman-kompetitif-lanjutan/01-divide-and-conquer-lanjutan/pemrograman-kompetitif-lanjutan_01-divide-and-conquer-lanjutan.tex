\documentclass{beamer}
\usetheme{tokitex}

\usepackage{tikz}
\usepackage{graphics}
\usepackage{multirow}
\usepackage{tabto}
\usepackage{xspace}
\usepackage{amsmath}
\usepackage{hyperref}

\usepackage{tikz}
\usepackage{clrscode3e}

\usepackage[english,bahasa]{babel}
\newtranslation[to=bahasa]{Section}{Bagian}
\newtranslation[to=bahasa]{Subsection}{Subbagian}

\usepackage{listings, lstautogobble}
\usepackage{color}

\definecolor{dkgreen}{rgb}{0,0.6,0}
\definecolor{gray}{rgb}{0.5,0.5,0.5}
\definecolor{mauve}{rgb}{0.58,0,0.82}

\lstset{frame=tb,
  language=pascal,
  aboveskip=1mm,
  belowskip=1mm,
  showstringspaces=false,
  columns=fullflexible,
  keepspaces=true,
  basicstyle={\small\ttfamily},
  numbers=none,
  numberstyle=\tiny\color{gray},
  keywordstyle=\color{blue},
  commentstyle=\color{dkgreen},
  stringstyle=\color{mauve},
  breaklines=true,
  breakatwhitespace=true,
  autogobble=true
}

\usepackage{caption}
\captionsetup[figure]{labelformat=empty}

\newcommand{\progTerm}[1]{\textbf{#1}}
\newcommand{\foreignTerm}[1]{\textit{#1}}
\newcommand{\newTerm}[1]{\alert{\textbf{#1}}}
\newcommand{\emp}[1]{\alert{#1}}
\newcommand{\statement}[1]{"#1"}

% Getting tired of writing \foreignTerm all the time
\newcommand{\farray}{\foreignTerm{array}\xspace}
\newcommand{\fArray}{\foreignTerm{Array}\xspace}
\newcommand{\foverhead}{\foreignTerm{overhead}\xspace}
\newcommand{\fOverhead}{\foreignTerm{Overhead}\xspace}
\newcommand{\fsubarray}{\foreignTerm{subarray}\xspace}
\newcommand{\fSubarray}{\foreignTerm{Subarray}\xspace}
\newcommand{\fbasecase}{\foreignTerm{base case}\xspace}
\newcommand{\fBasecase}{\foreignTerm{Base case}\xspace}
\newcommand{\ftopdown}{\foreignTerm{top down}\xspace}
\newcommand{\fTopdown}{\foreignTerm{Top down}\xspace}
\newcommand{\fbottomup}{\foreignTerm{bottom up}\xspace}
\newcommand{\fBottomup}{\foreignTerm{Bottom up}\xspace}
\newcommand{\fpruning}{\foreignTerm{pruning}\xspace}
\newcommand{\fPruning}{\foreignTerm{Pruning}\xspace}

\newcommand{\fgraph}{\foreignTerm{graph}\xspace}
\newcommand{\fGraph}{\foreignTerm{Graph}\xspace}
\newcommand{\fnode}{\foreignTerm{node}\xspace}
\newcommand{\fNode}{\foreignTerm{Node}\xspace}
\newcommand{\fedge}{\foreignTerm{edge}\xspace}
\newcommand{\fEdge}{\foreignTerm{Edge}\xspace}
\newcommand{\fdegree}{\foreignTerm{degree}\xspace}
\newcommand{\fDegree}{\foreignTerm{Degree}\xspace}
\newcommand{\fadjacencylist}{\foreignTerm{adjacency list}\xspace}
\newcommand{\fAdjacencylist}{\foreignTerm{Adjacency list}\xspace}
\newcommand{\fadjacencymatrix}{\foreignTerm{adjacency matrix}\xspace}
\newcommand{\fAdjacencymatrix}{\foreignTerm{Adjacency matrix}\xspace}
\newcommand{\fedgelist}{\foreignTerm{edge list}\xspace}
\newcommand{\fEdgelist}{\foreignTerm{Edge list}\xspace}
\newcommand{\flist}{\foreignTerm{list}\xspace}
\newcommand{\fList}{\foreignTerm{List}\xspace}
\newcommand{\fgraphtraversal}{\foreignTerm{graph traversal}\xspace}
\newcommand{\fGraphtraversal}{\foreignTerm{Graph traversal}\xspace}
\newcommand{\ftree}{\foreignTerm{tree}\xspace}
\newcommand{\fTree}{\foreignTerm{Tree}\xspace}
\newcommand{\fsubtree}{\foreignTerm{subtree}\xspace}
\newcommand{\fSubtree}{\foreignTerm{Subtree}\xspace}

\newcommand{\fDivideAndConquer}{\foreignTerm{Divide and conquer}\xspace}
\newcommand{\fdivideAndConquer}{\foreignTerm{divide and conquer}\xspace}
\newcommand{\fMergeSort}{\foreignTerm{Merge sort}\xspace}
\newcommand{\fmergeSort}{\foreignTerm{merge sort}\xspace}
\newcommand{\fQuickSort}{\foreignTerm{Quicksort}\xspace}
\newcommand{\fquickSort}{\foreignTerm{quicksort}\xspace}
\newcommand{\fpivot}{\foreignTerm{pivot}\xspace}
\newcommand{\fPivot}{\foreignTerm{Pivot}\xspace}
\newcommand{\fbruteForce}{\foreignTerm{brute force}\xspace}
\newcommand{\fBruteForce}{\foreignTerm{Brute force}\xspace}
\newcommand{\fCompleteSearch}{\foreignTerm{complete search}\xspace}
\newcommand{\fExhaustiveSearch}{\foreignTerm{exhaustive search}\xspace}
\newcommand{\fBinarySearch}{\foreignTerm{binary search}\xspace}
\newcommand{\fGreedy}{\foreignTerm{greedy}\xspace}
\newcommand{\fGreedyChoice}{\foreignTerm{greedy choice}\xspace}

\newcommand{\pheap}{\foreignTerm{heap}\xspace}
\newcommand{\pHeap}{\foreignTerm{Heap}\xspace}
\newcommand{\pBinaryHeap}{\foreignTerm{Binary Heap}\xspace}
\newcommand{\pbinaryHeap}{\foreignTerm{binary heap}\xspace}
\newcommand{\pHeapSort}{\foreignTerm{Heap Sort}\xspace}
\newcommand{\pdjs}{\foreignTerm{disjoint set}\xspace}
\newcommand{\pDjs}{\foreignTerm{Disjoint set}\xspace}

\title{Divide and Conquer Lanjutan}
\author{Tim Olimpiade Komputer Indonesia}
\date{}

\begin{document}

\begin{frame}
\titlepage
\end{frame}

\begin{frame}
\frametitle{Pendahuluan}
Melalui dokumen ini, kalian akan:
\begin{itemize}
  \item Memahami konsep binary search the answer
  \item Memahami konsep ternary search
  \item Memahami konsep meet in the middle
  \item Memahami konsep perpangkatan matriks
\end{itemize}
~\newline
Mulai dari bab ini, seluruh kode program akan dituliskan dalam bahasa pseudo-C++.
\end{frame}

\begin{frame}
\frametitle{Binary Search the Answer}
\begin{itemize}
  \item Pada dasarnya, teknik ini menggunakan konsep binary search dalam menebak suatu jawaban, kemudian membuat decision untuk memperkecil search space kita dalam mencari jawaban yang optimal.
  \item Teknik ini sebenarnya tidak sulit untuk diaplikasikan jika kita sudah memahami teknik dasar binary search. Bahkan, beberapa orang ada yang dapat mengaplikasikan teknik ini secara tidak sadar sebelum mempelajarinya.
\end{itemize}
\end{frame}

\begin{frame}
\frametitle{Binary Search}
\begin{itemize}
  \item Mari kita lihat kembali bagaimana konsep dasar dari binary search itu sendiri.
  \item Contoh soal: Diberikan suatu \farray $A$ sepanjang $N$ yang elemennya terurut dari kecil ke besar. Tentukan index dengan value terkecil pada \farray tersebut yang nilainya $ \ge X$ dalam waktu $O(log N)$.
\end{itemize}
\end{frame}

\begin{frame}
\frametitle{Binary Search (lanj.)}
\begin{itemize}
  \item Soal ini dapat diselesaikan menggunakan binary search:
  \begin{itemize}
    \item Misalkan diberikan $N = 8$, $A = [1, 2, 3, 5, 8, 13, 21, 34]$, dan $X = 7$.
    \item Mari kita definisikan $B$ sebagai \farray sepanjang $N$ dengan $B[i]$ bernilai:
    \begin{itemize}
      \item $0$ jika $A[i] < X$, atau
      \item $1$ jika $A[i] \ge X$.
    \end{itemize}
    \item Dengan menggunakan \farray $B$, kita dapat mengubah soal yang diberikan menjadi sebagai berikut:
    \begin{itemize}
      \item Tentukan indeks pertama yang bernilai $1$ pada \farray yang hanya berisikan angka $0$ dan $1$, serta nilainya terurut dari kecil ke besar.
    \end{itemize}
  \end{itemize}
\end{itemize}
\end{frame}

\begin{frame}[fragile]
\frametitle{Binary Search (lanj.)}
\begin{itemize}
  \item Soal sebelumnya dapat diselesaikan menggunakan kode berikut.
\end{itemize}
\begin{lstlisting}
void check(int v, int X) {
  return v >= x;
}

int solve(int N, vector<int> A, int X) {
  int l = 0, r = N - 1;
  while (l < r) {
    int mid = (l + r) >> 1;
    if (check(A[i], X)) r = mid;
    else l = mid + 1;
  }
  return l;
}
\end{lstlisting}
\end{frame}

\begin{frame}
\frametitle{Binary Search the Answer}
\begin{itemize}
  \item Secara tidak langsung, dalam menyelesaikan soal tersebut, kita sudah mengaplikasikan teknik ini.
  \item Kita melakukan \fbinarySearch pada index \farray $B$, dengan index tersebut akan menjadi jawaban.
  \item Untuk memperkecil kemungkinan jawaban, kita memeriksa suatu tebakan dengan menggunakan fungsi \lstinline{check} yang membantu kita dalam membuat keputusan apakah harus memeriksa index di kiri atau kanan dari tebakan pada saat tersebut.
\end{itemize}
\end{frame}

\begin{frame}
\frametitle{Contoh Soal}
\begin{itemize}
  \item Terdapat $N$ barang dengan harga yang berbeda-beda dinomori $1$ sampai $N$.
  \item Kita diminta untuk membagi $N$ barang ini menjadi $K (2 \le K \le N)$ kelompok, sedemikian sehingga setiap barang berada dalam tepat satu kelompok, dan setiap kelompok berisi setidaknya satu barang yang memiliki nomor yang berurutan.
  \item Kemudian, $K - 1$ teman kita secara satu per satu akan memilih satu kelompok yang belum dipilih sebelumnya, dan mengambil seluruh barang yang berada di dalam kelompok tersebut.
  \item Kita akan mengambil seluruh barang di satu kelompok yang tersisa. Karena teman kita rakus, kita akan mendapatkan kelompok dengan total harga barang yang paling kecil.
  \item Tentukan total harga barang maksimum yang dapat kita peroleh dengan mengatur pembagian kelompok barang seoptimal mungkin.
\end{itemize}
\end{frame}

\begin{frame}
\frametitle{Solusi}
\begin{itemize}
  \item Kita dapat menggunakan fungsi $f$ sedemikian sehingga $f(x) = 1$ jika kita dapat mendapatkan kelompok dengan total harga barang setidaknya $x$, atau $f(x) = 0$ jika tidak.
  \item Perhatikan bahwa fungsi ini terurut dari besar ke kecil.
  \begin{itemize}
    \item Jika kita dapat mendapatkan kelompok dengan total harga barang setidaknya $x$, maka kita juga dapat mendapatkan kelompok dengan total harga barang setidaknya $x - 1$.
    \item Sebaliknya, jika kita tidak dapat mendapatkan kelompok dengan total harga barang setidaknya $x$, maka kita juga tidak dapat mendapatkan kelompok dengan total harga barang setidaknya $x + 1$.
  \end{itemize}
\end{itemize}
\end{frame}

\begin{frame}
\frametitle{Solusi (lanj.)}
\begin{itemize}
  \item Jika kita ingin mendapatkan kelompok dengan total harga barang setidaknya $x$, maka kita harus membagi kelompok barang sedemikian sehingga seluruh kelompok memiliki total harga barang setidaknya $x$.
  \begin{itemize}
    \item Sehingga, fungsi ini dapat diimplementasikan dengan memeriksa apakah kita dapat membagi kelompok barang yang memenuhi kondisi tersebut.
    \item Kita dapat menggunakan algoritme \fgreedy.
  \end{itemize}
  \item Kompleksitas solusi ini adalah $O(N \times \log(\sum H))$ dengan $\sum H$ adalah total harga barang.
\end{itemize}
\end{frame}

\begin{frame}[fragile]
\frametitle{Solusi (lanj.)}
\begin{lstlisting}
bool f(int N, int K, const vector<int>& H, int X) {
  int current_sum = 0;
  for (int i = 0; i < N; ++i) {
    if (current_sum + H[i] >= X) {
      // Dibentuk kelompok dengan total harga barang >= X.
      --K; current_sum = 0;
    } else {
      current_sum += H[i];
    }
  }
  return (K <= 0 ? 1 : 0);
}

int solve(int N, int K, vector<int> H) {
  int l = 0, r = accumulate(H.begin(), H.end(), 0);
  while (l < r) {
    int mid = (l + r) >> 1;
    if (f(N, K, H, mid)) l = mid;
    else r = mid - 1;
  }
  return l;
}
\end{lstlisting}
\end{frame}

\begin{frame}
\frametitle{Ternary Search}
\begin{itemize}
  \item Jika \farray yang dimiliki tidak terurut namun \funimodal, kita dapat menggunakan \fternarySearch untuk mencari sebuah elemen pada \farray tersebut dalam $O(\log(N))$.
  \item \fArray \newTerm{unimodal} adalah \farray yang memiliki hanya satu puncak. Lebih resminya, \fArray $A$ adalah \funimodal jika terdapat suatu indeks $x$ yang memenuhi:
  \begin{itemize}
    \item Untuk seluruh $a, b$ yang memenuhi $a < b \le x$, $A[a] < A[b]$.
    \item Untuk seluruh $a, b$ yang memenuhi $x \le a < b$, $A[a] > A[b]$.
  \end{itemize}
  \item Teknik serupa juga dapat digunakan bila \farray memiliki hanya satu lembah.
\end{itemize}
\end{frame}

\begin{frame}
\frametitle{Ternary Search (lanj.)}
\begin{itemize}
  \item Jika kita tahu bahwa puncak \farray berada di antara indeks $L$ dan $R$, maka jika kita memiliki dua indeks $M_1$ dan $M_2$ $(L < M_1 < M_2 < R)$
  \begin{itemize}
    \item Jika $A[M_1] < A[M_2]$, maka puncak \farray tidak mungkin berada di sisi kiri (antara indeks $L$ dan $M_1$). Maka kita dapat fokus melakukan pencarian di antara indeks $M_1 + 1$ dan $R$.
    \item Jika $A[M_1] > A[M_2]$, maka puncak \farray tidak mungkin berada di sisi kanan (antara indeks $M_2$ dan $R$). Maka kita dapat fokus melakukan pencarian di antara indeks $L$ dan $M_2 - 1$.
    \item Jika $A[M_1] = A[M_2]$, maka puncak \farray tidak mungkin berada di kedua sisi. Maka kita dapat fokus melakukan pencarian di antara indeks $M_1 + 1$ dan $M_2 - 1$.
  \end{itemize}
  \item Agar ukuran rentang pencarian berkurang menjadi paling banyak $\frac{2}{3}$ kali ukuran rentang pencarian sebelumnya pada seluruh kasus, kita dapat menggunakan $M_1 = L + \frac{1}{3}(R - L)$ dan $M_2 = R - \frac{1}{3}(R - L)$.
\end{itemize}
\end{frame}

\begin{frame}
\frametitle{Meet in the Middle}
\begin{itemize}
  \item Teknik \newTerm{Meet in the Middle} digunakan dengan membagi masalah menjadi dua submasalah.
  \item Pencarian lengkap akan dilakukan untuk setiap submasalah dan memeriksa apakah terdapat irisan pada kedua pencarian lengkap tersebut.
  \item Contoh soal: Diberikan suatu himpunan $S$ berisi $N$ bilangan. Tentukan apakah terdapat subhimpunan dari $S$ yang berjumlah $0$.
  \item Tentunya terdapat solusi menggunakan pencarian penuh dalam waktu $O(2^N)$, namun kita akan menggunakan solusi yang lebih cepat.
\end{itemize}
\end{frame}

\begin{frame}
\frametitle{Meet in the Middle (lanj.)}
\begin{itemize}
  \item Soal ini dapat diselesaikan dengan membagi himpunan $S$ menjadi dua subhimpunan $A$ dan $B$ sedemikian sehingga $|A| = \floor{\frac{N}{2}}$ dan $|B| = \ceil{\frac{N}{2}}$.
  \item Terdapat subhimpunan dari $S$ yang berjumlah $0$ jika dan hanya jika terdapat subhimpunan $A' \subseteq A$ dan subhimpunan $B' \subseteq B$ sedemikian sehingga jumlah seluruh bilangan pada $A'$ adalah negasi dari jumlah seluruh bilangan pada $B'$.
  \item Untuk setiap subhimpunan $B' \subseteq B$, kita dapat menyimpan jumlah seluruh bilangannya dalam \farray $P(B)$ dan mengurutkannya.
  \item Untuk setiap subhimpunan $A' \subseteq A$, kita dapat memeriksa apakah negasi dari jumlah seluruh bilangannnya terdapat pada \farray$P(B)$ menggunakan binary search.
  \item Kompleksitas dari solusi ini adalah $O(2^{\floor{\frac{N}{2}}} \times \log(2^{\ceil{\frac{N}{2}}}))$.
\end{itemize}
\end{frame}

\begin{frame}
\frametitle{Perpangkatan Matriks}
\begin{itemize}
  \item Teknik perpangkatan matriks dapat menghitung nilai dari matriks $M^K$ untuk sebuah matriks$M$ berukuran $N \times N$ dalam waktu $O(N^3 \times \log(K))$.
  \item Teknik ini sebenarnya serupa dengan perpangkatan bilangan bulat.
\end{itemize}
\[
    M^x= 
\begin{dcases}
    M,                                    & \text{jika } x = 1\\
    (M^{\frac{x}{2}})^2,                  & \text{jika } x \text{ bernilai genap}\\
    (M^{\floor{\frac{x}{2}}})^2 \times M, & \text{jika tidak}
\end{dcases}
\]
\end{frame}

\begin{frame}[fragile]
\frametitle{Perpangkatan Matriks}
\begin{lstlisting}
template<typename T>
T exp(T M, int K) {
  if (K == 1) {
    return M;
  }
  T res = exp(M, K >> 1);
  res = res * res;
  if (K & 1) {
    res = res * M;
  }
  return res;
}
\end{lstlisting}
\end{frame}

\begin{frame}
\frametitle{Perpangkatan Matriks: Fibonaci}
\begin{itemize}
  \item Salah satu penggunaan perpangkatan matriks adalah untuk menghitung $f_N$ (bilangan fibonaci ke-$N$) dalam $O(\log(N))$.
  \item Untuk seluruh bilangan bulat positif $n$, perhatikan bahwa $f_{n+2} = f_n + f_{n+1}$, sehingga
\[
  \begin{bmatrix} f_n & f_{n+1} \end{bmatrix} \times
  \begin{bmatrix} 0 & 1 \\ 1 & 1  \end{bmatrix} =
  \begin{bmatrix} f_{n+1} & f_{n+2} \end{bmatrix}
\]
  \item Sehingga, untuk setiap bilangan bulat positif $k$
\[
  \begin{bmatrix} f_n & f_{n+1} \end{bmatrix} \times
  \begin{bmatrix} 0 & 1 \\ 1 & 1  \end{bmatrix}^k =
  \begin{bmatrix} f_{n+k} & f_{n+k} \end{bmatrix}
\]
\end{itemize}
\end{frame}

\begin{frame}
\frametitle{Perpangkatan Matriks: Fibonaci (lanj.)}
\begin{itemize}
  \item, Sehingga, nilai $f_N$ dapat diperoleh dengan menghitung nilai
  \[
    \begin{bmatrix} f_1 & f_2 \end{bmatrix} \times
    \begin{bmatrix} 0 & 1 \\ 1 & 1  \end{bmatrix}^{N - 1} =
    \begin{bmatrix} f_N & f_{N+1} \end{bmatrix}
  \]
  \item Nilai $\begin{bmatrix} 0 & 1 \\ 1 & 1  \end{bmatrix}^{N - 1}$ dapat dihitung menggunakan perpangkatan matriks dalam $O(\log N)$.
\end{itemize}
\end{frame}

\end{document}
